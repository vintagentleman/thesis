\chapter{Представление грамматической информации в славянских исторических корпусах}

На сегодняшний день славянских диахронических корпусов существует крайне мало. Так, соответствующий
перечень, приведённый на сайте\footnote{\url{http://ruscorpora.ru}} Национального корпуса
русского языка (НКРЯ), состоит из всего четырёх наименований (не включая СКАТ):
\begin{inparaenum}[(1)]
    \item Регенсбургский диахронический корпус русского языка,
    \item Рукописные памятники Древней Руси,
    \item корпус "<Манускрипт"> Удмуртского государственного университета,
    \item корпус русских публицистических текстов второй половины XIX века Петрозаводского государственного университета (ввиду своей специфики он далее рассматриваться не будет);
\end{inparaenum}
помимо этого перечислены два старославянских корпуса: университетов Хельсинки и Южной Калифорнии.
Краткий обзор большинства названных корпусов (включая исторические подкорпуса самого НКРЯ) приведён
в статье \autocite{mitrenina:2014}; наше рассмотрение будет сосредоточено на реализованных в них
принципах и инструментах грамматической разметки и лемматизации. (Обсуждение технологий NLP~"---
в~т.",ч.\ морфологических модулей~"--- в зарубежных диахронических корпусах см.\ в седьмой главе
монографии \textcite[85--101]{passarotti:2010}).

\section{Исторические подкорпуса НКРЯ}

