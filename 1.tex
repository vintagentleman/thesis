\chapter{Представление грамматической информации в~славянских исторических корпусах}

На сегодняшний день славянских диахронических корпусов существует крайне мало. Так, соответствующий
перечень, приведённый на сайте Национального корпуса русского
языка\footnote{\url{http://ruscorpora.ru} (дата обр.\ \today)} (НКРЯ) в разделе "<Другие корпуса">,
включает в себя всего четыре наименования (помимо СКАТ):

\begin{compactenum}
    \item Регенсбургский диахронический корпус русского
    языка\footnote{\url{http://rhssl1.uni-regensburg.de/SlavKo/korpus/rrudi-new} (дата обр.\ \today)};
    \item Рукописные памятники Древней Руси\footnote{\url{http://www.lrc-lib.ru} (дата обр.\ \today)}
    (этот ресурс далее особо не рассматривается: собственно лингвистический компонент базы
    берестяных грамот представлен соответствующим подкорпусом НКРЯ; собрание же русских летописей
    к настоящему времени полностью интегрировано в древнерусский подкорпус);
    \item корпус "<Манускрипт">\footnote{\url{http://manuscripts.ru} (дата обр.\ \today)}
    Удмуртского государственного университета;
    \item корпус русских публицистических текстов второй половины
    XIX~в.\footnote{\url{http://smalt.karelia.ru/corpus/index.phtml} (дата обр.\ \today)}
    Петрозаводского государственного университета (не рассматривается).
\end{compactenum}

% Помимо этого перечислены два старославянских корпуса: университетов Хельсинки и Южной Калифорнии.
Краткий обзор названных корпусов (включая исторические подкорпуса самого НКРЯ), а также
корпуса "<Великие Минеи-Четьи">\footnote{\url{http://www.vmc.uni-freiburg.de/Mens/} (дата обр.\
\today)} Фрайбургского университета содержится в статье \autocite{mitrenina:2014}; наше
рассмотрение будет сосредоточено на реализованных в них принципах и инструментах грамматической
разметки и лемматизации. (Обсуждение технологий NLP~"--- в~т.",ч.\ морфологических модулей~"---
в зарубежных диахронических корпусах см.\ в \autocite[85---101]{passarotti:2010}).

\section{Национальный корпус русского языка}

\subsection{Древнерусский корпус и корпус берестяных грамот}

Грамматическая разметка древнерусского подкорпуса НКРЯ производится вручную~"--- с частичной
автоматизацией на основе прецедентных разборов \autocite[102]{mishina_pichkhadze:2015}, что
естественно ввиду значительного объёма корпуса в целом (к настоящему времени он превышает 500~тыс.\
словоупотреблений). Для разметки и прочего взаимодействия с лингвистическим обеспечением корпуса
используется специально разработанная система Morphy \autocite{pichkhadze:2015}. Что касается
корпуса берестяных грамот, то хотя его содержимое не описано в отдельных публикациях, будучи
полностью основанным на материалах сборников "<Новгородские грамоты на бересте"> и привязанным
к сайту "<Древнерусские берестяные грамоты">\footnote{\url{http://gramoty.ru} (дата обр.\
\today)},~"--- его малый объём и достаточно высокое качество аннотации (повсеместно снята омонимия;
многим словам приписана семантическая информация, автоматизации явно не поддающаяся) также позволяют
с уверенностью судить о её ручном характере. Оба корпуса размечены весьма сходным образом (аппарат
разметки последнего "<усовершенствован в связи с большим количеством фрагментарно сохранившихся
слов, а также мест, трактуемых лишь предположительно"> \autocite[227]{sichinava:2014}) и обладают
общим движком лексико-грамматического поиска.

Разработчики справедливо отмечают, что "<разбор древнерусского текста~"--- работа, не вполне
поддающаяся стандартизации">, и признают, что "<[п]о\-сколь\-ку потребность добавлять релевантные
пометы возникает при изучении древних текстов постоянно, система разметки разрабатывалась таким
образом, чтобы исследователь мог легко вводить новые признаки по собственному усмотрению">
\autocite[103---104]{mishina_pichkhadze:2015}. Как следствие, грамматическая аннотация разных
памятников, составляющих древнерусский корпус, порой обнаруживает известную степень неоднородности:
расхождения проявляются, например, в разборе глагольных форм будущего~I, которые трактуются то как
аналитические формы, то как свободные словосочетания фазовых либо модальных глаголов с инфинитивом,
и будущего~II, иногда проявляющих весьма разнообразные оттенки значения, плохо укладывающиеся
в семантику традиционного "<преждебудущего">. На грамматические данные порой наслаиваются пометы
другого рода, системность приписывания которых также вызывает некоторые вопросы: так, имена
собственные следующих классов: отчества, этнонимы, обозначения жены или вдовы по мужу~"--- в одних
текстах корпуса получают соответствующие теги, в других же их описание исчерпывается базовой
морфологической типизацией. Тем не менее, нельзя упрекнуть разметчиков в тотальной
рассогласованности их усилий: даже признаки из таких, казалось бы, периферийных групп, как
"<Комментарий">, "<В составе"> и "<Употребление"> (см.\ диалоговое окно выбора признаков на
рис.~\ref{ncr_old:1}), регулярно имеют место практически во всех текстах корпуса.

В отдельных переводных памятниках (в частности, это касается "<Александрии"> и "<Пчелы">,
воспроизводимых в корпусе по спискам XV в.)\ словоформам по возможности приписаны греческие
аналоги~"--- причём как сами оригинальные формы, так и их леммы (cм.\ рис.~\ref{ncr_old:2}).
Собственно же древнерусские леммы составляют общий для всех памятников словарь, где они представлены
в унифицированной древнерусской орфографии, которая, с одной стороны, не различает орфографические
варианты графем (например, узкие и широкие разновидности \textsc{Е} и \textsc{О}, \textsc{И}
восьмеричное и десятеричное), а с другой стороны, отражает состояние до падения и прояснения
редуцированных. Таким образом, сводный словарь лемм одновременно характеризует "<омолаживание">
и "<состаривание"> материала, лежащего в его основе; далее мы увидим, что к задаче нормализации
некодифицированной орфографии исторических языков могут быть применимы и иные подходы.

\subsection{Церковнославянский корпус}

В состав церковнославянского подкорпуса НКРЯ включены лишь те церковнославянские тексты, которые
были созданы или отредактированы в период книгопечатания,~"--- а именно в XVII---XX~вв.\
\autocite[117---118]{polyakov:2015}. Этим обстоятельством, в частности, объясняется объём корпуса,
весьма внушительный даже по меркам синхронических специализированных корпусов: в него включены
более 1250 документов, охватывающих порядка 4,7~млн словоупотреблений. Однако ручная аннотация
столь обширных массивов текстовых данных, очевидно, принципиально нереализуема.

Модуль грамматической разметки церковнославянского корпуса призван одновременно решать как задачу
лемматизации~"--- приведения словоформы к лемме и определения её постоянных признаков (части речи,
рода, вида, переходности), так и собственно грамматического анализа~"--- определения грамматических
свойств самой словоформы \autocite[250---251]{polyakov:2014}. Разметка опирается на разработанную
коллективом проекта модель словоизменения церковнославянского языка, состоящую из двух основных
компонентов:

\begin{compactenum}
    \item грамматический словарь~"--- перечень лексем с приписанными им словоизменительными
    параметрами, как то:
    \begin{inparaenum}[(1)]
        \item лемма и её варианты (при наличии),
        \item постоянные признаки лексемы,
        \item код парадигмы,
        \item краткое толкование (по необходимости);
    \end{inparaenum}
    \item грамматическая модель~"--- совокупность таблиц словоизменительных типов (парадигм),
    в которых задаются системные соотношения между множествами грамматических значений
    и соответствующих им форм, закодированные при помощи специальных кодов (индексов).
\end{compactenum}

Обе составляющие модели словоизменения не задаются априорно на основе существующих грамматических
описаний, но эмпирически и итеративно выводятся из содержимого самого корпуса. Таким образом,
разработка словаря и модели ведётся параллельно, и взаимные наработки постоянно корректируются
и согласуются между собой: с одной стороны, из корпуса постепенно извлекаются очередные наиболее
частотные слова, которым далее вручную приписываются леммы и коды парадигматических шаблонов;
с другой стороны, по мере обнаружения ранее неучтённых словоизменительных явлений в анализируемом
лексическом материале обновляется номенклатура парадигм, пополняется состав грамматических
признаков, правил морфонологических чередований и~т.",д.\ \autocite[129---130]{polyakov:2015}.

\begin{table}[t]
    \small
    \begin{tabularx}{\textwidth}{XXXXX} \toprule
        \thead{Парадигма} & \thead{N1t} & \thead{N1t*} & \thead{N1j} & \thead{N1k}   \\ \midrule
        Пример            & рабъ        & сонъ         & конь        & отрокъ        \\ \midrule
        Основа            & раб+ъ       & со*н+ъ       & кон+ь       & отро(к|ц|ч)+ъ \\ \midrule
        sg,nom            & ъ           & 2ъ           & ь           & ъ             \\ \midrule
        sg,gen            & а           & а            & я           & а             \\ \midrule
        sg,voc            & е           & е            & ю           & 3е            \\ \midrule
        pl,acc            & ы/=gen      & ы/=gen       & и/=gen      & ы/=gen        \\ \midrule
        pl,loc            & \old ѣхъ    & \old ѣхъ     & ехъ         & \old 2ѣхъ     \\ \midrule
        du,dat/ins        & ома         & ома          & ема         & ома           \\ \bottomrule
    \end{tabularx}
    \normalsize
    \caption{Фрагмент формальной записи парадигм (воспроизводится по \autocite[252]{polyakov:2014})}
\end{table}

\subsection{Старорусский корпус}

\verb|pass|

\section{Регенсбургский диахронический корпус русского языка}

\verb|pass|

\section{Великие Минеи-Четьи}

В разделе цитированной выше обзорной статьи О.",В.~Митрениной \autocite{mitrenina:2014},
посвящённом корпусу "<Великие Минеи-Четьи"> (ВМЧ), указано, что грамматической разметкой последний
не располагает и, как следствие, производить поиск по грамматическим формам не позволяет.
Аналогичные утверждения приводят и сами разработчики \autocites[349]{VMC:2012}[30]{VMC:2015},
подчёркивая, что электронное издание ВМЧ, составленных под руководством митрополита Макария
в XVI~в., прежде всего призвано выступать в качестве дополнения к дипломатическому изданию их
доселе неизданных фрагментов, выпускаемому кафедрой славянской филологии Фрайбургского университета.

Тем не менее, на момент написания данных строк в расширенной версии поискового движка на сайте
корпуса (рис.~\ref{vmc:1}) поле для ввода грамматических тегов, как ни странно, содержится. Данное
обстоятельство никак не задокументировано ни в актуальных работах по корпусу, ни в интерактивной
справке по режимам поиска на самом сайте. При наведении курсора на любую словоформу из выдачи
по любому же поисковому запросу выводится всплывающая подсказка с набором символов, в структуре
которого узнаётся нечто смутно напоминающее формат морфосинтаксического описания в спецификации
проекта MULTEXT-East \autocite{multext_east:2003}. Однако создаётся такое впечатление, что данные
символьные цепочки разбросаны по словоформам совершенно хаотично: поиск по любому подобному
"<тегу"> выдаёт результаты с самыми различными грамматическими характеристиками~"--- а порой и вовсе
не словоформы: так, во второй строке на рис.~\ref{vmc:2} подсвечена одна из выносных букв в составе
послелога \textsc{ра(ди)}.

О реальном состоянии работ по внедрению в корпус ВМЧ грамматической разметки и лемматизации нам
остаётся лишь догадываться.

\section{Манускрипт}

\verb|pass|

\section{Санкт-Петербургский корпус агиографических текстов}

Принятый в СКАТ формат грамматической аннотации был разработан выпускницей кафедры математической
лингвистики Е.",С.~Ивановой \autocite{ivanova:2006} и впоследствии уточнён Е.",Л.~Алексеевой. Он
предназначен для ручного ввода грамматических данных (в течение последнего десятилетия разметка
производилась студентами 1---2 курсов в ходе филологической практики) и представлен в виде таблиц,
где каждой словоформе приписаны соответствующие ей морфологические (в случае аналитических форм
глаголов~"--- также и некоторые синтаксические) характеристики. К настоящему времени таким образом
размечены 3 жития общим объёмом около 30~тыс.\ словоупотреблений.

Всего для внесения грамматических сведений предусмотрено 6 столбцов, однако фактическое число
и значение заполняемых позиций варьирует в зависимости от первой характеристики~"--- части речи.
Так, слова знаменательных именных частей речи (существительные, прилагательные и числительные),
а также неличные местоимения размечаются практически однотипно: для них указываются тип склонения,
падеж, число и род (в случае невозможности вывести какую-либо характеристику из контекста данная
позиция принимает значение \textit{0}); наполнение глагольных тегсетов зависит от наклонения,
в случае изъявительного~"--- ещё и от морфологического типа использованного времени (синтетического
или аналитического).

Кроме того, формат разметки призван учесть то обстоятельство, что представленные в корпусе житийные
тексты, будучи написанными на церковнославянском языке достаточно поздней редакции, отражают живые
процессы развития архаичных черт старославянской грамматики: смешение типов склонения, становление
категории одушевлённости, обособление деепричастий в особую глагольную форму и~т.",д. Для фиксации
переходных явлений подобного рода в соответствующей позиции тегсета приводятся два значения
категории, разделённые косой чертой,~"--- парадигматически ожидаемое и реальное. Например:

\begin{compactitem}
    \item тип склонения \textit{es/o} у существительного \textsc{т+ла\footnotemark} обозначает,
    что исторически его основа относится к одному из подтипов на согласный (*es), но реально
    употреблённая флексия соответствует типу *\u{o};
    \item падеж \textit{вин/род} у существительного \textsc{бга\#} показывает, что в значении
    винительного падежа здесь использован родительный, в чём проявляется категория одушевлённости;
    \item род \textit{ж/м} у причастия \textsc{блгодарr\#} свидетельствует об употреблении формы
    мужского рода вместо женского~"--- так отражается процесс образования деепричастий.
\end{compactitem}

\footnotetext{%
    Здесь и далее языковые примеры даются капителью с соблюдением транслитерационных соглашений,
    принятых в коллективе СКАТ: вышедшие из употребления буквы обозначаются при помощи символов
    латиницы, а также знака \textsc{+} (для буквы "<ять">); знак \textsc{\#} маркирует наличие
    в слове титла; выносные буквы заключаются в скобки.
}

\begin{table}[t]
    \small
    \begin{tabularx}{\textwidth}{Xp{1.5cm}p{1.5cm}p{1.5cm}p{1.5cm}p{1.5cm}p{1.5cm}} \toprule
        \textsc{житiю}         & сущ      & jo       & дат      & ед  & ср  & \\ \midrule
        \textsc{блгоwбразна\#} & прил     & o        & вин/род  & ед  & м   & \\ \midrule
        \textsc{болшiа}        & прил/ср  & м/тв     & род      & ед  & ж   & \\ \midrule
        \textsc{исполнь}       & прил/н   &          &          &     &     & \\ \midrule
        \textsc{осмьдесr(т)}   & числ     & i        & вин      & ед  & 0   & \\ \midrule
        \textsc{перваго}       & числ/п   & тв       & род      & ед  & м   & \\ \midrule
        \textsc{ми}            & мест     & личн     & 1        & дат & ед  & \\ \midrule
        \textsc{вс+мъ}         & мест     & р/скл    & дат      & мн  & м   & \\ \midrule
        \textsc{б+}            & гл       & сосл     & 3        & ед  & св  & \\ \midrule
        \textsc{сотворилъ}     & гл       & сосл     & м        & ед  & пр  & \\ \midrule
        \textsc{возвращdсr}    & гл/в     & изъяв    & н/б      & 1   & ед  & 4 \\ \midrule
        \textsc{воорdжи(в)сr}  & прич/в   & jo       & прош     & им  & ед  & м \\ \bottomrule
    \end{tabularx}
    \normalsize
    \caption{Примеры грамматической аннотации в корпусе СКАТ}
\end{table}

В отличие от тех рассмотренных выше корпусов, где также предусмотрен в первую очередь ручной ввод
грамматических данных, в формате разметки СКАТ не отведено специального поля для лемм: задача
их корректного и взаимно согласованного определения отнюдь не является тривиальной для студентов
младших курсов, и её постановка потребовала бы значительного увеличения временных затрат на проверку
студенческих работ квалифицированным специалистом коллектива. Однако главный недостаток описанной
системы разметки, на наш взгляд, заключается в том, что её содержимое изначально полностью отделено
от остальных компонентов лингвистического обеспечения корпуса~"--- в частности, от словоуказателя.
Файлы с таблицами хранятся исключительно локально; полноценный поиск по граммемам или их сочетаниям,
отбор коллокатов на основании их грамматических признаков~"--- фактически любая операция,
необходимая для исследования церковнославянской грамматики на материале СКАТ, всякий раз требует
написания специальных программ (ср.\ \autocite[24---26]{kasjanenko:2010}).

В.",А.~Алексеев в качестве одного из аспектов практической части своей магистерской диссертации
\autocite{alexeev:2011} разработал механизм внедрения грамматических данных в структуру
XML-представления текстов корпуса; кроме того, им была предпринята попытка создать полноценную среду
для работы с корпусом через интернет, свободную от обозначенных выше недостатков электронного
словоуказателя и обеспечивающую полнофункциональный поиск как по структурным частям рукописей,
так и по грамматическим тегам. Онлайн-версия\footnote{\url{http://scat.v-alexeev.ru}
(дата обр.\ \today)} данной системы доступна для работы в тестовом режиме и предоставляет доступ
к небольшому размеченному фрагменту жития Корнилия Комельского (объёмом порядка 10 листов). Пример
работы поискового движка приведён на рис.~\ref{va:1}---\ref{va:2}.

Следует признать, что данная среда отнюдь не лишена объективных недостатков. Механизм
формулировки самих поисковых запросов фактически тождественен соответствующему функционалу
на официальном сайте проекта: искать можно по вхождениям или совпадениям~"--- начальным, конечным
либо полным (строгим); однако не представляется возможным, например, поиск по маске или поиск
нескольких терминов подряд. Контекстные окна, отображаемые в поле выдачи, отчасти неинформативны
ввиду их жёсткой привязки к границам строк~"--- в результате они оказываются чересчур узкими, и
для анализа контекста искомого вхождения неизбежно приходится обращаться ко всему тексту; режим
выравнивания окон типа KWIC (\foreignlanguage{english}{Key Word in Context}) также отсутствует.
Наконец, чисто технические ограничения использованной для размещения системы интернет-площадки
не позволяют автоматически производить такие ресурсоёмкие операции, как загрузка новых рукописей
\autocite[54]{alexeev:2011}~"--- и, судя по всему, поддерживать данный ресурс в дальнейшем
В.",А.~Алексеев не намерен.

Тем не менее, важность его наработок по приведению XML-разметки СКАТ в соответствие со стандартами
Unicode и XML-TEI, по интеграции в неё грамматических данных нельзя переоценить; в ходе дальнейшего
обсуждения они будут рассмотрены более пристально.

\begin{figure}[p]
    \centering
    \includegraphics[width=\textwidth]{ncr_old_search}
    \caption{Древнерусский подкорпус НКРЯ: поисковый интерфейс}
    \label{ncr_old:1}
\end{figure}

\begin{figure}[p]
    \centering
    \includegraphics[width=\textwidth]{ncr_old_output}
    \caption{Древнерусский подкорпус НКРЯ: выдача по запросу на личные имена в звательном падеже}
    \label{ncr_old:2}
\end{figure}

\begin{figure}[p]
    \centering
    \includegraphics[width=\textwidth]{vmc_search}
    \caption{ВМЧ: поисковый интерфейс}
    \label{vmc:1}
\end{figure}

\begin{figure}[p]
    \centering
    \includegraphics[width=\textwidth]{vmc_output}
    \caption{ВМЧ: выдача по запросу на тег "<Vmpis-p">}
    \label{vmc:2}
\end{figure}

\begin{figure}[p]
    \centering
    \includegraphics{va_search}
    \caption{Система В.",А.: выдача по запросу на существительные типа склонения *en}
    \label{va:1}
\end{figure}

\begin{figure}[p]
    \centering
    \includegraphics[width=\textwidth]{va_output}
    \caption{Система В.",А.: просмотр полного контекста вхождения \textsc{камень} (КК 77~об./14)}
    \label{va:2}
\end{figure}
