\chapter{Представление грамматической информации в славянских исторических корпусах}

На сегодняшний день славянских диахронических корпусов существует крайне мало. Так, соответствующий
перечень, приведённый на сайте\footnote{\url{http://ruscorpora.ru/corpora-other.html}}
Национального корпуса русского языка (НКРЯ), состоит из всего четырёх наименований (помимо СКАТ):
\begin{inparaenum}[(1)]
    \item Регенсбургский диахронический корпус русского языка,
    \item Рукописные памятники Древней Руси,
    \item корпус "<Манускрипт"> Удмуртского государственного университета,
    \item корпус русских публицистических текстов второй половины XIX века Петрозаводского
    государственного университета (ввиду своей специфики он далее рассматриваться не будет);
\end{inparaenum}
помимо этого перечислены два старославянских корпуса: университетов Хельсинки и Южной Калифорнии.
Краткий обзор большинства названных корпусов (включая исторические подкорпуса самого НКРЯ) приведён
в статье \autocite{mitrenina:2014}; наше рассмотрение будет сосредоточено на реализованных в них
принципах и инструментах грамматической разметки и лемматизации. (Обсуждение технологий NLP~"---
в~т.",ч.\ морфологических модулей~"--- в зарубежных диахронических корпусах см.\ в монографии
\autocite[85--101]{passarotti:2010}).

\section{Исторические подкорпуса НКРЯ}

\subsection{Древнерусский корпус и корпус берестяных грамот}



\section{Регенсбургский диахронический корпус русского языка}



\section{Рукописные памятники Древней Руси}



\section{Манускрипт}



\section{СКАТ}

Принятый в СКАТ формат грамматической аннотации был разработан выпускницей кафедры математической
лингвистики Е.",С.~Ивановой \autocite{ivanova:2006} и впоследствии уточнён Е.",Л.~Алексеевой. Он
предназначен для ручного ввода грамматических данных (в течение последнего десятилетия разметка
производилась студентами 1--2 курсов в ходе филологической практики) и представлен в виде таблиц,
где каждой словоформе приписаны соответствующие ей морфологические (в случае аналитических форм
глаголов~"--- также и некоторые синтаксические) характеристики. К настоящему времени таким образом
размечены 3 жития общим объёмом около 30~тыс.\ словоупотреблений.

Всего для внесения грамматических сведений предусмотрено 6 столбцов, однако фактическое число
и значение заполняемых позиций варьирует в зависимости от первой характеристики~"--- части речи.
Так, слова знаменательных именных частей речи (существительные, прилагательные и числительные),
а также неличные местоимения размечаются практически однотипно: для них указываются тип склонения,
падеж, число и род (в случае невозможности вывести какую-либо характеристику из контекста данная
позиция принимает значение \textit{0}); наполнение глагольных тегсетов зависит от наклонения,
в случае изъявительного~"--- ещё и от морфологического типа использованного времени (синтетического
или аналитического).

Кроме того, формат разметки призван учесть то обстоятельство, что представленные в корпусе житийные
тексты, будучи написанными на церковнославянском языке достаточно поздней редакции, отражают живые
процессы развития архаичных черт старославянской грамматики: смешение типов склонения, становление
категории одушевлённости, обособление деепричастий в особую глагольную форму и~т.",д. Для фиксации
переходных явлений подобного рода в соответствующей позиции тегсета приводятся два значения
категории, разделённые косой чертой,~"--- парадигматически ожидаемое и реальное. Например:

\begin{itemize}
    \item тип склонения \textit{es/o} у существительного \textsc{т+ла\footnotemark} обозначает,
    что исторически его основа относится к одному из подтипов на согласный (*es), но реально
    употреблённая флексия соответствует типу *\u{o};
    \item падеж \textit{вин/род} у существительного \textsc{бга\#} показывает, что в значении
    винительного падежа здесь использован родительный, в чём проявляется категория одушевлённости;
    \item род \textit{ж/м} у причастия \textsc{блгодарr\#} свидетельствует об употреблении формы
    мужского рода вместо женского~"--- так отражается процесс образования деепричастий.
\end{itemize}

\footnotetext{
    Здесь и далее языковые примеры даются капителью с соблюдением транслитерационных соглашений,
    принятых в коллективе СКАТ: вышедшие из употребления буквы обозначаются при помощи символов
    латиницы, а также знака "<+"> (для буквы "<ять">); "<\#"> маркирует наличие в слове титла;
    выносные буквы заключаются в скобки.
}

\begin{table}[t]
    \small
    \begin{tabularx}{\textwidth}{Xp{1.5cm}p{1.5cm}p{1.5cm}p{1.5cm}p{1.5cm}p{1.5cm}} \toprule
        \textsc{житiю}         & сущ      & jo       & дат      & ед  & ср  & \\ \midrule
        \textsc{блгоwбразна\#} & прил     & o        & вин/род  & ед  & м   & \\ \midrule
        \textsc{болшiа}        & прил/ср  & м/тв     & род      & ед  & ж   & \\ \midrule
        \textsc{исполнь}       & прил/н   &          &          &     &     & \\ \midrule
        \textsc{осмьдесr(т)}   & числ     & i        & вин      & ед  & 0   & \\ \midrule
        \textsc{перваго}       & числ/п   & тв       & род      & ед  & м   & \\ \midrule
        \textsc{ми}            & мест     & личн     & 1        & дат & ед  & \\ \midrule
        \textsc{вс+мъ}         & мест     & р/скл    & дат      & мн  & м   & \\ \midrule
        \textsc{б+}            & гл       & сосл     & 3        & ед  & св  & \\ \midrule
        \textsc{сотворилъ}     & гл       & сосл     & м        & ед  & пр  & \\ \midrule
        \textsc{возвращdсr}    & гл/в     & изъяв    & н/б      & 1   & ед  & 4 \\ \midrule
        \textsc{воорdжи(в)сr}  & прич/в   & jo       & прош     & им  & ед  & м \\ \bottomrule
    \end{tabularx}
    \normalsize
    \caption{Примеры грамматической аннотации в корпусе СКАТ}
\end{table}

В отличие от всех рассмотренных выше корпусов, где также предусмотрен в первую очередь ручной ввод
грамматических данных, в формате разметки СКАТ не отведено специального поля для лемм: задача
их корректного и взаимно согласованного определения отнюдь не является тривиальной для студентов
младших курсов, и её постановка потребовала бы значительного увеличения временных затрат на проверку
студенческих работ квалифицированным специалистом коллектива. Однако главный недостаток описанной
системы разметки, на наш взгляд, заключается в том, что её содержимое изначально полностью отделено
от остальных компонентов лингвистического обеспечения корпуса~"--- в частности, от словоуказателя.
Файлы с таблицами хранятся исключительно локально; полноценный поиск по граммемам или их сочетаниям,
отбор коллокатов на основании их грамматических признаков~"--- фактически любая операция,
необходимая для исследования церковнославянской грамматики на материале СКАТ, всякий раз требует
написания специальных программ (ср.\ \autocite[24--26]{kasjanenko:2010}).

\begin{figure}[t]
    \centering
    \includegraphics{1_va_search}
    \caption{В.",А.~Алексеев: выдача по запросу на существительные типа склонения *en}
    \label{1:scat_1}
\end{figure}

В.",А.~Алексеев в качестве одного из аспектов практической части своей магистерской диссертации
разработал механизм внедрения грамматических данных в структуру XML-представления текстов корпуса
\autocite[50--54]{alexeev:2011}; кроме того, им была предпринята попытка создать полноценную среду
для работы с корпусом через интернет, свободную от обозначенных выше недостатков электронного
словоуказателя и обеспечивающую полнофункциональный поиск как по структурным частям рукописей,
так и по грамматическим тегам. Онлайн-версия\footnote{\url{http://scat.v-alexeev.ru}} данной
системы доступна для работы в тестовом режиме и предоставляет доступ к небольшому размеченному
фрагменту жития Корнилия Комельского (объёмом порядка 10 листов). Пример работы поискового движка
приведён на рис.~\ref{1:scat_1}--\ref{1:scat_2}.

\begin{figure}[t]
    \centering
    \includegraphics[width=\linewidth]{1_va_output}
    \caption{В.",А.~Алексеев: просмотр полного контекста вхождения \textsc{камень} (КК 77~об./14)}
    \label{1:scat_2}
\end{figure}

Следует признать, что данная среда отнюдь не лишена объективных недостатков. Механизм
формулировки самих поисковых запросов фактически тождественен соответствующему функционалу
на официальном сайте проекта: искать можно по вхождениям или совпадениям~"--- начальным, конечным
либо полным (строгим); однако не представляется возможным, например, поиск по маске или поиск
нескольких терминов подряд. Контекстные окна, отображаемые в поле выдачи, отчасти неинформативны
ввиду их жёсткой привязки к границам строк~"--- в результате они оказываются чересчур узкими, и
для анализа контекста искомого вхождения неизбежно приходится обращаться ко всему тексту; режим
выравнивания окон типа KWIC (\foreignlanguage{english}{Key Word in Context}) также отсутствует.

Складывается впечатление, что поддерживать данный ресурс в дальнейшем В.",А.~Алексеев не намерен.
Тем не менее, важность его наработок по приведению XML-разметки СКАТ в соответствие со стандартами
Unicode и TEI, по интеграции в неё грамматических данных нельзя переоценить; более пристально они
будут рассмотрены в ходе дальнейшего обсуждения.
