\chapter{Представление грамматической информации в~восточнославянских исторических корпусах}

Ассортимент современных восточнославянских диахронических корпусов\footnote{Термины "<диахронический корпус"> и "<исторический корпус"> мы употребляем как синонимы.} на сайте Национального корпуса русского языка\footnote{\url{http://ruscorpora.ru/corpora-other.html} (дата обр.\ \today)} (НКРЯ) представлен следующими наименованиями (помимо СКАТ):

\begin{compactenum}
    \item Регенсбургский диахронический корпус русского языка\footnote{\url{http://rhssl1.uni-regensburg.de/SlavKo/korpus/rrudi-new} (дата обр.\ \today)};
    \item Рукописные памятники Древней Руси\footnote{\url{http://www.lrc-lib.ru} (дата обр.\ \today). Отметим, что базы древнерусских берестяных грамот и летописей, входящие в архив данного ресурса, к настоящему времени полностью интегрированы в древнерусский сегмент НКРЯ и потому в особом рассмотрении не нуждаются.};
    \item корпус "<Манускрипт">\footnote{\url{http://manuscripts.ru} (дата обр.\ \today)};
    \item корпус русских публицистических текстов второй половины XIX~в.\footnote{\url{http://smalt.karelia.ru/corpus/index.phtml} (дата обр.\ \today). Ввиду своей временной специфики он выбивается из общего ряда восточнославянских исторических корпусов и потому останется за рамками настоящего обзора.}
\end{compactenum}

В своём большинстве перечисленные корпуса (включая исторические подкорпуса самого НКРЯ, а также корпус "<Великие Минеи Четьи">\footnote{\url{http://www.vmc.uni-freiburg.de/Mens/} (дата обр.\ \today). Грамматическая разметка в нём отсутствует, и потому далее рассматриваться он не будет.}) обзорно освещены в статье \autocite{mitrenina:2014}: каждый корпус охарактеризован с точки зрения характера текстов, лежащих в их основе, суммарного объёма (актуального на момент написания статьи), возможностей поиска, наличия морфосинтаксической аннотации и иных релевантных признаков. Поэтому в нижеследующем обсуждении мы не будем затрагивать общие места и рассмотрим описанные О.",В.~Митрениной корпуса в интересующем нас аспекте реализованных в них принципов и инструментов грамматической разметки и лемматизации.

\section{Национальный корпус русского языка}

\subsection{Древнерусский корпус и корпус берестяных грамот}

Эти подкорпуса НКРЯ будут рассмотрены совместно: исторически их разработка ведётся в рамках единой программы президиума РАН "<Корпусная лингвистика"> \autocite[226]{sichinava:2014}, а составляющие их тексты написаны на одном (пусть и значительно разрозненном) языке и датируются сходными хронологическими периодами~"--- XI--XIV и XI--XV вв.\ соответственно. Грамматическая разметка текстов, объединённых в древнерусский корпус, ведётся ещё с середины 2000-х~гг.\ в рамках проекта "<Рукописные памятники Древней Руси">; содержимое же корпуса берестяных грамот основывается на материалах сборников "<Новгородские грамоты на бересте"> и размечается для сайта "<Древнерусские берестяные грамоты">\footnote{\url{http://gramoty.ru} (дата обр.\ \today)}.

Разметка обоих корпусов производится вручную со снятием грамматической омонимии. Аннотация древнерусского корпуса ввиду его существенно большего объёма (на сегодняшний день он составляет около 500~тыс.\ словоупотреблений~"--- в противовес 20~тыс.\ в случае корпуса берестяных грамот) частично автоматизирована при помощи прецедентных разборов \autocite[102]{mishina_pichkhadze:2015}; аппарат разметки корпуса берестяных грамот по сравнению с древнерусским "<усовершенствован в связи с большим количеством фрагментарно сохранившихся слов, а также мест, трактуемых лишь предположительно"> \autocite[227]{sichinava:2014}, однако в остальном оба корпуса размечены в соответствии с едиными принципами.

\begin{figure}[p]
    \begin{subfigure}{\textwidth}
        \centering
        \includegraphics[width=\linewidth]{ncr_old_search}
        \caption{Поисковый интерфейс}
        \label{fig:ncr_old:1}
    \end{subfigure}
    \vfill
    \begin{subfigure}{\textwidth}
        \centering
        \includegraphics[width=\linewidth]{ncr_old_output}
        \caption{Выдача по запросу на личные имена в звательном падеже}
        \label{fig:ncr_old:2}
    \end{subfigure}
    \caption{Древнерусский подкорпус НКРЯ}
\end{figure}

Помимо повсеместных граммем, актуальных отнюдь не только для древнеписьменных языков (часть речи; падеж, число, род; наклонение, время, залог и~т.",д.), рассматриваемые корпуса располагают средствами разметки специфически древнерусских морфологических характеристик: например, для личных местоимений в дат.\ и вин.~п.\ предусмотрены специальные пометы для ударных и клитических форм. Кроме того, в формат аннотации словоформ включены элементы традиционной филологической адресации по номерам листа и строки, а также ряд помет принципиально иного рода, например ономастических (имена собственные могут размечаться как личные имена и отчества, обозначения жены или вдовы по мужу, топонимы и этнонимы) и текстологических (для мест с неясной интерпретацией введены пометы "<зачёркнуто">, "<лишнее">, "<порча">). Запросная форма и диалоговый интерфейс выбора грамматических признаков проиллюстрированы на рис.~\ref{fig:ncr_old:1}.

Разработчики отмечают, что "<разбор древнерусского текста~"--- работа, не вполне поддающаяся стандартизации. \ldots Поскольку потребность добавлять релевантные пометы возникает при изучении древних текстов постоянно, система разметки разрабатывалась таким образом, чтобы исследователь мог легко вводить новые признаки по собственному усмотрению"> \autocite[103--104]{mishina_pichkhadze:2015}. Как следствие, разные памятники в своей аннотации порой обнаруживают известную степень неоднородности, обусловленной личными предпочтениями и взглядами исследователей на размечаемые лингвистические феномены: в частности, подобные расхождения проявляются при разборе имён собственных (в одних памятниках соответствующие пометы проставляются, в других же разметчики ограничиваются базовой частеречной типизацией) и форм аналитических будущих времён, по сей день недостаточно изученных и неоднозначно трактуемых в исторической русистике \autocite[104--106]{mishina_pichkhadze:2015}.

В отдельных переводных памятниках (в частности, это касается "<Александрии"> и "<Пчелы">, воспроизводимых в корпусе по спискам XV~в.)\ словоформам по возможности приписаны греческие аналоги~"--- причём как сами эквивалентные формы, так и их леммы (cм.\ контекстное окно при существительном \textsc{*александръ}\footnotemark{} на рис.~\ref{fig:ncr_old:2}). Собственно же древнерусские леммы составляют общий для всех памятников словарь, где они представлены в унифицированной (не содержащей дублетных графем) древнерусской орфографии, отражающей состояние до падения и прояснения редуцированных \autocite[102--103]{mishina_pichkhadze:2015}.

\footnotetext{%
    Здесь и далее языковые примеры даются капителью с соблюдением транслитерационных соглашений, принятых в коллективе СКАТ: вышедшие из употребления графемы обозначаются при помощи символов латиницы; октоторп маркирует наличие в слове титла; выносные буквы заключаются в скобки; именам собственным предшествует астериск. Лишь в одном отношении мы склонны отступить от настоящих конвенций: в угоду читабельности примеров для обозначения буквы "<ять"> вместо знака \textsc{+} используется соответствующий символ Unicode.
}

\subsection{Церковнославянский корпус}

В состав церковнославянского подкорпуса НКРЯ включены лишь те церковнославянские тексты, которые были созданы или отредактированы в период книгопечатания,~"--- а именно в XVII--XX~вв.\ \autocite[117--118]{polyakov:2015}. Этим обстоятельством, в частности, объясняется объём корпуса, весьма внушительный даже по меркам синхронических специализированных корпусов: в него включены более 1250 документов, охватывающих порядка 4,7~млн словоупотреблений. Однако ручная аннотация столь обширных массивов текстовых данных, очевидно, принципиально нереализуема.

Модуль грамматической разметки церковнославянского корпуса призван одновременно решать как задачу лемматизации~"--- приведения словоформы к лемме и определения её постоянных признаков (части речи, рода, вида, переходности), так и собственно грамматического анализа~"--- определения грамматических свойств самой словоформы \autocite[250--251]{polyakov:2014}. Разметка опирается на разработанную коллективом проекта модель словоизменения церковнославянского языка, состоящую из двух основных компонентов:

\begin{compactenum}
    \item грамматический словарь~"--- перечень лексем с приписанными им словоизменительными параметрами, как то:
    \begin{inparaenum}[(1)]
        \item лемма и её варианты (при наличии),
        \item постоянные признаки лексемы,
        \item код парадигмы,
        \item краткое толкование (по необходимости);
    \end{inparaenum}
    \item грамматическая модель~"--- совокупность таблиц словоизменительных типов (парадигм), в которых задаются системные соотношения между множествами грамматических значений и соответствующих им форм, закодированные при помощи специальных кодов (индексов).
\end{compactenum}

Обе составляющие модели словоизменения не задаются априорно на основе существующих грамматических описаний, но эмпирически и итеративно выводятся из содержимого самого корпуса. Таким образом, разработка словаря и модели ведётся параллельно, и взаимные наработки постоянно корректируются и согласуются между собой: с одной стороны, из корпуса постепенно извлекаются очередные наиболее частотные слова, которым далее вручную приписываются леммы и коды парадигматических шаблонов; с другой стороны, по мере обнаружения ранее неучтённых словоизменительных явлений в анализируемом лексическом материале обновляется номенклатура парадигм, пополняется состав грамматических признаков, правил морфонологических чередований и~т.",д.\ \autocite[129--130]{polyakov:2015}.

\begin{table}[t]
    \small
    \begin{tabularx}{\textwidth}{XXXXX}
        \toprule
        \thead{Парадигма} & \thead{N1t} & \thead{N1t*} & \thead{N1j} & \thead{N1k}   \\ \midrule
        \midrule
        \textit{Пример}   & рабъ        & сонъ         & конь        & отрокъ        \\ \midrule
        \textit{Основа}   & раб+ъ       & со*н+ъ       & кон+ь       & отро(к|ц|ч)+ъ \\ \midrule
        sg,nom            & ъ           & 2ъ           & ь           & ъ             \\ \midrule
        sg,gen            & а           & а            & я           & а             \\ \midrule
        sg,voc            & е           & е            & ю           & 3е            \\ \midrule
        pl,acc            & ы/=gen      & ы/=gen       & и/=gen      & ы/=gen        \\ \midrule
        pl,loc            & ѣхъ         & ѣхъ          & ехъ         & 2ѣхъ          \\ \midrule
        du,dat/ins        & ома         & ома          & ема         & ома           \\ \bottomrule
        \caption{Фрагмент формальной записи парадигм (воспроизводится по \autocite[252]{polyakov:2014})}
    \end{tabularx}
\end{table}

\foreignlanguage{english}{(To be concluded!)}

\subsection{Старорусский корпус}

\verb|pass|

\section{Регенсбургский диахронический корпус русского языка}

\verb|pass|

\section{Манускрипт}

\verb|pass|

\section{Санкт-Петербургский корпус агиографических текстов}
\label{sec:scat}

Принятый в СКАТ формат грамматической аннотации был разработан выпускницей кафедры математической лингвистики СПбГУ Е.",С.~Ивановой \autocite{ivanova:2006} и впоследствии видоизменён и уточнён Е.",Л.~Алексеевой. Он используется для ручного ввода грамматических данных (в течение последнего десятилетия разметка производилась студентами 1--2~курсов в ходе филологической практики) и представляет собой таблицы, где каждой словоформе приписаны соответствующие ей морфологические (в случае аналитических глагольных форм~"--- также и некоторые синтаксические) характеристики. К настоящему времени таким образом размечено 5~житий общим объёмом более 50~тыс.\ словоупотреблений.

Всего для внесения грамматических сведений предусмотрено 6~столбцов, однако фактическое число и значение заполняемых позиций варьирует в зависимости от первой характеристики~"--- части речи. Так, слова знаменательных именных частей речи (существительные, прилагательные и числительные), а также неличные местоимения размечаются единообразно: для них последовательно указываются тип склонения, падеж, число и род; наполнение глагольных тегсетов зависит от наклонения, в случае изъявительного~"--- ещё и от морфологического типа использованного времени (простого или сложного).

Кроме того, формат разметки призван учесть то обстоятельство, что представленные в корпусе житийные тексты, будучи написанными на церковнославянском языке достаточно поздней редакции, отражают живые процессы развития архаичных черт старославянской грамматики: смешение типов склонения, становление категории одушевлённости, обособление деепричастий в особую глагольную форму и~т.",д. Для фиксации переходных явлений подобного рода в соответствующей позиции тегсета приводятся два категориальных значения, разделённые косой чертой,~"--- парадигматически ожидаемое и реально встретившееся \autocite[70--71]{alexeev_alexeeva_kasjanenko:2011}. Например:

\begin{compactitem}
    \item тип склонения \textit{es/o} у существительного \textsc{тѣла} обозначает, что исторически его основа относится к одному из подтипов на согласный (*es), но реально употреблённая флексия соответствует типу *\u{o};
    \item падеж \textit{вин/род} у существительного \textsc{бга\#} показывает, что в значении винительного падежа здесь использован родительный, в чём проявляется категория одушевлённости;
    \item род \textit{ж/м} у причастия \textsc{блгодарr\#} свидетельствует об употреблении формы мужского рода вместо женского~"--- так отражается процесс образования деепричастий.
\end{compactitem}

\begin{figure}[p]
    \begin{subfigure}{\textwidth}
        \centering
        \includegraphics{va_search}
        \caption{Выдача по запросу на существительные, склоняющиеся по типу *en}
    \end{subfigure}
    \vfill
    \begin{subfigure}{\textwidth}
        \centering
        \includegraphics[width=\textwidth]{va_output}
        \caption{Просмотр полного контекста вхождения \textsc{камень} (КК $-$77/14)}
    \end{subfigure}
    \caption{Система В.",А.~Алексеева}
    \label{fig:va}
\end{figure}

В приложении~\ref{app:grm} приведён иллюстративный пример морфологической аннотации начального фрагмента жития Димитрия Прилуцкого. Отметим, что в настоящее время массив размеченных текстов хранится сугубо автономно от прочих компонентов лингвистического обеспечения СКАТ и фактически представляет собой материалы, ожидающие дальнейшей интеграции в основной корпус. Лемматизация в ходе разметки не производится.

В.",А.~Алексеев в качестве одного из аспектов практической части своей магистерской диссертации \autocite{alexeev:2011} разработал механизм внедрения грамматических данных в структуру XML-представления текстов корпуса; кроме того, им был создан пилотный вариант полноценной среды для работы с корпусом через интернет, обеспечивающей полнофункциональный поиск как по структурным частям рукописей, так и по грамматическим признакам. Однако в полном объёме система реализована не была: её онлайн-версия\footnote{\url{http://scat.v-alexeev.ru} (дата обр.\ \today)} предоставляет доступ лишь к небольшому размеченному фрагменту жития Корнилия Комельского объёмом порядка 10 листов (технические ограничения использованной для размещения системы интернет-площадки не позволяют производить такие ресурсоёмкие операции, как загрузка новых рукописей, автоматически \autocite[54]{alexeev:2011}). Пример взаимодействия со средой В.",А.~Алексеева приведён на рис.~\ref{fig:va}.

\section*{Выводы}
\addcontentsline{toc}{section}{Выводы}

