\chapter{Представление грамматической информации в~восточнославянских исторических корпусах}

Список восточнославянских исторических корпусов, приведённый на сайте Национального корпуса русского языка\footnote{\url{http://ruscorpora.ru/corpora-other.html} (дата обр.\ 01.06.2018)} (НКРЯ), включает в себя следующие наименования (помимо СКАТ):

\begin{compactenum}
    \item Регенсбургский диахронический корпус русского языка\footnote{\url{http://rhssl1.uni-regensburg.de/SlavKo/korpus/rrudi-new} (дата обр.\ 01.06.2018)};
    \item Рукописные памятники Древней Руси\footnote{\url{http://www.lrc-lib.ru} (дата обр.\ 01.06.2018) Отметим, что базы древнерусских берестяных грамот и летописей, входящие в архив данного ресурса, к настоящему времени полностью интегрированы в древнерусский сегмент НКРЯ и потому в особом рассмотрении не нуждаются.};
    \item система "<Манускрипт">\footnote{\url{http://manuscripts.ru} (дата обр.\ 01.06.2018)};
    \item корпус русских публицистических текстов второй половины XIX~в.\footnote{\url{http://smalt.karelia.ru/corpus/index.phtml} (дата обр.\ 01.06.2018) Ввиду своей временной специфики он выбивается из общего ряда восточнославянских исторических корпусов и поэтому останется за рамками настоящего обзора.}
\end{compactenum}

Среди перечисленных корпусов первый и третий, а также СКАТ, исторические подкорпуса самого НКРЯ и корпус "<Великие Минеи Четьи">\footnote{\url{http://www.vmc.uni-freiburg.de/Mens/} (дата обр.\ 01.06.2018)} обзорно освещены в статье \autocite{mitrenina:2014}: каждый охарактеризован с точки зрения характера текстов, лежащих в их основе, суммарного объёма (актуального на момент написания статьи), возможностей поиска, наличия морфосинтаксической аннотации и прочих релевантных признаков. В данной главе мы рассмотрим описанные О.",В.~Митрениной корпуса в аспекте реализованных в них принципов и инструментов грамматической разметки и лемматизации несколько более детально (исключая лишь корпус "<Великие Минеи Четьи">, в котором последние отсутствуют \autocite[30]{VMC:2015}).

\section{Национальный корпус русского языка}

\subsection{Древнерусский корпус и корпус берестяных грамот}
\label{subsec:ncr_old}

Эти подкорпуса НКРЯ будут рассмотрены совместно: их разработка ведётся по единой программе президиума РАН "<Корпусная лингвистика">, а составляющие их тексты написаны на одном (пусть и значительно неоднородном) языке и датируются сходными хронологическими периодами~"--- XI--XIV и XI--XV~вв.\ соответственно. Грамматическая разметка текстов, объединённых в древнерусский корпус, ведётся ещё с середины 2000-х~гг.\ в рамках проекта "<Рукописные памятники Древней Руси">; содержимое же корпуса берестяных грамот основывается на материалах сборников "<Новгородские грамоты на бересте"> и размечается для сайта "<Древнерусские берестяные грамоты">\footnote{\url{http://gramoty.ru} (дата обр.\ 01.06.2018)} (в настоящее время доступ к разметке с него не предоставляется) \autocite[226]{sichinava:2014}.

Разметка обоих корпусов производится вручную со снятием грамматической омонимии. При аннотации древнерусского корпуса ввиду его существенно большего объёма (на сегодняшний день он составляет около 500~тыс.\ словоупотреблений~"--- в противовес 20~тыс.\ в случае корпуса берестяных грамот) производится дополнительное обращение к базе прецедентных разборов \autocite[102]{mishina_pichkhadze:2015}; аппарат разметки корпуса берестяных грамот по сравнению с древнерусским "<усовершенствован в связи с большим количеством фрагментарно сохранившихся слов, а также мест, трактуемых лишь предположительно"> \autocite[227]{sichinava:2014}, однако в остальном оба корпуса размечены в соответствии с едиными принципами. Решение собственно технических задач ввода разметки, построения словоуказателей, работы со словарями и прочими компонентами лингвистического обеспечения осуществляется разработчиками посредством специальной среды Morphy \autocite{archangel_mishina_pichkhadze:2014}.

\begin{figure}[p]
    \begin{subfigure}{\textwidth}
        \centering
        \includegraphics[width=\linewidth]{ncr_old_search}
        \caption{Поисковый интерфейс}
        \label{fig:ncr_old:1}
    \end{subfigure}
    \par\medskip
    \begin{subfigure}{\textwidth}
        \centering
        \includegraphics[width=\linewidth]{ncr_old_output}
        \caption{Выдача по запросу на личные имена в звательном падеже}
        \label{fig:ncr_old:2}
    \end{subfigure}
    \caption{Древнерусский подкорпус НКРЯ}
\end{figure}

Помимо повсеместных граммем, актуальных отнюдь не только для древнеписьменных языков (часть речи; падеж, число, род; наклонение, время, залог и~т.",д.), рассматриваемые корпуса располагают средствами разметки специфически древнерусских морфологических характеристик: например, для личных местоимений в дат.\ и вин.~п.\ предусмотрены специальные пометы для ударных и клитических форм. Кроме того, в формат аннотации словоформ включены элементы традиционной филологической адресации по номерам листа и строки, а также ряд помет принципиально иного рода, например ономастических (имена собственные могут размечаться как личные имена и отчества, обозначения жены или вдовы по мужу, топонимы и этнонимы) и текстологических (для мест с неясной интерпретацией введены пометы "<зачёркнуто">, "<лишнее">, "<порча">). Запросная форма и диалоговый интерфейс выбора грамматических признаков проиллюстрированы на рис.~\ref{fig:ncr_old:1}.

Разработчики отмечают, что "<разбор древнерусского текста~"--- работа, не вполне поддающаяся стандартизации. \ldots{} Поскольку потребность добавлять релевантные пометы возникает при изучении древних текстов постоянно, система разметки разрабатывалась таким образом, чтобы исследователь мог легко вводить новые признаки по собственному усмотрению"> \autocite[103--104]{mishina_pichkhadze:2015}. Как следствие, разные памятники в своей аннотации порой обнаруживают известную степень неоднородности, обусловленной личными предпочтениями и взглядами исследователей на размечаемые лингвистические феномены: в частности, подобные расхождения проявляются при разборе имён собственных (в одних памятниках соответствующие пометы проставляются, в других же разметчики ограничиваются базовой частеречной типизацией) и форм аналитических будущих времён, по сей день недостаточно изученных и неоднозначно трактуемых в исторической русистике \autocite[104--106]{mishina_pichkhadze:2015}.

В отдельных переводных памятниках (в частности, это касается "<Александрии"> и "<Пчелы">, воспроизводимых в корпусе по спискам XV~в.)\ словоформам по возможности приписаны греческие аналоги~"--- причём как сами эквивалентные формы, так и их леммы (cм.\ контекстное окно при существительном \textsc{*александръ}\footnotemark на рис.~\ref{fig:ncr_old:2}). Собственно же древнерусские леммы составляют общий для всех памятников словарь, где они представлены в унифицированной (не содержащей дублетных графем) древнерусской орфографии, отражающей состояние до падения и прояснения редуцированных \autocite[102--103]{mishina_pichkhadze:2015}. К сожалению, суммарный объём словаря разработчиками не уточняется.

\footnotetext{%
    Здесь и далее языковые примеры даются капителью с соблюдением транслитерационных соглашений, принятых в коллективе СКАТ: вышедшие из употребления графемы обозначаются при помощи символов латиницы; октоторп маркирует наличие в слове титла; выносные буквы заключаются в скобки; именам собственным предшествует астериск. Лишь в одном отношении мы отступим от настоящих конвенций: в угоду читабельности примеров для обозначения буквы "<ять"> вместо знака \textsc{+} используется соответствующий символ Unicode.
}

\subsection{Церковнославянский корпус}
\label{subsec:ncr_chu}

Церковнославянский подкорпус НКРЯ охватывает лишь те тексты на церковнославянском языке, которые были созданы или отредактированы уже в период книгопечатания~"--- в XVII--XX~вв., причём основная доля (60",\%) приходится на современные тексты, т.",е.\ используемые в современной богослужебной практике. Этим обстоятельством объясняется объём корпуса, по меркам исторических корпусов весьма внушительный: его составляют более 1250~документов, охватывающих все основные типы и жанры церковнославянской литературы и включающих более 4,7~млн словоупотреблений, которые группируются в 150~тыс.\ различных словоформ \autocite[246--247]{polyakov:2014}. Очевидно, ручной морфологический анализ столь обширных текстовых массивов принципиально нереализуем.

Грамматическая аннотация церковнославянского корпуса одновременно нацелена как на процедуру лемматизации~"--- приведение словоформы к лемме и определение её постоянных признаков (части речи, рода, вида, переходности), так и на собственно грамматический анализ~"--- выявление грамматических свойств самой словоформы (падежа, числа, времени, лица, наклонения). В основе автоматической разметки лежит формальная модель словоизменения церковнославянского языка, включающая в себя два основных компонента \autocite[250--251]{polyakov:2014}:

\begin{compactenum}
    \item грамматический словарь~"--- перечень лексем с приписанными им словоизменительными параметрами. Это, как минимум, \begin{inparaenum}[(1)]
        \item словарная форма и её варианты (при наличии),
        \item постоянные признаки лексемы,
        \item код парадигмы и частные особенности словоизменения,
        \item краткое толкование (по необходимости);
    \end{inparaenum}
    \item грамматическая модель~"--- совокупность таблиц словоизменительных типов (парадигм), в которых задаются системные соотношения между множествами грамматических значений и соответствующих им форм, обозначенные при помощи специальных кодов (индексов).
\end{compactenum}

Обе составляющие модели словоизменения не задаются априорно на основе существующих грамматик и словарей, но эмпирически и итеративно выводятся из содержимого самого корпуса. Таким образом, разработка словаря и модели ведётся параллельно, и взаимные наработки постоянно корректируются и согласуются между собой: с одной стороны, из корпуса постепенно извлекаются ранее не описанные слова, которым при занесении в словарь вручную приписываются леммы и коды парадигматических шаблонов; с другой стороны, по мере обнаружения ранее неучтённых словоизменительных явлений в анализируемом лексическом материале обновляется номенклатура парадигм, пополняется состав грамматических признаков, правил морфонологических чередований и~т.",д. Суммарный объём словника по состоянию на 2015~г.\ насчитывал около 35~тыс.\ лемм и 60~тыс.\ отдельных словоформ \autocite[129--131]{polyakov:2015}.

\begin{table}[t]
    \small
    \begin{tabularx}{\textwidth}{XXXXX}
        \toprule
        \thead{Парадигма} & \thead{N1t} & \thead{N1t*} & \thead{N1j} & \thead{N1k}   \\ \midrule\midrule
        \textit{Пример}   & рабъ        & сонъ         & конь        & отрокъ        \\ \midrule
        \textit{Основа}   & раб+ъ       & со*н+ъ       & кон+ь       & отро(к|ц|ч)+ъ \\ \midrule
        sg,nom            & ъ           & 2ъ           & ь           & ъ             \\ \midrule
        sg,gen            & а           & а            & я           & а             \\ \midrule
        sg,voc            & е           & е            & ю           & 3е            \\ \midrule
        pl,acc            & ы/=gen      & ы/=gen       & и/=gen      & ы/=gen        \\ \midrule
        pl,loc            & ѣхъ         & ѣхъ          & ехъ         & 2ѣхъ          \\ \midrule
        du,dat/ins        & ома         & ома          & ема         & ома           \\ \bottomrule
        \caption{Фрагмент формальной записи парадигм в церковнославянском корпусе}
        \label{tab:chu_grm}
    \end{tabularx}
\end{table}

Структура грамматической модели проиллюстрирована в таблице~\ref{tab:chu_grm} (воспроизводится по \autocite[252]{polyakov:2014}). В заголовочной строке указаны коды парадигм, в двух последующих~"--- типовые примеры лемм в обыкновенной записи и в формате грамматического словаря; далее наборам грамматических признаков в первом столбце сопоставлены наборы соответствующих флексий. Вариативность словарных основ отражается при помощи специальных помет (астериск обозначает беглость предшествующего гласного, в скобки заключаются морфонологические альтернанты), а целочисленные префиксы при флексиях обозначают порядковый номер сочетающегося варианта основы (основным считается первый~"--- без чередований). Всего к настоящему времени составлено не менее \begin{inparaitem}[]
    \item 43~таблиц для существительных,
    \item 8 для прилагательных,
    \item 9 для местоимений"=прилагательных,
    \item 7 для местоимений"=существительных,
    \item 50 для глаголов \autocite[133--134]{polyakov:2015}.
\end{inparaitem}

\begin{figure}[p]
    \begin{subfigure}{\textwidth}
        \centering
        \includegraphics[width=\linewidth]{ncr_chu_search}
        \caption{Выпадающий список лемм в поисковой форме}
        \label{fig:ncr_chu:1}
    \end{subfigure}
    \par\medskip
    \begin{subfigure}{\textwidth}
        \centering
        \includegraphics[width=\linewidth]{ncr_chu_output}
        \caption{Неразрешённая морфологическая неоднозначность}
        \label{fig:ncr_chu:2}
    \end{subfigure}
    \caption{Церковнославянский подкорпус НКРЯ}
\end{figure}

Также в цитируемой статье за 2015~г.\ обсуждается непосредственно эмпирическая реализация модели словоизменения~"--- частотный грамматический словарь церковнославянского языка. Каждой вокабуле частотного словаря (т.",е.\ лемме) помимо словоизменительных параметров, перечисленных выше, сопоставлено суммарное число её употреблений во всех текстах корпуса, за которым следует перечень конкретных словоформ~"--- членов её словоизменительной парадигмы, встретившихся в корпусе (также снабжённых абсолютными частотами встречаемости). От каждой леммы и словоформы можно перейти к цитатам, в которых они содержатся \autocite[131]{polyakov:2015}. Следует отметить, что страница, запрашиваемая по приводимой А.",Е.~Поляковым ссылке\footnote{\url{http://feb-web.ru/febupd/slavonic/dicgram/} (дата обр.\ 01.06.2018)}, уже в течение весьма продолжительного времени выдаёт лишь ошибку~404; с другой стороны, при вводе запросов в поисковую форму на основном сайте выпадающий список лемм работает исправно (рис.~\ref{fig:ncr_chu:1}). Вероятно, это свидетельствует о том, что на сегодняшний день частотный словарь уже интегрирован в основной корпус.

Морфологическая дизамбигуация в церковнославянском подкорпусе НКРЯ не производится (рис.~\ref{fig:ncr_chu:2}).

\subsection{Старорусский корпус}
\label{subsec:ncr_mid}

В количественном отношении старорусский (среднерусский) подкорпус НКРЯ превосходит все вышеперечисленные корпуса, вместе взятые: его составляют около 5~тыс.\ документов XV--XVII~вв.\ суммарным объёмом более 7~млн словоупотреблений. Морфологическая разметка этих текстов также предусмотрена в рамках общей программы развития исторического сегмента НКРЯ, однако, как отмечает Д.",В.~Сичинава, "<её выполнение как вручную, так и автоматически наталкивается на известные сложности, связанные с огромным объёмом текстов и их орфографической и языковой пестротой"> \autocite[227]{sichinava:2014}. Поэтому публичного доступа к грамматической информации данный корпус не предоставляет: организация поиска возможна только по точным совпадениям.

В настоящее время разработки, направленные на изменение такого положения вещей путём автоматизации лексико"=грамматической разметки старорусского подкорпуса НКРЯ, активно ведутся сотрудниками школы лингвистики НИУ "<Высшая школа экономики"> под руководством О.",Н.~Ляшевской. В статье \autocite{lyashevskaya:2016} описывается опыт построения двух альтернативных модулей морфологического анализа тестовой выборки из среднерусского корпуса объёмом порядка 2~млн словоупотреблений~"--- словарно"=правилового и гибридного.

Первый подход, которому и посвящена основная часть публикации, основывается на церковнославянском грамматическом словаре, обсуждавшемся выше (см.\ \ref{subsec:ncr_chu}): ввиду отсутствия подобных ресурсов, ориентированных на древнерусский языковой материал (притом сравнительно поздней редакции), авторами делается "<сильное допущение, что деление лексики на словоизменительные типы в древнерусском языке в достаточной мере соотносится с делением лексики в церковнославянском"> \autocite[14]{lyashevskaya:2016}. При этом из системы А.",Е.~Полякова заимствуется исключительно словарная составляющая: в качестве алгоритмической основы морфологического анализатора используется программа "<Юни-парсер"> Т.",А.~Архангельского, предназначенная для разметки текстов на языках различных структурных типов. Таким образом, церковнославянский грамматический словарь потребовалось специально адаптировать к механизму работы Юни-парсера~"--- путём решения, в частности, следующих задач \autocite[14--18]{lyashevskaya:2016}:

\begin{compactenum}
    \item составление правил для порождения косвенных (т.",е.\ альтернирующих) основ для каждого релевантного словоизменительного типа;
    \item введение новых типов склонения, возникших в результате перестройки системы именного словоизменения раннедревнерусского периода;
    \item автоматическое порождение основ и предсказание парадигм для глаголов, обладающих неполной грамматической аннотацией в словаре;
    \item обработка глаголов, изменяющихся по изолированным типам спряжения либо обладающих нерегулярными особенностями в парадигме.
\end{compactenum}

Второй предлагаемый подход к лексико"=грамматической аннотации старорусского корпуса, напротив, не связан с привлечением каких-либо грамматических словарей и опирается исключительно на уже существующие ресурсы. В качестве его методологической основы выступает простое предположение, что, с одной стороны, разбор среднерусских словоформ, имеющих непосредственные современные аналоги, можно без значительного качественного ущерба доверить морфологическим анализаторам для современного русского языка; с другой стороны, прецедентная совокупность ручных разборов, произведённых в рамках древнерусского подкорпуса НКРЯ (см.\ \ref{subsec:ncr_old}), может в значительной мере покрыть "<архаичные"> словоформы (включая исторические формы таких частотных лемм, как глагол \textsc{быти} или местоимение \textsc{иже}), с которыми анализаторы для современного русского языка заведомо не справятся. Для анализа условно"=современных словоформ были использованы парсеры MyStem и TreeTagger, адаптированные для НКРЯ \autocite[18--19]{lyashevskaya:2016}.

Заметим, что перед обоими подходами не ставилась дополнительная задача разрешения неоднозначности, а успех каждого конкретного разбора определялся на основании критерия "<широкого охвата">: если хотя бы один разбор являлся корректным, всё множество порождаемых разборов также принималось за правильное. С учётом данного обстоятельства приводимые авторами оценки качества как словарного, так и гибридного модуля являются весьма оптимистичными: на золотом стандарте, состоящем из двух вручную размеченных текстов XVI--XVII~вв.\ ("<Жития Сергия Радонежского"> и "<Наказа Афанасию Филипповичу Пашкову на воеводство в Даурской земле">), оба парсера показали точность не ниже 93",\% и 89",\% в случае частеречной разметки и лемматизации соответственно. Показатели полноты и аккуратности словарного парсера ожидаемо ниже, чем у гибридного, однако по точности он неизменно выигрывает~"--- в особенности применительно к более архаичному тексту "<Жития"> \autocite[19--21]{lyashevskaya:2016}.

Таким образом, не исключено, что в ближайшие годы старорусский корпус, подобно остальным историческим подкорпусам НКРЯ, будет также обогащён собственным лексико"=грамматическим инструментарием.

\section{Регенсбургский диахронический корпус русского языка}
\label{sec:rrudi}

Регенсбургский диахронический корпус русского языка (\foreignlanguage{english}{Regensburg Russian Diachronic Corpus}, далее RRuDi) во многих отношениях стоит особняком от отечественных восточнославянских корпусов. Его разработка велась при поддержке гранта Немецкого научно"=исследовательского общества в рамках проекта "<\foreignlanguage{english}{Corpus Linguistics and Diachronic Syntax: The Grammaticalization of Non-Canonical Subjects in Slavonic Languages}">; таким образом, RRuDi с самого начала проектировался как эмпирическая база для конкретных исследовательских задач и позиционировался как прикладной ресурс, не стремящийся охватить восточнославянские письменные источники какого-либо временного пласта во всей полноте,~"--- как \textit{диахронический} корпус в противовес \textit{историческим}. Как следствие, документальная основа RRuDi обнаруживает весьма значительное хронологическое и типологическое разнообразие, включая в себя \begin{inparaitem}[]
    \item хроники,
    \item деловые документы,
    \item письма,
    \item путевые записки,
    \item жития
\end{inparaitem} и тексты иных литературных жанров, созданные на протяжении X--XVIII~вв.\ \autocite[36--37]{meyer:2012}. Однако в настоящее время для открытого пользования доступен лишь подкорпус общим объёмом порядка 115~тыс.\ словоупотреблений~"--- для работы с корпусом в полном объёме необходимо заполнить и отправить лицензионное соглашение.

Технологической базой для грамматической аннотации RRuDi является платформа GATE. Разметка производится полуавтоматически~"--- в том смысле, что первичный морфологический анализ осуществляется полностью автоматизированными средствами, но впоследствии его результаты корректируются вручную или путём задания специальных правил \autocite[43]{meyer:2012}. Здесь необходимо оговорить то немаловажное обстоятельство, что с точки зрения непосредственного материального источника тексты в составе RRuDi подразделяются на две неравнозначные группы: \begin{inparaenum}[(1)]
    \item специально оцифрованные коллективом в процессе проектирования корпуса;
    \item изначально опубликованные в серии "<Библиотека литературы Древней Руси">.
\end{inparaenum} На этапе первичного морфологического анализа данные группы обрабатываются по-разному.

Оригинальные тексты RRuDi размечаются полностью автоматически при помощи последовательного применения трёх морфологических анализаторов: старославянского и древнерусского, разработанных в Регенсбургском университете и представляющих собой конечные автоматы типа Xerox, а также статистического разметчика TreeTagger в версии С.",А.~Шарова \autocite[44]{meyer:2012}. Парсеры для обоих древнеписьменных языков создавались независимо от основного корпуса и могут работать автономно: так, старославянский морфологический анализатор был интегрирован в отдельное веб"=приложение\footnote{\url{http://rhssl1.uni-regensburg.de:8080/OCS} (дата обр.\ 01.06.2018)}, способное производить грамматический разбор и строить парадигмы по пользовательским запросам.

Механизм обработки несобственных текстов RRuDi несколько менее прямолинеен \autocite[270--278]{meyer:2011}. Поскольку в изданиях серии "<Библиотека литературы Древней Руси"> наряду с самими древнерусскими текстами содержатся их переводы на современный русский язык, разработчиками было принято решение рассматривать данные текстовые массивы как параллельные корпуса и при их морфологическом анализе дополнительно задействовать широко известные в соответствующем домене корпусной лингвистики методы проецирования (\foreignlanguage{english}{annotation projection}). Вначале тексты на современном русском и на древнерусском подвергаются процедуре автоматического выравнивания по абзацам и предложениям, а далее подвергаются обработке при помощи парсеров, упомянутых выше,~"--- TreeTagger и древнерусского соответственно. Ввиду того что последний функционирует исключительно на основе правил и не основывается ни на каких словарных ресурсах, множества порождаемых им неверных разборов бывают весьма многочисленными; так, \textcite[272]{meyer:2011} приводит следующие примеры того, как им анализируется словоформа \textsc{начати}: \begin{inparaenum}[(1)]
    \item инфинитив,
    \item краткое страдательное причастие прош.~вр.\ в форме им.~п.\ мн.~ч.\ м.~р.,
    \item существительное ж.~р.\ в форме дат.~п.\ ед.~ч.,
    \item оно же в форме мест.~п.\ ед.~ч.,
    \item оно же в форме им.~п.\ дв.~ч.
\end{inparaenum}

\begin{figure}[t]
    \centering
    \includegraphics[width=\linewidth]{rrudi}
    \caption{RRuDi: запрос на словоформы в винительном падеже}
    \label{fig:rrudi}
\end{figure}

После промежуточного этапа выравнивания по словам анализатор стремится уменьшить число полученных таким образом древнерусских разборов, проецируя на них те разборы, которые были ранее присвоены выявленным на предыдущем шаге современным русским эквивалентам. При этом он руководствуется рядом специфических правил, например \autocite[273--274]{meyer:2011}:

\begin{compactenum}
    \item если разбор анализируемой древнерусской словоформы единственен, то её разметка остаётся без изменений;
    \item если множество древнерусских разборов пусто, на подобную словоформу переносятся все разборы современного русского эквивалента;
    \item если у части древнерусских и современных русских разборов совпадают частеречные теги, то все прочие разборы удаляются;
    \item если множество современных русских разборов является подмножеством древнерусских, то иные разборы среди последних удаляются.
\end{compactenum}

К сожалению, перечень принятых в RRuDi грамматических тегов нигде специально не приводится, что затрудняет организацию сложных поисковых запросов в корпус"=менеджере. Снятие грамматической омонимии не предусмотрено (рис.~\ref{fig:rrudi}).

\section{Манускрипт}
\label{sec:mns}

Информационно"=аналитическая система (ИАС) "<Манускрипт">, объединяющая в рамках своих основных коллекций более 140~старославянских (включая 5~глаголических) и древнерусских текстов объёмом более 3,5~млн словоупотреблений, для автоматизированного морфологического анализа последних опирается на электронный грамматический словарь древнерусского языка (ГСДЯ). Формально он представляет из себя "<базу данных, содержащую лингвистические единицы, их значения и связи"> \autocite{baranov:2010}; среди базовых лингвистических единиц в свою очередь выделяются следующие основные объекты: \begin{inparaitem}[]
    \item основа,
    \item окончание,
    \item тип изменения,
    \item вариант основы,
    \item парадигма
    \item и субпарадигма.
\end{inparaitem}

\afterpage{
    \clearpage
    
    \begin{table}[t]
        \small
        \begin{tabularx}{\textwidth}{p{2.5cm}XX}
            \toprule
            \thead{Часть речи} & \thead{Параметры типов изменения} & \thead{Параметры окончаний} \\ \midrule\midrule
            \textit{сущ} & род & число, падеж \\ \midrule
            \textit{прил} & членность, разряд & род, число, падеж \\ \midrule
            \textit{гл} & наклонение & изменяемость, время, число, лицо \\ \midrule
            \textit{прич} & \textit{гл} $+$ время, залог, членность & \textit{гл} $+$ род, падеж \\ \bottomrule
            \caption{Параметры типов изменения и окончаний в ИАС "<Манускрипт">}
            \label{tab:mns_type}
        \end{tabularx}
    \end{table}
    
    \begin{figure}[b]
        \centering\small
        \begin{tikzpicture}[node distance = 5cm, every edge/.style = {link}]
            \node[relationship] (used) {Употребляться с};
            \node[entity] (infl) [above left of = used] {Окончание};
            \node[relationship] (inc1) [above of = used] {Входить в} edge (infl);
            \node[entity] (type) [above right of = used] {Тип изменения} edge [<-] (inc1);
            \node[relationship] (inc2) [right of = used] {Входить в} edge (type);
            \node[entity] (dict) [below right of = used] {Словарь} edge [<-] (inc2);
            \node[relationship] (inc3) [below of = used] {Входить в} [->] edge (dict);
            \node[entity] (stem) [below left of = used] {Основа} edge (inc3);
            \path (used) edge (stem) edge [->] (type);
        \end{tikzpicture}
        \caption{ER-модель ГСДЯ}
        \label{fig:mns_data}
    \end{figure}
    
    \clearpage
}

С точки зрения ER-модели (\foreignlanguage{english}{entity--relationship model}) ГСДЯ, воспроизведённой на рис.~\ref{fig:mns_data} по \autocite{baranov:2007}, центральное место среди перечисленных компонентов занимают типы изменения. Они представляют собой полноценные словарные единицы, фигурирующие в словнике как индексированные аббревиатуры типа \textit{2а\_г}, \textit{а1\_прич2} и~т.",п., и обладают собственными лексико"=грамматическими характеристиками; с другой стороны, входящие в них окончания имеют уже непосредственно морфологические параметры (см.\ таблицу~\ref{tab:mns_type}). Основы описываются с учётом типа изменения (а также номера омонима и различных лексико"=семантических свойств: одушевлённости, собирательности и~т.",д.), а построение парадигм осуществляется путём конкатенации основ и окончаний одинаковых типов. Выделение в качестве особых словарных единиц вариантов основ и субпарадигм направлено на учёт алломорфирования (ср.\ \textsc{"~куп"~}~// \textsc{"~купл"~}) и варьирования типов изменения (например, по признаку членности).

Имплементация морфологического анализатора представлена на сайте ИАС "<Манускрипт"> (раздел "<Инструменты">) в четырёх версиях\footnotemark, отличающихся друг от друга функциональными возможностями. Первая версия умеет \begin{inparaenum}[(1)]
    \item приводить словоформы к начальной форме,
    \item выводить грамматические признаки основ и окончаний,
    \item строить полные словоизменительные парадигмы,
    \item осуществлять поиск словоформ по маске.
\end{inparaenum} При этом запросы должны вводиться в поисковую форму исключительно в нормализованном виде: именно такое представление словоформы имеют в базе данных ГСДЯ, а поисковая машина первой версии морфологического анализатора работает только в режиме полного совпадения.

\footnotetext{%
    Доступ к пятой версии имеют только создатели транскрипций.
}

\begin{figure}[p]
    \begin{subfigure}{\textwidth}
        \centering
        \includegraphics[width=\linewidth]{mns_output}
        \caption{Поиск по коллекции славянских миней: выделение запроса \textsc{чудо}}
        \label{fig:mns:1}
    \end{subfigure}
    \par\medskip
    \begin{subfigure}{\textwidth}
        \centering
        \includegraphics[width=\linewidth]{mns_search}
        \caption{Список начальных форм типа изменения \textit{м1б3}}
        \label{fig:mns:2}
    \end{subfigure}
    \caption{ИАС "<Манускрипт">}
\end{figure}

Вторая версия является шагом вперёд по сравнению со своей предшественницей, позволяя приводить к лемме также и графико"=орфографические варианты: с одной стороны, для ввода в поисковую форму доступны не только современные кириллические символы, но и исторические; с другой стороны, наряду с поисковыми запросами пользователю предоставляется возможность задавать параметры стандартных орфографических преобразований их внешнего вида. Кроме того, предусмотрено ограничение выводимых результатов необходимыми грамматическими признаками.

Основное новшество третьей версии заключается в возможности морфологического анализа целых текстовых фрагментов~"--- причём как уже разделённых на словоформы, так и нет (т.",н.\ \foreignlanguage{english}{scriptum continuum}~"--- подавляющее большинство текстов из коллекций ИАС "<Манускрипт"> именно таково). В последнем случае словоделение производится автоматически на основе расчленения поисковых запросов на компоненты, имеющие соответствия в ГСДЯ; ненадёжные варианты устраняются из выдачи. Также в третьей версии реализована локальная дизамбигуация именных групп: при анализе последовательно идущих друг за другом существительных либо местоимений и прилагательных либо причастий, у которых множества разборов пересекаются, в выдачу попадают лишь результаты подобного пересечения.

Наконец, четвёртая версия морфологического анализатора призвана синтезировать наработки, реализованные в двух предыдущих: она позволяет задавать множественные запросы с уточнением расстояния между терминами, а также их грамматических и иных параметров. Однако её наиболее значимое отличие от более ранних версий состоит в возможности искать не только по ГСДЯ, но и по самим текстовым коллекциям: от результатов поиска доступен непосредственный переход к соответствующим рукописным фрагментам, где каждое искомое вхождение выделено подчёркиванием (рис.~\ref{fig:mns:1}). Не менее важной представляется и возможность просмотра полных перечней основ, имеющих те же типы изменения, что и термины запроса (рис.~\ref{fig:mns:2}).

\section{Санкт-Петербургский корпус агиографических текстов}
\label{sec:scat}

Принятый в СКАТ формат грамматической аннотации был разработан выпускницей кафедры математической лингвистики СПбГУ Е.",С.~Ивановой \autocite{ivanova:2006} и впоследствии видоизменён и уточнён Е.",Л.~Алексеевой. Он используется для ручного ввода грамматических данных (в течение последнего десятилетия разметка производилась студентами 1--2~курсов в ходе филологической практики) и представляет собой таблицы, где каждой словоформе приписаны соответствующие ей морфологические (в случае аналитических глагольных форм~"--- также и некоторые синтаксические) характеристики; лемматизация в ходе разметки не осуществляется. К настоящему времени таким образом размечено 5~житий общим объёмом более 50~тыс.\ словоупотреблений.

Всего для внесения грамматических сведений предусмотрено 6~столбцов, однако фактическое число и значение заполняемых позиций варьирует в зависимости от первой характеристики~"--- части речи. Так, слова знаменательных именных частей речи (существительные, прилагательные и числительные), а также неличные местоимения размечаются единообразно: для них последовательно указываются тип склонения, падеж, число и род; наполнение глагольных тегсетов зависит от наклонения, в случае изъявительного~"--- ещё и от морфологического типа использованного времени (простого или сложного).

\begin{figure}[p]
    \begin{subfigure}{\textwidth}
        \centering
        \includegraphics[height=0.5\textheight]{va_search}
        \caption{Выдача по запросу на существительные *\u{\i}-склонения}
    \end{subfigure}
    \par\medskip
    \begin{subfigure}{\textwidth}
        \centering
        \includegraphics[width=\textwidth]{va_output}
        \caption{Просмотр полного контекста вхождения \textsc{памrтiю} (КК 70~об./2)}
    \end{subfigure}
    \caption{Система В.",А.~Алексеева}
    \label{fig:va}
\end{figure}

Кроме того, формат разметки призван учесть то обстоятельство, что представленные в корпусе житийные тексты, будучи написанными на церковнославянском языке достаточно поздней редакции, отражают живые процессы развития архаичных черт старославянской грамматики: смешение типов склонения, становление категории одушевлённости, обособление деепричастий в самостоятельную глагольную форму и~т.",д. Для фиксации переходных явлений подобного рода в соответствующей позиции тегсета приводятся два категориальных значения, разделённые косой чертой,~"--- парадигматически ожидаемое и реально встретившееся \autocite[70--71]{alexeev_alexeeva_kasjanenko:2011}. Например:

\begin{compactitem}
    \item тип склонения \textit{es/o} у существительного \textsc{тѣла} обозначает, что исторически его основа относится к одному из подтипов на согласный (*es), но употреблённая флексия соответствует типу *\u{o};
    \item падеж \textit{вин/род} у существительного \textsc{бга\#} показывает, что в значении винительного падежа здесь использован родительный, в чём проявляется категория одушевлённости;
    \item род \textit{ж/м} у причастия \textsc{блгодарr\#} свидетельствует об употреблении формы мужского рода вместо женского~"--- так отражается процесс образования деепричастий.
\end{compactitem}

В приложении~\ref{app:grm} приведён иллюстративный фрагмент морфологической аннотации жития Димитрия Прилуцкого.

В настоящее время массив размеченных текстов хранится сугубо автономно от прочих компонентов лингвистического обеспечения СКАТ, представляя собой материалы, по-прежнему ожидающие интеграции в основной корпус. Задаче исправления подобной ситуации была посвящена практическая часть магистерской диссертации В.",А.~Алексеева \autocite[54--64]{alexeev:2011}: в частности, им был разработан механизм внедрения грамматической информации в структуру XML"=представления текстов корпуса, а также тестовый вариант полноценной среды для работы с корпусом через интернет, обеспечивающей полнофункциональный поиск как по структурным частям рукописей, так и по грамматическим признакам.

К сожалению, в полном объёме система реализована не была: её онлайн"=версия\footnote{\url{http://scat.v-alexeev.ru} (дата обр.\ 01.06.2018)} предоставляет доступ лишь к небольшому размеченному фрагменту жития Корнилия Комельского объёмом порядка 10~листов (В.",А.~Алексеев отмечает, что технические ограничения использованной для размещения системы интернет"=площадки не позволяют производить такие ресурсоёмкие операции, как загрузка новых рукописей, автоматически \autocite[54]{alexeev:2011}). Пример работы с данной средой приведён на рис.~\ref{fig:va}.

\section*{Выводы}
\addcontentsline{toc}{section}{Выводы}

В таблице~\ref{tab:corpora} представлено сопоставление всех рассмотренных в данной главе корпусов по основным релевантным для нас основаниям.

\begin{table}[t]
    \small
    \begin{tabularx}{\linewidth}{XXXXXX}
        \toprule
        \thead{Корпус} & \thead{Период, \\ вв.} & \thead{Объём, \\ с/у} & \thead{Разметка} & \thead{Дизамби"= \\ гуация} & \thead{Леммати"= \\ зация} \\ \midrule\midrule
        НКРЯ: др.-р. & XI--XIV & 500~тыс. & Ручная & Полная & Есть \\ \midrule
        НКРЯ: б.~гр. & XI--XV & 20~тыс. & Ручная & Полная & Есть \\ \midrule
        НКРЯ: ц.-сл. & XVII--XX & 4,7~млн & Словарная & Нет & Есть \\ \midrule
        НКРЯ: ст.-р. & XV--XVII & 7~млн & Нет & Нет & Нет \\ \midrule
        RRuDi & X--XVIII & 115+~тыс. & Гибридная & Нет & Есть \\ \midrule
        Манускрипт & X--XIV & 3,5~млн & Словарная & Локальная & Есть \\ \midrule
        СКАТ & XV--XVII & 500~тыс. & Ручная & Полная & Нет \\ \bottomrule
        \caption{Восточнославянские исторические корпуса: резюме}
        \label{tab:corpora}
    \end{tabularx}
\end{table}

Нетрудно заметить, что грамматическая аннотация всех перечисленных корпусов (исключая неразмеченный старорусский подкорпус НКРЯ, а также особый во многих отношениях корпус RRuDi) либо опирается на грамматический словарь, либо производится вручную, причём ручная разметка закономерно является необходимым условием для полной дизамбигуации. С другой стороны, лемматизация в каком-либо виде отсутствует лишь в корпусе СКАТ, что подтверждает актуальность поставленной перед настоящей работой цели и крайнюю важность её достижения для приведения СКАТ в соответствие с глобальным уровнем развития восточнославянских исторических корпусов.
