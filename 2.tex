\chapter{Лемматизация церковнославянских словоформ на~основе морфологической разметки}

В данной главе поэтапно описывается алгоритм лемматизации морфологически размеченных текстов СКАТ, разработанный в ходе практической части настоящей работы; изложение проблем церковнославянского словоизменения и иных теоретических вопросов в известной мере вторично и прежде всего обусловлено необходимостью их алгоритмического решения. Программный пакет, реализующий весь описанный в данной и следующей главах функционал, написан на языке Python (версии 3.6); исходный код находится в открытом доступе по ссылке: \url{https://github.com/vintagentleman/SCAT}.

\section{Нормализация}
\label{sec:norm}

В литературе по компьютерной морфологии имеет место определённая терминологическая путаница, связанная с терминами "<лемматизация"> и "<нормализация">: порой они употребляются взаимозаменяемо и считаются абсолютными синонимами \autocite[75]{koval:2005}, порой "<нормализации"> приписывают такое определение ("<постановка или словосочетания в каноническую форму">~"--- т.",е.\ базовую форму, или лемму \autocite[14]{mono_morphology}), которое исходя из одних словообразовательных соображений в большей степени подошло бы "<лемматизации">. Однако в контексте исторических языков, на наш взгляд, между данными терминами необходимо проводить чёткую границу: статус процедуры АОТ на морфологическом языковом уровне надлежит приписывать термину "<лемматизация">~"--- и только ему; "<нормализация">, с другой стороны, прежде всего связана с графическим слоем представления текста (исходя из чего производить её следует строго \textit{до} лемматизации и прочих этапов АОТ на более высоких языковых уровнях) и заключается в приведении различных орфографических вариантов одного и того же слова к некоторому инварианту. Например, элементы следующего ряда:
\begin{inparaitem}[]
    \item \textsc{блаженаго},
    \item \textsc{блаженна(г)},
    \item \textsc{бл(а)женнаго},
    \item \textsc{бла(ж)еннаго},
    \item \textsc{бла(ж)нна(г)},
    \item \textsc{бла(ж)ннаго},
    \item \textsc{блжена(г)\#},
    \item \textsc{блженаго\#},
    \item \textsc{блжен(н)аго\#},
    \item \textsc{блаженна(г)\#}~"---
\end{inparaitem}
представляют собой варианты записи одной и той же словоформы, "<нормальная форма"> которой имеет вид \textsc{блаженаго} или \textsc{блаженнаго}. В случае необходимости выбирать между различными орфографическими вариантами инвариантной единицы выбор во многом обусловлен методологическими установками конкретного исследователя или коллектива: так, при ориентации на близость к орфографии соответствующего современного языка в рассматриваемом примере предпочтение должно быть отдано последнему члену пары. Установку на современную ("<каноническую">) орфографию кладёт в основу своего определения нормализации \textcite[69]{piotrowski:2012}.

Е.",Г.~Уфлянд, ранее работавшая над проблемами нормализации в рамках СКАТ, также~"--- с незначительными отступлениями~"--- придерживалась курса именно на орфографию современного русского литературного языка \autocite[41--42]{uflyand:2008}. В своём дипломном сочинении она очерчивает круг причин и характер вариативных написаний в исследованных 10 текстах корпуса общим объёмом порядка 144 тыс.\ словоупотреблений, а далее описывает разработанный ей алгоритм автоматического сведения орфографических вариантов словоформ к основному (т.",е.\ к нормальной форме), реализованный на Python 2.7 в качестве функционального ядра программы для уменьшения объёма сводного словоуказателя. Функция, о которой идёт речь (\texttt{modif}), работает следующим образом: вначале она предварительно очищает входные символьные цепочки от дублетных символов, различиями в которых на момент создания составляющих корпус рукописей можно пренебречь (так, \textsc{u}, \textsc{d} и \textsc{g}~"--- ук диграфный, лигатурный и юс большой~"--- заменяются на \textsc{у}, \textsc{w} (омега) на \textsc{о}, \textsc{i}~десятеричное на \textsc{и}~восьмеричное; из некириллических знаков оставляется лишь \textsc{+}, обозначающий ять), а затем последовательно проверяет их на вхождения ненормализованных буквосочетаний различных типов, которые при обнаружении подвергаются соответствующим заменам, перечисленным в словарном компоненте модуля.

Перечень типов заменяемых буквосочетаний, выделяемых Е.",Г.~Уфлянд, весьма обширен; упомянем лишь некоторые:
\begin{inparaenum}[(1)]
    \item буквосочетания в сокращённых словах под титлом: \textsc{бомтр}~$\to$ \textsc{богоматер}, \textsc{мчнк}~$\to$ \textsc{мученик}, \textsc{срц}~$\to$ \textsc{сердц};
    \item буквосочетания в сокращённых словах с выносными буквами: \textsc{б(д)ц}~/ \textsc{б(ди)ц}~$\to$ \textsc{богородиц}, \textsc{г(с)дрв}~$\to$ \textsc{государев}, \textsc{г(с)др}~$\to$ \textsc{государ};
    \item отдельные буквосочетания в разных внутрисловных позициях:
    \begin{inparaitem}[]
        \item сочетания заднеязычных согласных с \textsc{ь} везде, кроме абсолютного конца слова: \textsc{кь}~/ \textsc{гь}~/ \textsc{хь}~$\to$ \textsc{к}~/ \textsc{г}~/ \textsc{х};
        \item сочетания типа \textit{*TorT} с метатезой срединных плавных: \textsc{прьст}~/ \textsc{пръст}~$\to$ \textsc{перст};
        \item регулярные окончания с выносными буквами: \textsc{а(ш)}~/ \textsc{я(ш)}~$\to$ \textsc{аше}~/ \textsc{яше} и~мн.",др.
    \end{inparaitem}
\end{inparaenum}

В ходе интеграции функции \texttt{modif} в нашу программу мы сочли необходимым привнести некоторые новшества в механизм её работы. Непосредственно в код было внесено множество технологических улучшений: словарная составляющая отныне отделена от алгоритмической (они разнесены по разным файлам), и для хранения информации о буквосочетаниях, подлежащих замене, в ней используется естественным образом напрашивающаяся структура данных типа "<словарь">, а не пары синхронизированных между собой массивов (вносить в подобную структуру любые изменения крайне проблематично). Замены были переписаны на языке регулярных выражений, что позволило более полно и экономно охватить вариативность плана выражения сводимых орфографических вариантов; их перечень также регулярно пополнялся по мере обнаружения нового релевантного языкового материала. Так, в список замен буквосочетаний в сокращённых словах под титлом были добавлены замены \textsc{др(в)н} (либо \textsc{дрвн}) на \textsc{деревн}, \textsc{пр(с)нодв} на \textsc{приснодев}, при замене \textsc{ннѣ} (и окказионального \textsc{нне}) на \textsc{нынѣ} дополнительно учтены префиксальные дериваты (\textsc{донынѣ}, \textsc{о(т)нынѣ}, \textsc{понынѣ}) и~т.",д.

Кроме того, доступ к данным морфологической разметки позволил корректно производить замены неоднозначных сокращений с выносными буквами и под титлом, которые необходимо раскрывать по-разному в зависимости от принадлежности нормализуемой словоформы к тому или иному лексико-грамматическому классу: сочетания \textsc{гн} (под титлом) и \textsc{г(с)дн} в начальной позиции форм существительных подлежат замене на \textsc{господин}, в случае же прилагательных~"--- на \textsc{господн} (при этом сначала осуществляется замена надстрок \textsc{гнь} и \textsc{г(с)днь} на вариант без редукции~"--- \textsc{господень}); аналогичным образом \textsc{ч(с)т} приводится к написанию \textsc{чест} либо \textsc{чист}. Стандартное сокращение адъективной флексии род.~п.\ ед.~ч.\ м.\ и ср.~р.\ при помощи выносного \textsc{(г)} раскрывается как \textsc{гъ} в абсолютном конце словоформ иных частей речи с целью учёта окказиональных написаний типа \textsc{вра(г)} и \textsc{*выпря(г)}.

В результате работы Е.",Г.~Уфлянд вариативность в словоуказателе не была устранена полностью: вне её рассмотрения остались такие явления, как
\begin{inparaenum}[(1)]
    \item чередование редуцированных и гласных полного образования в корнях и префиксах (\textsc{въсхитити}~"--- \textsc{восхитити}),
    \item наличие дублетов с удвоенными согласными (\textsc{воистину}~"--- \textsc{воистинну}),
    \item непоследовательное написание ятя, весьма характерное для поздних рукописей и отражающее фонологическую нестабильность соответствующей фонемы, проявлявшуюся в речи писцов и переписчиков того времени (\textsc{грѣхъ}~"--- \textsc{грехъ})
\end{inparaenum}
\autocite[378]{uflyand_alexeeva:2008}. Поскольку же решение обозначенных и связанных с ними проблем не входило в круг наших непосредственных задач, постольку орфографический разнобой частично "<процеживается"> через сито дальнейших этапов разработанного алгоритма и влияет на графический облик некоторых из получаемых на выходе лемм. Это обстоятельство, однако, лишь пуще подтверждает актуальность данной проблематики, диктующую необходимость дальнейших разработок в этой области с привлечением нового корпусного материала.

\section{Стемминг}

\subsection{Методологические замечания}

Традиционно среди методов автоматизированного морфологического анализа на основе правил (\foreignlanguage{english}{rule-based}) ключевое место отводится лемматизации и стеммингу. В статье В.",В.~Бочарова и О.",В.~Митрениной \autocite[21]{mono_morphology} различие между ними проводится на основании того, используется ли в процессе анализа какой-либо словарный ресурс (полноценный грамматический словарь, словарь основ либо отдельных морфем и~т.",д.)\ или же разбор текстовых форм сводится к обработке словоизменительных формантов с точечным привлечением словарей ограниченного объёма (в основном для учёта всевозможных исключений). Однако в условиях того, что разработки каких-либо грамматических словарных ресурсов на материале СКАТ ранее не велись, производить лемматизацию в обозначенном смысле нам не представляется возможным. Более широко под лемматизацией обычно подразумевается "<идентификация инвариантов лексических единиц (выражение ЛО [лексикографического описания.~"--- \textit{К.",С.}]\ с точностью до отдельной лексемы)"> \autocite[76]{koval:2005}; очевидно, при таком понимании на то, каким образом осуществляется переход от словоформ к леммам, не накладывается никаких существенных ограничений, и именно его мы склонны далее придерживаться.

В нашем подходе лемматизация осуществляется опосредованно~"--- через промежуточный этап стемминга. Однако в отличие от "<классического"> стемминга представление о псевдоосновах и псевдофлексиях (неизменяемых начальных и конечных сегментах) не находит в нём своего воплощения: известные из разметки морфологические свойства анализируемых словоформ позволяют максимально точно отделить собственно основы от собственно флексий, лингвистически интерпретируемых и несущих полноценное грамматическое значение, и избежать ошибок \foreignlanguage{english}{overstemming} и \foreignlanguage{english}{understemming}, типичных для "<слепого"> стемминга без учёта семантики.

Таким образом, применение данной процедуры к словоформе \textsc{день} (пример, приводимый в \autocite[20]{mono_morphology}) призвано отсечь не псевдофлексию \textsc{-ень} от псевдооссновы \textsc{д-}, но \textit{флексию} \textsc{-ь} от \textit{основы} \textsc{ден-}. Преобразование получаемых таким образом основ с целью их отождествления с основами соответствующих лемм (ср.\ \textsc{ден-} и \textsc{дн-}) происходит уже на следующем этапе алгоритма.

\subsection{Классы словоизменительных парадигм}

Словоизменительные парадигмы, сопоставляющие грамматические значения выражаемым ими финальным сегментам, были составлены с опорой на авторитетные учебно-научные пособия по старославянскому языку \autocites{stsl:2013}{ivanova:1998}{khaburgaev:1986}; составляющие их флексии для учёта неустранимой на этапе нормализации орфографической вариативности записаны в виде регулярных выражений. Условно все парадигмы можно отнести к одному из следующих классов.

\paragraph{Именные парадигмы}

При помощи именных парадигм производится стемминг
\begin{inparaenum}[(1)]
    \item существительных,
    \item нечленных (кратких) прилагательных и причастий,
    \item несоставных количественных числительных,
    \item неличных местоимений в именительном и винительном падежах.
\end{inparaenum}
Они характеризуются наибольшим качественным разнообразием и значительной вариативностью плана выражения флексий, и для их составления было привлечено множество дополнительных материалов, в частности \autocite{agio:1990}.

Например, частная именная парадигма *j\={a}-склонения во множественном числе мужского (и одновременно женского) рода имеет следующий вид:

\begin{Verbatim}[fontsize=\small, gobble=4, xleftmargin=5ex]
    ('ja', 'им', 'мн', 'м'): '[+АЕИЫЯ]',
    ('ja', 'род', 'мн', 'м'): '[+ЕИЫ]И|[ЪЬ]',
    ('ja', 'дат', 'мн', 'м'): '[АЯ]М[ЪЬ`]',
    ('ja', 'вин', 'мн', 'м'): '[+АЕИЫЯ]',
    ('ja', 'тв', 'мн', 'м'): '[АЯ]МИ',
    ('ja', 'мест', 'мн', 'м'): '[АЯ]Х[ЪЬ`]',
    ('ja', 'зв', 'мн', 'м'): '[+АЕИЫЯ]',
\end{Verbatim}

\paragraph{Местоименные парадигмы}

Парадигмы местоименного класса задействуются при анализе членных (полных) форм причастий и прилагательных, к числу которых мы относим и порядковые (последние, однако, размечаются тегом \textit{числ/п}), а также неличных местоимений в косвенных падежах (кроме винительного). Собственно адъективные флексии фонетически и орфографически весьма близки (их стяжённые разновидности~"--- практически идентичны) местоименным, и уже рукописи X--XI~вв. обнаруживают результаты их взаимодействия и уподобления \autocite[170--171]{khaburgaev:1986}; исходя из этих соображений мы сочли приемлемым объединить их в общий класс.

В качестве примера приведём парадигму местоименного склонения по твёрдому типу в единственном числе среднего рода:

\begin{Verbatim}[fontsize=\small, gobble=4, xleftmargin=5ex]
    ('тв', 'им', 'ед', 'ср'): 'О?Е',
    ('тв', 'род', 'ед', 'ср'): 'А?[АЕО]?ГО',
    ('тв', 'дат', 'ед', 'ср'): 'У?[ЕОУ]?МУ',
    ('тв', 'вин', 'ед', 'ср'): 'О?Е',
    ('тв', 'тв', 'ед', 'ср'): '[ИЫ]?[+ЕИЫ]М[ЪЬ`]',
    ('тв', 'мест', 'ед', 'ср'): '[+Е]?[+ЕО]М[ЪЬ`]',
    ('тв', 'зв', 'ед', 'ср'): 'О?Е',
\end{Verbatim}

\paragraph{Глагольные парадигмы}

Обращение к глагольным парадигмам происходит при анализе спрягаемых форм глаголов, а также эловых причастий. Церковнославянское спряжение в целом характеризуется большей стройностью и меньшей вариативностью, нежели склонение; основные трудности при лемматизации глагольных форм (в~т.",ч.\ вербоидов) связаны со следующим этапом восстановления словарных основ.

Так, например, выглядит парадигма спряжения глаголов в форме сигматического аориста:

\begin{Verbatim}[fontsize=\small, gobble=4, xleftmargin=5ex]
    ('1', 'ед'): 'Х?[ЪЬ`]',
    ('2', 'ед'): '[+Е]',
    ('3', 'ед'): '[+Е]',
    ('1', 'дв'): 'Х?ОВ[+Е]',
    ('2', 'дв'): 'С?Т[+АЕ]',
    ('3', 'дв'): 'С?Т[+АЕ]',
    ('1', 'мн'): 'Х?ОМ[ЪЬ`]',
    ('2', 'мн'): 'С?ТЕ',
    ('3', 'мн'): 'Ш?[АЯ]',
\end{Verbatim}

\subsection{Особые случаи}

Морфологически неизменяемые формы~"--- а именно, несклоняемые прилагательные (близкие по значению к наречиям и весьма немногочисленные \autocite[140--141]{ivanova:1998}: в каждом из размеченных текстов единожды употреблено лишь прилагательное \textsc{исполнь} `полный'), инфинитивы и супины (последние на рассмотренном материале не встречаются вовсе), наречия, предлоги и послелоги, союзы, частицы, междометия~"--- стеммингу естественным образом не подлежат: в качестве лемм им присваиваются их нормализованные формы. Однако лемматизация посредством стемминга также затруднена либо принципиально невозможна и применительно к некоторым отдельным группам изменяемых слов, заслуживающих отдельного рассмотрения.

\subsubsection{Составные существительные}

На материале рассмотренных житийных текстов выделяются две семантико-морфологические группировки существительных со склонением нескольких частей в их составе.

\begin{asparaenum}
    \item Во-первых, речь о названиях населённых пунктов со второй корневой морфемой \textsc{-град-} либо \textsc{-город-} (полногласный вариант низкочастотен, но встречается в некоторых текстах корпуса): \textsc{*костянтинъградъ}, \textsc{*новъградъ}. При анализе подобных имён собственных мы исходим из посылки, что их первая составляющая всегда представляет собой существительное или прилагательное мужского рода *\u{o}- (см.\ примеры выше) или *j\u{o}-склонения (ср.\ \textsc{*царьградъ}), т.",е.\ изменяется аналогично либо сходно существительному \textsc{градъ}. Таким образом, тип склонения первой части можно считать известным, а процедуру стемминга единообразно производить над обоими компонентами: \textsc{*новѣградѣ}~$\to$ \textsc{-нов-}, \textsc{-град-}.

    \item Во-вторых, спецификой в рассматриваемом аспекте обладают наименования времён суток \textsc{полдень} и \textsc{полнощь} (русизм \textsc{полночь} в корпусе не встречается). Здесь типы склонения составных частей не совпадают: существительные \textsc{день} и \textsc{нощь} склоняются по типам *en и *\u{\i} соответственно, а \textsc{полъ} (в значении `половина')~"--- по типу *\u{u}. Однако несмотря на то, что последний нам априорно известен, и при соответствующей поправке первая часть также подлежит стеммингу, ввиду предельной ограниченности и закрытости данной группы существительных было решено проверять их начальные подстроки на совпадение простым регулярным выражениям: \verb|ПОЛ.*Д[ЕЬ]?Н| либо \verb|ПОЛ.*НОЩ|~"--- и при положительном результате приписывать готовые леммы без какого-либо анализа грамматических данных.
\end{asparaenum}

\subsubsection{Составные числительные}

В церковнославянском языке сложносоставным количественным числительным присущи специфические словоизменительные особенности, отчасти похожие на таковые в современном русском.

\begin{asparaenum}
    \item Числительные, обозначающие числа от 11 до 19, представляют собой сочетания единиц первого десятка с предложно-падежной группой \textsc{на десяте} (мест.~п.). Формальный тип синтаксической связи внутри подобных структур~"--- предложно-падежное примыкание, и изменению подвержена только первая часть: \textsc{пятьнадесяте}~"--- \textsc{пятинадесяте}.

    \item Названия чисел 20, 30, 40 и 200, 300, 400 являются сочетаниями имён соответствующих единиц с существительными \textsc{десять} (склоняется по типу *ent) либо \textsc{сто} (типа *\u{o}). Тип связи~"--- согласование, а следовательно, при склонении изменяются оба компонента: \textsc{двадесяти}~"--- \textsc{двудесяту} (дв.~ч.), \textsc{триста}~"--- \textsc{трехъсотъ}  (мн.~ч.).

    \item Обозначения чисел от 50 до 90 и от 500 до 900 лексически подобны названиям десятков и сотен меньших порядков, однако синтаксически ведут себя иначе: здесь наименования единиц управляют существительными \textsc{десять} или \textsc{сто} в форме род.~п.\ мн.~ч. Изменяется только первая часть: \textsc{седмьдесятъ}~"--- \textsc{седмидесятъ}, \textsc{осмьсотъ}~"--- \textsc{осмисотъ}.
\end{asparaenum}

Принимая во внимание закрытость множества составных числительных и стремясь избежать излишних технологических затруднений, которые могли бы возникнуть при их прямолинейном анализе, здесь мы также ограничились описанием морфемной структуры в виде регулярных выражений:

\begin{Verbatim}[fontsize=\small, gobble=4, xleftmargin=5ex]
    'ЕДИН.*НАДЕСЯТ': 'ЕДИННАДЕСЯТЬ',
    'Д[ЪЬ]?В.*НАДЕСЯТ': 'ДВАНАДЕСЯТЬ',
    'ТР.*НАДЕСЯТ': 'ТРИНАДЕСЯТЬ',
    'ЧЕТЫР.*ДЕСЯТ': 'ЧЕТЫРЕДЕСЯТЕ',
    'ПЯТ.*ДЕСЯТ': 'ПЯТЬДЕСЯТЪ',
    'ШЕСТ.*ДЕСЯТ': 'ШЕСТЬДЕСЯТЪ',
    'СЕДМ.*С[ЪО]?Т': 'СЕДМЬСОТЪ',
    'ОСМ.*С[ЪО]?Т': 'ОСМЬСОТЪ',
    'ДЕВЯТ.*С[ЪО]?Т': 'ДЕВЯТЬСОТЪ',
\end{Verbatim}

\subsubsection{Местоимения нерегулярного склонения}

Словоизменению местоимений указанных разрядов, а именно:
\begin{inparaenum}[(1)]
    \item личных \textsc{азъ}, \textsc{ты} (ед.~ч.), \textsc{вѣ}, \textsc{ва} (дв.~ч.), \textsc{мы}, \textsc{вы} (мн.~ч.);
    \item возвратного \textsc{себе};
    \item вопросительных \textsc{кто} и \textsc{что}~"---
\end{inparaenum}
присуще множество глубоко архаических черт (ярко выраженный супплетивизм, особая система флексий), из которых следует принципиальная невозможность их анализа по общим правилам. Их парадигмы обособлены от остальных классов и имеют несколько иную организацию; описывающие их словоизменение регулярные выражения представляют собой полные покрытия соответствующих символьных цепочек, а обращение к разметке производится исключительно с целью выявления возможных ошибок.

Примеры парадигм местоимений всех трёх разрядов:

\begin{Verbatim}[fontsize=\small, gobble=4, xleftmargin=5ex]
    ('1', 'им', 'ед'): ('АЗ[ЪЬ`]?$', 'АЗЪ'),
    ('1', 'род', 'ед'): ('М([ЕЪЬ]?Н)?[+ЕЯ]$', 'АЗЪ'),
    ('1', 'дат', 'ед'): ('М([ЕЪЬ]?Н[+Е]|И)$', 'АЗЪ'),
    ('1', 'вин', 'ед'): ('М([ЕЪЬ]?Н)?[+ЕЯ]$', 'АЗЪ'),
    ('1', 'тв', 'ед'): ('М[ЪЬ]?НОЮ$', 'АЗЪ'),
    ('1', 'мест', 'ед'): ('М[ЪЬ]?Н[+Е]$', 'АЗЪ'),

    'род': ('С([ЕО]Б)?[+ЕЯ]$', 'СЕБЕ'),
    'дат': ('С([ЕО]Б[+Е]|И)$', 'СЕБЕ'),
    'вин': ('С([ЕО]Б)?[+ЕЯ]$', 'СЕБЕ'),
    'тв': ('СОБОЮ$', 'СЕБЕ'),
    'мест': ('С[ЕО]Б[+Е]$', 'СЕБЕ'),

    ('м', 'им'): ('Ч[ЪЬ]?ТО$', 'ЧТО'),
    ('м', 'род'): ('Ч[ЕЬ]?(СО)?(ГО)?$', 'ЧТО'),
    ('м', 'дат'): ('Ч[ЕЬ]?(СО)?МУ$', 'ЧТО'),
    ('м', 'вин'): ('Ч[ЪЬ]?ТО$', 'ЧТО'),
    ('м', 'тв'): ('ЧИМ[ЪЬ`]?$', 'ЧТО'),
    ('м', 'мест'): ('Ч[ЕЬ]?(СО)?М[ЪЬ`]?$', 'ЧТО'),
\end{Verbatim}

Полностью аналогично вопросительным местоимениям обрабатываются неопределённые (\textsc{нѣкто}, \textsc{нѣчто}) и отрицательные (\textsc{никтоже}, \textsc{ничтоже}).

Наконец, в связи с вопросительными местоимениями заслуживает упоминания особое определительное местоимение \textsc{кождо}\footnote{%
    В "<Материалах"> И.",И.~Срезневского также приводятся примеры с конечным сегментом \textsc{-жде-}, но в корпусе подобных употреблений зафиксировано не было.
} `каждый', в косвенных падежах изменяющееся по образцу вопросительного \textsc{кто} (при этом сегмент \textsc{-ждо-} ведёт себя подобно слитной частице \textsc{-же-} и не оказывает влияния на словоизменение): \textsc{когождо}, \textsc{комуждо} \autocite[I,][1389]{sreznevsky}. Отметим также, что наряду с ним существует синонимичное местоимение \textsc{кииждо} (с вариантом \textsc{коиждо}), однако оно, в отличие от \textsc{кождо}, имеет регулярную парадигму на основе вопросительного \textsc{кии}: \textsc{коегождо}, \textsc{коемуждо} \autocite[I,][1417]{sreznevsky}~"--- и потому обрабатывается по общему правилу.

\subsubsection{Аналитические глагольные формы}

Некоторые церковнославянские глагольные формы образуются с использованием вспомогательных глаголов, т.",е.\ аналитическим способом. Соответствующий перечень включает в себя следующие формы:
\begin{compactenum}
    \item формы будущего~I, представляющие собой сочетания одного из следующих вспомогательных глаголов в настоящем-будущем времени: \textsc{имѣти}, \textsc{начати} или \textsc{хотѣти}~"--- с инфинитивом основного глагола;
    \item формы будущего~II, перфекта и плюсквамперфекта, образующиеся при помощи глагола-связки \textsc{быти} в одном из синтетических времён (соответственно простом будущем, настоящем-будущем, в случае плюсквамперфекта~"--- имперфекте либо аористе от имперфектной основы \textsc{-бѣ-}) и элового причастия смыслового глагола;
    \item формы сослагательного наклонения, также выражаемые эловым причастием смыслового глагола в сочетании со связкой \textsc{быти} в аористе.
\end{compactenum}

Поскольку глагольные связки в аналитических формах является десемантизированными, их лемматизация наравне с омонимичными полнозначными формами привела бы к неоправданному искажению статистики по соответствующим леммам. Поэтому в случае аналитических форм стандартным образом обрабатываются только смысловые глаголы, в то время как вспомогательным в качестве лемм приписываются специальные теги: \textsc{aux-ft1}, \textsc{aux-ft2}, \textsc{aux-prf}, \textsc{aux-pqp} и \textsc{aux-sbj}.

\section{Восстановление словарной основы и финализация имён}

Под "<именами"> в настоящем разделе понимаются все словоформы, над которыми процедура стемминга производится при помощи именных или местоименных парадигм, за исключением причастий~"--- они рассматриваются в следующем разделе наряду с личными глагольными формами. Не вполне общеупотребительный термин "<финализация"> используется вместо описательного оборота "<добавление словарной финали"> (будь то флексия или суффикс).

\subsection{Нововведения в~формат морфологической разметки}

В ходе нашей работы выяснилось, что степень подробности формата морфологической разметки в корпусе СКАТ, описанного в разделе~\ref{sec:scat}, не всегда позволяет в полной мере учесть словоизменительные особенности имён, на которые необходимо делать поправку для того, чтобы лемматизация была произведена корректно. Вследствие этого мы сочли необходимым внести в формат аннотации определённые новшества и уточнения, а также обновить существующие разметки житий Димитрия Прилуцкого, Дионисия Глушицкого и Кирилла Новоезерского с их учётом.

Примеры реализации описанных нововведений при морфологической разметке приведены в таблице~\ref{tab:new}.

\begin{table}[p]
    \small
    \begin{tabularx}{\linewidth}{Xp{1.5cm}p{1.5cm}p{1.5cm}p{1.5cm}p{1.5cm}p{1.5cm}}
        \toprule
        \textsc{братir}      & сущ  & o/ja & им   & мн & м  &       \\ \midrule
        \textsc{братiами}    & сущ  & o/ja & тв   & мн & м  &       \\ \midrule
        \midrule
        \textsc{хр(с)тiане}  & сущ  & o/en & им   & мн & м  &       \\ \midrule
        \textsc{вrтчrне}     & сущ  & o/en & тв   & мн & м  &       \\ \midrule
        \midrule
        \textsc{dста}        & сущ  & o    & вин  & pt & ср &       \\ \midrule
        \textsc{перси}       & сущ  & i    & вин  & pt & ж  &       \\ \midrule
        \midrule
        \textsc{рdцѣ}        & сущ  & a    & вин  & дв & ж  & *     \\ \midrule
        \textsc{страсѣ}      & сущ  & o    & мест & ед & м  & *     \\ \midrule
        \textsc{мнwзи}       & прил & o    & им   & мн & м  & *     \\ \midrule
        \textsc{еллинстiи}   & прил & тв   & им   & мн & м  & *     \\ \midrule
        \midrule
        \textsc{dглѣ}        & сущ  & o    & мест & ед & м  & $+$о  \\ \midrule
        \textsc{помыслы}     & сущ  & o    & вин  & мн & м  & $+$е  \\ \midrule
        \textsc{зwлъ}        & сущ  & o    & род  & мн & ср & $-$о  \\ \midrule
        \textsc{сdдебъ}      & сущ  & a    & род  & мн & ж  & $-$е  \\ \midrule
        \textsc{старцd}      & сущ  & jo   & дат  & ед & м  &       \\ \midrule
        \textsc{wвець}       & сущ  & ja   & род  & мн & ж  &       \\ \midrule
        \textsc{тоно(к)}     & прил & o    & вин  & ед & м  & $-$о  \\ \midrule
        \textsc{ра(до)стенъ} & прил & о    & им   & ед & м  & $-$е  \\ \midrule
        \midrule
        \textsc{хр(с)тiаны}  & сущ  & o    & вин  & мн & м  & $+$ин \\ \midrule
        \textsc{татары}      & сущ  & o    & вин  & мн & м  & $+$ин \\ \midrule
        \textsc{по(с)}       & сущ  & o    & вин  & ед & м  & $+$т  \\ \midrule
        \textsc{пѣ(с)ми}     & сущ  & i    & тв   & мн & ж  & $+$н  \\ \midrule
        \textsc{роже(н)и}    & сущ  & jo   & мест & ед & ср & $+$и  \\ \midrule
        \textsc{бранiи}      & сущ  & ja   & вин  & ед & ж  & $-$и  \\ \bottomrule
        \caption{Примеры морфологической разметки с учётом нововведений}
        \label{tab:new}
    \end{tabularx}
\end{table}

\subsubsection{Спецификации типов склонения}

\paragraph{Тип \textit{o/ja}}

На материале исследованных текстов этот смешанный тип был зафиксирован только у форм существительного \textsc{братъ}, чья основа во множественном числе представлена корневым алломорфом \textsc{-братi-}. Исходя из предположения о том, что и другие существительные с таким смешением обнаруживают подобное алломорфирование (ср.\ рус.\ \textit{лист}~"--- \textit{листья}), данный тип предлагается использовать как маркер наличия йотового наращения у основы и необходимости его удаления при лемматизации.

\paragraph{Тип \textit{o/en}}

Данный тип присваивается формам множественного числа существительных, обозначающих человека по роду деятельности, происхождению или вероисповеданию с суффиксом \textsc{-ин-} в единственном числе. Во множественном последний подвергается утрате, и тогда тип \textit{o/en} свидетельствует о том, что в ходе лемматизации его необходимо восстановить.

\subsubsection{Существительные \foreignlanguage{english}{pluralia tantum}}

Существительным, употребляющимся только во множественном числе (\foreignlanguage{english}{pluralia tantum}), в данной позиции вместо тега \textit{мн} присваивается специальный тег \textit{pt}. Его наличие далее (см.\ раздел~\ref{subsec:fin}) позволяет добавлять к словарным основам подобных существительных флексии множественного числа вместо единственного.

\subsubsection{Дополнительные пометы}

Ввиду того, что в составляющих разметку таблицах последний, шестой столбец при аннотации имён не используется (он задействован только для морфологического описания форм глаголов в форме настоящего-будущего времени и причастий, а также для морфосинтаксического описания компонентов форм аналитических времён), мы воспользовались этим обстоятельством и отвели его под ряд принципиально новых помет, фиксирующих регулярные морфонологические явления на стыке основы и флексии.

\paragraph{Помета \textit{*}}

Астериск в последнем столбце свидетельствует о том, что в абсолютном конце основы размечаемой словоформы действует закон второй палатализации~"--- переход заднеязычных согласных \textsc{к}, \textsc{г}, \textsc{х} в мягкие свистящие \textsc{ц}, \textsc{з}, \textsc{с} перед \textsc{ѣ} или \textsc{и} дифтонгического происхождения.

У существительных (а также у нечленных прилагательных) вторая палатализация имеет место в следующих парадигматических позициях:

\begin{compactenum}
    \item *\={a}-склонение: \begin{inparaenum}[(1)]
        \item дат.~п.\ ед.~ч.;
        \item мест.~п.\ ед.~ч.;
        \item им.-вин.~п.\ дв.~ч.\
    \end{inparaenum} (\textsc{рdка}~"--- \textsc{рdцѣ});
    
    \item *\u{o}-склонение м.~р.: \begin{inparaenum}[(1)]
        \item мест.~п.\ ед.~ч.;
        \item им.~п.\ мн.~ч.;
        \item мест.~п.\ мн.~ч.\
    \end{inparaenum} (\textsc{rзыкъ}~"--- \textsc{rзыцѣ}, \textsc{rзыцы}, \textsc{rзыцѣхъ});
    
    \item *\u{o}-склонение ср.~р.: \begin{inparaenum}[(1)]
        \item мест.~п.\ ед.~ч.;
        \item им.-вин.~п.\ дв.~ч.;
        \item мест.~п.\ мн.~ч.\
    \end{inparaenum} (\textsc{вѣко}~"--- \textsc{вѣцѣ}, \textsc{вѣцѣхъ}).
\end{compactenum}

У прилагательных местоименного склонения (речь только о твёрдом типе~"--- ввиду твёрдости согласных \textsc{к}, \textsc{г}, \textsc{х}) вторая палатализация происходит в
\begin{inparaenum}[(1)]
    \item дат.~п.\ ед.~ч.\ ж.~р.;
    \item мест.~п.\ ед.~ч.\ м., ж.\ и ср.~р.;
    \item им.-вин.~п.\ дв.~ч.\ ж.\ и ср.~р.;
    \item им.~п.\ мн.~ч.\ м.~р.\
\end{inparaenum} (\textsc{благии}~"--- \textsc{блазѣи}, \textsc{блазѣмъ}, \textsc{блазiи}). Кроме того, в местоименном склонении имеет место особая разновидность второй палатализации с чередованием \textsc{-ск-}~// \textsc{-ст-}: \textsc{члческии\#}~"--- \textsc{члчестѣи\#} и~т.",д.\ \autocite[139]{stsl:2013}.

Фиксировать случаи второй палатализации на конце основ существительных и прилагательных абсолютно необходимо для того, чтобы лемматизация была корректной, поскольку по формальным признакам невозможно отличить сибилянты, возникшие в результате палатализации, от этимологических: \textsc{нозѣ}~"--- лемма \textsc{нога}, но \textsc{трапезѣ}~"--- лемма \textsc{трапеза}; \textsc{дуси}~"--- \textsc{духъ}, но \textsc{бѣси}~"--- \textsc{бѣсъ}.

\paragraph{Пометы \textit{$\pm$о} и \textit{$\pm$е}}

Наличие одной из указанных четырёх помет указывает на то, что в последнем слоге словарной основы по сравнению с основой размеченной словоформы имеет место прояснение ($+$) либо падение ($-$) этимологического редуцированного (соответственно \textsc{ъ} или \textsc{ь}).

Указанные процессы обусловлены тем, что в последнем слоге формообразующих основ сильные и слабые позиции (см.\ \autocite[50--51]{khaburgaev:1986}) могут чередоваться: \textsc{сонъ}~"--- \textsc{сна}, \textsc{веренъ}~"--- \textsc{верна}; подобное чередование возможно только при наличии в словоизменительной парадигме форм с односложными редуцированными флексиями (современными нулевыми), а следовательно, актуально исключительно для именных парадигм. Более того, размечать его целесообразно отнюдь не для всех словоформ, склоняющихся по именному типу: так, среди несоставных количественных числительных и неличных местоимений в данном аспекте выделяются лишь лексемы \textsc{сто} (род.~п.\ мн.~ч.\ \textsc{сотъ}) и \textsc{весь} (род.~п.\ мн.~ч.\ \textsc{всѣхъ}), легко поддающиеся словарному заданию.

Таким образом, приписыванию помет \textit{$\pm$о} и \textit{$\pm$е} подлежат лишь существительные и нечленные прилагательные. При этом:

\begin{compactenum}
    \item высокочастотные существительные с суффиксом \textsc{-ц-}~// \textsc{-ец-} всегда обнаруживают процессы прояснения и падения редуцированного \textsc{ь} и потому обрабатываются автоматически;
    \item поскольку все прилагательные стандартно (см.\ \ref{subsec:fin}) приводятся к членным формам, где словарные флексии всегда ненулевые, здесь надлежит бороться только с результатами прояснения.
\end{compactenum}

\paragraph{Семейство помет \textit{$\pm$X}}

Если \textit{X}~"--- произвольная последовательность символов, отличная от \textit{o} и \textit{е}, то соответствующая помета служит для восстановления пропущенных ($+$) или удаления избыточных ($-$) букв и буквосочетаний на конце основы при лемматизации.

Подобные пропуски и вставки в большинстве своём носят идиосинкратический характер и во многом обусловлены орфографическим контекстом и привычками конкретного писца; тем не менее, иногда постановку указанных помет определяют и причины иного рода. Например, помета \textit{$+$ин} регулярно приписывается формам множественного числа существительных, закономерно утративших суффикс деятеля \textsc{-ин-}, но склоняющихся не по ожидаемому типу на согласный (такие случаи предусматривает спецификация типа склонения \textit{o/en}, о которой говорилось выше), а по образцу *\u{o}-склонения.

\subsection{Прочие преобразования основ и финализация}
\label{subsec:fin}

\paragraph{Существительные}

Из существительных следующих подтипов склонения на согласный: *ent, *men, *es, *er~"--- удаляются тематические суффиксы: \textsc{-врѣмен-}~"--- \textsc{-врѣм-}, \textsc{-словес-}~"--- \textsc{-слов-} и~т.",д. С другой стороны, в случае отсутствия тематических суффиксов в составе основ существительных подтипов *en и *uu последние, напротив, восстанавливаются из соображений модернизации лемм (архаичные формы им.~п.\ ед.~ч.\ без осложнения довольно рано начали замещаться формами вин.~п.\ \autocite[126]{ivanova:1998}): \textsc{камы}~"--- \textsc{камень}, \textsc{любы}~"--- \textsc{любовь}. Сходным образом восстановлению подлежит субморф \textsc{-ос-} у частотного имени собственного \textsc{*христосъ}, подвергающийся утрате в косвенных падежах.

Кроме того, основы существительных (и среди имён только их) в отдельных парадигматических позициях подвергаются закону первой палатализации~"--- переходу заднеязычных согласных \textsc{к}, \textsc{г}, \textsc{х} в мягкие шипящие \textsc{ч}, \textsc{ж}, \textsc{ш} в положении перед гласными переднего ряда, а также свистящего \textsc{ц} в составе суффикса деятеля мужского пола и \textsc{з} в слове \textsc{князь}. Фактически сфера действия данного закона ограничена
\begin{inparaenum}[(1)]
    \item формой зв.~п.\ ед.~ч.\ в парадигме *\u{o}-склонения м.~р.: \textsc{бгъ\#}~"--- \textsc{бже\#}, \textsc{отець}~"--- \textsc{отче} (в последнем случае имеет место смешение *j\u{o}/*\u{o}),
    \item формами дв. и мн.~ч.\ двух существительных, обозначавших части тела: \textsc{око}~"--- \textsc{очи} (смешение *es/*\u{\i})~"--- \textsc{очеса} и аналогично \textsc{ухо}~"--- \textsc{уши}~"--- \textsc{ушеса}.
\end{inparaenum}
Первая палатализация, в отличие от второй, всегда предсказуема и потому устраняется автоматически.

На этапе финализации к основам прибавляются флексии им.~п.\ ед.~ч.\ (мн.~ч., если данное существительное размечено как \foreignlanguage{english}{plurale tantum}) с последовательным учётом типа склонения и родовой принадлежности.

\paragraph{Прилагательные}

Из форм прилагательных сравнительной степени (в позиции части речи они размечаются особым тегом \textit{прил/ср}) удаляется суффикс \textsc{-ш-} (включая алломорфы), присущий всем членам словоизменительной парадигмы, кроме им.~п.\ ед.~ч.\ м.\ и ср.~р. Заметим, однако, что мы считаем компаратив самостоятельной грамматической категорией и никаких дальнейших преобразований, нацеленных на перевод сравнительной степени в положительную (в~т.",ч.\ устранение супплетивизма типа \textsc{болшии}~"--- \textsc{великии}), не производим.

В ходе финализации к основам всех прилагательных прибавляются \textit{местоименные} флексии им.~п.\ ед.~ч.\ м.~р.\ \textsc{-ыи} (к основам твёрдой разновидности, кроме основ на заднеязычные) или \textsc{-ии} (ко всем прочим): \textsc{непорочныи}, \textsc{божии}; к несупплетивным формам компаратива (на гласную) добавляется \textsc{-и}: \textsc{грѣшнѣи}. Исключение составляют имена собственные: к их основам наряду с местоименными флексиями могут добавляться и именные, а также сохраняется их родовая характеристика: \textsc{бѣлыхъ} (\textsc{ризахъ})~"--- лемма \textsc{бѣлыи}, но \textsc{*бѣла} (\textsc{*езера})~"--- лемма \textsc{*бѣло}.

\paragraph{Местоимения и числительные}

Основы местоимений внутри словоизменительных парадигм практически не обнаруживают вариативности. Точечные модификации требуются в отдельных частных случаях: \textsc{-ко-}~$\to$ \textsc{-к-} (\textsc{кии}~"--- \textsc{коего}, \textsc{коему} и~т.",д.), \textsc{-с-}~$\to$ \textsc{-се-} (унификация альтернантов \textsc{сеи} и \textsc{сии} в пользу более частотного); из предложных форм местоимения \textsc{и}: \textsc{него}, \textsc{нему} и~т.",д.~"--- удаляется наращение \textsc{-н-} (его основа, таким образом, считается нулевой). Финализация происходит аналогично прилагательным, и лишь нескольким группам местоимений приписываются специфические флексии:
\begin{inparaenum}[(1)]
    \item \textsc{вашь}, \textsc{весь}, \textsc{нашь}, \textsc{сиць};
    \item \textsc{онъ}, \textsc{самъ};
    \item \textsc{елико};
    \item \textsc{онсица};
    \item \textsc{тои};
    \item \textsc{и}.
\end{inparaenum}

Среди числительных вариативный характер имеет лишь основа \textsc{-об-} (ср.\ \textsc{оба}~"--- \textsc{обоихъ}); в прочих преобразованиях они не нуждаются и сразу дополняются соответствующими именными флексиями им.~п.: дв.~ч.~"--- числительные \textsc{два} и \textsc{оба}, мн.~ч.~"--- \textsc{три} и \textsc{четыре}, ед.~ч.~"--- все прочие.

\section{Восстановление словарной основы и финализация глаголов}

В качестве леммы всех без исключения глагольных форм принимается форма инфинитива. Соответственно, этап финализации основ как личных форм, так и вербоидов осуществляется одинаково и далее не оговаривается: к восстановленным основам стандартно прибавляется инфинитивный суффикс \textsc{-ти-} и (при необходимости) возвратный постфикс \textsc{-ся-}, предварительно удаляемый из состава словоформ ещё до этапа стемминга.

\subsection{Преобразования основ прошедшего времени}
\label{subsec:past}

Традиционно в системе церковнославянского глагольного словоизменения выделяются две формообразующие основы: настоящего времени и инфинитива. От последней, помимо супина и собственно инфинитива, образуются все личные и неличные формы, обладающие семантикой предшествования, а именно:
\begin{inparaenum}[(1)]
    \item формы аориста,
    \item формы имперфекта,
    \item причастия прошедшего времени (как действительные, так и страдательные),
    \item эловые причастия.
\end{inparaenum}

\subsection{Преобразования основ настоящего времени}
\label{subsec:pres}

\section{Анализ работы алгоритма}
