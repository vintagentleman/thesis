\chapter{Лемматизация церковнославянских словоформ на~основе морфологической разметки}

В данной главе описывается алгоритм лемматизации, реализованный в ходе практической части настоящей работы; изложение теоретических проблем моделирования церковнославянского словоизменения и иных побочных вопросов в известной мере вторично и обусловлено необходимостью их алгоритмического решения. Разработанный программный пакет написан на языке Python (версии 3.6); исходный код находится в открытом доступе по ссылке: \url{https://github.com/vintagentleman/SCAT}.

\section{Нормализация}
\label{sec:norm}

В литературе по компьютерной морфологии имеет место определённая терминологическая путаница, связанная с терминами "<лемматизация"> и "<нормализация">: порой они употребляются взаимозаменяемо и считаются абсолютными синонимами \autocite[75]{koval:2005}, порой "<нормализации"> приписывают такое определение ("<постановка или словосочетания в каноническую форму">~"--- т.",е.\ базовую форму, или лемму \autocite[14]{mono_morphology}), которое исходя из словообразовательных соображений в большей степени подошло бы "<лемматизации">. Однако в контексте исторических языков, на наш взгляд, между данными терминами необходимо проводить чёткую границу: статус процедуры АОТ на морфологическом языковом уровне надлежит приписывать термину "<лемматизация">~"--- и только ему; "<нормализация">, с другой стороны, прежде всего связана с графическим слоем представления текста (исходя из чего производить её следует строго \textit{до} лемматизации и прочих этапов АОТ на более высоких языковых уровнях) и заключается в приведении различных орфографических вариантов одного и того же слова к некоторому инварианту \autocite[267]{mono_criticism}. Например, элементы следующего ряда:
\begin{inparaenum}[(1)]
    \item \textsc{блаженаго},
    \item \textsc{блаженна(г)},
    \item \textsc{бл(а)женнаго},
    \item \textsc{бла(ж)еннаго},
    \item \textsc{бла(ж)нна(г)},
    \item \textsc{бла(ж)ннаго},
    \item \textsc{блжена(г)\#},
    \item \textsc{блженаго\#},
    \item \textsc{блжен(н)аго\#},
    \item \textsc{блаженна(г)\#}~"---
\end{inparaenum}
представляют собой варианты записи одной и той же словоформы, "<нормальная форма"> которой имеет вид \textsc{блаженаго} или \textsc{блаженнаго}. В случае необходимости выбирать между различными орфографическими вариантами инвариантной единицы выбор во многом обусловлен методологическими установками конкретного исследователя или коллектива: так, при ориентации на близость к орфографии соответствующего современного языка в рассматриваемом примере предпочтение должно быть отдано последнему члену пары. Установку на современную ("<каноническую">) орфографию кладёт в основу своего определения нормализации \textcite[69]{piotrowski:2012}.

Е.",Г.~Уфлянд, ранее работавшая над проблемами нормализации в рамках СКАТ, также придерживалась курса на современную орфографию \autocite[41---42]{uflyand:2008}. В своём дипломном сочинении она очерчивает круг причин и характер вариативных написаний в исследованных текстах корпуса, а далее описывает разработанный ей алгоритм автоматического сведения орфографических вариантов словоформ к основному (т.",е.\ к нормальной форме), реализованный на Python 2.7 в качестве функционального ядра программы для уменьшения объёма сводного словоуказателя. Функция, о которой идёт речь (\texttt{modif}), последовательно проверяет входные символьные цепочки на наличие в них ненормализованных буквосочетаний различных типов (титлованных, содержащих в своём составе выносные буквы, сочетания определённых согласных и~т.",д.)\ и при их наличии осуществляет соответствующие замены, перечисленные в словарном компоненте модуля.

В ходе интеграции функции \texttt{modif} в нашу программу мы сочли необходимым привнести некоторые новшества в механизм её работы. Непосредственно в код было внесено множество технологических улучшений: словарная составляющая отныне отделена от алгоритмической (они разнесены по разным файлам), и для хранения информации о буквосочетаниях, подлежащих замене, в ней используется естественным образом напрашивающаяся структура данных типа "<словарь">, а не пары синхронизированных между собой массивов (вносить в подобную структуру любые изменения крайне проблематично). Замены были переписаны на языке регулярных выражений, что позволило более полно и экономно охватить вариативность плана выражения сводимых орфографических вариантов; их перечень также регулярно пополнялся по мере обнаружения нового релевантного языкового материала. Так, в список замен буквосочетаний в сокращённых словах под титлом были добавлены замены \textsc{др(в)н} (либо \textsc{дрвн}) на \textsc{деревн}, \textsc{пр(с)нодв} на \textsc{приснодев}, при замене \textsc{нн+} (и окказионального \textsc{нне}) на \textsc{нын+} дополнительно учтены префиксальные дериваты \textsc{донын+}, \textsc{о(т)нын+}, \textsc{понын+}~"--- и~т.",д.

Кроме того, доступ к данным морфологической разметки позволил корректно производить замены неоднозначных сокращений с выносными буквами и под титлом, обусловленных принадлежностью отдельно взятой словоформы к тому или иному лексико-грамматическому классу: сочетания \textsc{гн} (под титлом) и \textsc{г(с)дн} в начальной позиции форм существительных подлежат замене на \textsc{господин}, в случае же прилагательных~"--- на \textsc{господн} (при этом сначала осуществляется замена надстрок \textsc{гнь} и \textsc{г(с)днь} на вариант без редукции~"--- \textsc{господень}); аналогичным образом \textsc{ч(с)т} приводится к написанию \textsc{чест} либо \textsc{чист}. Стандартное сокращение адъективной флексии родительного падежа единственного числа при помощи выносного \textsc{(г)} раскрывается как \textsc{гъ} в абсолютном конце словоформ иных частей речи с целью учёта окказиональных написаний типа \textsc{вра(г)} и \textsc{*выпря(г)}.

В результате работы Е.",Г.~Уфлянд вариативность в словоуказателе не была устранена полностью: вне её рассмотрения остались такие явления, как
\begin{inparaenum}[(1)]
    \item чередование редуцированных и гласных полного образования в корнях и префиксах (\textsc{въсхитити}~"--- \textsc{восхитити}),
    \item наличие дублетов с удвоенными согласными (\textsc{воистину}~"--- \textsc{воистинну}),
    \item непоследовательное написание ятя, весьма характерное для поздних рукописей и отражающее фонологическую нестабильность соответствующей фонемы, проявлявшуюся в речи писцов и переписчиков того времени (\textsc{гр+хъ}~"--- \textsc{грехъ})
\end{inparaenum}
\autocite[378]{uflyand_alexeeva:2008}. Поскольку же решение обозначенных и связанных с ними проблем не входило в круг наших непосредственных задач (о единственном значимом исключении см.\ раздел~\ref{subsec:new}), постольку орфографический разнобой частично "<процеживается"> через сито дальнейших этапов разработанного алгоритма и влияет на графический облик некоторых из получаемых на выходе лемм,~"--- что, однако, лишь пуще подтверждает актуальность данной проблематики и необходимость дальнейших исследований и разработок в этой области.

\section{Стемминг}

* без словаря больше делать нечего
* источники флексий

\section{Восстановление основы леммы}

\subsection{Нововведения в~формат морфологической разметки}
\label{subsec:new}

\subsection{Имена}

\subsection{Глаголы}

\subsubsection{Основа прошедшего времени}

\subsubsection{Основа настоящего времени}

\section{Особые случаи}

\section{Анализ работы алгоритма}
