\chapter{Лемматизация церковнославянских словоформ на~основе морфологической разметки}

В данной главе поэтапно описывается алгоритм лемматизации морфологически размеченных текстов СКАТ, разработанный в ходе практической части настоящей работы; изложение проблем церковнославянского словоизменения и иных теоретических вопросов в известной мере вторично и прежде всего обусловлено необходимостью их алгоритмического решения. Программный пакет к текущей и следующей главам написан на языке Python (версии~3.6) и находится в открытом доступе на GitHub\footnote{\url{https://github.com/vintagentleman/SCAT}}.

В монографии С.",А.~Коваля лемматизация понимается как "<идентификация инвариантов лексических единиц (выражение ЛО [лексикографического описания.~"--- \textit{К.",С.}]\ с точностью до отдельной лексемы)"> \autocite[76]{koval:2005}. Иначе говоря, лемматизация~"--- такая аналитическая процедура, которая для любой входной словоформы позволяет определить, к парадигме какой лексемы она принадлежит, и на выходе эксплицировать словарное наименование этой последней~"--- лемму.

Сразу сделаем два замечания касательно границ применимости реализованного алгоритма. Во-первых, он ориентирован только на морфологически аннотированные словоформы: лемматизация без опоры на какие-либо уже имеющиеся ресурсы (будь то прецедентная разметка или грамматический словарь) принципиально не может быть надёжной и лингвистически корректной в 100",\% случаев. Во-вторых, на данный момент алгоритм способен приводить к леммам только словоформы именных частей речи, а также обрабатывает неизменяемые слова; попытки подступиться к глаголам были предприняты, однако формальный учёт их словоизменительных особенностей (в особенности это касается форм, образованных от основы настоящего времени) с целью приведения к форме инфинитива оказался весьма нетривиальной задачей, требующей дополнительного исследования.

Процедурно в ходе алгоритма лемматизации, реализованного в настоящей работе, ко входным словоформам последовательно применяются следующие преобразования: \begin{inparaenum}[(1)]
    \item орфографическая нормализация,
    \item стемминг,
    \item восстановление леммы;
\end{inparaenum} кроме того, на предварительном этапе подготовки морфологической разметки к обработке в неё вносятся определённые коррективы.

\section{Корректировка формата морфологической разметки}
\label{sec:corr}

В ходе нашей работы выяснилось, что степень подробности формата морфологической разметки в корпусе СКАТ, описанного в разделе~\ref{sec:scat}, не позволяет учесть все словоизменительные особенности имён, на которые необходимо делать поправку для того, чтобы в дальнейшем лемматизация была произведена корректно. Вследствие этого мы сочли необходимым внести в формат аннотации определённые новшества и уточнения, а также обновить существующие разметки житий Димитрия Прилуцкого, Дионисия Глушицкого и Кирилла Новоезерского с их учётом.

Примеры реализации описанных нововведений при морфологической разметке приведены в таблице~\ref{tab:new}.

\begin{table}[p]
    \small
    \begin{tabularx}{\linewidth}{Xp{1.5cm}p{1.5cm}p{1.5cm}p{1.5cm}p{1.5cm}p{1.5cm}}
        \toprule
        \textsc{братir}      & сущ  & o/ja & им   & мн & м  &       \\ \midrule
        \textsc{братiами}    & сущ  & o/ja & тв   & мн & м  &       \\ \midrule
        \midrule
        \textsc{хр(с)тiане}  & сущ  & o/en & им   & мн & м  &       \\ \midrule
        \textsc{вrтчrне}     & сущ  & o/en & тв   & мн & м  &       \\ \midrule
        \midrule
        \textsc{dста}        & сущ  & o    & вин  & pt & ср &       \\ \midrule
        \textsc{перси}       & сущ  & i    & вин  & pt & ж  &       \\ \midrule
        \midrule
        \textsc{рdцѣ}        & сущ  & a    & вин  & дв & ж  & *     \\ \midrule
        \textsc{страсѣ}      & сущ  & o    & мест & ед & м  & *     \\ \midrule
        \textsc{мнwзи}       & прил & o    & им   & мн & м  & *     \\ \midrule
        \textsc{еллинстiи}   & прил & тв   & им   & мн & м  & *     \\ \midrule
        \midrule
        \textsc{dглѣ}        & сущ  & o    & мест & ед & м  & $+$о  \\ \midrule
        \textsc{помыслы}     & сущ  & o    & вин  & мн & м  & $+$е  \\ \midrule
        \textsc{зwлъ}        & сущ  & o    & род  & мн & ср & $-$о  \\ \midrule
        \textsc{сdдебъ}      & сущ  & a    & род  & мн & ж  & $-$е  \\ \midrule
        \textsc{старцd}      & сущ  & jo   & дат  & ед & м  &       \\ \midrule
        \textsc{wвець}       & сущ  & ja   & род  & мн & ж  &       \\ \midrule
        \textsc{тоно(к)}     & прил & o    & вин  & ед & м  & $-$о  \\ \midrule
        \textsc{ра(до)стенъ} & прил & о    & им   & ед & м  & $-$е  \\ \midrule
        \midrule
        \textsc{хр(с)тiаны}  & сущ  & o    & вин  & мн & м  & $+$ин \\ \midrule
        \textsc{татары}      & сущ  & o    & вин  & мн & м  & $+$ин \\ \midrule
        \textsc{по(с)}       & сущ  & o    & вин  & ед & м  & $+$т  \\ \midrule
        \textsc{пѣ(с)ми}     & сущ  & i    & тв   & мн & ж  & $+$н  \\ \midrule
        \textsc{роже(н)и}    & сущ  & jo   & мест & ед & ср & $+$и  \\ \midrule
        \textsc{бранiи}      & сущ  & ja   & вин  & ед & ж  & $-$и  \\ \bottomrule
        \caption{Примеры морфологической разметки с учётом нововведений}
        \label{tab:new}
    \end{tabularx}
\end{table}

\subsection{Спецификации типов склонения}

\paragraph{Тип \textit{o/ja}} На материале исследованных текстов этот смешанный тип был зафиксирован только у форм существительного \textsc{братъ}, чья основа во множественном числе представлена корневым алломорфом \textsc{-братi-}. Исходя из предположения о том, что и другие существительные с таким смешением обнаруживают подобное алломорфирование (ср.\ рус.\ \textit{лист}~"--- \textit{листья}), данный тип предлагается использовать как маркер наличия йотового наращения у основы и необходимости его удаления при лемматизации.

\paragraph{Тип \textit{o/en}} Данный тип присваивается формам множественного числа существительных, обозначающих человека по роду деятельности, происхождению или вероисповеданию с суффиксом \textsc{-ин-} в единственном числе. В парадигме множественного числа данный суффикс утрачивается, и тогда тип \textit{o/en} свидетельствует о том, что в ходе лемматизации его необходимо восстановить.

\subsection{Существительные \foreignlanguage{english}{pluralia tantum}}

Существительным, употребляющимся только во множественном числе (\foreignlanguage{english}{pluralia tantum}), в данной позиции вместо пометы \textit{мн} присваивается специальная помета \textit{pt}. Её наличие далее (см.\ \ref{subsec:fin_nn}) позволяет добавлять к словарным основам подобных существительных флексии множественного числа вместо единственного.

\subsection{Дополнительные пометы}

Ввиду того что в составляющих разметку таблицах последний, шестой столбец при аннотации имён не используется (он задействован только для морфологического описания форм глаголов в форме настоящего"=будущего времени и причастий, а также при морфосинтаксическом описании компонентов форм аналитических времён), мы воспользовались этим обстоятельством и отвели его под ряд принципиально новых помет, фиксирующих регулярные морфонологические явления на стыке основы и флексии.

\paragraph{Помета \textit{*}} Астериск в последнем столбце свидетельствует о том, что в абсолютном конце основы размечаемой словоформы действует закон второй палатализации~"--- переход заднеязычных согласных \textsc{к}, \textsc{г}, \textsc{х} в мягкие свистящие \textsc{ц}, \textsc{з}, \textsc{с} перед \textsc{ѣ} или \textsc{и} дифтонгического происхождения. У существительных, а также у кратких прилагательных вторая палатализация имеет место в следующих парадигматических позициях:

\begin{compactenum}
    \item тип *\={a}: \begin{inparaitem}[]
        \item дат.~п.\ ед.~ч.,
        \item мест.~п.\ ед.~ч.,
        \item им.-вин.~п.\ дв.~ч.\
    \end{inparaitem} (например, \textsc{рdка}~"--- \textsc{рdцѣ});
    
    \item тип *\u{o}, м.~р.: \begin{inparaitem}[]
        \item мест.~п.\ ед.~ч.,
        \item им.~п.\ мн.~ч.,
        \item мест.~п.\ мн.~ч.\
    \end{inparaitem} (\textsc{rзыкъ}~"--- \textsc{rзыцѣ}, \textsc{rзыцы}, \textsc{rзыцѣхъ});
    
    \item тип *\u{o}, ср.~р.: \begin{inparaitem}[]
        \item мест.~п.\ ед.~ч.,
        \item им.-вин.~п.\ дв.~ч.,
        \item мест.~п.\ мн.~ч.\
    \end{inparaitem} (\textsc{вѣко}~"--- \textsc{вѣцѣ}, \textsc{вѣцѣхъ}).
\end{compactenum}

У полных прилагательных (речь только о твёрдом типе~"--- ввиду исторической твёрдости заднеязычных согласных) вторая палатализация происходит в \begin{inparaenum}[(1)]
    \item дат.~п.\ ед.~ч.\ ж.~р.;
    \item мест.~п.\ ед.~ч.\ м., ж.\ и ср.~р.;
    \item им.-вин.~п.\ дв.~ч.\ ж.\ и ср.~р.;
    \item им.~п.\ мн.~ч.\ м.~р.\
\end{inparaenum} (\textsc{благии}~"--- \textsc{блазѣи}, \textsc{блазѣмъ}, \textsc{блазiи}). Кроме того, в местоименном склонении имеет место особая разновидность второй палатализации с чередованием \textsc{-ск-}~// \textsc{-ст-}: \textsc{члческии\#}~"--- \textsc{члчестѣи\#} и~т.",д.\ \autocite[139]{stsl:2013}.

Фиксировать случаи второй палатализации на конце основ существительных и прилагательных абсолютно необходимо для корректности процедуры лемматизации, поскольку по формальным признакам невозможно отличить сибилянты, возникшие в результате палатализации, от этимологических, ср.: \textsc{нозѣ}~"--- лемма \textsc{нога}, но \textsc{трапезѣ}~"--- \textsc{трапеза}; \textsc{дуси}~"--- \textsc{духъ}, но \textsc{бѣси}~"--- \textsc{бѣсъ}.

\paragraph{Пометы \textit{$\pm$о} и \textit{$\pm$е}} Наличие одной из указанных четырёх помет указывает на то, что в последнем слоге словарной основы по сравнению с основой размеченной словоформы имеет место прояснение ($+$) либо падение ($-$) этимологического редуцированного (соответственно \textsc{ъ} или \textsc{ь}).

Указанные процессы обусловливаются тем, что в последнем слоге формообразующих основ сильные и слабые позиции (см.\ \autocite[50--51]{khaburgaev:1986}) могут чередоваться: \textsc{сонъ}~"--- \textsc{сна}, \textsc{веренъ}~"--- \textsc{верна}. Подобное чередование возникает только при наличии в словоизменительной парадигме форм с односложными редуцированными флексиями (современными нулевыми), а следовательно, актуально исключительно для именных парадигм. Более того, размечать его целесообразно отнюдь не для всех словоформ, склоняющихся по именному типу: так, среди несоставных количественных числительных и неличных местоимений данному чередованию подвержены лишь лексемы \textsc{сто} (род.~п.\ мн.~ч.\ \textsc{сотъ}) и \textsc{весь} (род.~п.\ мн.~ч.\ \textsc{всѣхъ}), которые легко поддаются словарному заданию.

Таким образом, пометы \textit{$\pm$о} и \textit{$\pm$е} приписываются лишь существительным и кратким прилагательным. Реальных случаев употребления указанных помет относительно немного (всего 131 случай на около 11~тыс. словоформ); кроме того, на их употребление накладываются следующие ограничения:

\begin{compactenum}
    \item высокочастотные существительные с суффиксом \textsc{-ц-}~// \textsc{-ец-} всегда обнаруживают процессы прояснения и падения редуцированного \textsc{ь} и потому обрабатываются автоматически;
    \item все прилагательные стандартно приводятся к полным формам (см.\ \ref{subsec:fin_aj}), где словарные флексии всегда ненулевые~"--- а потому при них следует отмечать только наличие прояснения.
\end{compactenum}

\paragraph{Семейство помет \textit{$\pm$x}} Если \textit{x}~"--- произвольная последовательность символов, отличная от \textit{o} и \textit{е}, то соответствующая помета служит для восстановления пропущенных ($+$) или удаления избыточных ($-$) букв и буквосочетаний на конце основы при лемматизации.

Подобные пропуски и вставки в большинстве своём носят идиосинкратический характер и во многом обусловлены орфографическим контекстом (например, концом строки) и привычками конкретного писца; тем не менее, иногда постановку указанных помет определяют и причины иного рода. Например, помета \textit{$+$ин} регулярно приписывается формам множественного числа существительных, закономерно утративших суффикс деятеля \textsc{-ин-}, но склоняющихся не по ожидаемому типу на согласный (такие случаи предусматривает спецификация типа склонения \textit{o/en}, о которой говорилось выше), а по образцу *\u{o}-склонения.

\section{Орфографическая нормализация}
\label{sec:norm}

\subsection{Методологические замечания}

Общеизвестно, что в рукописную эпоху орфография не была кодифицирована, и написание многих слов могло существенно варьироваться даже в пределах одного текста. Рассмотрим элементы следующего ряда: \begin{inparaitem}[]
    \item \textsc{блаженаго},
    \item \textsc{блаженна(г)},
    \item \textsc{бл(а)женнаго},
    \item \textsc{бла(ж)еннаго},
    \item \textsc{бла(ж)нна(г)},
    \item \textsc{бла(ж)ннаго},
    \item \textsc{блжена(г)\#},
    \item \textsc{блженаго\#},
    \item \textsc{блжен(н)аго\#},
    \item \textsc{блаженна(г)\#}~"---
\end{inparaitem} очевидно, все они представляют собой варианты записи одной и той же словоформы, которые в целях единообразия обработки на высших языковых уровнях (в~т.",ч.\ морфологическом) необходимо предварительно унифицировать путём приведения к единой "<нормальной форме">~"--- иначе говоря, подвергнуть процедуре орфографической нормализации\footnotemark.

\footnotetext{%
    В компьютерной морфологии термины "<нормализация"> и "<лемматизация"> нередко употребляются как абсолютные синонимы \autocite[75]{koval:2005}. Во избежание терминологической путаницы в настоящей работе речь о нормализации идёт исключительно в орфографическом смысле.
}

Проблема осложняется тем, что выбор подобной инвариантной единицы также бывает множественным (так, нормальная форма членов вышеприведённого ряда может иметь вид \textsc{блаженаго} или \textsc{блаженнаго}) и всякий раз обусловливается методологическими установками конкретного исследователя или коллектива. В том случае, если обрабатываемые тексты относятся к раннему периоду жизни древнеписьменного языка, то вне зависимости от датировки всякой конкретной рукописи, как правило, восстановление идёт по линии канонической орфографии (и тогда предпочтение было бы отдано варианту \textsc{блаженаго}); в противном случае выбор осуществляется в пользу современной орфографической нормы (\textsc{блаженнаго}).

Е.",Г.~Уфлянд, в 2004--2008~гг.\ работавшая над проблемами нормализации в рамках СКАТ, при обосновании того, что подход, ориентированный на современную орфографию, в контексте рукописей XV--XVII~вв.\ является более целесообразным, опирается на принципы, сформулированные в проекте "<Словаря языка житий русских святых XVI--XVII~вв.">\ и во введении к "<Словарю русского языка XI--XVII~вв.">\ \autocite[41--42]{uflyand:2008}: поскольку до XVIII~в.\ церковнославянской орфографической нормы не существовало, нет никакой возможности принимать в качестве эталонного способ написания слов на каком-либо историческом временном срезе; обращаться же к нормативному написанию XIX--XX~вв., нежели к современному русскому, не только менее практично, но и бессмысленно.

\subsection{Модуль нормализации Е.~Г.~Уфлянд}

В практической части своей дипломной работы Е.",Г.~Уфлянд разработала алгоритм автоматического сведения орфографических вариантов словоформ к основному (т.",е.\ к нормальной форме), реализованный на Python~2.7 в качестве функционального ядра программы для уменьшения объёма сводного словоуказателя. Программой последовательно обрабатывается ряд стандартных ситуаций, в которых наблюдается орфографическое варьирование, например \autocite[46--70]{uflyand:2008}:

\begin{compactitem}
    \item написания под титлом: \textsc{бомтр}~$\to$ \textsc{богоматер}, \textsc{мчнк}~$\to$ \textsc{мученик}, \textsc{хв}~$\to$ \textsc{христов};
    \item написания с выносными буквами: \textsc{б(д)ц}~/ \textsc{б(ди)ц}~$\to$ \textsc{богородиц}, \textsc{ис(с)в}~/ \textsc{ис(о)в}~$\to$ \textsc{иисусов};
    \item сочетания типа \textit{*TorT} с метатезой срединных плавных: \textsc{млъч}~/ \textsc{мльч}~$\to$ \textsc{молч}, \textsc{прьст}~/ \textsc{пръст}~$\to$ \textsc{перст};
    \item регулярные окончания с выносными буквами: \textsc{а(ш)}~/ \textsc{я(ш)}~$\to$ \textsc{аше}~/ \textsc{яше}, \textsc{бы(с)}~$\to$ \textsc{бысть}.
\end{compactitem}

В ходе интеграции функции, написанной Е.",Г.~Уфлянд, в нашу собственную программу мы сочли необходимым привнести определённые новшества в механизм её работы. Во-первых, непосредственно в исходный код были внедрены некоторые технологические улучшения:

\begin{compactenum}
    \item словарная составляющая отныне изолирована от алгоритмической и вынесена в отдельный файл;
    \item для хранения информации о буквосочетаниях, подлежащих замене, используются структуры данных типа "<словарь"> вместо пар синхронизированных между собой массивов;
    \item замены были переписаны на языке регулярных выражений, что позволило более полно и экономно охватить вариативность плана выражения сводимых орфографических вариантов.
\end{compactenum}

Во-вторых, сам перечень замен регулярно пополнялся по мере обнаружения нового релевантного языкового материала. Так, при замене \textsc{ннѣ} (и окказионального \textsc{нне}) на \textsc{нынѣ} дополнительно учтены префиксальные дериваты \textsc{донынѣ}, \textsc{о(т)нынѣ}, \textsc{понынѣ}; вообще же в перечень замен сокращённых слов были внесены, например, следующие добавления:

\begin{compactitem}
    \item \textsc{слв}~$\to$ \textsc{слав}, \textsc{срц}~$\to$ \textsc{сердц}, \textsc{стл}~$\to$ \textsc{святител};
    \item \textsc{блг(д)т}~/ \textsc{блго(д)т}~$\to$ \textsc{благодат}, \textsc{др(в)н}~/ \textsc{дрвн}~$\to$ \textsc{деревн};
    \item \textsc{кр(с)тл}~$\to$ \textsc{крестител}, \textsc{мдр(с)т}~$\to$ \textsc{мудрост}, \textsc{пр(с)нодв}~$\to$ \textsc{приснодев}.
\end{compactitem}

В-третьих, доступ к данным морфологической разметки позволил корректно производить замены неоднозначных сокращений с выносными буквами и под титлом, которые необходимо раскрывать по-разному в зависимости от принадлежности нормализуемой словоформы к тому или иному лексико"=грамматическому классу. Так, сочетания \textsc{гн} (под титлом) и \textsc{г(с)дн} в начальной позиции форм существительных подлежат замене на \textsc{господин}, в случае же прилагательных~"--- на \textsc{господн}; аналогичным образом \textsc{ч(с)т} приводится к написанию \textsc{чест} либо \textsc{чист}. Выносное \textsc{(г)} на конце прилагательных является стандартным сокращением адъективной флексии род.~п.\ ед.~ч.\ м.\ и ср.~р.\ и раскрывается как \textsc{го}, однако в абсолютном конце словоформ иных частей речи~"--- как \textsc{гъ}: ср.\ \textsc{вра(г)}, \textsc{*выпря(г)}.

Отметим, однако, что в рамках настоящей работы задача окончательного решения проблемы орфографической вариативности в СКАТ не ставилась. Как следствие, такие явления, как \begin{inparaenum}[(1)]
    \item чередование редуцированных и гласных полного образования в корнях и префиксах (\textsc{въсхитити}~"--- \textsc{восхитити}),
    \item наличие дублетов с одиночными либо удвоенными согласными (\textsc{воистину}~"--- \textsc{воистинну}),
    \item непоследовательное написание ятя (\textsc{грѣхъ}~"--- \textsc{грехъ})
\end{inparaenum} \autocite[378]{uflyand_alexeeva:2008}~"--- в модуле нормализации остаются неучтёнными, частично "<процеживаясь"> через сито дальнейших этапов алгоритма лемматизации и отражаясь на графическом облике соответствующих лемм.

\section{Стемминг}

\subsection{Методологические замечания}

На втором этапе алгоритма лемматизации нормализованные словоформы подвергаются процедуре стемминга. Идеологически стеммер, разработанный в рамках данной работы, наследует принципы бессловарно"=правиловых стеммеров, основанных на методе усечения окончаний (в частности, речь о классическом стеммере Портера \autocite{porter:1980}). Однако в отличие от сугубо формальных подходов, опирающихся исключительно на план выражения и фактически выделяющих в анализируемых словоформах псевдоосновы и псевдофлексии (неизменяемые начальные и конечные последовательности символов), реализованный нами алгоритм, напротив, исходит из плана содержания: известные из разметки морфологические свойства обрабатываемых словоформ позволяют максимально точно отделять собственно основы от собственно флексий, лингвистически интерпретируемых и несущих полноценное грамматическое значение, и при этом избегать таких типичных ошибок "<слепого"> стемминга без опоры на семантику, как \foreignlanguage{english}{over-} и \foreignlanguage{english}{understemming}.

Так, применение данной процедуры к словоформе \textsc{день} призвано отсечь не псевдофлексию \textsc{-ень} от псевдооссновы \textsc{д-}, но \textit{флексию} \textsc{-ь} от \textit{основы} \textsc{ден-}. Преобразование получаемых таким образом основ с целью их отождествления с основами соответствующих лемм (ср.\ \textsc{ден-} и \textsc{дн-}) происходит уже на следующем, заключительном этапе алгоритма.

Морфологически неизменяемые формы: несклоняемые прилагательные (близкие по значению к наречиям и весьма немногочисленные \autocite[140--141]{ivanova:1998}: в каждом из размеченных текстов единожды употреблено лишь прилагательное \textsc{исполнь} `полный'), инфинитивы и супины (последние на рассмотренном материале не встречаются вовсе), наречия, предлоги и послелоги, союзы, частицы, междометия~"--- стеммингу естественным образом не подлежат; в качестве лемм им присваиваются их нормализованные формы.

\subsection{Классы словоизменительных парадигм}

Словоизменительные парадигмы, сопоставляющие грамматические значения выражаемым ими финальным сегментам, были составлены с опорой на авторитетные учебно"=научные пособия по старославянскому языку \autocites{stsl:2013}{ivanova:1998}{khaburgaev:1986}; составляющие их флексии для учёта неустранимой орфографической вариативности записаны в виде регулярных выражений. Условно все парадигмы относятся к одному из следующих классов.

\subsubsection{Именные парадигмы}

При помощи именных парадигм производится стемминг \begin{inparaenum}[(1)]
    \item существительных,
    \item нечленных (кратких) прилагательных,
    \item несоставных количественных числительных,
    \item неличных местоимений в именительном и винительном падежах.
\end{inparaenum} Они характеризуются наибольшим качественным разнообразием и значительной вариативностью плана выражения флексий, и для их составления были привлечены дополнительные сведения из монографии \autocite[257--314]{agio:1990}.

Например, частная именная парадигма *j\={a}-склонения во множественном числе мужского (и одновременно женского) рода имеет следующий вид:

\begin{Verbatim}[fontsize=\small, gobble=4, xleftmargin=\parindent]
    ('ja', 'им', 'мн', 'м'): '[+АЕИЫЯ]',
    ('ja', 'род', 'мн', 'м'): '[+ЕИЫ]И|[ЪЬ]',
    ('ja', 'дат', 'мн', 'м'): '[АЯ]М[ЪЬ`]',
    ('ja', 'вин', 'мн', 'м'): '[+АЕИЫЯ]',
    ('ja', 'тв', 'мн', 'м'): '[АЯ]МИ',
    ('ja', 'мест', 'мн', 'м'): '[АЯ]Х[ЪЬ`]',
    ('ja', 'зв', 'мн', 'м'): '[+АЕИЫЯ]',
\end{Verbatim}

\subsubsection{Местоименные парадигмы}

Парадигмы местоименного класса используются при анализе неличных местоимений в косвенных падежах (кроме винительного), а также членных (полных) форм прилагательных и порядковых числительных. Собственно адъективные флексии фонетически и орфографически весьма близки (их стяжённые разновидности~"--- практически идентичны) местоименным, и уже рукописи X--XI~вв. обнаруживают результаты их взаимодействия и уподобления \autocite[170--171]{khaburgaev:1986}; исходя из этих соображений мы сочли приемлемым объединить их в общий класс.

В качестве примера рассмотрим парадигму местоименного склонения по твёрдому типу в единственном числе среднего рода:

\begin{Verbatim}[fontsize=\small, gobble=4, xleftmargin=\parindent]
    ('тв', 'им', 'ед', 'ср'): 'О?Е',
    ('тв', 'род', 'ед', 'ср'): 'А?[АЕО]?ГО',
    ('тв', 'дат', 'ед', 'ср'): 'У?[ЕОУ]?МУ',
    ('тв', 'вин', 'ед', 'ср'): 'О?Е',
    ('тв', 'тв', 'ед', 'ср'): '[ИЫ]?[+ЕИЫ]М[ЪЬ`]',
    ('тв', 'мест', 'ед', 'ср'): '[+Е]?[+ЕО]М[ЪЬ`]',
    ('тв', 'зв', 'ед', 'ср'): 'О?Е',
\end{Verbatim}

\subsubsection{Особые местоименные парадигмы}

Словоизменению таких местоимений, как \begin{inparaenum}[(1)]
    \item личные \textsc{азъ}, \textsc{ты}, \textsc{вѣ}, \textsc{ва}, \textsc{мы}, \textsc{вы},
    \item возвратное \textsc{себе},
    \item вопросительные \textsc{кто} и \textsc{что},~"---
\end{inparaenum} присущ ряд глубоко архаических черт (ярко выраженный супплетивизм, особая система флексий), из чего следует принципиальная невозможность их анализа по общим правилам. Их словоизменительные парадигмы обособлены от парадигм прочих классов и имеют несколько иную структурную организацию; описывающие их словоизменение регулярные выражения представляют собой полные покрытия соответствующих символьных цепочек, а обращение к разметке производится исключительно с целью выявления в ней возможных ошибок.

Примеры парадигм местоимений всех трёх разрядов:

\begin{Verbatim}[fontsize=\small, gobble=4, xleftmargin=\parindent]
    ('1', 'им', 'ед'): ('АЗ[ЪЬ`]?$', 'АЗЪ'),
    ('1', 'род', 'ед'): ('М([ЕЪЬ]?Н)?[+ЕЯ]$', 'АЗЪ'),
    ('1', 'дат', 'ед'): ('М([ЕЪЬ]?Н[+Е]|И)$', 'АЗЪ'),
    ('1', 'вин', 'ед'): ('М([ЕЪЬ]?Н)?[+ЕЯ]$', 'АЗЪ'),
    ('1', 'тв', 'ед'): ('М[ЪЬ]?НОЮ$', 'АЗЪ'),
    ('1', 'мест', 'ед'): ('М[ЪЬ]?Н[+Е]$', 'АЗЪ'),

    'род': ('С([ЕО]Б)?[+ЕЯ]$', 'СЕБЕ'),
    'дат': ('С([ЕО]Б[+Е]|И)$', 'СЕБЕ'),
    'вин': ('С([ЕО]Б)?[+ЕЯ]$', 'СЕБЕ'),
    'тв': ('СОБОЮ$', 'СЕБЕ'),
    'мест': ('С[ЕО]Б[+Е]$', 'СЕБЕ'),

    ('м', 'им'): ('Ч[ЪЬ]?ТО$', 'ЧТО'),
    ('м', 'род'): ('Ч[ЕЬ]?(СО)?(ГО)?$', 'ЧТО'),
    ('м', 'дат'): ('Ч[ЕЬ]?(СО)?МУ$', 'ЧТО'),
    ('м', 'вин'): ('Ч[ЪЬ]?ТО$', 'ЧТО'),
    ('м', 'тв'): ('ЧИМ[ЪЬ`]?$', 'ЧТО'),
    ('м', 'мест'): ('Ч[ЕЬ]?(СО)?М[ЪЬ`]?$', 'ЧТО'),
\end{Verbatim}

Полностью аналогично вопросительным местоимениям обрабатываются неопределённые \textsc{нѣкто}, \textsc{нѣчто} и отрицательные \textsc{никтоже}, \textsc{ничтоже}.

Наконец, в связи с вопросительными местоимениями заслуживает упоминания определительное местоимение \textsc{кождо} `каждый'\footnotemark, в косвенных падежах изменяющееся по образцу вопросительного \textsc{кто} (при этом сегмент \textsc{-ждо-} ведёт себя подобно слитной частице \textsc{же} и не оказывает влияния на словоизменение): \textsc{когождо}, \textsc{комуждо} \autocite[I,][1389]{sreznevsky}. Отметим также, что наряду с ним существует синонимичное местоимение \textsc{кииждо} (с вариантом \textsc{коиждо}), однако оно, в отличие от \textsc{кождо}, имеет регулярную парадигму на основе вопросительного \textsc{кии}: \textsc{коегождо}, \textsc{коемуждо} \autocite[I,][1417]{sreznevsky}~"--- и потому обрабатывается по общему правилу.

\footnotetext{%
    В "<Материалах"> И.",И.~Срезневского также приводятся примеры с конечным сегментом \textsc{-жде-}, но в корпусе подобных употреблений зафиксировано не было.
}

\subsection{Обработка составных форм}

\subsubsection{Составные существительные}

На материале рассмотренных житийных текстов выделяются две семантико"=морфологические группы составных существительных со склонением обеих частей в их составе.

Во-первых, речь о названиях населённых пунктов со второй корневой морфемой \textsc{-град-} либо \textsc{-город-} (полногласный вариант низкочастотен, но встречается в некоторых текстах корпуса): \textsc{*костянтиньградъ}, \textsc{*новъградъ}\footnotemark. При анализе подобных имён собственных мы исходим из посылки, что их первая составляющая всегда представляет собой краткое прилагательное мужского рода *\u{o}- или *j\u{o}-склонения, т.",е.\ изменяется аналогично существительному \textsc{градъ}. Таким образом, тип склонения первой части можно считать известным, а процедуру стемминга единообразно производить над обоими компонентами: \textsc{*новѣградѣ}~$\to$ \textsc{-нов-}, \textsc{-град-}.

\footnotetext{%
    Слитное написание подобных топонимов~"--- результат модернизации орфографии: в самих рукописях XV--XVII~вв.\ они представляют собой словосочетания вида "<краткое прилагательное~$+$ существительное"> с последовательным склонением обеих лексем.
}

Во-вторых, спецификой в рассматриваемом аспекте обладают наименования времён суток \textsc{полдень} и \textsc{полнощь} (русизм \textsc{полночь} в корпусе не встречается). Здесь типы склонения составных частей не совпадают: существительные \textsc{день} и \textsc{нощь} склоняются по типам *en и *\u{\i} соответственно, а \textsc{полъ} (в значении `половина')~"--- по типу *\u{u}. Однако несмотря на то, что последний нам априорно известен, и при соответствующей поправке первая часть также подлежит стеммингу, ввиду предельной ограниченности и закрытости данной группы существительных было решено проверять их начальные подстроки на соответствие простым регулярным выражениям: \texttt{ПОЛ.*Д[ЕЬ]?Н} либо \texttt{ПОЛ.*НО[ЧЩ]}~"--- и при положительном результате приписывать готовые леммы без какого-либо анализа грамматических данных.

\subsubsection{Составные числительные}

В церковнославянском языке сложносоставные количественные числительные функционируют как словосочетания, и поэтому при слитном написании им присущи специфические словоизменительные особенности, отчасти присутствующие и в современном русском.

\begin{asparaenum}
    \item Числительные, обозначающие числа от 11 до 19, представляют собой сочетания единиц первого десятка с предложно"=падежной группой \textsc{на десяте} (мест.~п.). Формальный тип синтаксической связи внутри подобных структур~"--- предложно"=падежное примыкание; изменению подвержена только первая часть: \textsc{пятьнадесяте}~"--- \textsc{пятинадесяте}.

    \item Названия чисел 20, 30, 40 и 200, 300, 400 являются сочетаниями имён соответствующих единиц с существительными \textsc{десять} (склоняется по типу *ent) либо \textsc{сто} (типа *\u{o}). Тип связи~"--- согласование, а следовательно, при склонении изменяются оба компонента: \textsc{двадесяти}~"--- \textsc{двудесяту} (дв.~ч.), \textsc{триста}~"--- \textsc{трехъсотъ}  (мн.~ч.).

    \item Обозначения чисел от 50 до 90 и от 500 до 900 лексически подобны названиям десятков и сотен меньших порядков, однако синтаксически ведут себя иначе: здесь наименования единиц управляют существительными \textsc{десять} или \textsc{сто} в форме род.~п.\ мн.~ч. Изменяется только первая часть: \textsc{седмьдесятъ}~"--- \textsc{седмидесятъ}, \textsc{осмьсотъ}~"--- \textsc{осмисотъ}.
\end{asparaenum}

Принимая во внимание закрытость множества составных числительных и стремясь избежать излишних технологических затруднений, которые могли бы возникнуть при их прямолинейном анализе, здесь мы также ограничились описанием морфемной структуры в виде регулярных выражений:

\begin{Verbatim}[fontsize=\small, gobble=4, xleftmargin=\parindent]
    'ЕДИН.*НАДЕСЯТ': 'ЕДИННАДЕСЯТЕ',
    'Д[ЪЬ]?В.*НАДЕСЯТ': 'ДВАНАДЕСЯТЕ',
    'ТР.*НАДЕСЯТ': 'ТРИНАДЕСЯТЕ',

    'ЧЕТЫР.*ДЕСЯТ': 'ЧЕТЫРЕДЕСЯТЕ',
    'ПЯТ.*ДЕСЯТ': 'ПЯТЬДЕСЯТЪ',
    'ШЕСТ.*ДЕСЯТ': 'ШЕСТЬДЕСЯТЪ',

    'СЕДМ.*С[ЪО]?Т': 'СЕДМЬСОТЪ',
    'ОСМ.*С[ЪО]?Т': 'ОСМЬСОТЪ',
    'ДЕВЯТ.*С[ЪО]?Т': 'ДЕВЯТЬСОТЪ',
\end{Verbatim}

\section{Восстановление леммы}

Основы, получаемые в результате стемминга, не всегда совпадают с основами соответствующих лемм и потому до прибавления словарных флексий зачастую нуждаются в дополнительных преобразованиях. Первоочерёдными среди них являются те, которые продиктованы специальными пометами в самой скорректированной разметке (см.\ \ref{sec:corr}); однако наряду со словоизменительными явлениями, для лингвистически корректной обработки требующими обязательной ручной фиксации, существуют и такие, последствия которых заведомо известны и могут учитываться программой автоматически.

\subsection{Существительные}
\label{subsec:fin_nn}

Из существительных следующих подтипов склонения на согласный: *ent, *men, *es, *er~"--- удаляются тематические суффиксы: \textsc{-врѣмен-}~$\to$ \textsc{-врѣм-}, \textsc{-словес-}~$\to$ \textsc{-слов-} и~т.",д. С другой стороны, в случае отсутствия тематических суффиксов в составе основ существительных подтипов *en и *uu последние, напротив, восстанавливаются из соображений модернизации лемм (архаичные формы им.~п.\ ед.~ч.\ без осложнения довольно рано начали замещаться формами вин.~п.\ \autocite[126]{ivanova:1998}): \textsc{-кам-}~$\to$ \textsc{-камен-}, \textsc{-люб-}~$\to$ \textsc{-любов-}. Сходным образом восстановлению подлежит субморф \textsc{-ос-} у частотного имени собственного \textsc{*христосъ}, подвергающийся утрате в косвенных падежах.

Кроме того, основы существительных (и среди имён только их) в отдельных парадигматических позициях подвергаются закону первой палатализации~"--- переходу заднеязычных согласных \textsc{к}, \textsc{г}, \textsc{х} в мягкие шипящие \textsc{ч}, \textsc{ж}, \textsc{ш} в положении перед гласными переднего ряда, а также свистящего \textsc{ц} в составе суффикса деятеля мужского пола и \textsc{з} в слове \textsc{князь}. Фактически сфера действия данного закона ограничена \begin{inparaenum}[(1)]
    \item формой зв.~п.\ ед.~ч.\ в парадигме *\u{o}-склонения м.~р.: \textsc{бгъ\#}~"--- \textsc{бже\#}, \textsc{отець}~"--- \textsc{отче} (в последнем случае имеет место смешение *j\u{o}/*\u{o});
    \item формами дв.\ и мн.~ч.\ двух существительных, обозначающих части тела: \textsc{око}~"--- \textsc{очи} (смешение *es/*\u{\i})~"--- \textsc{очеса} и аналогично \textsc{ухо}~"--- \textsc{уши}~"--- \textsc{ушеса}.
\end{inparaenum} Первая палатализация, в отличие от второй, всегда предсказуема и потому устранима автоматически.

К словарным основам прибавляются флексии им.~п.\ ед.~ч.\ (мн.~ч., если данное существительное размечено как \foreignlanguage{english}{plurale tantum}) с последовательным учётом типа склонения и родовой принадлежности.

\subsection{Прилагательные}
\label{subsec:fin_aj}

Из форм прилагательных сравнительной степени (они размечаются при помощи особого частеречного тега \textit{прил/ср}) удаляется суффикс \textsc{-ш-} (включая алломорфы), присущий всем членам словоизменительной парадигмы, кроме им.~п.\ ед.~ч.\ м.\ и ср.~р. Заметим, однако, что мы считаем компаратив самостоятельной грамматической категорией и никаких дальнейших преобразований, нацеленных на перевод сравнительной степени в положительную (в~т.",ч.\ устранение супплетивизма типа \textsc{болшии}~"--- \textsc{великии}), не производим.

К словарным основам всех прилагательных прибавляются местоименные флексии им.~п.\ ед.~ч.\ м.~р.\ \textsc{-ыи} (к основам твёрдой разновидности, кроме основ на заднеязычные) или \textsc{-ии} (ко всем прочим): \textsc{непорочныи}, \textsc{божии}; к несупплетивным формам компаратива (на гласную) добавляется \textsc{-и}: \textsc{грѣшнѣи}. Исключение здесь составляют имена собственные: к их основам наряду с местоименными флексиями могут добавляться и именные, а также сохраняется их родовая характеристика: \textsc{бѣлыхъ} (\textsc{ризахъ})~"--- лемма \textsc{бѣлыи}, но \textsc{*бѣла} (\textsc{*езера})~"--- \textsc{*бѣло}.

\subsection{Местоимения}

Основы местоимений внутри словоизменительных парадигм практически не обнаруживают вариативности. Точечные модификации требуются в отдельных частных случаях: \textsc{-ко-}~$\to$ \textsc{-к-} (\textsc{кии}~"--- \textsc{коего}, \textsc{коему} и~т.",д.), \textsc{-с-}~$\to$ \textsc{-се-} (унификация альтернантов \textsc{сеи} и \textsc{сии} в пользу более частотного); из предложных форм местоимения \textsc{и}: \textsc{него}, \textsc{нему} и~т.",д.~"--- удаляется наращение \textsc{-н-} (его основа, таким образом, формально считается нулевой). В остальном построение местоименных лемм осуществляется аналогично прилагательным, и лишь нескольким группам местоимений приписываются специфические конечные сегменты: \begin{inparaenum}[(1)]
    \item \textsc{вашь}, \textsc{весь}, \textsc{нашь}, \textsc{сиць};
    \item \textsc{онъ}, \textsc{самъ};
    \item \textsc{елико};
    \item \textsc{онсица};
    \item \textsc{тои};
    \item \textsc{и}.
\end{inparaenum}

\subsection{Числительные}

Среди числительных вариативный характер имеет лишь основа \textsc{-об-}: \textsc{оба}~"--- \textsc{обоихъ}. В иных преобразованиях они не нуждаются и сразу дополняются соответствующими именными флексиями им.~п.: дв.~ч.~"--- числительные \textsc{два} и \textsc{оба}, мн.~ч.~"--- \textsc{три} и \textsc{четыре}, ед.~ч.~"--- все прочие.

\section*{Выводы}
\addcontentsline{toc}{section}{Выводы}

В сумме жития Димитрия Прилуцкого, Дионисия Глушицкого и Кирилла Новоезерского содержат 29617~словоупотреблений, из которых написанной программой обрабатываются 24319~"--- в том числе 14858~имён и 9461~неизменяемое слово. Все необработанные случаи приходятся на личные формы глаголов и причастия, которым предполагается посвятить отдельное исследование.

Реализованному алгоритму присущи два основных недостатка. Во-первых, ввиду отсутствия каких-либо золотых стандартов, пригодных для формальной оценки качества лемматизации церковнославянских текстов XV--XVII~вв., достоверно судить о безошибочности полученных результатов нам представляется возможным лишь экспертным путём~"--- исходя из собственного многолетнего опыта работы над данной проблематикой. Во-вторых, разработанный алгоритм является принципиально немасштабируемым: опора на морфологическую разметку, которая и позволяет производить лемматизацию максимально точно и лингвистически корректно, в то же время исключает возможность его приложения к неразмеченным текстам, составляющих от общего объёма корпуса СКАТ несравненно более значительную долю.

С другой стороны, сами размеченные тексты, на материале которых была проведена настоящая работа и которые в её результате были тщательнейшим образом выверены и дополнены слоем лемм, представляется возможным и рассматривать как прецедентную совокупность, пригодную для частичной автоматизации ручной морфологической разметки~"--- либо же для оценки качества автоматических морфологических анализаторов для СКАТ, если таковые появятся в будущем.
