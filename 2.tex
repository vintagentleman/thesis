\chapter{Лемматизация церковнославянских словоформ на~основе морфологической разметки}

В данной главе описывается алгоритм лемматизации, реализованный в ходе практической части настоящей работы; изложение теоретических проблем моделирования церковнославянского словоизменения и иных побочных вопросов в известной мере вторично и прежде всего обусловлено необходимостью их алгоритмического решения. Разработанный программный пакет написан на языке Python (версии 3.6); исходный код находится в открытом доступе по ссылке: \url{https://github.com/vintagentleman/SCAT}.

\section{Нормализация}
\label{sec:norm}

В литературе по компьютерной морфологии имеет место определённая терминологическая путаница, связанная с терминами "<лемматизация"> и "<нормализация">: порой они употребляются взаимозаменяемо и считаются абсолютными синонимами \autocite[75]{koval:2005}, порой "<нормализации"> приписывают такое определение ("<постановка или словосочетания в каноническую форму">~"--- т.",е.\ базовую форму, или лемму \autocite[14]{mono_morphology}), которое исходя из словообразовательных соображений в большей степени подошло бы "<лемматизации">. Однако в контексте исторических языков, на наш взгляд, между данными терминами необходимо проводить чёткую границу: статус процедуры АОТ на морфологическом языковом уровне надлежит приписывать термину "<лемматизация">~"--- и только ему; "<нормализация">, с другой стороны, прежде всего связана с графическим слоем представления текста (исходя из чего производить её следует строго \textit{до} лемматизации и прочих этапов АОТ на более высоких языковых уровнях) и заключается в приведении различных орфографических вариантов одного и того же слова к некоторому инварианту. Например, элементы следующего ряда:
\begin{inparaenum}[(1)]
    \item \textsc{блаженаго},
    \item \textsc{блаженна(г)},
    \item \textsc{бл(а)женнаго},
    \item \textsc{бла(ж)еннаго},
    \item \textsc{бла(ж)нна(г)},
    \item \textsc{бла(ж)ннаго},
    \item \textsc{блжена(г)\#},
    \item \textsc{блженаго\#},
    \item \textsc{блжен(н)аго\#},
    \item \textsc{блаженна(г)\#}~"---
\end{inparaenum}
представляют собой варианты записи одной и той же словоформы, "<нормальная форма"> которой имеет вид \textsc{блаженаго} или \textsc{блаженнаго}. В случае необходимости выбирать между различными орфографическими вариантами инвариантной единицы выбор во многом обусловлен методологическими установками конкретного исследователя или коллектива: так, при ориентации на близость к орфографии соответствующего современного языка в рассматриваемом примере предпочтение должно быть отдано последнему члену пары. Установку на современную ("<каноническую">) орфографию кладёт в основу своего определения нормализации \textcite[69]{piotrowski:2012}.

Е.",Г.~Уфлянд, ранее работавшая над проблемами нормализации в рамках СКАТ, также~"--- с незначительными отступлениями~"--- придерживалась курса на современную орфографию \autocite[41---42]{uflyand:2008}. В своём дипломном сочинении она очерчивает круг причин и характер вариативных написаний в исследованных 10 текстах корпуса общим объёмом порядка 144 тыс.\ словоупотреблений, а далее описывает разработанный ей алгоритм автоматического сведения орфографических вариантов словоформ к основному (т.",е.\ к нормальной форме), реализованный на Python 2.7 в качестве функционального ядра программы для уменьшения объёма сводного словоуказателя. Функция, о которой идёт речь (\texttt{modif}), предварительно очищает входные символьные цепочки от дублетных символов, различиями в которых на момент создания составляющих корпус рукописей можно пренебречь (так, \textsc{u}, \textsc{d} и \textsc{g}~"--- ук диграфный, лигатурный и юс большой~"--- заменяются на \textsc{у}, \textsc{w} (омега) на \textsc{о}, \textsc{i}~десятеричное на \textsc{и}~восьмеричное; из некириллических знаков оставляется лишь \textsc{+}, обозначающий ять), а затем последовательно проверяет их на вхождения ненормализованных буквосочетаний различных типов, которые при обнаружении подвергаются соответствующим заменам, перечисленным в словарном компоненте модуля. Среди различных видов замен выделяются, например, следующие (полный перечень намного шире):

\begin{compactenum}
    \item замены буквосочетаний в сокращённых словах под титлом: \textsc{бомтр}~$\to$ \textsc{богоматер}; \textsc{дхв}, \textsc{дхм}~$\to$ \textsc{духов}, \textsc{духом}; \textsc{црквн}, \textsc{црк(в)н}~$\to$ \textsc{церковн};

    \item замены буквосочетаний в сокращённых словах с выносными буквами: \textsc{б(д)ц}, \textsc{б(ди)ц}~$\to$ \textsc{богородиц}; \textsc{иер(с)лм}~$\to$ \textsc{иерусалим}; \textsc{иер(с)л}~$\to$ \textsc{иерусал};

    \item замены отдельных буквосочетаний в разных внутрисловных позициях:
    \begin{inparaenum}[(1)]
        \item сочетания заднеязычных с \textsc{ь} везде, кроме абсолютного конца слова: \textsc{гь}, \textsc{кь}, \textsc{хь}~$\to$ \textsc{г}, \textsc{к}, \textsc{х};
        \item сочетания типа \textit{*TorT} с метатезой срединных плавных: \textsc{прьст}, \textsc{пръст}~$\to$ \textsc{перст};
        \item регулярные окончания с выносными буквами: \textsc{а(ш)}, \textsc{я(ш)}~$\to$ \textsc{аше}, \textsc{яше}.
    \end{inparaenum}
\end{compactenum}

В ходе интеграции функции \texttt{modif} в нашу программу мы сочли необходимым привнести некоторые новшества в механизм её работы. Непосредственно в код было внесено множество технологических улучшений: словарная составляющая отныне отделена от алгоритмической (они разнесены по разным файлам), и для хранения информации о буквосочетаниях, подлежащих замене, в ней используется естественным образом напрашивающаяся структура данных типа "<словарь">, а не пары синхронизированных между собой массивов (вносить в подобную структуру любые изменения крайне проблематично). Замены были переписаны на языке регулярных выражений, что позволило более полно и экономно охватить вариативность плана выражения сводимых орфографических вариантов; их перечень также регулярно пополнялся по мере обнаружения нового релевантного языкового материала. Так, в список замен буквосочетаний в сокращённых словах под титлом были добавлены замены \textsc{др(в)н} (либо \textsc{дрвн}) на \textsc{деревн}, \textsc{пр(с)нодв} на \textsc{приснодев}, при замене \textsc{нн+} (и окказионального \textsc{нне}) на \textsc{нын+} дополнительно учтены префиксальные дериваты \textsc{донын+}, \textsc{о(т)нын+}, \textsc{понын+}~"--- и~т.",д.

Кроме того, доступ к данным морфологической разметки позволил корректно производить замены неоднозначных сокращений с выносными буквами и под титлом, обусловленных принадлежностью отдельно взятой словоформы к тому или иному лексико-грамматическому классу: сочетания \textsc{гн} (под титлом) и \textsc{г(с)дн} в начальной позиции форм существительных подлежат замене на \textsc{господин}, в случае же прилагательных~"--- на \textsc{господн} (при этом сначала осуществляется замена надстрок \textsc{гнь} и \textsc{г(с)днь} на вариант без редукции~"--- \textsc{господень}); аналогичным образом \textsc{ч(с)т} приводится к написанию \textsc{чест} либо \textsc{чист}. Стандартное сокращение адъективной флексии родительного падежа единственного числа неженского рода при помощи выносного \textsc{(г)} раскрывается как \textsc{гъ} в абсолютном конце словоформ иных частей речи с целью учёта окказиональных написаний типа \textsc{вра(г)} и \textsc{*выпря(г)}.

В результате работы Е.",Г.~Уфлянд вариативность в словоуказателе не была устранена полностью: вне её рассмотрения остались такие явления, как
\begin{inparaenum}[(1)]
    \item чередование редуцированных и гласных полного образования в корнях и префиксах (\textsc{въсхитити}~"--- \textsc{восхитити}),
    \item наличие дублетов с удвоенными согласными (\textsc{воистину}~"--- \textsc{воистинну}),
    \item непоследовательное написание ятя, весьма характерное для поздних рукописей и отражающее фонологическую нестабильность соответствующей фонемы, проявлявшуюся в речи писцов и переписчиков того времени (\textsc{гр+хъ}~"--- \textsc{грехъ})
\end{inparaenum}
\autocite[378]{uflyand_alexeeva:2008}. Поскольку же решение обозначенных и связанных с ними проблем не входило в круг наших непосредственных задач (о единственном значимом исключении см.\ раздел~\ref{subsec:new}), постольку орфографический разнобой частично "<процеживается"> через сито дальнейших этапов разработанного алгоритма и влияет на графический облик некоторых из получаемых на выходе лемм,~"--- что, однако, лишь пуще подтверждает актуальность данной проблематики, диктующую необходимость дальнейших разработок в этой области с привлечением нового корпусного материала.

\section{Стемминг}

Традиционно среди методов автоматизированного морфологического анализа на основе правил (\foreignlanguage{english}{rule-based}) ключевое место отводится лемматизации и стеммингу. В статье В.",В.~Бочарова и О.",В.~Митрениной \autocite[21]{mono_morphology} различие между ними проводится на основании того, используется ли в процессе анализа какой-либо словарный ресурс (полноценный грамматический словарь, словарь основ либо отдельных морфем и~т.",д.)\ или же разбор текстовых форм сводится к обработке словоизменительных формантов с точечным привлечением словарей ограниченного объёма (в основном для учёта всевозможных исключений). Однако в условиях того, что разработки каких-либо грамматических словарных ресурсов на материале СКАТ ранее не велись, производить лемматизацию в обозначенном смысле нам не представляется возможным. Более широко под лемматизацией обычно подразумевается "<идентификация инвариантов лексических единиц (выражение ЛО [лексикографического описания.~"--- \textit{К.",С.}]\ с точностью до отдельной лексемы)"> \autocite[76]{koval:2005}; очевидно, при таком понимании на то, каким образом осуществляется переход от словоформ к леммам, не накладывается никаких существенных ограничений, и именно его мы склонны далее придерживаться.

В нашем подходе лемматизация осуществляется опосредованно~"--- через промежуточный этап стемминга. Однако в отличие от "<классического"> стемминга представление о псевдоосновах и псевдофлексиях (неизменяемых начальных и конечных сегментах) не находит в нём своего воплощения: известные из разметки морфологические свойства анализируемых словоформ позволяют максимально точно отделить собственно основы от собственно флексий, лингвистически интерпретируемых и несущих полноценное грамматическое значение; кроме того, подобный подход позволяет избежать ошибок \foreignlanguage{english}{overstemming} и \foreignlanguage{english}{understemming}, типичных для "<слепого"> стемминга без учёта семантики. Таким образом, применение данной процедуры к словоформе \textsc{день} (пример, приводимый в \autocite[20]{mono_morphology}) призвано отсечь не псевдофлексию \textsc{-ень} от псевдооссновы \textsc{д-}, но \textit{флексию} \textsc{-ь} от \textit{основы} \textsc{ден-}. Преобразование получаемых таким образом основ с целью их отождествления с основами соответствующих лемм (ср.\ \textsc{ден-} и \textsc{дн-}) происходит на следующем этапе алгоритма.

Словоизменительные парадигмы, сопоставляющие грамматические значения выражаемым ими финальным сегментам, были составлены с опорой на авторитетные учебно-научные пособия по старославянскому языку \autocites{ivanova:1998}{khaburgaev:1986}; составляющие их флексии для учёта неустранимой на этапе нормализации орфографической вариативности записаны в виде регулярных выражений. Условно все парадигмы можно отнести к одному из следующих классов:

\begin{compactenum}
    \item \textbf{именные} парадигмы, при помощи которых производится стемминг
    \begin{inparaenum}[(1)]
        \item существительных,
        \item нечленных (кратких) прилагательных и причастий,
        \item несоставных количественных числительных,
        \item неличных местоимений в именительном и винительном падежах.
    \end{inparaenum}
    Они характеризуются наибольшим качественным разнообразием и значительной вариативностью плана выражения флексий, и для их составления было привлечено множество дополнительных материалов, в частности \autocite{agio:1990};

    \item \textbf{местоименные} парадигмы~"--- задействуются при анализе членных (полных) форм причастий и прилагательных, к числу которых мы относим и порядковые (последние, однако, размечаются тегом \textit{числ/п}), а также неличных местоимений в косвенных падежах (кроме винительного). Собственно адъективные флексии фонетически весьма близки (их стяжённые разновидности~"--- практически идентичны) местоименным, и уже рукописи X---XI~вв. обнаруживают результаты их взаимодействия и уподобления \autocite[170---171]{khaburgaev:1986}. Исходя из этих соображений мы сочли приемлемым объединить их в общий класс;

    \item \textbf{глагольные} парадигмы, обращение к которым происходит при анализе спрягаемых форм глаголов. Церковнославянское спряжение в целом характеризуется большей стройностью и меньшей вариативностью, нежели склонение: основные трудности при лемматизации форм глаголов (в~т.",ч.\ вербоидов) сопряжены с восстановлением леммных основ.
\end{compactenum}

Применительно к некоторым отдельным классам слов лемматизация посредством стемминга принципиально невозможна~"--- такие случаи будут рассмотрены в разделе~\ref{sec:spec}.

\section{Восстановление основы леммы}

\subsection{Нововведения в~формат морфологической разметки}
\label{subsec:new}

\subsection{Имена}

\subsection{Глаголы}

\subsubsection{Основа прошедшего времени}

\subsubsection{Основа настоящего времени}

\section{Особые случаи}
\label{sec:spec}

\section{Анализ работы алгоритма}
