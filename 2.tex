\chapter{Лемматизация церковнославянских словоформ на~основе морфологической разметки}

В данной главе описывается алгоритм лемматизации, реализованный в ходе практической части настоящей работы; изложение теоретических проблем церковнославянского словоизменения и иных побочных вопросов в известной мере вторично и прежде всего обусловлено необходимостью их алгоритмического решения. Разработанный программный пакет написан на языке Python (версии 3.6); исходный код находится в открытом доступе по ссылке: \url{https://github.com/vintagentleman/SCAT}.

\section{Нормализация}
\label{sec:norm}

В литературе по компьютерной морфологии имеет место определённая терминологическая путаница, связанная с терминами "<лемматизация"> и "<нормализация">: порой они употребляются взаимозаменяемо и считаются абсолютными синонимами \autocite[75]{koval:2005}, порой "<нормализации"> приписывают такое определение ("<постановка или словосочетания в каноническую форму">~"--- т.",е.\ базовую форму, или лемму \autocite[14]{mono_morphology}), которое исходя из словообразовательных соображений в большей степени подошло бы "<лемматизации">. Однако в контексте исторических языков, на наш взгляд, между данными терминами необходимо проводить чёткую границу: статус процедуры АОТ на морфологическом языковом уровне надлежит приписывать термину "<лемматизация">~"--- и только ему; "<нормализация">, с другой стороны, прежде всего связана с графическим слоем представления текста (исходя из чего производить её следует строго \textit{до} лемматизации и прочих этапов АОТ на более высоких языковых уровнях) и заключается в приведении различных орфографических вариантов одного и того же слова к некоторому инварианту. Например, элементы следующего ряда:
\begin{inparaenum}[(1)]
    \item \textsc{блаженаго},
    \item \textsc{блаженна(г)},
    \item \textsc{бл(а)женнаго},
    \item \textsc{бла(ж)еннаго},
    \item \textsc{бла(ж)нна(г)},
    \item \textsc{бла(ж)ннаго},
    \item \textsc{блжена(г)\#},
    \item \textsc{блженаго\#},
    \item \textsc{блжен(н)аго\#},
    \item \textsc{блаженна(г)\#}~"---
\end{inparaenum}
представляют собой варианты записи одной и той же словоформы, "<нормальная форма"> которой имеет вид \textsc{блаженаго} или \textsc{блаженнаго}. В случае необходимости выбирать между различными орфографическими вариантами инвариантной единицы выбор во многом обусловлен методологическими установками конкретного исследователя или коллектива: так, при ориентации на близость к орфографии соответствующего современного языка в рассматриваемом примере предпочтение должно быть отдано последнему члену пары. Установку на современную ("<каноническую">) орфографию кладёт в основу своего определения нормализации \textcite[69]{piotrowski:2012}.

Е.",Г.~Уфлянд, ранее работавшая над проблемами нормализации в рамках СКАТ, также~"--- с незначительными отступлениями~"--- придерживалась курса на современную орфографию \autocite[41---42]{uflyand:2008}. В своём дипломном сочинении она очерчивает круг причин и характер вариативных написаний в исследованных 10 текстах корпуса общим объёмом порядка 144 тыс.\ словоупотреблений, а далее описывает разработанный ей алгоритм автоматического сведения орфографических вариантов словоформ к основному (т.",е.\ к нормальной форме), реализованный на Python 2.7 в качестве функционального ядра программы для уменьшения объёма сводного словоуказателя. Функция, о которой идёт речь (\texttt{modif}), предварительно очищает входные символьные цепочки от дублетных символов, различиями в которых на момент создания составляющих корпус рукописей можно пренебречь (так, \textsc{u}, \textsc{d} и \textsc{g}~"--- ук диграфный, лигатурный и юс большой~"--- заменяются на \textsc{у}, \textsc{w} (омега) на \textsc{о}, \textsc{i}~десятеричное на \textsc{и}~восьмеричное; из некириллических знаков оставляется лишь \textsc{+}, обозначающий ять), а затем последовательно проверяет их на вхождения ненормализованных буквосочетаний различных типов, которые при обнаружении подвергаются соответствующим заменам, перечисленным в словарном компоненте модуля. Среди типов заменяемых буквосочетаний выделяются, например, следующие (полный перечень несравненно шире):

\begin{compactenum}
    \item буквосочетания в сокращённых словах под титлом: \textsc{бомтр}~$\to$ \textsc{богоматер}; \textsc{дхв}, \textsc{дхм}~$\to$ \textsc{духов}, \textsc{духом}; \textsc{црквн}, \textsc{црк(в)н}~$\to$ \textsc{церковн};

    \item буквосочетания в сокращённых словах с выносными буквами: \textsc{б(д)ц}, \textsc{б(ди)ц}~$\to$ \textsc{богородиц}; \textsc{ев(г)л}, \textsc{ева(г)л}~$\to$ \textsc{евангел};

    \item отдельные буквосочетания в разных внутрисловных позициях:
    \begin{inparaenum}[(1)]
        \item сочетания заднеязычных согласных с \textsc{ь} везде, кроме абсолютного конца слова: \textsc{гь}, \textsc{кь}, \textsc{хь}~$\to$ \textsc{г}, \textsc{к}, \textsc{х};
        \item сочетания типа \textit{*TorT} с метатезой срединных плавных: \textsc{прьст}, \textsc{пръст}~$\to$ \textsc{перст};
        \item регулярные окончания с выносными буквами: \textsc{а(ш)}, \textsc{я(ш)}~$\to$ \textsc{аше}, \textsc{яше}.
    \end{inparaenum}
\end{compactenum}

В ходе интеграции функции \texttt{modif} в нашу программу мы сочли необходимым привнести некоторые новшества в механизм её работы. Непосредственно в код было внесено множество технологических улучшений: словарная составляющая отныне отделена от алгоритмической (они разнесены по разным файлам), и для хранения информации о буквосочетаниях, подлежащих замене, в ней используется естественным образом напрашивающаяся структура данных типа "<словарь">, а не пары синхронизированных между собой массивов (вносить в подобную структуру любые изменения крайне проблематично). Замены были переписаны на языке регулярных выражений, что позволило более полно и экономно охватить вариативность плана выражения сводимых орфографических вариантов; их перечень также регулярно пополнялся по мере обнаружения нового релевантного языкового материала. Так, в список замен буквосочетаний в сокращённых словах под титлом были добавлены замены \textsc{др(в)н} (либо \textsc{дрвн}) на \textsc{деревн}, \textsc{пр(с)нодв} на \textsc{приснодев}, при замене \textsc{нн+} (и окказионального \textsc{нне}) на \textsc{нын+} дополнительно учтены префиксальные дериваты \textsc{донын+}, \textsc{о(т)нын+}, \textsc{понын+}~"--- и~т.",д.

Кроме того, доступ к данным морфологической разметки позволил корректно производить замены неоднозначных сокращений с выносными буквами и под титлом, обусловленных принадлежностью отдельно взятой словоформы к тому или иному лексико-грамматическому классу: сочетания \textsc{гн} (под титлом) и \textsc{г(с)дн} в начальной позиции форм существительных подлежат замене на \textsc{господин}, в случае же прилагательных~"--- на \textsc{господн} (при этом сначала осуществляется замена надстрок \textsc{гнь} и \textsc{г(с)днь} на вариант без редукции~"--- \textsc{господень}); аналогичным образом \textsc{ч(с)т} приводится к написанию \textsc{чест} либо \textsc{чист}. Стандартное сокращение адъективной флексии родительного падежа единственного числа неженского рода при помощи выносного \textsc{(г)} раскрывается как \textsc{гъ} в абсолютном конце словоформ иных частей речи с целью учёта окказиональных написаний типа \textsc{вра(г)} и \textsc{*выпря(г)}.

В результате работы Е.",Г.~Уфлянд вариативность в словоуказателе не была устранена полностью: вне её рассмотрения остались такие явления, как
\begin{inparaenum}[(1)]
    \item чередование редуцированных и гласных полного образования в корнях и префиксах (\textsc{въсхитити}~"--- \textsc{восхитити}),
    \item наличие дублетов с удвоенными согласными (\textsc{воистину}~"--- \textsc{воистинну}),
    \item непоследовательное написание ятя, весьма характерное для поздних рукописей и отражающее фонологическую нестабильность соответствующей фонемы, проявлявшуюся в речи писцов и переписчиков того времени (\textsc{гр+хъ}~"--- \textsc{грехъ})
\end{inparaenum}
\autocite[378]{uflyand_alexeeva:2008}. Поскольку же решение обозначенных и связанных с ними проблем не входило в круг наших непосредственных задач (о единственном значимом исключении см.\ раздел~\ref{subsec:index}), постольку орфографический разнобой частично "<процеживается"> через сито дальнейших этапов разработанного алгоритма и влияет на графический облик некоторых из получаемых на выходе лемм,~"--- что, однако, лишь пуще подтверждает актуальность данной проблематики, диктующую необходимость дальнейших разработок в этой области с привлечением нового корпусного материала.

\section{Стемминг}

\subsection{Методологические замечания}

Традиционно среди методов автоматизированного морфологического анализа на основе правил (\foreignlanguage{english}{rule-based}) ключевое место отводится лемматизации и стеммингу. В статье В.",В.~Бочарова и О.",В.~Митрениной \autocite[21]{mono_morphology} различие между ними проводится на основании того, используется ли в процессе анализа какой-либо словарный ресурс (полноценный грамматический словарь, словарь основ либо отдельных морфем и~т.",д.)\ или же разбор текстовых форм сводится к обработке словоизменительных формантов с точечным привлечением словарей ограниченного объёма (в основном для учёта всевозможных исключений). Однако в условиях того, что разработки каких-либо грамматических словарных ресурсов на материале СКАТ ранее не велись, производить лемматизацию в обозначенном смысле нам не представляется возможным. Более широко под лемматизацией обычно подразумевается "<идентификация инвариантов лексических единиц (выражение ЛО [лексикографического описания.~"--- \textit{К.",С.}]\ с точностью до отдельной лексемы)"> \autocite[76]{koval:2005}; очевидно, при таком понимании на то, каким образом осуществляется переход от словоформ к леммам, не накладывается никаких существенных ограничений, и именно его мы склонны далее придерживаться.

В нашем подходе лемматизация осуществляется опосредованно~"--- через промежуточный этап стемминга. Однако в отличие от "<классического"> стемминга представление о псевдоосновах и псевдофлексиях (неизменяемых начальных и конечных сегментах) не находит в нём своего воплощения: известные из разметки морфологические свойства анализируемых словоформ позволяют максимально точно отделить собственно основы от собственно флексий, лингвистически интерпретируемых и несущих полноценное грамматическое значение, и избежать ошибок \foreignlanguage{english}{overstemming} и \foreignlanguage{english}{understemming}, типичных для "<слепого"> стемминга без учёта семантики.

Таким образом, применение процедуры стемминга к словоформе \textsc{день} (пример, приводимый в \autocite[20]{mono_morphology}) призвано отсечь не псевдофлексию \textsc{-ень} от псевдооссновы \textsc{д-}, но \textit{флексию} \textsc{-ь} от \textit{основы} \textsc{ден-}. Преобразование получаемых таким образом основ с целью их отождествления с основами соответствующих лемм (ср.\ \textsc{ден-} и \textsc{дн-}) происходит уже на следующем этапе алгоритма (см.\ раздел~\ref{sec:lem}).

\subsection{Классы словоизменительных парадигм}

Словоизменительные парадигмы, сопоставляющие грамматические значения выражаемым ими финальным сегментам, были составлены с опорой на авторитетные учебно-научные пособия по старославянскому языку \autocites{ivanova:1998}{khaburgaev:1986}; составляющие их флексии для учёта неустранимой на этапе нормализации орфографической вариативности записаны в виде регулярных выражений. Условно все парадигмы можно отнести к одному из следующих классов.

\paragraph{Именные парадигмы}

При помощи именных парадигм производится стемминг
\begin{inparaenum}[(1)]
    \item существительных,
    \item нечленных (кратких) прилагательных и причастий,
    \item несоставных количественных числительных,
    \item неличных местоимений в именительном и винительном падежах.
\end{inparaenum}
Они характеризуются наибольшим качественным разнообразием и значительной вариативностью плана выражения флексий, и для их составления было привлечено множество дополнительных материалов, в частности \autocite{agio:1990}. Например, частная именная парадигма *j\={a}-склонения во множественном числе мужского (и одновременно женского) рода имеет следующий вид:

\begin{Verbatim}[fontsize=\small, gobble=4, xleftmargin=5ex]
    ('ja', 'им', 'мн', 'м'): '[+АЕИЫЯ]',
    ('ja', 'род', 'мн', 'м'): '[+ЕИЫ]И|[ЪЬ]',
    ('ja', 'дат', 'мн', 'м'): '[АЯ]М[ЪЬ`]',
    ('ja', 'вин', 'мн', 'м'): '[+АЕИЫЯ]',
    ('ja', 'тв', 'мн', 'м'): '[АЯ]МИ',
    ('ja', 'мест', 'мн', 'м'): '[АЯ]Х[ЪЬ`]',
    ('ja', 'зв', 'мн', 'м'): '[+АЕИЫЯ]',
\end{Verbatim}

\paragraph{Местоименные парадигмы}

Парадигмы местоименного класса задействуются при анализе членных (полных) форм причастий и прилагательных, к числу которых мы относим и порядковые (последние, однако, размечаются тегом \textit{числ/п}), а также неличных местоимений в косвенных падежах (кроме винительного). Собственно адъективные флексии фонетически и орфографически весьма близки (их стяжённые разновидности~"--- практически идентичны) местоименным, и уже рукописи X---XI~вв. обнаруживают результаты их взаимодействия и уподобления \autocite[170---171]{khaburgaev:1986}; исходя из этих соображений мы сочли приемлемым объединить их в общий класс. В качестве примера приведём парадигму местоименного склонения по твёрдому типу в единственном числе среднего рода:

\begin{Verbatim}[fontsize=\small, gobble=4, xleftmargin=5ex]
    ('тв', 'им', 'ед', 'ср'): 'О?Е',
    ('тв', 'род', 'ед', 'ср'): 'А?[АЕО]?ГО',
    ('тв', 'дат', 'ед', 'ср'): 'У?[ЕОУ]?МУ',
    ('тв', 'вин', 'ед', 'ср'): 'О?Е',
    ('тв', 'тв', 'ед', 'ср'): '[ИЫ]?[+ЕИЫ]М[ЪЬ`]',
    ('тв', 'мест', 'ед', 'ср'): '[+Е]?[+ЕО]М[ЪЬ`]',
    ('тв', 'зв', 'ед', 'ср'): 'О?Е',
\end{Verbatim}

\paragraph{Глагольные парадигмы}

Обращение к глагольным парадигмам происходит при анализе спрягаемых форм глаголов, а также эловых причастий. Церковнославянское спряжение в целом характеризуется большей стройностью и меньшей вариативностью, нежели склонение; основные трудности при лемматизации глагольных форм (в~т.",ч.\ вербоидов) связаны со следующим этапом восстановления леммных основ. Так, например, выглядит парадигма спряжения глаголов в форме сигматического аориста:

\begin{Verbatim}[fontsize=\small, gobble=4, xleftmargin=5ex]
    ('1', 'ед'): 'Х?[ЪЬ`]',
    ('2', 'ед'): '[+Е]',
    ('3', 'ед'): '[+Е]',
    ('1', 'дв'): 'Х?ОВ[+Е]',
    ('2', 'дв'): 'С?Т[+АЕ]',
    ('3', 'дв'): 'С?Т[+АЕ]',
    ('1', 'мн'): 'Х?ОМ[ЪЬ`]',
    ('2', 'мн'): 'С?ТЕ',
    ('3', 'мн'): 'Ш?[АЯ]',
\end{Verbatim}

\subsection{Особые случаи}

Морфологически неизменяемые формы~"--- а именно, несклоняемые прилагательные (близкие по значению к наречиям и весьма немногочисленные \autocite[140---141]{ivanova:1998}: в каждом из размеченных текстов единожды употреблено лишь прилагательное \textsc{исполнь} `полный'), инфинитивы и супины (последние на рассмотренном материале не встречаются вовсе), наречия, предлоги и послелоги, союзы, частицы, междометия~"--- стеммингу естественным образом не подлежат: в качестве лемм им присваиваются их нормализованные формы. Однако лемматизация посредством стемминга также затруднена либо принципиально невозможна и применительно к некоторым отдельным группам изменяемых слов, заслуживающих отдельного рассмотрения.

\paragraph{Составные существительные}

На материале рассмотренных житийных текстов выделяются две семантико-морфологические группировки существительных со склонением нескольких частей в их составе.

Во-первых, речь о названиях населённых пунктов со второй корневой морфемой \textsc{-град-} либо \textsc{-город-} (полногласный вариант низкочастотен, но встречается в некоторых текстах корпуса): \textsc{*костянтинъградъ}, \textsc{*новъградъ}. При анализе подобных имён собственных мы исходим из посылки, что их первая составляющая всегда представляет собой существительное или прилагательное мужского рода *\u{o}- (см.\ примеры выше) или *j\u{o}-склонения (ср.\ \textsc{*царьградъ}), т.",е.\ изменяется аналогично либо сходно существительному \textsc{градъ}. Таким образом, тип склонения первой части можно считать известным, а процедуру стемминга единообразно производить над обоими компонентами: \textsc{*нов+град+}~$\to$ \textsc{-нов-}, \textsc{-град-}.

Во-вторых, спецификой в рассматриваемом аспекте обладают наименования времён суток \textsc{полдень} и \textsc{полнощь} (русизм \textsc{полночь} в корпусе не встречается). Здесь типы склонения составных частей принципиально не совпадают: существительные \textsc{день} и \textsc{нощь} склоняются по типам *en и *\u{\i} соответственно, а \textsc{полъ} (в значении `половина')~"--- по типу *\u{u}. Однако несмотря на то, что последний нам априорно известен, и при соответствующей поправке первая часть также подлежит стеммингу, ввиду предельной ограниченности и закрытости данной группы существительных было решено проверять их начальные подстроки на совпадение простым регулярным выражениям: \verb|ПОЛ.*Д[ЕЬ]?Н| либо \verb|ПОЛ.*НОЩ|~"--- и при положительном результате приписывать готовые леммы без какого-либо анализа грамматических данных.

\paragraph{Составные числительные}

В церковнославянском языке сложносоставным количественным числительным присущи специфические словоизменительные особенности, отчасти похожие на таковые в современном русском:

\begin{compactenum}
    \item числительные, обозначающие числа от 11 до 19, представляют собой сочетания единиц первого десятка с предложно-падежной группой \textsc{на десяте} (мест.\ п.). Формальный тип синтаксической связи внутри подобных структур~"--- предложно-падежное примыкание, и изменению подвержена только первая часть: \textsc{пятьнадесяте}~"--- \textsc{пятинадесяте};

    \item названия чисел 20, 30, 40 и 200, 300, 400 являются сочетаниями имён соответствующих единиц с существительными \textsc{десять} (склоняется по типу *ent) либо \textsc{сто} (типа *\u{o}). Тип связи~"--- согласование, а следовательно, при склонении изменяются оба компонента: \textsc{двадесяти}~"--- \textsc{двудесяту} (дв.\ ч.), \textsc{триста}~"--- \textsc{трехъсотъ}  (мн.\ ч.);

    \item обозначения чисел от 50 до 90 и от 500 до 900 лексически подобны названиям десятков и сотен меньших порядков, однако синтаксически ведут себя иначе: здесь наименования единиц управляют существительными \textsc{десять} или \textsc{сто} в форме род.\ п.\ мн.\ ч. Изменяется только первая часть: \textsc{седмьдесятъ}~"--- \textsc{седмидесятъ}, \textsc{осмьсотъ}~"--- \textsc{осмисотъ}.
\end{compactenum}

Принимая во внимание закрытость множества составных числительных и стремясь избежать излишних технологических затруднений, которые могли бы возникнуть при их прямолинейном анализе, здесь мы также ограничились описанием морфемной структуры в виде регулярных выражений:

\begin{Verbatim}[fontsize=\small, gobble=4, xleftmargin=5ex]
    'ЕДИН.*НАДЕСЯТ': 'ЕДИННАДЕСЯТЬ',
    'Д[ЪЬ]?В.*НАДЕСЯТ': 'ДВАНАДЕСЯТЬ',
    'ТР.*НАДЕСЯТ': 'ТРИНАДЕСЯТЬ',
    'ЧЕТЫР.*ДЕСЯТ': 'ЧЕТЫРЕДЕСЯТЕ',
    'ПЯТ.*ДЕСЯТ': 'ПЯТЬДЕСЯТЪ',
    'ШЕСТ.*ДЕСЯТ': 'ШЕСТЬДЕСЯТЪ',
    'СЕДМ.*С[ЪО]?Т': 'СЕДМЬСОТЪ',
    'ОСМ.*С[ЪО]?Т': 'ОСМЬСОТЪ',
    'ДЕВЯТ.*С[ЪО]?Т': 'ДЕВЯТЬСОТЪ',
\end{Verbatim}

\paragraph{Личные, возвратное и вопросительные местоимения}

Словоизменению местоимений указанных разрядов, а именно:
\begin{inparaenum}[(1)]
    \item \textsc{азъ}, \textsc{ты} (ед.\ ч.), \textsc{в+}, \textsc{ва} (дв.\ ч.), \textsc{мы}, \textsc{вы} (мн.\ ч.);
    \item \textsc{себе};
    \item \textsc{кто}, \textsc{что}~"---
\end{inparaenum}
присуще множество глубоко архаических черт (ярко выраженный супплетивизм, особая система флексий), и линейный морфемный анализ их плана выражения неизбежно сопряжён со значительными трудностями, аналогичными таковым в рамках современного русского языка. В связи с этим описывающие их словоизменение регулярные выражения представляют собой полные покрытия соответствующих символьных цепочек, а обращение к разметке производится исключительно с целью выявления возможных ошибок. Примеры парадигм местоимений всех трёх разрядов:

\begin{Verbatim}[fontsize=\small, gobble=4, xleftmargin=5ex]
    ('1', 'им', 'ед'): ('АЗ[ЪЬ`]?$', 'АЗЪ'),
    ('1', 'род', 'ед'): ('М([ЕЪЬ]?Н)?[+ЕЯ]$', 'АЗЪ'),
    ('1', 'дат', 'ед'): ('М([ЕЪЬ]?Н[+Е]|И)$', 'АЗЪ'),
    ('1', 'вин', 'ед'): ('М([ЕЪЬ]?Н)?[+ЕЯ]$', 'АЗЪ'),
    ('1', 'тв', 'ед'): ('М[ЪЬ]?НОЮ$', 'АЗЪ'),
    ('1', 'мест', 'ед'): ('М[ЪЬ]?Н[+Е]$', 'АЗЪ'),

    'род': ('С([ЕО]Б)?[+ЕЯ]$', 'СЕБЕ'),
    'дат': ('С([ЕО]Б[+Е]|И)$', 'СЕБЕ'),
    'вин': ('С([ЕО]Б)?[+ЕЯ]$', 'СЕБЕ'),
    'тв': ('СОБОЮ$', 'СЕБЕ'),
    'мест': ('С[ЕО]Б[+Е]$', 'СЕБЕ'),

    ('м', 'им'): ('Ч[ЪЬ]?ТО$', 'ЧТО'),
    ('м', 'род'): ('Ч[ЕЬ]?(СО)?(ГО)?$', 'ЧТО'),
    ('м', 'дат'): ('Ч[ЕЬ]?(СО)?МУ$', 'ЧТО'),
    ('м', 'вин'): ('Ч[ЪЬ]?ТО$', 'ЧТО'),
    ('м', 'тв'): ('ЧИМ[ЪЬ`]?$', 'ЧТО'),
    ('м', 'мест'): ('Ч[ЕЬ]?(СО)?М[ЪЬ`]?$', 'ЧТО'),
\end{Verbatim}

Наконец, в связи с вопросительными местоимениями нельзя не отметить особое местоимение \textsc{кождо} `каждый': оно обладает уникальной словоизменительной парадигмой, совмещающей в себе как элементы, основанные на вопросительном местоимении \textsc{кии}, так и более архаичные формы на основе вопросительного \textsc{кто} (финальный сегмент при этом остаётся неизменным). Так, в им.\ п.\ наряду с приведённой выше возможны формы \textsc{кииждо} и \textsc{коиждо}, в род.\ п.~"--- \textsc{коегождо} либо \textsc{когождо}; в дат.\ п., однако, на рассмотренном материале встречается лишь форма \textsc{комуждо}. Ввиду столь исключительной нерегулярности всем местоимениям, оканчивающимся на сегмент \textsc{-ждо-}, лемма присваивается автоматически без какого-либо учёта информации, содержащейся в разметке.

\section{Нововведения в~формат морфологической разметки}

В ходе нашей работы выяснилось, что степень подробности формата морфологической разметки в корпусе СКАТ, описанного в разделе~\ref{sec:scat}, не всегда позволяет в полной мере учесть словоизменительные особенности текстовых форм, на которые необходимо делать поправку в ходе процедуры определения леммы (в~т.",ч.\ на этапах восстановления леммной основы и прибавления флексии). Вследствие этого мы сочли необходимым внести в формат аннотации определённые новшества и уточнения, а также обновить существующие разметки житий Димитрия Прилуцкого, Дионисия Глушицкого и Кирилла Новоезерского с их учётом.

\subsection{Тип склонения}

\paragraph{Тип \textit{o/ja}}

На материале исследованных текстов этот смешанный тип был зафиксирован только у форм существительного \textsc{братъ}, чья основа во множественном числе представлена корневым алломорфом \textsc{-братi-}. Исходя из предположения о том, что и другие существительные с таким смешением обнаруживают подобное алломорфирование (ср.\ рус.\ \textit{лист}~"--- \textit{листья}), данный тип можно использовать как маркер наличия йотового наращения у основы и необходимости его удаления при лемматизации.

\begin{table}[h]
    \small
    \begin{tabularx}{\textwidth}{Xp{1.5cm}p{1.5cm}p{1.5cm}p{1.5cm}p{1.5cm}p{1.5cm}}
        \toprule
        \textsc{братir}   & сущ & o/ja & им & мн & м &  \\ \midrule
        \textsc{братiами} & сущ & o/ja & тв & мн & м &  \\ \bottomrule
    \end{tabularx}
\end{table}

\paragraph{Тип \textit{o/en}}

Данный тип присваивается формам множественного числа существительных, обозначающих человека по роду деятельности, происхождению или вероисповеданию с суффиксом \textsc{-ин-} в единственном числе. Во множественном последний подвергается утрате, и тогда тип \textit{o/en} свидетельствует о том, что в ходе лемматизации его необходимо восстановить.

\begin{table}[h]
    \small
    \begin{tabularx}{\textwidth}{Xp{1.5cm}p{1.5cm}p{1.5cm}p{1.5cm}p{1.5cm}p{1.5cm}}
        \toprule
        \textsc{хр(с)тiане} & сущ & o/en & им & мн & м &  \\ \midrule
        \textsc{вrтчrне}    & сущ & o/en & тв & мн & м &  \\ \bottomrule
    \end{tabularx}
\end{table}

\subsection{Число}

Существительным, употребляющимся только во множественном числе (\foreignlanguage{english}{pluralia tantum}), в данной позиции вместо тега \textit{мн} присваивается специальный тег \textit{pt}; его наличие далее позволяет добавлять к леммным основам подобных существительных флексии множественного числа вместо единственного.

\begin{table}[h]
    \small
    \begin{tabularx}{\textwidth}{Xp{1.5cm}p{1.5cm}p{1.5cm}p{1.5cm}p{1.5cm}p{1.5cm}}
        \toprule
        \textsc{dста}  & сущ & o & вин & pt & ср &  \\ \midrule
        \textsc{перси} & сущ & i & вин & pt & ж  &  \\ \bottomrule
    \end{tabularx}
\end{table}

\subsection{"<Индекс">}
\label{subsec:index}

\paragraph{Астериск}



\paragraph{Пометы \textit{$\pm$о} и \textit{$\pm$е}}



\paragraph{Семейство помет \textit{$\pm$X}}



\section{Восстановление леммных основ}
\label{sec:lem}

\subsection{Именные основы}



\subsection{Глагольные основы прошедшего времени}

\subsection{Глагольные основы настоящего времени}

\section{Анализ работы алгоритма}

\section{Прецедентная лемматизация неразмеченных текстов}
