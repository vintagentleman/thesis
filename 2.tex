\chapter{Лемматизация церковнославянских словоформ на~основе морфологической разметки}

В данной главе описывается алгоритм лемматизации, реализованный в ходе практической части настоящей работы; изложение теоретических вопросов моделирования церковнославянского словоизменения и иных побочных проблем в известной мере вторично и обусловлено необходимостью их алгоритмического решения. Разработанный программный пакет написан на языке Python (версии 3.6); исходный код находится в свободном доступе по ссылке: \url{https://github.com/vintagentleman/SCAT}.

\section{Нормализация}
\label{sec:norm}

В литературе по компьютерной морфологии имеет место определённая терминологическая путаница, связанная с терминами "<лемматизация"> и "<нормализация">: порой они употребляются взаимозаменяемо и считаются абсолютными синонимами \autocite[75]{koval:2005}, порой "<нормализации"> приписывают такое определение ("<постановка или словосочетания в каноническую форму">~"--- т.",е.\ базовую форму, или лемму \autocite[14]{mono_morphology}), которое исходя из словообразовательных соображений в большей степени подошло бы "<лемматизации">. Однако в контексте исторических языков, на наш взгляд, между данными терминами необходимо проводить чёткую границу: статус процедуры АОТ на морфологическом языковом уровне надлежит приписывать термину "<лемматизация">; "<нормализация"> же, с другой стороны, прежде всего завязана на графическом уровне представления текста (исходя из чего производить её следует строго \textit{до} лемматизации и прочих этапов АОТ на высших уровнях языка) и заключается в приведении различных орфографических вариантов одного и того же слова к некоторому инварианту \autocite[267]{mono_criticism}. Например, элементы следующего ряда:
\begin{inparaenum}[(1)]
    \item \textsc{блаженаго},
    \item \textsc{блаженна(г)},
    \item \textsc{бл(а)женнаго},
    \item \textsc{бла(ж)еннаго},
    \item \textsc{бла(ж)нна(г)},
    \item \textsc{бла(ж)ннаго},
    \item \textsc{блжена(г)\#},
    \item \textsc{блженаго\#},
    \item \textsc{блжен(н)аго\#},
    \item \textsc{блаженна(г)\#}~"---
\end{inparaenum}
представляют собой варианты записи одной и той же словоформы, "<нормальная форма"> которой имеет вид \textsc{блаженаго} или \textsc{блаженнаго}. В случае необходимости выбирать между различными орфографическими вариантами инвариантной единицы выбор во многом обусловлен методологическими установками конкретного исследователя или коллектива: так, при ориентации на близость к орфографии соответствующего современного языка в рассматриваемом примере предпочтение должно быть отдано последнему члену пары. Установку на современную ("<каноническую">) орфографию кладёт в основу своего определения нормализации \textcite[69]{piotrowski:2012}.

Е.",Г.~Уфлянд, ранее работавшая над проблемами нормализации в рамках СКАТ, также придерживалась курса на современную орфографию \autocite[41---42]{uflyand:2008}. В своём дипломном сочинении она очерчивает круг причин и характер вариативных написаний в исследованных текстах корпуса, а также описывает разработанный ей алгоритм автоматического сведения орфографических вариантов словоформ к основному (т.",е.\ к нормальной форме), реализованный на Python 2.7 в качестве функционального ядра программы для уменьшения объёма сводного словоуказателя. Функция, о которой идёт речь (\texttt{modif}), последовательно проверяет входные символьные цепочки на наличие в них ненормализованных буквосочетаний различных типов (титлованных, содержащих в своём составе выносные буквы, сочетания определённых согласных и~т.",д.)\ и при их наличии осуществляет соответствующие замены, перечисленные в словарном компоненте программы.



\section{Стемминг}

* без словаря больше делать нечего
* источники флексий

\section{Восстановление основы леммы}

\subsection{Нововведения в~формат морфологической разметки}

\subsection{Имена}

\subsection{Глаголы}

\subsubsection{Основа прошедшего времени}

\subsubsection{Основа настоящего времени}

\section{Особые случаи}

\section{Анализ работы алгоритма}
