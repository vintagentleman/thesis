\chapter{Портирование корпуса СКАТ на~платформу TXM}

Платформа TXM\footnote{\url{http://textometrie.ens-lyon.fr} (дата обр.\ \today)}~"--- это свободно распространяемое программное обеспечение для работы с текстовыми корпусами, разработанное в лаборатории IHRIM (\foreignlanguage{french}{Institut d'Histoire des Repr\'{e}sentations et des Id\'{e}es dans les Modernit\'{e}s}) Национального центра научных исследований Франции \autocite{heiden:2010}. TXM предоставляет в распоряжение пользователя широкий набор инструментов количественного и качественного анализа текстов: получение конкордансов в формате KWIC и частотных списков лексических единиц на основе любого приписанного им параметра; построение частотных графиков динамики вхождений единиц, удовлетворяющих пользовательскому запросу (для статистических расчётов используется вычислительный движок R); сбор данных о совместной встречаемости, о лексических шаблонах и~мн.",др. Также платформа приспособлена для обработки текстовой метаинформации, что позволяет пользователю строить подкорпуса (\foreignlanguage{english}{subcorpora}) и разбиения (\foreignlanguage{english}{partitions}) корпусов, введённых в платформу, по различным метатекстовым основаниям. TXM поддерживает множество входных форматов (TXT, ODT/DOC/RTF, XML, различные проприетарные форматы), однако для внутреннего представления содержимого введённых корпусов используется XML-представление.

По инициативе А.",М.~Лаврентьева, одного из главных разработчиков платформы, на протяжении нескольких лет активно сотрудничавшего с коллективом СКАТ и впервые написавшего программу для автоматической конвертации текстовых файлов житий в формат XML \autocite[20]{alexeeva_lavrentiev_azarova_zakharova:2004}, фрагмент корпуса СКАТ объёмом 12 житийных текстов (включая 2 похвальных слова) был загружен на демонстрационный портал TXM\footnote{\url{http://portal.textometrie.org/demo/} (дата обр.\ \today)}, открытый для пользования в режиме онлайн. Однако сотрудники СКАТ участия в этой работе фактически не принимали, вследствие чего корпус был не вполне качественно адаптирован к реалиям платформы: в частности, сами тексты доступны для чтения лишь в упрощённой графике и содержат ошибки перекодирования (в особенности это касается цифирных обозначений чисел).

В ходе настоящего этапа работы мы предприняли попытку внедрить в представление корпуса СКАТ все позднейшие технологические наработки, включая обогащение морфологической аннотации размеченных текстов слоем лемм, и максимально приспособить его к комфортной обработке посредством стационарной версии платформы TXM.

\section{Адаптация структуры XML к~режиму импортирования XTZ}
\label{sec:xml}



\section{Внедрение грамматических данных в~XML-представление}



\section{Поддержка Unicode~6.1}



\section{Результаты, проблемы и перспективы}


