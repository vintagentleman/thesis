\chapter{Портирование корпуса СКАТ на~платформу TXM}

Платформа TXM\footnote{\url{http://textometrie.ens-lyon.fr} (дата обр.\ \today)}~"--- это свободно распространяемое программное обеспечение для работы с текстовыми корпусами, разработанное в лаборатории IHRIM (\foreignlanguage{french}{Institut d'Histoire des Repr\'{e}sentations et des Id\'{e}es dans les Modernit\'{e}s}) Национального центра научных исследований Франции \autocite{heiden:2010}. TXM предоставляет в распоряжение пользователя широкий набор инструментов количественного и качественного анализа текстов: получение конкордансов в формате KWIC и частотных списков лексических единиц на основе любого приписанного им параметра; построение частотных графиков динамики вхождений единиц, удовлетворяющих пользовательскому запросу (для статистических расчётов используется вычислительный движок R); сбор данных о совместной встречаемости, о лексических шаблонах и~мн.",др. Также платформа приспособлена для обработки текстовой метаинформации, что позволяет пользователю строить подкорпуса (\foreignlanguage{english}{subcorpora}) и разбиения (\foreignlanguage{english}{partitions}) корпусов, введённых в платформу, по различным метатекстовым основаниям. TXM поддерживает множество входных форматов (TXT, ODT/DOC/RTF, XML, различные проприетарные форматы), однако для внутреннего представления содержимого введённых корпусов используется XML-представление.

По инициативе А.",М.~Лаврентьева, одного из главных разработчиков платформы, на протяжении нескольких лет активно сотрудничавшего с коллективом СКАТ и впервые написавшего программу для автоматической конвертации текстовых файлов житий в формат XML \autocite[21]{alexeeva_lavrentiev_azarova_zakharova:2004}, фрагмент корпуса СКАТ объёмом 12 житийных текстов (включая 2 похвальных слова) был загружен на демонстрационный портал TXM, открытый для пользования в режиме онлайн\footnote{\url{http://portal.textometrie.org/demo/} (дата обр.\ \today)}. Однако сотрудники СКАТ участия в этой работе фактически не принимали, вследствие чего корпус был не вполне качественно адаптирован к реалиям платформы: в частности, сами тексты доступны для чтения лишь в упрощённой графике и содержат ошибки перекодирования (в особенности это касается цифирных обозначений чисел).

В ходе настоящего этапа работы мы предприняли попытку внедрить в представление корпуса СКАТ все позднейшие технологические наработки, включая обогащение морфологической аннотации размеченных текстов слоем лемм, и максимально приспособить его к комфортной обработке посредством стационарной версии платформы TXM.

\section{Режим импортирования XTZ}

Как было отмечено ранее, платформа TXM приспособлена к импорту текстовых корпусов во множестве различных форматов, однако де-факто стандартным и наиболее активно совершенствуемым в последних версиях платформы способом загрузки входных текстов в формате XML является режим XTZ~"--- \foreignlanguage{english}{XML TEI Zero} \autocite[76--87]{txm}.

Помимо универсальных средств обработки импортируемых документов (включая транспонирование различных уровней разметки во внутреннее TXM-представление, благодаря которому пользователь получает возможность строить подкорпусы и разбиения по любым интересующим его размеченным текстовым структурам, многоаспектное индексирование словоформ и~мн.",др.), режиму XTZ также присуща заточенность под определённый минимальный ("<нулевой">) набор тегов, наиболее часто используемых при разметке текстовых данных с опорой на рекомендации консорциума TEI, и способность учитывать их семантику при конструировании HTML-версий документов, непосредственно доступных для чтения.

\foreignlanguage{english}{(To be concluded!)}

\section{Адаптация XML-представления СКАТ к режиму XTZ}
\label{sec:xml}

\subsection{Обновление структуры XML, предложенной В.",А.~Алексеевым}



\begin{table}[h]
    \footnotesize
    \begin{tabularx}{\linewidth}{XX}
        \toprule
        \thead{Старый тег}                       & \thead{Новый тег} \\ \midrule
        
        \verb|<div1 type='part' n='1'></div1>|   & \verb|<milestone/>| \\ \midrule
        
        \verb|<div2 type='page' n='1'>|          & \\
        \verb| <div3 type='back'></div3>|        & \verb|<pb n='-1'/>| \\
        \verb|</div2>|                           & \\ \midrule
        
        \verb|<div3 type='front'>|               & \\
        \verb| <div4 type='col' n='1'></div4>|   & \verb|<pb n='1a'/>| \\
        \verb|</div3>|                           & \\ \midrule
        
        \verb|<l n='1'></l>|                     & \verb|<lb n='2'/>| \\ \midrule
        
        \verb|<w>|                               & \\
        \verb| <orig>ѿ</orig>|                   & \\
        \verb| <reg>О(Т)</reg>|                  & \verb|<w reg='о(т)' src='W(Т)'>ѿ</w>| \\
        \verb| <src>W(Т)</src>|                  & \\
        \verb|</w>|                              & \\ \midrule
        
        \verb|<w>|                               & \\
        \verb| <orig><choice><sic>го</sic>|      & \\
        \verb| <corr>его</corr></choice></orig>| & \verb|<w reg='его' src='~ГО &lt;ЕГО&gt;'>ег</w>| \\
        \verb| <reg>~ГО &lt;ЕГО&gt;</reg>|       & \verb|<note type='corr'>его</note>| \\
        \verb| <src>~ГО &lt;ЕГО&gt;</src>|       & \\
        \verb|</w>|                              & \\ \midrule
        
        \verb|<c type='punctuation'></c>|        & \verb|<pc></pc>| \\ \bottomrule
        \caption{Предлагаемые замены тегов в XML-представлении}
    \end{tabularx}
\end{table}

\subsection{Проблемы совместимости с TEI}



\subsection{Поддержка Unicode 6.1}



\section{Результаты, проблемы и перспективы}


