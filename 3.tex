\chapter{Портирование корпуса СКАТ на~платформу TXM}

Платформа TXM\footnote{\url{http://textometrie.ens-lyon.fr} (дата обр.\ \today)}~"--- это свободно распространяемое программное обеспечение для работы с текстовыми корпусами, разработанное в лаборатории IHRIM (\foreignlanguage{french}{Institut d'Histoire des Repr\'{e}sentations et des Id\'{e}es dans les Modernit\'{e}s}) Национального центра научных исследований Франции \autocite{heiden:2010}. TXM предоставляет в распоряжение пользователя широкий набор инструментов количественного и качественного анализа текстов: получение конкордансов в формате KWIC и частотных списков лексических единиц на основе любого приписанного им параметра; построение частотных графиков динамики вхождений единиц, удовлетворяющих пользовательскому запросу (для статистических расчётов используется вычислительный движок R); сбор данных о совместной встречаемости, о лексических шаблонах и многое другое. Также платформа приспособлена для обработки текстовой метаинформации, что позволяет пользователю строить подкорпуса (\foreignlanguage{english}{subcorpora}) и разбиения (\foreignlanguage{english}{partitions}) корпусов, введённых в платформу, по различным метатекстовым основаниям. TXM поддерживает множество входных форматов (TXT, ODT/DOC/RTF, XML, различные проприетарные форматы), однако для внутреннего представления содержимого введённых корпусов используется XML"=представление.

По инициативе А.",М.~Лаврентьева, одного из главных разработчиков платформы, на протяжении нескольких лет активно сотрудничавшего с коллективом СКАТ и впервые написавшего программу для автоматической конвертации текстовых файлов житий в формат XML \autocite[21]{alexeeva_lavrentiev_azarova_zakharova:2004}, фрагмент корпуса СКАТ объёмом 12 житийных текстов (включая 2 похвальных слова) был загружен на демонстрационный портал TXM, открытый для пользования в режиме онлайн\footnote{\url{http://portal.textometrie.org/demo/} (дата обр.\ \today)}. Однако сотрудники СКАТ участия в этой работе фактически не принимали, вследствие чего корпус был не вполне качественно адаптирован к реалиям платформы: в частности, сами тексты доступны для чтения лишь в упрощённой графике и содержат ошибки перекодирования (в особенности это касается цифирных обозначений чисел).

Настоящий этап работы нацелен на устранение всех подобных недостатков и максимальное приспособление корпуса СКАТ к комфортному использованию при помощи стационарной версии платформы TXM, а также на внедрение в TXM"=совместимое представление текстов корпуса слоя грамматических данных и лемм.

\section{Режим импортирования XTZ}

Как было отмечено ранее, платформа TXM приспособлена к импорту текстовых корпусов во множестве различных форматов, однако де-факто стандартным и наиболее активно совершенствуемым в позднейших версиях платформы способом загрузки входных текстов в формате XML является режим XTZ~"--- \foreignlanguage{english}{XML TEI Zero} \autocite[76]{txm}.

Помимо универсальных средств обработки импортируемых документов (включая транспонирование различных уровней разметки во внутреннее TXM"=представление, благодаря которому пользователь получает возможность строить подкорпусы и разбиения по любым интересующим его размеченным текстовым структурам, многоаспектное индексирование словоформ и многое другое), режиму XTZ также присуща ориентированность на определённый минимальный ("<нулевой">) набор тегов, наиболее часто используемых при разметке текстовых данных с опорой на рекомендации консорциума TEI, и способность учитывать их семантику при конструировании HTML"=изданий, непосредственно доступных для чтения.

\begin{table}[t]
    \small
    \begin{tabularx}{\linewidth}{Xp{4cm}X}
        \toprule
        \thead{XML} & \thead{HTML} & \thead{Пояснение} \\ \midrule\midrule
        \texttt{<head>} & \texttt{<h2>} & Заголовок \\ \midrule
        \texttt{<p>} & \texttt{<p>} & Абзац \\ \midrule
        \texttt{<hi>} & \texttt{<b>} & Полужирное начертание \\ \midrule
        \texttt{<emph>} & \texttt{<i>} & Курсивное начертание \\ \midrule
        \texttt{<list type='unordered'>} & \texttt{<ul>} & Маркированный список \\ \midrule
        \texttt{<list type='ordered'>} & \texttt{<ol>} & Нумерованный список \\ \midrule
        \texttt{<item>} & \texttt{<li>} & Элемент списка \\ \midrule
        \texttt{<table>} & \texttt{<table>} & Таблица \\ \midrule
        \texttt{<row>} & \texttt{<tr>} & Табличная строка \\ \midrule
        \texttt{<cell>} & \texttt{<td>} & Табличная ячейка \\ \midrule
        \texttt{<graphic>} & \texttt{<img>} & Рисунок \\ \midrule
        \texttt{<ref>} & \texttt{<a>} & Гиперссылка \\ \midrule
        \texttt{<note>} & \texttt{<a>}, \texttt{<span>} & Сноска \\ \midrule
        \texttt{<w>} & \texttt{<span>} & Токен \\ \bottomrule
        \caption{Преобразования тегов при импорте в режиме XTZ (по \autocite[78--80]{txm})}
        \label{tab:edition}
    \end{tabularx}
\end{table}

Так, определяемые TEI маркеры начала новой строки~"--- \texttt{<lb/>} (\foreignlanguage{english}{line beginning})~"--- при генерации HTML преобразуются в теги \texttt{<br/>}, позволяющие форсировать разрыв строки в любом необходимом месте. Кроме того, если они дополнительно снабжены глобальным атрибутом \texttt{@n}, указывающим на порядковый номер соответствующей строки, то напротив строк через определённые интервалы автоматически вставляются их порядковые номера, подобно тому как нумеруются стихи в академических изданиях античной поэзии. Аналогично обрабатывается тег \texttt{<pb/>} (\foreignlanguage{english}{page beginning}); в тех случаях, когда пагинация текстов корпуса на уровне разметки не предусмотрена, TXM фрагментирует их самостоятельно, исходя из максимального числа токенов на каждой странице (этот параметр задаётся пользователем при импорте).

Перечень прочих XML"=тегов и их HTML"=эквивалентов приведён в таблице~\ref{tab:edition}.

\section{Адаптация XML"=представления СКАТ к режиму XTZ}
\label{sec:xml}

\subsection{Проблемы существующей XML"=структуры}

Последним, кто работал над СКАТ в рассматриваемом аспекте, был В.",А.~Алексеев. В рамках своей магистерской диссертации \autocite[41--54]{alexeev:2011} он предпринял ряд серьёзных мер, направленных на модернизацию XML"=представления текстов СКАТ в соответствии с современными стандартами электронного представления текстовых данных.

Нестандартные сущности, теги и атрибуты, ранее использовавшиеся для отображения графем, отсутствующих в современном русском языке, были заменены на символы Unicode~5.1. Данная мера была продиктована как нормативными, так и прагматическими соображениями, поскольку XML"=представление СКАТ образца нулевых было весьма громоздким и неудобочитаемым; так, результат преобразования в XML такой словоформы, как \textsc{ра(д)уасr}, в нём выглядел следующим образом:

\begin{Verbatim}[fontsize=\small, gobble=4, xleftmargin=5ex]
    ра<osl_letter type='overline'>
      д
    </osl_letter>уас&cyr-littleyus;
\end{Verbatim}

Те немногие графемы, которые не были определены в кодовой таблице Unicode~5.1, В.",А.~Алексеев предложил по-прежнему кодировать как сущности~"--- например, \texttt{\&i8-overline;} в случае выносного \textsc{и} восьмеричного. При этом все сущности были определены в отдельном файле определения типа документа (DTD), а также снабжены формальной декларацией (\texttt{<charDecl>}) на уровне описания кодировки TEI"=документа (\texttt{<encodingDesc>}). Этот механизм был впервые включён в рекомендации TEI в версии P5 \autocite[39, 192--201]{tei}, полноценное обновление до которой и строгое следование соответствующим нормам также входило в круг задач диссертационного исследования.

Тем не менее, разработанная В.",А.~Алексеевым версия XML"=представления СКАТ по ряду причин не является TXM"=совместимой. Во-первых, для разметки мельчайших структурных частей рукописи (страниц, колонок и строк) было предложено использовать сразу два синонимичных набора элементов:

\begin{compactenum}
    \item парные теги \texttt{<div2>}, \texttt{<div3>}\footnotemark, \texttt{<div4>}, \texttt{<l>};
    \item пустые теги \texttt{<pb/>}, \texttt{<cb/>}, \texttt{<lb/>}.
\end{compactenum}

\footnotetext{
    Тег \texttt{<div2>} маркирует лист, а \texttt{<div3>}~"--- страницу, т.",е.\ одну из сторон листа (лицевую либо оборотную).
}

Если первый набор нацелен на описание формально"=иерархической организации XML"=документа, то последний скорее предназначен для его семантического структурирования: вместо разбивки на строго непересекающиеся блоки в текст вносятся маркеры (\foreignlanguage{english}{milestones}), попросту указывающие на окончание одной структурной единицы и начало другой. Оба способа одновременно консорциум TEI предписывает задействовать лишь тогда, когда размечаемых структур более одной и они являются соперничающими \autocite[123--124]{tei}, т.",е.\ синонимичными, но не идентичными; в противном случае большей простотой и практичностью, невзирая на меньшую экспрессивность, обладает маркерная аннотация. Если же кодированию подлежит множество разнородных структур, то её использование для разметки таких базовых единиц, как строки, колонки и страницы, тем более предпочтительно. Кроме того, спецификации режима XTZ затрагивают именно пустые теги, а использования их парных аналогов (и шире~"--- всех элементов и атрибутов с целочисленными суффиксами) ввиду особенностей функционирования поисковой машины CQP, напротив, рекомендуется избегать \autocite[78]{txm}.

Во-вторых, XML"=разметка элементарных лексических единиц (токенов) была призвана учесть множество различных вариантов их графического представления. При этом все подобные варианты определялись как потомки базового тега \texttt{<w>}:

\begin{Verbatim}[fontsize=\small, gobble=4, xleftmargin=5ex]
    <w xml:id='CrlNvz.1'>
      <orig>мѣсѧца</orig>
      <reg>М+СЯЦА</reg>
      <src>М+СRЦА</src>
    </w>
\end{Verbatim}

Здесь внутри вложенного тега \texttt{<src>}\footnote{
    Это единственный случай отступления В.",А.~Алексеевым от рекомендаций TEI. Паронимический тег \texttt{<source>} имеет совершенно иную семантику и иное назначение \autocite[356]{tei}.
} первое слово жития Кирилла Новоезерского представлено в оригинальном 8-битном формате, внутри \texttt{<reg>}~"--- в упрощённой графике; наконец, в теге \texttt{<orig>} оно записано с использованием символов Unicode~5.1. В случае ошибочных написаний иерархия получает дальнейшее усложнение: \texttt{<orig>} в качестве потомка приобретает тег \texttt{<choice>}, обозначающий наличие альтернантов\footnote{
    Строго говоря, триада из \texttt{<orig>}, \texttt{<reg>} и \texttt{<src>} также требует обрамления тегом \texttt{<choice>}, однако подобный шаг, очевидно, ознаменовал бы собой ещё большее осложнение XML"=структуры.
}, а внутрь него в свою очередь заносится ошибка (\texttt{<sic>}) и исправление (\texttt{<corr>}). Например:

\begin{Verbatim}[fontsize=\small, gobble=4, xleftmargin=5ex]
    <w xml:id='CrlNvz.90'>
      <orig><choice>
          <sic>человѣчетвѡ</sic>
          <corr>человѣчествѡ</corr>
      </choice></orig>
      <reg>~ЧЕЛОВ+ЧЕТВО &lt;ЧЕЛОВ+ЧЕСТВО&gt;</reg>
      <src>~ЧЕЛОВ+ЧЕТВW &lt;ЧЕЛОВ+ЧЕСТВW&gt;</src>
    </w>
\end{Verbatim}

Между тем режим XTZ не предполагает наличия у ядерных лексических единиц столь развитой иерархической организации. Он ориентирован на обработку тегов \texttt{<w>} в простейшем виде, когда в качестве их содержимого выступает единственный вариант графического представления токена, а все альтернативные вкупе с прочими сопутствующими сведениями записаны в атрибуты \autocite[77]{txm}. Иначе говоря, предполагается, что элементы \texttt{<w>} являются терминальными узлами XML"=структуры и потомков не имеют; если в действительности это не так, то при импорте последние игнорируются, а содержимым родительского тега считается результат конкатенации содержимого всех дочерних.

Таким образом, при подготовке HTML"=издания первый пример из приведённых выше считался бы тождественным следующему (что было бы нежелательно):

\begin{Verbatim}[fontsize=\small, gobble=4, xleftmargin=5ex]
    <w xml:id='CrlNvz.1'>
      мѣсѧцаМ+СЯЦАМ+СRЦА
    </w>
\end{Verbatim}

\subsection{Структурные нововведения}

Предлагаемые нами нововведения в XML"=структуру текстов СКАТ, призванные обеспечить их полную совместимость с режимом импортирования XTZ, обобщены в таблице~\ref{tab:new_xml}.

\begin{table}[t]
    \footnotesize
    \begin{tabularx}{\linewidth}{XX}
        \toprule
        \thead{Старый тег} & \thead{Новый тег} \\ \midrule\midrule
        
        \texttt{<div1 type='part' n='1'></div1>} & \texttt{<ab></ab>} \\ \midrule
        
        \texttt{<div2 type='page' n='1'>} & \\
        \texttt{~~<div3 type='back'></div3>} & \texttt{<pb n='-1'/>} \\
        \texttt{</div2>} & \\ \midrule
        
        \texttt{<div3 type='front'>} & \\
        \texttt{~~<div4 type='col' n='1'></div4>} & \texttt{<pb n='1a'/>} \\
        \texttt{</div3>} & \\ \midrule
        
        \texttt{<l n='1'></l>} & \texttt{<lb n='1'/>} \\ \midrule
        
        \texttt{<w>} & \\
        \texttt{~~<orig>ѿ</orig>} & \\
        \texttt{~~<reg>О(Т)</reg>} & \texttt{<w reg='о(т)' src='W(Т)'>ѿ</w>} \\
        \texttt{~~<src>W(Т)</src>} & \\
        \texttt{</w>} & \\ \midrule
        
        \texttt{<w>} & \\
        \texttt{~~<orig><choice>} & \\
        \texttt{~~~~<sic>мъ</sic>} & \\
        \texttt{~~~~<corr>мѧ</corr>} & \texttt{<w reg='мя' src='\~{}МЪ \&lt;МR\&gt;'>мъ</w>} \\
        \texttt{~~</choice></orig>} & \texttt{<note type='corr'>мѧ</note>} \\
        \texttt{~~<reg>\~{}МЪ \&lt;МЯ\&gt;</reg>} & \\
        \texttt{~~<src>\~{}МЪ \&lt;МR\&gt;</src>} & \\
        \texttt{</w>} & \\ \midrule
        
        \texttt{<c type='punctuation'></c>} & \texttt{<pc></pc>} \\ \bottomrule
        \caption{Предлагаемые замены тегов}
        \label{tab:new_xml}
    \end{tabularx}
\end{table}

С чисто формальной точки зрения замещение структурных подразделений верхнего уровня \texttt{<div1>} анонимными блоками \texttt{<ab>} (\foreignlanguage{english}{anonymous block}) обусловлено обозначенным выше стремлением избавиться от тегов с целочисленными суффиксами; содержательная же подоплёка данного нововведения состоит в том, что так житийные тексты представляются как нерасчленённые,~"--- иначе говоря, делается имплицитное утверждение, что никаких промежуточных смысловых блоков внутри них не выделяется. Однако в будущем такое положение вещей, вероятно, изменится, поскольку разработки формата сюжетной разметки внутри коллектива СКАТ также ведутся \autocite{rogozina:2015}.

Формальную разбивку документов на листы (\texttt{<div2>}) и страницы (\texttt{<div3>}) предлагается полностью заменить смысловой и для маркировки границ между ними пользоваться исключительно тегом \texttt{<pb/>} с обязательным атрибутом \texttt{@n}, обозначающим порядковый номер соответствующей страницы. Номера лицевых и оборотных сторон листа в соответствии с транслитерационными соглашениями СКАТ отличаются между собой по наличию при них специального префикса~"--- дефиса.

Поскольку в режиме XTZ отсутствует поддержка специализированного тега"=разделителя между колонками (\texttt{<cb/>}, \foreignlanguage{english}{column beginning}), последние видится необходимым рассматривать как отдельные страницы и также отграничивать друг от друга при помощи элемента \texttt{<pb/>}. При этом формат атрибута \texttt{@n} получает дополнительное расширение в виде суффикса \texttt{a} для первой колонки или \texttt{b} для второй. Отметим, что рукописи с тремя колонками и более чрезвычайно редки и в корпусе СКАТ не представлены, а рукопись с двумя колонками всего одна (житие Александра Свирского; РНБ, Пог.~874, XVI~в.).

Замена элементов \texttt{<l>} на \texttt{<lb/>} осуществляется по аналогичному принципу; присваиваемые им порядковые номера являются простыми натуральными числами.

Направление преобразования тегов элементарных лексических единиц (\texttt{<w>}) было обосновано выше: в качестве их содержимого отныне выступает единственный вариант графического представления (совместимый с Unicode), а прочие конвертируются в одноимённые атрибуты. Для ошибочных написаний имеют место следующие спецификации: \begin{inparaenum}[(1)]
    \item в атрибут \texttt{@reg} записывается нормализованная форма исправленного варианта~"--- и только его;
    \item внутри тега \texttt{<w>} содержится Unicode"=совместимое представление оригинального написания;
    \item исправление обрамляется типизированным тегом \texttt{<note>}, непосредственно следующим за токеном.
\end{inparaenum} Далее это позволяет конструировать HTML"=издания житийных текстов в первозданном виде, исправления же отображать как сноски.

Наконец, \texttt{<pc>} введён вместо типизированного тега \texttt{<c>} в угоду краткости и соответствию нормам TEI \autocite[575--577]{tei}.

\subsection{Обновление до Unicode~6.1}

Выше было упомянуто, что стараниями В.",А.~Алексеева между собственной кодировкой исторических символов кириллицы, принятой в проекте СКАТ, и стандартом Unicode~5.1 было установлено практически полное взаимно однозначное соответствие. Исключение составляет ряд выносных букв (\textsc{ь}, \textsc{ы}, \textsc{у}, \textsc{u}, \textsc{и}, \textsc{i}, \textsc{w}, а также \textsc{е} широкое и его йотированный аналог), к моменту окончания диссертационного исследования В.",А.~Алексеева не успевших войти в Unicode; однако им отмечалось, что предложение по внесению соответствующих дополнений в стандарт к тому времени уже было составлено и находилось на рассмотрении одной из рабочих групп ISO (\foreignlanguage{english}{International Organization for Standardization}) \autocite[21]{alexeev:2011}.

В обновлении Unicode до версии~6.1, увидевшем свет в январе 2012~г., данное предложение \autocite{proposal:2010} было принято: все перечисленные выносные буквы стали доступны в составе блока \foreignlanguage{english}{Cyrillic Extended-B}. Следовательно, отныне кодировать недостающие символы как сущности и приписывать им формальную декларацию нет необходимости, и всем им были поставлены в соответствие их интернациональные эквиваленты.

\subsection{Нормализация и лемматизация}

В обновлённое XML"=представление были интегрированы все технологические наработки, составившие предмет обсуждения предыдущей главы. А именно: \begin{inparaenum}[(1)]
    \item в содержимое атрибута \texttt{@reg} словоформы отныне записываются не просто в упрощённой графике, но в нормализованном виде (см.\ \ref{sec:norm});
    \item морфологически размеченные словоформы дополнительно снабжаются атрибутом \texttt{@ana}, где позиции разметки последовательно перечислены через точку с запятой;
    \item леммы в случае их успешного определения попадают в атрибут \texttt{@lemma}.
\end{inparaenum}

Приложение~\ref{app:xml} иллюстрирует фрагмент XML"=представления начального фрагмента жития Димитрия Прилуцкого.

\section{Проблемы совместимости с TEI}



\section*{Выводы}
\addcontentsline{toc}{section}{Выводы}


