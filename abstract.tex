\afterpage{
    \clearpage\vspace*{\fill}
    
    \begin{abstract}
        В данной выпускной квалификационной работе исследуется проблема разработки автоматизированных инструментов для лемматизации морфологически размеченных житий в составе Санкт"=Петербургского корпуса агиографических текстов (СКАТ). В рамках теоретической части исследования производится обзор существующих ныне восточнославянских исторических корпусов в аспекте реализованных в них технологий грамматической разметки. Практическая составляющая работы посвящена проблемам словоизменения в церковнославянском языке позднесредневекового извода и методам их формального учёта применительно к задаче лемматизации, а также организации полноценного доступа к корпусу СКАТ (включая его размеченный и лемматизированный сегмент) при помощи платформы TXM.
        
        \paragraph{\small Ключевые слова:} древнерусская агиография, грамматическая разметка, исторический корпус, церковнославянское словоизменение, электронное представление рукописей
    \end{abstract}
    
    \selectlanguage{english}
    
    \begin{abstract}
        This graduation paper deals with the problem of developing automatic tools for lemmatizing morphologically annotated vitae comprising the Saint Petersburg Corpus of Hagiographic Texts (SCAT). As a theoretical background of the present work, a survey of existing East Slavic historical corpora is carried out, with special attention paid to the technological aspects of their grammatical annotation. The experimental part addresses the issues of inflectional morphology in late medieval Church Slavonic and the procedures involved in their formalized solution as applied to the problem of lemmatization, as well as the provision of full access to SCAT (including its annotated and lemmatized subcorpus) by means of the TXM platform.
        
        \paragraph{\small Keywords:} Old Russian hagiography, grammatical annotation, historical corpus, inflection in Church Slavonic, digital representation of manuscripts
    \end{abstract}
    
    \selectlanguage{russian}
    \vspace*{\fill}\clearpage
}
