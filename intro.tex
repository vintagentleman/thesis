\intro

Славянская рукописная традиция зародилась уже более тысячи лет назад. От времён, последовавших за просветительской деятельностью преподобных Константина (Кирилла) и Мефодия в середине IX~в., до сегодняшних дней дошли десятки тычяч рукописей, созданных писцами и переписчиками в монастырях Восточной Европы,~"--- как списков Священного Писания, служебников, часословов и прочих богослужебных книг, непосредственно обслуживавших запросы церкви, так и оригинальных произведений, предназначенных для индивидуального чтения: поучений, сказаний, житий святых. Тем не менее, значительная доля данных текстовых массивов по сей день изучена недостаточно и по-прежнему нуждается во всесторонней исследовательской обработке~"--- исторической, этнографической, лингвистической.

Несколько десятилетий назад ситуация начала качественно преображаться в связи с появлением, а впоследствии и массовым распространением компьютеров и цифровых технологий: средства представления рукописей в электронном виде ознаменовали собой принципиально новые возможности их сохранения и изучения вне стен отдельных библиотек и архивных фондов, регулярным доступом к которым обладает далеко не каждый исследователь.

Соответствующие разработки начали появляться уже в конце третьей четверти XX~в.~"--- в том числе на кафедре математической лингвистики Ленинградского государственного университета. С конца 1970-х~гг.\ при участии сотрудников кафедры русского языка ЛГУ, а также ИРЛИ АН СССР и ГПБ им.~М.",Е.~Салтыкова-Щедрина на кафедре начал создаваться фонд фото- и ксерокопий списков древнерусских житий и похвальных слов XV--XVII вв.\ \autocite[512]{averina_alexeeva_gerd:1996}, впоследствии получивший название "<Санкт"=Петербургский корпус агиографических текстов"> (СКАТ). Для представления содержимого фонда в памяти ЭВМ каждую копию рукописного текста было необходимо транслитерировать~"--- перевести в машиночитаемый формат при помощи специальной системы кодирования. Однако в те годы фактически единственным средством ввода символьных цепочек в память компьютера являлись 8-битные кодировки на базе ASCII (\foreignlanguage{english}{American Standard Code for Information Interchange}), очевидно не предназначенные для размещения в диапазоне кодируемых символов знаков устаревших и экзотических систем письменности (в~т.",ч.\ кириллической). Вследствие этого для набора текстов, составляющих фонд, на кафедре была выработана собственная кодировка, в которой для вышедших из употребления символов кириллицы были введены замены (преимущественно буквы латинского алфавита): так, юс большой и юс малый обозначаются соответственно "<\textsc{g}"> и "<\textsc{r}">, кси~"--- "<\textsc{l}"> и~т.",д. Тексты вводимых в память ЭВМ рукописей набираются квалифицированными специалистами"=филологами вручную при помощи специально разработанного шрифта AGIO и затем автоматически переводятся в данную кодировку; при этом в текст вставляются словоразделы (в соответствии с принципами, разработанными проф.\ А.",А.~Алексеевым для издания серии "<Библиотека литературы Древней Руси">), а также маркируются границы составных частей рукописи~"--- строк, колонок и страниц. Всего к настоящему времени в базу данных введено более полусотни рукописей общим объёмом около полумиллиона словоупотреблений \autocite{gerd_alexeeva_azarova_zakharova:2004}.

Сегодня доступ к результатам работы коллектива проекта обеспечивается двояко. С одной стороны, с конца 1990-х~гг.\ издательством Санкт"=Петербургского государственного университета ведётся публикация изданий серии "<Памятники русской агиографической литературы">, в каждом из которых содержится один или несколько подготовленных к печати житийных текстов, набранных упомянутым выше шрифтом AGIO, полный словоуказатель словоформ, а также текстологические статьи об истории публикуемых житий, биографии святых, сведения об обителях. Последний, одиннадцатый выпуск увидел свет в 2012~г.; там же приведён перечень всех предыдущих публикаций серии \autocite[4]{coll:2012}.

С другой стороны, всё более повсеместное распространение онлайн"=технологий в 2000-х~гг.\ дало импульс к тому, чтобы обеспечить доступ к опубликованным материалам через интернет: был создан сайт проекта\footnote{\url{http://project.phil.spbu.ru/scat/} (дата обр.\ \today)}, а корпус получил своё нынешнее наименование. На сегодняшний день около полутора десятков житий доступны для загрузки с сайта в двух форматах: PDF, воспроизводящем их представление в печатных сборниках, и XML, где с помощью системы тегов производится формальное членение рукописей на структурные элементы. XML"=разметка текстов СКАТ соответствует международному стандарту оформления электронных изданий~"--- \foreignlanguage{english}{Text Encoding Initiative} (TEI).

\begin{figure}[t!]
    \centering
    \begin{subfigure}[t]{0.495\textwidth}
        \includegraphics[width=\linewidth]{scat_search}
        \caption{Выдача по запросу \textsc{бц} (режим нестрогого соответствия)}
        \label{fig:scat:1}
    \end{subfigure}
    \hfill
    \begin{subfigure}[t]{0.495\textwidth}
        \includegraphics[width=\linewidth]{scat_output}
        \caption{Контекстное окно вхождения словоформы \textsc{бголюбци\#} (ГП 323/19)}
        \label{fig:scat:2}
    \end{subfigure}
    \caption{Поиск по словоуказателю СКАТ}
\end{figure}

Также на сайте имеется возможность поиска по корпусу~"--- вернее, по той его части, для которой построен сводный словоуказатель. Это центральный компонент лингвистического обеспечения СКАТ, представляющий собой список словарных статей, в каждой из которых указана словоформа в нормализованном виде, абсолютная частота её встречаемости по всем проиндексированным рукописям и адреса вхождений. Адрес состоит из сокращённого наименования рукописи и сочетания порядковых номеров листа (с уточнением стороны~"--- лицевой либо оборотной), колонки и строки, разделённых косой чертой. При нажатии на адрес в поисковой выдаче (рис.~\ref{fig:scat:1}) пользователю предлагается "<нарезка"> из соответствующего PDF"=документа (рис.~\ref{fig:scat:2}), в которую попадает искомое вхождение; отыскивать его приходится самостоятельно~"--- путём отсчитывания от межстраничной либо межколонной границы с номером, указанным в адресе, необходимого числа строк.

Однако современный электронный корпус~"--- в отличие от простой коллекции текстов~"--- должен располагать определённым набором автоматизированных инструментов, применимых в ходе решения конкретных лингвистических задач. В ряде зарубежных работ по языкам с ограниченными ресурсами в последние годы вошло в обиход понятие BLARK~"--- \foreignlanguage{english}{Basic Language Resource Toolkit} (базовый набор лингвистических ресурсов), которое определяется как "<\foreignlanguage{english}{the minimal set of language resources that is necessary to do any precompetitive research and education at all}"> \autocite[11]{krauwer:2003} (минимальный набор лингвистических ресурсов, необходимый для любых базовых исследовательских и образовательных нужд). BLARK может включать в себя как традиционные одно- и двуязычные словари и грамматики, так и специфические ресурсы, вошедшие в лингвистический обиход лишь в последние десятилетия: модули распознавания и синтеза речи, морфосинтаксические анализаторы и~т.",д. Притом отмечается, что этот список не закрытый и может варьироваться от языка к языку: очевидно, для древнеписьменных языков, в число которых входит и церковнославянский, неактуальна задача обработки устной речи, однако вследствие некодифицированного характера орфографии зачастую требуются модули её нормализации.

\textcite[28]{passarotti:2010} предлагает вариант BLARK ("<\foreignlanguage{english}{a BLARK-like set}">) для латинского языка, который, как кажется, в равной степени приложим к другим древнеписьменным языкам. В нём предусмотрены инструменты, направленные на решение следующих основных задач: \begin{inparaenum}[(1)]
    \item предобработка текстовых данных: токенизация и распознавание именованных сущностей;
    \item морфологический анализ: лемматизация и разрешение морфосинтаксической неоднозначности;
    \item синтаксический анализ (поверхностный и глубинный);
    \item разрешение анафоры;
    \item семантический и прагматический анализ.
\end{inparaenum}

\textbf{Цель} настоящей работы заключается в том, чтобы в применении к корпусу СКАТ разработать комплекс инструментов для осуществления одной из подзадач морфологического анализа, специфицируемой в рамках базового набора лингвистических ресурсов,~"--- процедуры лемматизации. \textbf{Задачи}, которые необходимо решить для достижения поставленной цели, таковы:

\begin{compactenum}
    \item ознакомление с системами представления грамматических сведений (в~т.",ч.\ данных по леммам) в существующих восточнославянских исторических корпусах;
    \item изучение теоретических предпосылок алгоритма лемматизации церковнославянского языкового материала с учётом всех релевантных морфологических особенностей и его программная реализация;
    \item организация доступа к лемматизированному подкорпусу СКАТ (и шире~"--- ко всей оцифрованной части корпуса) с использованием общедоступных технологических средств.
\end{compactenum}

\textbf{Объект} основной части исследования~"--- словоизменение в церковнославянском языке XV--XVII вв. \textbf{Предмет} изучения~"--- проблемы формализации феноменов церковнославянского словоизменения в ходе алгоритмизации перехода от словоформ в несловарных парадигматических позициях к словарным (т.",е.\ леммам). \textbf{Материалом} послужили морфологически размеченные тексты трёх агиографических текстов в составе корпуса СКАТ: жития Димитрия Прилуцкого, Дионисия Глушицкого и Кирилла Новоезерского~"--- суммарным объёмом около 30~тыс.\ словоупотреблений.

\textbf{Актуальность} работы обоснована тем, что в рамках СКАТ~"--- единственного в своём роде источника сведений по языку древнерусской агиографии эпохи позднего Средневековья и Нового времени~"--- серьёзные попытки разработки составных частей BLARK в целом и подсистем морфологического анализа в частности фактически не предпринимались.

\textbf{Структура} работы включает в себя введение, \total{chpnum}~главы, заключение, список литературы из \total{citnum}~наименований и \total{appnum}~приложения.
