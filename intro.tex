\intro

Славянская рукописная традиция зародилась уже более десяти столетий назад. За время, прошедшее со времён просветительской деятельности преподобных Константина (Кирилла) и Мефодия в середине IX~в., писцами и переписчиками в монастырях по всей Восточной Европе были созданы десятки тысяч рукописей "--- как списков Священного Писания, служебников, часословов и прочих богослужебных книг, непосредственно обслуживавших запросы церкви, так и оригинальных произведений, предназначенных непосредственно для чтения: поучений, сказаний, житий святых. Тем не менее, значительная доля данных текстовых массивов по сей день изучена недостаточно и по-прежнему нуждается во всесторонней исследовательской обработке "--- исторической, этнографической, лингвистической.

Однако несколько десятилетий назад ситуация начала качественно преображаться в связи с появлением, а впоследствии и массовым распространением компьютеров и компьютерных технологий: средства представления рукописей в электронном виде ознаменовали собой новые возможности их сохранения и изучения вне стен отдельных библиотек и архивных фондов, свободным доступом к которым обладает далеко не каждый исследователь.

Соответствующие разработки начали появляться уже в конце третьей четверти XX~в.\ "--- в том числе на кафедре математической лингвистики Ленинградского государственного университета. Начиная с 1980~г.\ при участии сотрудников кафедры русского языка ЛГУ, а также ИРЛИ АН СССР и ГПБ им.~М.\,Е.~Салтыкова-Щедрина на кафедре начал создаваться Санкт-Петербургский корпус агиографических текстов (СКАТ) "--- автоматизированный банк данных по спискам житий севернорусских святых и похвальных слов XVI--XVIII~вв.\ \autocite{averina_alexeeva_gerd:1996}. С каждой рукописи снимались фотокопии, каждую из которых впоследствии было необходимо транслитерировать "--- перевести в машиночитаемый формат. Транслитерация производилась вручную "--- и производится так и поныне; в целом же к настоящему времени в базу данных введено около 65 рукописей общим объёмом более полумиллиона словоупотреблений \autocite[157]{azarova_alexeeva:2013}.

Доступ к результатам работы коллектива проекта обеспечивается двояко. С одной стороны, с конца 1990-х~гг.\ издательством Санкт-Петербургского государственного университета ведётся публикация изданий серии <<Памятники русской агиографической литературы>>, каждое из которых содержит один или несколько подготовленных к печати житийных текстов, набранных специально разработанным шрифтом, полный словоуказатель словоформ, а также текстологические статьи об истории публикуемых житий, сведения о святых и их обителях. Последний, одиннадцатый по счёту, выпуск увидел свет в 2012 г.; там же приведён перечень всех предыдущих публикаций серии \autocite[4]{coll:2012}.

С другой стороны, всё более повсеместное распространение онлайн-технологий, пришедшееся на начало 2000-х~гг., дало импульс к тому, чтобы обеспечить доступ к материалам СКАТ через интернет; в конце концов был запущен \href{http://project.phil.spbu.ru/scat/}{сайт проекта}. На сегодняшний день около полутора десятков житий доступны для загрузки с сайта в двух форматах: PDF, воспроизводящем их представление в печати, и XML, где с помощью системы тегов производится формальное членение рукописей на структурные части. XML-разметка соответствует международному стандарту оформления электронных изданий "--- Text Encoding Initiative (TEI).

\begin{figure}[t!]
    \begin{subfigure}[t]{0.5\textwidth}
        \centering
        \includegraphics[width=\linewidth]{intro_scat_search}
        \caption{Выдача по запросу \textsc{бц} (нестрогое соответствие)}
        \label{intro:scat_1}
    \end{subfigure}
    ~
    \begin{subfigure}[t]{0.5\textwidth}
        \centering
        \includegraphics[width=\linewidth]{intro_scat_output}
        \caption{Пример контекстного окна для вхождения словоформы \textsc{бголюбци\#} (ГП 323/19)}
        \label{intro:scat_2}
    \end{subfigure}
    \caption{Поиск по словоуказателю СКАТ}
\end{figure}

Также на сайте имеется возможность поиска по корпусу "--- вернее, той его части, для которой построен сводный словоуказатель. Это центральный компонент лингвистического обеспечения СКАТ; он представляет собой список словарных статей, в каждой из которых указана словоформа в некотором нормализованном виде, абсолютная частота её встречаемости в рукописях и адреса вхождений. Адрес состоит из сокращённого наименования рукописи и сочетания порядковых номеров листа (с указанием стороны "--- лицевой либо оборотной), колонки и строки, разделённых косой чертой. При нажатии на адрес в поисковой выдаче (рис.~\ref{intro:scat_1}) пользователю выдаётся <<нарезка>> из соответствующего PDF-документа (рис.~\ref{intro:scat_2}), в которую попадает искомое вхождение; однако отыскивать его предлагается самостоятельно "--- путём отсчитывания от межстраничной (либо межколонной) границы с номером, указанным в адресе, необходимого числа строк.

Очевидно, что функционально электронный словоуказатель СКАТ не\-мно\-гим отличается от печатного: это весьма традиционный филологический ресурс, не предназначенный для неспециалистов и обладающий множеством очевидных неудобств и ограничений: отсутствует какое-либо выделение терминов запроса во фрагментах найденных документов, ширина отображаемых контекстных окон непомерно велика. Едва ли представляется возможным называть выдачу подобной системы <<конкордансом>> в современном понимании этого термина.

С другой стороны, современный электронный корпус "--- в отличие от простой коллекции текстов "--- должен располагать определённым набором автоматизированных инструментов, пригодных для эффективного решения конкретных лингвистических задач. В ряде зарубежных работ по языкам с ограниченными ресурсами в последние годы вошло в обиход понятие BLARK "--- Basic Language Resource Toolkit (базовый инструментарий для обработки естественного языка), которое определяется как <<\foreignlanguage{english}{the minimal set of language resources that is necessary to do any precompetitive research and education at all}>> \autocite[11]{krauwer:2003} (минимальный набор лингвистических инструментов, необходимых для любых предварительных исследовательских и образовательных нужд). BLARK может включать в себя как традиционные одно- и двуязычные словари и грамматики, так и специфические ресурсы, вошедшие в лингвистический обиход лишь в последние десятилетия: модули распознавания и синтеза речи, морфосинтаксические анализаторы и~т.\,д. Притом отмечается, что этот список не закрытый и может варьироваться от языка к языку: очевидно, для т.\,н. исторических языков (англ. \textit{historical languages}), в число которых входит и церковнославянский, неактуальна задача обработки устной речи, однако из-за некодифицированной орфографии зачастую требуются модули нормализации.

\textcite{passarotti:2010} предлагает вариант BLARK (<<a BLARK-like set>>) для латинского языка, который, как кажется, в равной степени применим к другим историческим языкам. Он включает в себя инструменты, направленные на решение следующих основных задач:
\begin{inparaenum}[1)]
\item предобработка текстовых данных: токенизация и распознавание именованных сущностей;
\item \label{intro:goal} морфологический анализ: лемматизация и разрешение морфосинтаксической неоднозначности;
\item синтаксический анализ (поверхностный и глубинный);
\item разрешение анафоры;
\item анализ семантики и прагматики.
\end{inparaenum}
Между тем в рамках СКАТ "--- единственного в своём роде источника сведений по церковнославянскому языку XVI--XVIII~вв.\ "--- маргинальный статус можно приписать разве что токенизации (транскрипции, как уже отмечалось выше, набираются вручную с соблюдением специальных соглашений); серьёзные же попытки решения остальных проблем фактически не предпринимались.

С учётом всех вышеперечисленных соображений в данной работе мы ставим перед собой две основополагающие цели. Первой целью является попытка подступиться к задаче \textnumero~\ref{intro:goal} из приведённого выше списка и разработать один из компонентов системы морфологического анализа текстов СКАТ "--- а именно, модуля лемматизации. Материалом послужили морфологически размеченные тексты трёх житий, введённых в корпус: Димитрия Прилуцкого, Дионисия Глушицкого и Кирилла Новоезерского "--- суммарным объёмом около 30~тыс.\ словоупотреблений. Среди конкретных задач этой части работы "--- ознакомление с системами представления грамматических сведений в других славянских исторических корпусах; выработка алгоритма лемматизации с учётом словоизменительных особенностей церковнославянского языка рассматриваемого периода; его программная реализация.

Второй аспект работы связан с приведением СКАТ в соответствие с реалиями современной корпусной лингвистики "--- с его портированием в среду, обеспечившую бы полноценный поиск по корпусу, в том числе по грамматическим тегам и по леммам, полученным на предыдущем этапе, построение конкордансов и частотных списков, разбиение на подкорпусы по различным основаниям и отправление прочего лингвистического функционала. В качестве такой среды была выбрана платформа TXM "--- свободно распространяемое программное обеспечение, обладающее большими техническими возможностями для работы с текстовыми корпусами, в том числе поддержкой стандартов Unicode и TEI-XML, а также рядом технологий автоматической обработки естественного языка \autocite{heiden:2010}. Таким образом, наша задача сводится к адаптации представления корпуса к TXM-совместимому формату, его загрузке в платформу и тестировании доступного инструментария.

Работа состоит из введения, трёх глав, заключения и списка литературы из \total{citnum} наименований.
