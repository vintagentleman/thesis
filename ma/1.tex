\chapter{Лемматизация церковнославянских глаголов}

В настоящей главе будет рассмотрено словоизменение церковнославянского глагола в аспекте проблем, сопряжённых с процедурой лемматизации его словоформ. Далее будет описан компонент разработанного грамматического модуля, программно реализующий данную процедуру.

Исходный код модуля открыт и доступен на GitHub\footnote{\url{https://github.com/vintagentleman/scat-v2}}.

\section{О термине "<лемматизация">}

Статус термина "<лемматизация"> нельзя назвать окончательно сложившимся в литературе.

Предметно лемматизация описывалась как "<идентификация инвариантов лексических единиц (выражение ЛО [лексикографического описания.~"--- \textit{К.",С.}]\ с точностью до отдельной лексемы)"> \autocite[76]{koval:2005}; "<приведение разных текстовых форм слова к его канонической, либо зарегистрированной в словаре как таковая, либо~"--- при включении всех словоформ в автоматический словарь словоформ~"--- назначенной в качестве главной, исходной, несущей основной состав информации"> \autocite[87]{marchuk:2007}. На основании того, что выводимое посредством лемматизации наименование фактически представляет собой интегральный атрибут соответствующей лексемы безотносительно членов её словоизменительной парадигмы, в вышеуказанных источниках лемматизация выносится за рамки собственно морфологического анализа и рассматривается в контексте проблем вычислительной лексикографии.

В более современных источниках лемматизации отводится статус отдельного метода в рамках морфологического анализа. С ней сосуществует стемминг~"--- также алгоритм приведения словоформ к некоторому инварианту ("<псевдооснове">), однако опирающийся на более простые эвристики, опирающиеся в основном на правила обработки суффиксов и флексий наряду с небольшими словарями исключений; лемматизации же, по крайней мере на материале современного русского языка, свойственен преимущественно словарный подход \autocite[21]{boch_mitr:2016}.

Наряду с "<лемматизацией"> в качестве абсолютного синонима иногда используется термин "<нормализация"> \autocite[75]{koval:2005}, а лемма, соответственно, именуется "<нормальной формой">. Однако в контексте исторических языков такое употребление приводит к нежелательной путанице с другим толкованием термина "<нормализация">, связанным с длительным отсутствием в большинстве исторических языков, в~т.",ч.\ церковнославянском, кодифицированной орфографической нормы и вытекающей из этого важной подзадачи автоматической обработки текстов~"--- сведения графико"=орфографических вариантов словоформ к унифицированному представлению. Это последнее и следует называть нормализацией. Иногда в этом значении также прибегают к термину "<каноникализация">, а результирующие инварианты называют каноническими формами \autocite[69]{piotrowski:2012}.

В терминах теории множеств лемматизацию можно определить как функцию из множества словоформ с вариативными грамматическими признаками во множество кортежей, элементы которого представляют собой словоформы из такого подмножества множества всех словоформ языка, у которых грамматические признаки фиксированы.

В лексикографических традициях разных языков длина кортежей~"--- лемм и грамматические признаки, присущие их элементам, может фиксироваться по-разному. Так, словарная форма латинского глагола состоит из четырёх форм, позволяющих однозначно определить его спряжение: \selectlanguage{english} \begin{inparaenum}[(1)]
    \item praesens indicativi activi personae primae singularis,
    \item perfectum indicativi activi personae primae singularis,
    \item supinum,
    \item infinitivus;
\end{inparaenum} \selectlanguage{russian} однако поскольку для восточнославянских языков подобная множественность мало релевантна, далее для простоты мы будем везде предполагать под результатами лемматизации кортежи единичной длины. С другой стороны, варьирование лемм по грамматическим признакам имело место и на восточнославянском материале: известно, что под влиянием "<Российской грамматики"> М.",В.~Ломоносова (1755~г.)\ в первом толковом словаре русского языка, "<Словаре Академии Российской"> (1789--1794~гг.), глаголы как заголовочные слова также фигурировали в форме 1~л.\ ед.",ч. Данная традиция, однако, не закрепилась, и вслед за подавляющим большинством лексикографических источников последних двух столетий мы признаём словарной формой глагола инфинитив.

В отношении причастий мы ориентируемся на современную интерпретацию вербоидов, принятую в академической грамматике и рассматривающую их как атрибутивные формы глагола, а не самостоятельную часть речи \autocite[662, 669]{russ_gram}. Следовательно, предполагается, что процедура их лемматизации также выдаёт на выход формы инфинитива.

Мы предполагаем, что для каждой лемматизируемой словоформы грамматическая информация уже известна в том объёме, который содержится в таблицах с грамматической разметкой корпуса СКАТ (см.\ \ref{sec:annotation}). Как будет показано ниже, даже при данном предположении лемматизация отдельно взятых классов глагольных форм со стопроцентной точностью крайне затруднена,~"--- что, на наш взгляд, исключает возможность решить обозначенную нами проблему сугубо автоматически.

\section{Грамматическая разметка глаголов в~корпусе СКАТ}
\label{sec:annotation}

Принятый в СКАТ формат грамматической разметки был разработан в дипломном сочинении Е.",С.~Ивановой \autocite{ivanova:2006} и впоследствии видоизменён и уточнён Е.",Л.~Алексеевой. Разметка представлена в виде таблиц, где каждой словоформе приписаны соответствующие ей морфологические (в случае аналитических глагольных форм~"--- также отчасти и синтаксические) характеристики. Разметка производится вручную студентами 1--2~курсов в ходе учебной филологической практики. К настоящему времени размеченный подкорпус включает в себя 5~житий общим объёмом порядка 50~тыс.\ словоупотреблений.

Разметка имеет позиционный характер: число и значение граммем в столбцах таблицы, формирующих тегсет, варьирует в зависимости от первой~"--- части речи. Большинство имён (существительные, прилагательные, числительные, а также неличные местоимения) размечаются единообразно путём последовательного указания типа склонения, падежа, числа и рода; для прочих словоформ, за исключением неизменяемых, формат тегсета зависит от наполнения отдельных позиций внутри него же.

Разметка причастий практически не отличается от разметки имён и фиксирует лишь одну добавочную граммему~"--- время. Залог не фиксируется, поскольку непосредственно выводится из типа склонения: все причастия, склоняющиеся по мягкому типу,~"--- действительные, а по твёрдому~"--- страдательные. Обозначение возвратности у всех глагольных форм совмещено с обозначением части речи: \texttt{гл/в}, \texttt{прич/в}, \texttt{инф/в}.

Инфинитив и супин иной разметкой, кроме частеречной, не обладают. Формы супина на рассматриваемом материале не встретились, а инфинитивы и леммы от них тождественны и подлежат только нормализации (см.\ \ref{subsec:norm}); потому и те, и другие далее не рассматриваются.

Что касается личных форм глаголов, то она сохраняет свой позиционный характер, однако правила распределения грамматических тегов в большей степени обусловлены значениями отдельных граммем, чем это касается причастий и имён (среди последних позиционное варьирование фактически имеет место только у местоимений), и зависят от значений наклонения, времени (в случае изъявительного), а также от части речи или синтаксической роли неличных элементов аналитических форм.

Примеры разметки приведены в табл.~\ref{tab:examples}.\footnote{%
    Здесь и далее языковые примеры даны капителью с соблюдением соглашений, принятых в СКАТ: \begin{inparaenum}[(1)]
        \item для графем, вышедших из употребления, используются символы латиницы;
        \item астериск (\textsc{*}) маркирует имена как собственные;
        \item октоторп (\textsc{\#}) обозначает наличие титла;
        \item в скобки заключены выносные буквы.
    \end{inparaenum}
}

\paragraph{Изъявительное наклонение.} Формы размечаются по-разному в зависимости от того, является ли форма времени синтетической или аналитической. В первом случае различия обусловлены собственно граммемой категории времени, во втором~"--- синтаксической ролью составляющих.

\begin{compactenum}
    \item Простые времена: \begin{compactenum}
        \item настоящее-будущее~"--- 6~позиций: \begin{inparaitem}
            \item часть речи,
            \item наклонение,
            \item время,
            \item лицо,
            \item число,
            \item класс (\textit{1--5}\footnote{\textit{1--4}~"--- тематические классы, 5~"--- атематический});
        \end{inparaitem}
        \item аорист и имперфект~"--- 5~позиций: \begin{inparaitem}
            \item часть речи,
            \item наклонение,
            \item время,
            \item лицо,
            \item число;
        \end{inparaitem}
        \item простое прошедшее~"--- 5~позиций: \begin{inparaitem}
            \item часть речи,
            \item наклонение,
            \item время,
            \item род,
            \item число.
        \end{inparaitem}
    \end{compactenum}
    \item Сложные времена: \begin{compactenum}
        \item связка~"--- 6~позиций: \begin{inparaitem}
            \item часть речи,
            \item наклонение,
            \item время,
            \item лицо,
            \item число,
            \item роль (\textit{св}\footnote{Расшифровка для ролей: \textit{св}~"--- связка, \textit{пр}~"--- причастие, \textit{инф}~"--- инфинитив.});
        \end{inparaitem}
        \item причастие~"--- 6~позиций: \begin{inparaitem}
            \item часть речи,
            \item наклонение,
            \item время,
            \item род,
            \item число,
            \item роль (\textit{пр} или \textit{пр-св});
        \end{inparaitem}
        \item инфинитив~"--- 4~позиции: \begin{inparaitem}
            \item часть речи,
            \item наклонение,
            \item время,
            \item роль (\textit{инф}).
        \end{inparaitem}
    \end{compactenum}
\end{compactenum}

\paragraph{Сослагательное наклонение.}

\begin{compactenum}
    \item Связка~"--- 5~позиций: \begin{inparaitem}
        \item часть речи,
        \item наклонение,
        \item лицо,
        \item число,
        \item роль (\textit{св});
    \end{inparaitem}
    \item Причастие~"--- 5~позиций: \begin{inparaitem}
        \item часть речи,
        \item наклонение,
        \item род,
        \item число,
        \item роль (\textit{пр}).
    \end{inparaitem}
\end{compactenum}

\paragraph{Повелительное наклонение.} 5~позиций: \begin{inparaitem}
    \item часть речи,
    \item наклонение,
    \item лицо,
    \item число,
    \item класс.
\end{inparaitem}

\begin{table}[t]
    \small
    \begin{tabularx}{\textwidth}{Xp{1.5cm}p{1.5cm}p{1.5cm}p{1.5cm}p{1.5cm}p{1.5cm}}
        \toprule
        \textsc{явисr}              & гл/в   & изъяв & аор гл  & 3   & ед &     \\ \midrule
        \textsc{имаши}              & гл     & изъяв & буд 1   & 2   & ед & св  \\
        \textsc{терпѣти}            & гл     & изъяв & буд 1   &     &    & инф \\ \midrule
        \textsc{сотворилъ}          & гл     & сосл  & м       & ед  & пр &     \\
        \textsc{бы}                 & гл     & сосл  & 3       & ед  & св &     \\ \midrule
        \textsc{бл(с)ви}            & гл     & повел & 2       & ед  & 4  &     \\ \midrule
        \textsc{радdrсr}            & прич/в & jo    & наст    & им  & ед & м   \\ \midrule
        \textsc{глати\#}            & инф    &       &         &     &    &     \\ \midrule
        \textsc{мdчитъ}             & суп    &       &         &     &    &     \\ \bottomrule
        \caption{Примеры разметки глагольных форм в корпусе СКАТ}
        \label{tab:examples}
    \end{tabularx}
\end{table}

\section{Морфосинтаксические оппозиции глагольных форм}
\label{sec:oppositions}

По признаку расчленённости выражения грамматических значений формы церковнославянского глагола делятся на синтетические и аналитические. Последние включают в себя следующие:

\begin{compactenum}
    \item будущее~I: образуется сочетанием инфинитива с одним из следующих глаголов в наст.-буд.~вр.: \begin{compactitem}
        \item \textsc{имѣти},
        \item \textsc{начати},
        \item \textsc{хотѣти};
    \end{compactitem}
    \item будущее~II: образуется сочетанием элового причастия со связкой \textsc{быти} в простом будущем;
    \item перфект: образуется сочетанием элового причастия со связкой \textsc{быти} в наст.-буд.~вр.;
    \item плюсквамперфект: образуется сочетанием элового причастия со связкой \textsc{быти} в любом другом прошедшем времени: \begin{compactenum}
        \item имперфект,
        \item аорист от имперфектной основы \textsc{"~бѣ"~},
        \item перфект;
    \end{compactenum}
    \item сослагательное наклонение: выражается сочетанием элового причастия со связкой \textsc{быти} в аористе.
\end{compactenum}

Морфологически связки в текстах СКАТ размечены так же, как полнозначные глагольные формы. Однако поскольку они являются десемантизированными и сами по себе не составляют парадигму соответствующих лексем, их лемматизацию мы считаем необходимым ограничить приписыванием их неких обобщённых псевдолемм, свидетельствующих об их особом статусе: \begin{inparablank}
    \item \textsc{aux"~ft1},
    \item \textsc{aux"~ft2},
    \item \textsc{aux"~prf},
    \item \textsc{aux"~pqp} и
    \item \textsc{aux"~sbj} соответственно.
\end{inparablank} Их лемматизация наравне с омонимичными полнозначными формами привела бы к неоправданному искажению статистических данных.

Для синтетических форм в свою очередь релевантным различительным критерием является тип формообразующей основы (ФОС). У подавляющего большинства глаголов их две: основа наст.~вр.\ и основа инфинитива. Вследствие относительной регулярности глагольных флексий и формообразующих суффиксов лемматизация во многом сводится к проблеме приведения первой ко второй.

\section{Лемматизация форм от основы настоящего времени}
\label{sec:lem_pres}

От основы наст.~вр.\ образуются следующие формы: \begin{inparaenum}[(1)]
    \item наст.-буд.~вр.,
    \item повелительное наклонение,
    \item причастия наст.~вр.
\end{inparaenum}

\subsection{Настоящее-будущее время}

Словоизменительная классификация церковнославянских глаголов, принятая в славистике и отражённая в разметке СКАТ, выделяет 5~глагольных классов. Различительным критерием является тематический суффикс, которым оканчивается ФОС формы наст.-буд.~вр.\ \autocite[148--150]{shulezhkova:2015}.

\begin{table}[p]
    \small
    \begin{tabularx}{\textwidth}{P{1.5cm}P{2cm}X}
        \toprule
        \thead{Класс} & \thead{Тема} & \thead{Примеры} \\ \midrule\midrule
        1 & \textit{*\u{e}}~// \textit{*\u{o}} &\textsc{зовеши}~"--- \textsc{звати},  \textsc{идеши}~"--- \textsc{ити}, \textsc{можеши}~"--- \textsc{мощи} \\ \midrule
        2 & \textit{*n\u{e}}~// \textit{*n\u{o}} & \textsc{сохнеши}~"--- \textsc{сохнути}, \textsc{станеши}~"--- \textsc{стати} \\ \midrule
        3 & \textit{*j\u{e}}~// \textit{*j\u{o}} & \textsc{копаеши}~"--- \textsc{копати}, \textsc{ищеши}~"--- \textsc{искати}, \textsc{цѣлуеши}~"--- \textsc{цѣловати} \\ \midrule
        4 & \textit{*\={\i}} & \textsc{водиши}~"--- \textsc{водити}, \textsc{гориши}~"--- \textsc{горѣти}, \textsc{съпиши}~"--- \textsc{съпати} \\ \midrule
        5 & "--- & \textsc{еси}~"--- \textsc{быти}, \textsc{даси}~"--- \textsc{дати} \\ \bottomrule
        \caption{Классы церковнославянского глагола}
        \label{tab:classes}
    \end{tabularx}
\end{table}

Как видно уже из табл.~\ref{tab:classes}, члены одного и того же класса по особенностям словоизменения могут значительно отличаться между собой, поскольку в основу классификации положен лишь косвенно релевантный для \textit{церковнославянского} словоизменения признак~"--- \textit{праславянская} тема. Как следствие, она не является достаточно точной для учёта всех интересующих нас процессов. Поэтому далее мы также будем обращаться к словоизменительной классификации глаголов для современного русского языка, принятой в \autocite[646--664]{russ_gram} (АГ), которая является более дробной и позволяет достигнуть большей точности при анализе. Мы делаем допущение, что системы церковнославянского и современного русского спряжения достаточно схожи, чтобы такое сближение было правомочно.

Сводная сопоставительная таблица двух классификаций приведена в табл.~\ref{tab:comparison}.\footnote{%
    Здесь и далее по тексту работы для глагольных классов принята следующая нотация: \begin{inparaenum}[(1)]
        \item классы, традиционно выделяемые в палеославистике для древнерусского, старо- и церковнославянского языков, обозначаются арабскими цифрами;
        \item для классов по АГ используются римские цифры, по необходимости дополненные номером подкласса (арабскими цифрами) и группы (кириллицей), разделёнными косой чертой.
    \end{inparaenum}
}

\begin{table}[p]
    \small
    \begin{tabularx}{\textwidth}{P{1.5cm}P{1.25cm}P{2cm}X}
        \toprule
        \thead{Класс \\ (ц.-сл.)} & \thead{Класс \\ (рус.)} & \thead{Подкласс \\ (рус.)} & \thead{Примеры \\ (рус.)} \\ \midrule\midrule

        & & 1/в & \textit{попрать}~"--- \textit{попрут}, \textit{рвать}~"--- \textit{рвут} \\ \cmidrule{3-4}
        & V & 1/г & \textit{брать}~"--- \textit{берут}, \textit{звать}~"--- \textit{зовут} \\ \cmidrule{3-4}
        & & 3 & \textit{реветь}~"--- \textit{ревут} \\ \cmidrule{2-4}
        1 & VI & & \textit{беречь}~"--- \textit{берегут}, \textit{грести}~"--- \textit{гребут}, \textit{умереть}~"--- \textit{умрут} \\ \cmidrule{2-4}
        & \multirow{2.5}{*}{VII} & 1 & \textit{блюcти}~"--- \textit{блюдут}, \textit{цвеcти}~"--- \textit{цветут} \\ \cmidrule{3-4}
        & & 2 & \textit{жить}~"--- \textit{живут}, \textit{слыть}~"--- \textit{слывут} \\ \cmidrule{2-4}
        & IX & & \textit{взять}~"--- \textit{возьмут}, \textit{начать}~"--- \textit{начнут} \\ \midrule

        & III & & \textit{двинуть}~"--- \textit{двинут}, \textit{рухнуть}~"--- \textit{рухнут} \\ \cmidrule{2-4}
        2 & IV & & \textit{разверзнуть}~"--- \textit{разверзнут}, \textit{вянуть}~"--- \textit{вянут} \\ \cmidrule{2-4}
        & VII & 3 & \textit{деть}~"--- \textit{денут}, \textit{стать}~"--- \textit{станут} \\ \midrule

        \multirow{9}{*}{3} & I & & \textit{вожделеть}~"--- \textit{вожделеют}, \textit{уповать}~"--- \textit{уповают} \\ \cmidrule{2-4}
        & II & & \textit{даровать}~"--- \textit{даруют}, \textit{помиловать}~"--- \textit{помилуют} \\ \cmidrule{2-4}
        & & 1/а & \textit{кликать}~"--- \textit{кличут}, \textit{прятать}~"--- \textit{прячут} \\ \cmidrule{3-4}
        & V & 1/б & \textit{веять}~"--- \textit{веют}, \textit{надеяться}~"--- \textit{надеются} \\ \cmidrule{3-4}
        & & 2 & \textit{бороться}~"--- \textit{борются}, \textit{пороть}~"--- \textit{порют} \\ \cmidrule{2-4}
        & VIII & & \textit{давать}~"--- \textit{дают}, \textit{познавать}~"--- \textit{познают} \\ \midrule

        4 & X & & \textit{любить}~"--- \textit{любят}, \textit{видеть}~"--- \textit{видят}, \textit{бояться}~"--- \textit{боятся} \\ \bottomrule
        \caption{Сопоставление классификаций ц.-сл.\ и рус.\ глаголов}
        \label{tab:comparison}
    \end{tabularx}
\end{table}

\subsubsection{Первый класс}
\label{subsec:cls_1}

По типам соотношения основ инфинитива и наст.~вр.\ церковнославянский 1~кл.\ весьма разнороден.

\paragraph{Основы инфинитива на согласную.} Наибольшие проблемы при лемматизации создают глаголы с основой инфинитива на согласную, поскольку в этом случае на стыке ФОС и формообразующего суффикса происходят различные морфонологические процессы, восходящие ещё к праславянскому времени и затемняющие общую морфемную структуру.

Соответствующие глаголы современного русского языка составляют VI~кл., в котором основы наст.\ и прош.~вр.\ совпадают, в инфинитиве же могут как совпадать, так и различаться. Большинство глаголов составляют 1~подкл., где инфинитивы обнаруживают упрощение праславянских сочетаний \textit{*kti}, \textit{*gti}: \textsc{пекu}~"---\textsc{пещи}, \textsc{могu}~"--- \textsc{мощи}. Со 2~подкл.\ в историческом плане ситуация более сложная. Так, в гр.~VI/2/а объединены глаголы с инфинитивами на \textit{-сти} (\textit{-сть}) и \textit{-зти} (\textit{-зть}). С точки зрения лемматизации проблему здесь представляют лишь три лексемы, где на стыке основы и инфинитивного суффикса произошла диссимиляция \textit{*bt} $\rightarrow$ \textit{*st}: \textsc{грести}, \textsc{погрести}, \textsc{скрести}; у прочих глаголов данной группы конечные свистящие этимологические: \textsc{везти}, \textsc{пасти}. С этой точки зрения включение их в одну группу не вполне оправдано.

В гр.~VI/2/б входят глаголы на \textit{-еть} с чередованием корневой \textit{е} с нулём, однако в их церковнославянских аналогах чередования нет, а особенности при лемматизации ограничиваются темой \textsc{ѣ}: \textit{умру}~"--- \textit{умереть}, но \textsc{умрu}~"--- \textsc{умрѣти}. Что касается гр.~VI/2/в, то её, согласно АГ, составляют префиксальные дериваты единственного связанного корня \textit{-шибить}, где конечная гласная в инфинитиве чередуется с нулём в финитных формах: \textit{сшибу}~"--- \textit{сшибить}. Однако фонетически закономерным инфинитивом от древнерусских примеров, приводимых, например, в \avolcite{3}[867]{sreznevsky} (\textsc{сшибесr}, \textsc{съшибошасr}), должна являться форма \textsc{съшитися}, а не указанная там \textsc{съшибитися}. Лемму с закономерной диерезой губного \textsc{б} фиксирует, в частности, \avolcite{29}[124]{green_dict},~"--- иллюстрируя её, правда, на ровно том же материале; однако правомерность её выделения подкрепляется в соседней статье на невозвратный глагол \textsc{сшити}, где примеров значительно больше. Поэтому как корректный мы рассматриваем инфинитив на \textsc{-шити}, в том числе для прочих производных.

К гр.~VI/2/а тесно примыкает подкл.~VII/1, члены которого проявляют в своей словоизменительной парадигме сразу два процесса упрощения праславянских консонантных групп: с одной стороны, в инфинитиве имеет место диссимиляция сочетаний \textit{*tt}, \textit{*dt} на стыке основы и формообразующего суффикса; с другой стороны, основы эловых причастий ввиду диерезы переднеязычных в составе праславянских сочетаний \textit{*tl}, \textit{*dl} оканчиваются на гласную: \textsc{обрести}~"--- \textsc{wбрѣлъ}, \textsc{пасти}~"--- \textsc{палъ}. Этим, в частности, и объясняется тот факт, что АГ помещает такие глаголы в VII~кл., ведь роль личных форм прош.~вр.\ в современном русском языке играют именно исторические эловые причастия; однако для нас данная историческая деталь мало релевантна, поэтому мы считаем необходимым объединить их с глаголами VI~кл.

Рассмотренные классы являются непродуктивными и весьма ограниченными в объёме, проблема несовпадения основ прош.~вр.\ и инфинитива входящих в них глаголов может быть решена словарным методом~"--- путём задания соответствующих списков и процедур дифференцированной обработки их элементов.\footnote{%
    Все словари, упомянутые здесь и далее, были составлены с опорой на АГ и дополнительно выверены по \autocite{srezn_index}. Они представлены в коде разработанного модуля файлом \href{https://github.com/vintagentleman/scat-v2/blob/master/src/utils/verbs.py}{\texttt{src/utils/verbs.py}}.
}

\paragraph{Основы инфинитива на гласную.} Были выделены чередования, присущие следующим классам по АГ:

\begin{compactenum}
    \item гр.~V/1/г (чередование с редуцированным гласным в корне): \textsc{береши}~"--- \textsc{бьрати}, \textsc{зовеши}~"--- \textsc{зъвати};
    \item подкл.~VII/2 (чередование \textit{*\={u}}~// \textit{*\u{o}v}, а также прибавление древнего суффикса \textsc{"~в"~}): \textsc{словеши}~"--- \textsc{слыти}, \textsc{живеши}~"--- \textsc{жити};
    \item IX~кл. (чередование праславянского \textit{*\k{e}}~// \textit{*\u{i}m}): \textsc{приимеши}~"--- \textsc{прияти}, \textsc{возьмеши}~"--- \textsc{възяти}.
\end{compactenum}

Также особой обработки требуют \begin{inparablank}
    \item префиксальные дериваты \textsc{быти} (\textsc{пребудеши}~"--- \textsc{пребыти})
    \item и глагол \textsc{идеши}~"--- \textsc{ити} (без неэтимологического \textsc{д}).
\end{inparablank} Прочие словоформы считаются принадлежащими гр.~V/1/в, за исключением форм с основой на \textsc{"~рев"~}~"--- они принадлежат подкл.~V/3 и оканчиваются на \textsc{ѣ}, а не \textsc{а}. В обоих случаях чередования отсутствуют.

\subsubsection{Второй класс}

Глаголы 2~кл.\ проблем при лемматизации практически не обнаруживают: их основа инфинитива фиксирована и всегда оканчивается суффиксом \textsc{"~ну"~}. Единственную группу исключений, где он отсутствует, составляет подкл.~VII/3; на рассматриваемом материале из него встретился только глагол \textsc{стати} и его производные (\textsc{престати}, \textsc{wстатися}).

\subsubsection{Третий класс}
\label{subsec:cls_3}

\paragraph{Основы продуктивных классов.} Глаголы продуктивных I и II~кл.\ трудностей при лемматизации не представляют: в I~кл.\ фонетическое различие между основами наст.~вр. и инфинитива не отражается на письме; соотношение \textsc{"~у"~}~//\textsc{"~ова"~} у глаголов II~кл.\ имеет регулярный характер и задаётся правилом.

\paragraph{Сочетания с \textit{*j}.} Типы соотношения основ в рамках оставшихся классов по АГ, соответствующих церковнославянскому 3-му, в меньшей степени поддаются регуляризации. Наиболее обширной среди них является гр.~V/1/а, в которую входит около сотни (современных русских) глаголов \autocite[651]{russ_gram}. Все члены данной группы обнаруживают позиционное смягчение, возникшее как результат праславянских сочетаний с \textit{*j}: это чередования "<одиночный губной согласный">~// "<губной согласный $+$ \textsc{л}"> (\textsc{сыплеши}~"--- \textsc{сыпати}, \textsc{приемлеши}~"--- \textsc{прияти} через посредство основы \textsc{"~прием"~}), а также различных согласных и их сочетаний с шипящими (в~т.",ч.\ со сложным согласным \textsc{жд}) \autocite[96--100]{shulezhkova:2015}. Однако если первые регулярны и предсказуемы, то соответствие между шипящими и согласными в положении не перед \textit{*j}, которые в них переходят, является много"=многозначным; таким образом, при йотации внешние различия между согласными нейтрализуются, и автоматически восстановить согласный по йотированной форме глагола оказывается невозможным.

Обозначенная проблема затрагивает всю парадигму глаголов 3~кл., а также актуальна для 4~кл.\ в форме 1~л.\ ед.~ч. (табл.~\ref{tab:jot}).

\begin{table}[t]
    \small
    \begin{tabularx}{\textwidth}{P{1.5cm}P{1cm}XX}
        \toprule
        \multirow{2.5}{*}{\thead{X $+$ \textit{*j}}} & \multirow{2.5}{*}{\thead{X}} & \multicolumn{2}{c}{\thead{Примеры}} \\
        & & \thead{3~кл.} & \thead{4~кл.} \\ \midrule\midrule

        \multirow{5}{*}{\textsc{ж}} & \textsc{г} & \textsc{движеши}~"--- \textsc{двигати} & \\ \cmidrule{2-4}
        & \textsc{д} & \textsc{гложеши}~"--- \textsc{глодати} & \textsc{распdжu}~"--- \textsc{распdдити} \\ \cmidrule{2-4}
        & \textsc{ж} & & \textsc{низложu}~"--- \textsc{низложити} \\ \cmidrule{2-4}
        & \textsc{з} & \textsc{рѣжеши}~"--- \textsc{рѣзати} & \textsc{сражu}~"--- \textsc{сразити} \\ \midrule

        & \textsc{к} & \textsc{алчеши}~"--- \textsc{алкати} & \\ \cmidrule{2-4}
        \textsc{ч} & \textsc{т} & \textsc{мечеши}~"--- \textsc{метати} & \textsc{uхвачю}~"--- \textsc{uхватити} \\ \cmidrule{2-4}
        & \textsc{ч} & & \textsc{w(т)лучю}~"--- \textsc{w(т)лучити} \\ \midrule

        & \textsc{х} & \textsc{машеши}~"--- \textsc{махати} & \\ \cmidrule{2-4}
        \textsc{ш} & \textsc{с} & \textsc{пишеши}~"--- \textsc{писати} & \textsc{вкdшu}~"--- \textsc{вкdсити} \\ \cmidrule{2-4}
        & \textsc{ш} & & \textsc{совершu}~"--- \textsc{совершити} \\ \midrule

        & \textsc{т} & & \textsc{посещю}~"--- \textsc{посетити} \\ \cmidrule{2-4}
        \textsc{щ} & \textsc{ск} & \textsc{ищu}~"--- \textsc{искати} & \\ \cmidrule{2-4}
        & \textsc{ст} & \textsc{блещеши}~"--- \textsc{блистати} & \textsc{причащю}~"--- \textsc{причастити} \\ \midrule

        \multirow{2.5}{*}{\textsc{жд}} & \textsc{д} & \textsc{страждu}~"--- \textsc{страдати} & \textsc{побѣждu}~"--- \textsc{побѣдити}  \\ \cmidrule{2-4}
        & \textsc{зд} & & \textsc{пригвождu}~"--- \textsc{пригвоздити} \\ \bottomrule
        \caption{Омонимичные сочетания с \textit{*j}}
        \label{tab:jot}
    \end{tabularx}
\end{table}

К счастью, хотя перечни глагольных основ с данными чередованиями являются достаточно обширными, но закрыты и поддаются словарному заданию по аналогии с основами на согласный (\ref{subsec:cls_1}).  Отметим, что словари были созданы только для основ с действительными чередованиями~"--- основы с исконными конечными шипящими, встречающиеся в 4~кл.\ (\textsc{"~низлож"~}, \textsc{"~соверш"~}, \textsc{"~w(т)луч"~}), при ненахождении в соответствующем словаре по остаточному принципу признаются нечередующимися. В рамках 3~кл.\ гласным, следующим за чередующимся согласным, всегда является \textsc{а}, за исключением глагола \textsc{хотѣти}.

\paragraph{Прочие случаи.} Глаголы гр.~V/1/б, V/2 и VIII~кл.\ также входят в ограниченные множества малой мощности и задаются словарным путём; их основы всегда оканчиваются гласной сами по себе.

\subsubsection{Четвёртый класс}
\label{subsec:cls_4}

Проблемы с лемматизацией форм 4~кл., помимо рассмотренных выше сочетаний с йотом, также связаны с вариативностью тематического суффикса, оканчивающего ФОС: это мог быть \textsc{"~и"~}, сохранявшийся в основе инфинитива (\textsc{ходиши}~"--- \textsc{ходити}), \textsc{"~ѣ"~} (\textsc{сѣдиши}~"--- \textsc{cѣдѣти}) или \textsc{"~а"~} (\textsc{спиши}~"--- \textsc{спати}).

При рассмотрении современного русского X~кл., в целом соответствующего церковнославянскому 4-му и имеющего аналогичные подклассы, в АГ утверждается, что "<к подкл.~1 [с основой на \textit{и}.~"--- К.",С.] относятся глаголы, мотивированные именами, [\ldots] и большая группа немотивированных глаголов">, в то время как глаголы на \textit{е} и на \textit{а} фактически задаются перечислением \autocite[657--659]{russ_gram}. Часть глаголов на \textit{а} также задаётся правилом (основа оканчивается на шипящую или |j|), однако данное правило с точки зрения исторической грамматики легко объяснимо фонологическим переходом из \textsc{ѣ} \autocite[54]{shulezhkova:2015} и потому должно быть признано регулярным и не знающим исключений.

Из этого очевидно, что глаголы на \textsc{"~и"~} составляют ядро класса, а прочие~"--- периферию, которую возможно задать словарным путём.

\subsubsection{Пятый класс}

Непродуктивный 5~кл.\ составляют всего четыре лексемы: \textsc{быти}, \textsc{дати}, \textsc{ясти}, \textsc{вѣдати}~"--- а также их дериваты (всех, кроме \textsc{быти}: ср.\ \textsc{предамь}, но \textsc{пребуду}) и глагол \textsc{имати} c атематической формой 1~л.\ ед.~ч. \textsc{имамь}. Закрытый характер множества глаголов 5~кл.\ позволяет решить проблему их лемматизации простым заданием соответствующего списка.

\subsection{Повелительное наклонение}

За исключением граммемы времени формы императива размечаются полностью аналогично формам наст.-буд.~вр.\ и по соотношению основ практически ничем от них не отличаются. Следовательно, их лемматизация осуществима использованием уже существующей процедуры. Нерегулярные формы императива части глаголов 5~кл.\ (\textsc{даждь}, \textsc{вѣждь}, \textsc{rждь}) известны и перечислимы.

Единственное значимое систематическое различие заключается в том, что суффиксы повелительного наклонения 1--3~кл.\ \textsc{"~ѣ"~} и \textsc{"~и"~}, как гласные дифтонгического происхождения, вызывают перед собой вторую палатализацию заднеязычных: \textsc{рещи}~"--- \textsc{рцы}, \textsc{помощи}~"--- \textsc{помози} \autocite[167]{shulezhkova:2015}. Ранее для имён было сочтено необходимым внести информацию о наличии палатализации на конце основы непосредственно в разметку, поскольку при её отсутствии отличить сибилянт, появившийся в результате второй палатализации, от этимологического в общем случае невозможно \autocite[37--38]{sipunin:2018}. Однако с глаголами ситуация иная: множество глагольных основ на заднеязычные не открыто, как обстоит дело у имён, а ограничено подкл.~VI/1, который уже задан для обработки соответствующих словоформ наст.-буд.~вр.\ (\ref{subsec:cls_1}). Таким образом, вопрос их лемматизации в императиве решается аналогично.

\subsection{Причастия настоящего времени}
\label{subsec:part_pres}

Задача лемматизации причастий наст.~вр.\ затруднена тем, что обозначение класса в их разметке отсутствует. Фактически единственным показателем классовой принадлежности лексемы оказывается формообразующий суффикс, на основе которого и приходится строить всю процедуру обработки.

\paragraph{Причастия действительного залога.} Суффиксы действительных причастий~"--- \textsc{"~ущ"~}, \textsc{"~ющ"~} и \textsc{"~ащ/ящ"~} во всей парадигме, кроме им.~п.\ ед.~ч.\ м.\ и ср.~р.: показателем этой позиции являются суффиксы \textsc{"~ы"~}, \textsc{"~я"~} или его вариант \textsc{"~а"~}, отражающий второе южнославянское влияние. Выделены следующие правила соотнесения суффиксов с классами:

\begin{compactenum}
    \item \textsc{"~ущ"~}: см.\ дерево решений на рис.~\ref{fig:decision_tree};
    \item \textsc{"~ющ"~}~"--- всегда 3~кл.;
    \item \textsc{"~ащ/ящ"~}~"--- всегда 4~кл.;
    \item \textsc{"~ы"~}: 2~кл., если основа на \textsc{н}, иначе 1~кл.
\end{compactenum}

\begin{figure}
    \small
    \centering
    \begin{forest}
        [основа на \textsc{н}?
        [2~кл.]
        [на шипящий \\ или сонант?
        [3~кл.]
        [1~кл.]
        ]
        ]
    \end{forest}
    \caption{Правила соотнесения словоформ на \textsc{"~ущ"~} с классами}
    \label{fig:decision_tree}
\end{figure}

Ситуация с суффиксом \textsc{"~я/а"~} более затруднительна: его присутствие в равной степени присуще глаголам 3 и 4~кл., поэтому выбрать между ними автоматически оказывается невозможным. Сопоставление результатов лемматизации всех словоформ, не обработанных правилами выше, как безусловно принадлежащих только 3 или 4~кл.\ показало, что "<меньшего зла"> в данной ситуации не существует: классы распределены по лексемам с различными результирующими леммами (одна из которых всегда оказывается некорректной) практически поровну, поэтому доля ошибок в любом случае оказывается велика.

За неимением универсального алгоритмического решения было сочтено необходимым прибегнуть к словарному: был составлен список основ на \textsc{"~и"~}, встретившихся на материале рассмотренных житий в составе форм причастий наст.~вр. Вкупе с аналогичными списками основ на \textsc{"~ѣ"~} и на \textsc{"~а"~} они позволяют корректно идентифицировать глаголы 4~кл., оставшиеся же причислять к 3-му.\footnote{%
    Предпочтение 4~кл.\ было отдано по той причине, что он представляет собой аналогичные проблемы при лемматизации форм имперфекта (\ref{subsec:imperfect}), где роль списка основ на \textsc{"~и"~} также оказывается решающей.
} Разумеется, составленный список не претендует на полноту и при привлечении материалов новых житий подлежит дополнению.

Также дополнительного учёта требуют действительные причастия от атематических основ (\textsc{сыи}~"--- \textsc{быти}), а также основ подкл.~VII/2 (\textsc{живыи}~"--- \textsc{жити}).

\paragraph{Причастия страдательного залога.} С суффиксами страдательных причастий ситуация проще, поскольку их соотношение с классами однозначно: \begin{inparablank}
    \item \textsc{"~ом"~}~"--- 1~кл.,
    \item \textsc{"~ем"~}~"--- 3~кл.,
    \item \textsc{"~им"~}~"--- 4~кл.
\end{inparablank} От глаголов 2~кл.\ страдательные причастия практически невозможны \autocite[178]{shulezhkova:2015} и на материале проанализированных текстов не встретились.

\section{Лемматизация форм от основы инфинитива}
\label{sec:lem_past}

Основа инфинитива в церковнославянском языке является формообразующей для форм, обладающих семантикой предшествования: \begin{inparaenum}[(1)]
    \item аорист,
    \item имперфект,
    \item причастия прош.~вр.,
    \item эловые причастия.
\end{inparaenum}

У подавляющего большинства глаголов основы инфинитива и перечисленных грамматических форм совпадают, что делает задачу их лемматизации тривиальной. Тем не менее, существует несколько групп глагольных лексем, у которых в этом отношении имеются расхождения; они проиллюстрированы в табл.~\ref{tab:past}.

\begin{table}[t]
    \small
    \begin{tabularx}{0.75\textwidth}{P{1.5cm}X}
        \toprule
        \thead{АГ} & \thead{Примеры} \\ \midrule\midrule
        IV & \textsc{навыкоша}~"--- \textsc{навыкнути}, \textsc{погрrзоша}~"--- \textsc{погрrзнути} \\ \midrule
        VI/1 & \textsc{возмогоша}~"--- \textsc{возмощи}, \textsc{постригоша}~"--- \textsc{пострищи} \\ \midrule
        VI/2/а & \textsc{wскребоша}~"--- \textsc{wскрести}, \textsc{погребоша}~"--- \textsc{погрести} \\ \midrule
        VI/2/б & \textsc{прострошасr}~"--- \textsc{простретисr}, \textsc{uмроша}~"--- \textsc{uмрети} \\ \midrule
        VII/1 & \textsc{wбрѣтоша}~"--- \textsc{wбрѣсти}, \textsc{падоша}~"--- \textsc{пасти} \\ \bottomrule
        \caption{Несовпадение основ прош.~вр.\ и инфинитива}
        \label{tab:past}
    \end{tabularx}
\end{table}

Нетрудно заметить, что все перечисленные классы, кроме IV, уже были рассмотрены выше при обсуждении 1~кл.\ в наст.-буд.~вр.\ (см.~\ref{subsec:cls_1}); следовательно, вопрос их лемматизации решается применением уже созданной процедуры. Прочие глаголы по остаточному принципу причисляются к IV~кл.\ и обрабатываются соответствующе~"--- прибавлением беглого суффикса \textsc{"~ну"~}.

Помимо "<глобальных"> замен также имеют место специфические преобразования, актуальные только для отдельных временных форм, а также причастий.

\subsection{Аорист}

Из форм простого аориста (в частности, из наиболее частотных форм 3~л.\ ед.~ч.)\ устраняются последствия первой палатализации (\textsc{рече}, \textsc{движе}), а из нового сигматического~"--- тематический гласный \textsc{"~о"~} (\textsc{внидохъ}, \textsc{падоша}). Среди частных случаев следует отметить немногочисленные формы употребительных глаголов от основы наст.~вр.\ (\textsc{даде}, \textsc{wживе}), а также формы древнего сигматического аориста с удлинением \textit{*\u{e}}~// \textit{*\={e}} (\textsc{рѣхъ}).

\subsection{Имперфект}
\label{subsec:imperfect}

С лемматизацией форм имперфекта связан ряд трудностей, специфичных только для него.

Прежде всего следует отметить, что уже в старославянских рукописях XI~в.\ обычны и наиболее употребительны формы имперфекта с упрощением вокалических групп на стыке основы и формообразующего суффикса, т.",е.\ стяжённые: \textsc{даяше} (из более раннего \textsc{даяаше}), \textsc{идѣше} (из \textsc{идѣаше}), \textsc{можаше} (из \textsc{можааше}) и~т.",д. Церковнославянским языком были закономерно усвоены именно они \autocite[202--203]{khaburgaev:1986}, однако немногочисленные архаичные формы без стяжения встречаются и на рассмотренном материале. Следовательно, на первом шаге восстановления основы инфинитива требуется удаление нестяжённых вокалических сочетаний.

Далее применяются правила учёта релевантных основ"=исключений 5~кл.\ и подкл.~VII/2 (\textsc{"~бя"~}, \textsc{"~жив"~}), а также сочетаний c йотом (см.\ \ref{subsec:cls_3}). Про основы, после этого оставшиеся непреобразованными и не причисленными ни к одной из "<проблемных"> групп, наверняка известно лишь то, что они оканчиваются на гласную; определение же того, какой именно гласный является конечным в каждом конкретном случае, требует дополнительного анализа.

Судьба стыкового гласного сводится к одной из следующих операций:

\begin{compactenum}
    \item \label{itm:imp_del} удаление: \textsc{биrше}~"--- \textsc{бити}, \textsc{пиrше}~"--- \textsc{пити};
    \item \label{itm:imp_save} сохранение: \textsc{восиrше}~"--- \textsc{восияти}, \textsc{чаrше}~"--- \textsc{чаrти};
    \item \label{itm:imp_repl} замена: \textsc{хотrше}~"--- \textsc{хотѣти}, \textsc{творrше}~"--- \textsc{творити}.
\end{compactenum}

Из рассмотрения лексем, составляющих группы на каждый из перечисленных случаев, можно сделать вывод, что \ref{itm:imp_repl}"~й~случай затрагивает только основы 4~кл.\ Для их обработки были привлечены справочники основ на \textsc{"~ѣ"~} и \textsc{"~а"~} (\ref{subsec:cls_4}), а список встретившихся основ на \textsc{"~и"~} (\ref{subsec:part_pres})~"--- дополнен основами релевантных форм имперфекта (его итоговый объём составил 30~основ). \ref{itm:imp_del}"~й случай затрагивает ограниченное множество глаголов, которые выделены АГ в подкл.~I/5 \autocite[647]{russ_gram} и также заданы отдельно. Наконец, прочие основы по остаточному принципу подлежат причислению ко \ref{itm:imp_save}"~му случаю.

\subsection{Причастия}

\paragraph{Причастия прошедшего времени.} Основные сложности в ходе определения основы инфинитива по причастной основе прош.~вр.\ возникают при анализе глаголов 4~кл.~"--- с праславянской темой \textit{*\={\i}}. Эта последняя в положении перед гласными суффиксов причастий (\textsc{"~ьш"~} у действительных, \textsc{"~ен"~} у страдательных) подвержена переходу в \textit{*j} \autocite[177, 179]{shulezhkova:2015}, исторические сочетания с которым входят в те же чередования, что были рассмотрены выше в~\ref{subsec:cls_3}: регулярные с меной йота на \textsc{л} (\textsc{возлюбль}~// \textsc{возлюбленъ}~"--- \textsc{возлюбити}, \textsc{wставль}~// \textsc{wставленъ}~"--- \textsc{wставити}) и специфические с меной на шипящие. Процедура обработки и тех, и других здесь подлежит переиспользованию.

Среди прочих аспектов несовпадения причастных основ с инфинитивными можно выделить следующие: \begin{inparaenum}[(1)]
    \item чередование праславянского \textit{*\k{e}} с сочетанием "<чистый гласный $+$ носовой согласный"> (у действительных причастий): \textsc{вземшаго}~"--- \textsc{взrти}, \textsc{распеншаго}~"--- \textsc{распrти};
    \item чередование \textit{*\={u}}~// \textit{*\u{o}v} (у страдательных): \textsc{вдохновенъ}~"--- \textsc{вдохнdти}, \textsc{забъвенъ}~"--- \textsc{забыти};
    \item супплетивизм: \textsc{шедше}~"--- \textsc{ити}.
\end{inparaenum}

\paragraph{Эловые причастия.} Среди эловых причастий особого упоминания заслуживает супплетивная форма от того же глагола~"--- \textsc{шелъ}.

Специфической особенностью при образовании эловых причастий обладают глаголы современного русского VI~кл.: суффикс \textsc{"~л"~} в формах мужского рода может сохраняться (как предписывает церковнославянская норма), но изредка может и выпадать (под влиянием русской): \textsc{реклъ}~// \textsc{рекъ}. Это обстоятельство также учитывается.

\paragraph{Отрицательные префиксы.} Согласно принципам сегментации текстов СКАТ, "<отрицательная частица \textit{не} пишется со следующим словом слитно или раздельно по правилам современной орфографии"> (\textsc{неизреченныи}, \textsc{неw(т)ступенъ}).\footnote{%
    \url{http://project.phil.spbu.ru/scat/page.php?page=txtprinciples} (дата обр.\ 01.06.2020).
} Для корректного приведения причастий к глаголам её следует отсекать. Исключения составляют случаи, когда деривационный префикс~"--- \textsc{недо-}, а не \textsc{не-} (\textsc{недостати}, \textsc{недомыслити}), либо когда соответствующий глагол без \textsc{не-} не употребляется (\textsc{ненавидѣти}, \textsc{негодовати}). Список последних был составлен по \autocite[119--120]{orth:1994}.

\section{Программная реализация компонента модуля}
\label{sec:module}

Компонент модуля, реализованный на языке Python, оформлен как "<конвертер"> для размеченных текстов (файл \foreignlanguage{english}{\texttt{converter.py}}) в различные форматы. Формат передаётся модулю при запуске как строковый аргумент:

\begin{compactenum}
    \item \texttt{tsv}: табличный формат, аналогичный формату разметки, но с добавочным столбцом под леммы. Предназначен для диагностики и отладки;
    \item \texttt{xml}: TXM"=совместимое XML"=представление текстов СКАТ, описанное в \autocite[54--64]{sipunin:2018};
    \item \texttt{pkl}: сериализованное хранилище под морфологические разборы (см.\ \ref{sec:precedent}).
\end{compactenum}

Собственно процедура лемматизации в ходе работы модуля сводится к последовательному применению над словоформами следующих преобразований: \begin{inparaenum}[(1)]
    \item нормализация,
    \item стемминг,
    \item восстановление леммы.
\end{inparaenum}

\subsection{Нормализация}
\label{subsec:norm}

Церковнославянская орфография в~XV--XVII~вв.\ не была кодифицирована, вследствие чего написание многих слов могло существенно варьироваться~"--- не только в разных текстах, созданных в одно время, но даже в пределах одного текста. Например, все элементы следующего ряда: \begin{inparaitem}[]
    \item \textsc{блаженаго},
    \item \textsc{блаженна(г)},
    \item \textsc{бл(а)женнаго},
    \item \textsc{бла(ж)еннаго},
    \item \textsc{бла(ж)нна(г)},
    \item \textsc{бла(ж)ннаго},
    \item \textsc{блжена(г)\#},
    \item \textsc{блженаго\#},
    \item \textsc{блжен(н)аго\#},
    \item \textsc{блаженна(г)\#}~"---
\end{inparaitem} представляют собой варианты записи одной словоформы. Для того чтобы излишняя неоднозначность не "<просочилась"> на высшие уровни лингвистического анализа, требуется процедура их сведения к графико"=орфографическим инвариантам.

В настоящей работе она реализована при помощи модуля Е.",Г.~Уфлянд, изначально предназначенного для уменьшения объёма сводного словоуказателя к печатным изданиям житий. Модуль интегрирован в конвертер в несколько модифицированном виде по сравнению с изначально описанным в \autocite{uflyand:2008}: внесён ряд технологических улучшений, пополнены перечни правил для замены орфографических вариантов различных типов \autocite[41--43]{sipunin:2018}.

\subsection{Стемминг}

На втором этапе обработки нормализованные словоформы подвергаются процедуре отсечения словоизменительных флексий. Отметим, что здесь она названа "<стеммингом"> исходя в большей степени из функциональной, нежели из содержательной близости с одноимённой процедурой в компьютерной морфологии. Классический стемминг, восходящий к работам \autocites{lovins:1968}{porter:1980}, является бессловарно"=правиловым и направлен на выявление у парадигматически связанных словоформ общих неизменяемых частей, которые могут и не соответствовать никаким реальным лингвистическим единицам (ср.\ пример выделения у словоформы \textit{день} псевдоосновы \textit{д-} и псевдофлексии \textit{-ень} \autocite[20--21]{boch_mitr:2016}).

В нашей ситуации осмыслен обратный, семасиологический подход: грамматические данные, известные из разметки, позволяют по возможности точно отделить флексию с соответствующим грамматическим значением от ФОС, избегая при этом проблем излишнего и недостаточного стемминга (англ.\ \textit{\foreignlanguage{english}{over-}} и \textit{\foreignlanguage{english}{understemming}}), типичных для чисто ономасиологического подхода. Таким образом, возвращаясь к примеру выше, применение процедуры "<стемминга"> в понимании, принятом в рамках данной работы, над словоформой \textsc{день} возвращает флексию \textsc{"~ь} и основу \textsc{ден"~}. Отождествление вариантов ФОС между собой (ср.\ \textsc{ден-} и \textsc{дн-})~"--- задача уже следующего этапа алгоритма.

Инструментом для программного отождествления грамматических значений с флексиями послужили парадигмы, в которых кортежам из граммем, записанным в формате разметки СКАТ, сопоставлены регулярные выражения для флексий. Парадигмы составлены с опорой на учебно"=научную литературу по старославянскому языку \autocites{khaburgaev:1986}{shulezhkova:2015} и учитывают графическую вариативность, не устранимую на этапе нормализации.

Примеры парадигм приведены на рис.~\ref{fig:paradigms}. Для причастий, склоняющихся по именному либо местоименному типу, были использованы уже разработанные парадигмы из \autocite[45--46]{sipunin:2018}.

\begin{figure}
    \centering
    \begin{subfigure}[t]{0.475\linewidth}
        \begin{Verbatim}[fontsize=\small, frame=single, framesep=5mm, gobble=8]
        {
          ("1", "ед"): "Х?[ЪЬ]",
          ("2", "ед"): "[+Е]",
          ("3", "ед"): "[+Е]",
          ("1", "дв"): "Х?ОВ[+Е]",
          ("2", "дв"): "С?Т[+АЕ]",
          ("3", "дв"): "С?Т[+АЕ]",
          ("1", "мн"): "Х?ОМ[ЪЬ]",
          ("2", "мн"): "С?ТЕ",
          ("3", "мн"): "Ш?[АЯ]",
        }
        \end{Verbatim}
        \caption{Парадигма сигматического аориста}
    \end{subfigure}
    \hfill
    \begin{subfigure}[t]{0.475\linewidth}
        \begin{Verbatim}[fontsize=\small, frame=single, framesep=5mm, gobble=8]
        {
          ("м", "ед"): "Л?[ЪЬ]",
          ("м", "дв"): "Л?А",
          ("м", "мн"): "Л?И",
          ("ж", "ед"): "Л?А",
          ("ж", "дв"): "Л?[+И]",
          ("ж", "мн"): "Л?[ЫИ]",
          ("ср", "ед"): "Л?О",
          ("ср", "дв"): "Л?[+И]",
          ("ср", "мн"): "Л?[АИ]",
        }
        \end{Verbatim}
        \caption{Парадигма эловых причастий}
    \end{subfigure}
    \caption{Примеры парадигм}
    \label{fig:paradigms}
\end{figure}

В случае \begin{inparaenum}[(1)]
    \item вхождения кортежа из необходимых граммем словоформы во множество ключей парадигмы и
    \item строкового совпадения соотнесённого с ним значения и конца анализируемой словоформы
\end{inparaenum} последний отсекается, а оставшееся подстрока признаётся основой. При невыполнении любого из этих условий результат лемматизации считается отрицательным (возвращается \texttt{None}).

\subsection{Восстановление леммы}

Последний этап процедуры лемматизации заключается в отсечении формообразующих суффиксов, практической реализации всех преобразований, описанных выше в разделах~\ref{sec:lem_pres}--\ref{sec:lem_past}, и прибавлении инфинитивного суффикса \textsc{-ти} (за исключением основ на согласный, у которых этот последний в большинстве случаев нестандартен). Возвратный постфикс \textsc{-ся} (при его наличии) отсекается в начале анализа и восстанавливается в конце.

\section{Оценка результатов}

\todo[inline]{TODO}

\section*{Выводы}
\addcontentsline{toc}{section}{Выводы}
