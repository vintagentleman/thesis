\chapter{Прецедентная разметка текстов СКАТ}

В настоящей главе пойдёт речь о втором компоненте разработанного грамматического модуля, направленном на частичную автоматизацию морфологической разметки текстов СКАТ.

\section{Опыт древнерусского подкорпуса НКРЯ}

Идея о том, что при определённом уровне накопленного материала дальнейшая лингвистическая разметка может осуществляться не с нуля, не нова. "<Базы данных текстовых прецедентов с приписанными вручную морфологическими пометами"> перечисляются в \autocite[47]{baranov:2015} первыми среди средств автоматизации разметки, распространённых в корпусной палеославистике.

Примером исторического корпуса русского языка, в котором данная идея успешно получила своё воплощение, может служить древнерусский подкорпус Национального корпуса русского языка (НКРЯ). Корпус составляют книжные тексты древнерусских рукописей XI--XIV~вв.\ суммарным объёмом порядка 500~тыс.\ словоупотреблений \autocite[101--102]{mishina_pichkhadze:2015}. Архаический характер их грамматики, огромная вариативность в орфографии и другие связанные проблемы исключают автоматизацию разметки, например, путём создания грамматических словарей (в отличие от более "<современных"> церковнославянского и старорусского подкорпусов, где данный вопрос либо решён \autocite[250--253]{polyakov:2014}, либо находится в процессе решения \autocite{lyashevskaya:2016}), поэтому их разметка ещё с середины 2000-х~гг.\ осуществляется вручную силами экспертов"=славистов. Очевидно, что затраты на столь скрупулёзный труд остаются крайне высокими.

В \autocite{archangel_mishina_pichkhadze:2014} описана среда Morphy, разработанная в Институте русского языка для грамматической разметки древних славянских текстов и используемая в древнерусском подкорпусе НКРЯ. Процесс аннотирования может осуществляться как вручную, так и полуавтоматически; в последнем случае "<программа использует информацию из уже размеченных текстов, т.",е.\ использует прецедентные разборы, а исследователь проверяет и редактирует предложенный вариант разбора. Если предложенных разборов несколько, а в данном контексте правильным является только один, исследователь убирает ненужные варианты разбора, возникающие из-за омонимии словоформ"> \autocite[29]{archangel_mishina_pichkhadze:2014}. В обоих случаях также привлекаются сведения из сводного словаря лемм: при вводе экспертом леммы, которая в нём присутствует, "<словарные грамматические признаки">, т.",е.\ граммемы классификационных категорий подставляются автоматически \autocite[28]{archangel_mishina_pichkhadze:2014}.

\section{Реализация прецедентной разметки для СКАТ}
\label{sec:precedent}

Реализованный компонент грамматического модуля СКАТ решает схожие задачи, но с поправкой на более привычный для коллектива СКАТ табличный формат представления разметки и ориентацией на студентов как конечных её исполнителей. С программной точки зрения компонент в свою очередь состоит из двух дочерних компонентов.

Первый интегрирован в конвертер для размеченных текстов как отдельный режим его работы (см. \ref{sec:module}). В данном режиме обрабатываемые словоформы нормализуются и сериализуются в хранилище данных типа "<ключ~"--- значение">. В качестве ключа выступают они сами, в качестве значений~"--- массивы из зафиксированных в разметке кортежей из морфологических разборов.

Второй подкомпонент (\foreignlanguage{english}{\texttt{annotator.py}}) принимает на вход ещё не размеченный текст и сегментирует его; далее каждая словоформа нормализуется и ищется в созданном ранее хранилище. Её присутствие позволяет использовать ассоциированные с ней кортежи граммем для прецедентной разметки. При разметке принимается во внимание грамматическая омонимия: все кортежи сверяются на предмет несовпадений~"--- между собой, если мощность их множества превышает единицу, а также с перечнем множеств всех кортежей, план выражения которых омонимичен. Перечень был составлен вручную с опорой на \autocites{khaburgaev:1986}{shulezhkova:2015} и приведён в приложении~\ref{app:homonyms}.

Результат записывается в таблицу формата \texttt{.xlsx}. В порядке организации разметки как работы в рамках филологической практики таблица сегментирована на листы, на каждый из которых попадает ограниченное множество словоформ (предполагается, что количество листов соответствует количеству студентов в группе). При запуске компонента оба числа настраиваются; также подлежит конфигурации порядковый номер токена, вплоть до которого содержимое анализируемого жития следует игнорировать,~"--- это вызвано тем, что большинство текстов частично уже размечены и требуют доразметки не с начала.

Все однозначно определяемые граммемы записываются в таблицу как есть; все потенциально омонимичные явно не записываются, но соответствующие ячейки выделяются цветом фона, а при наведении становятся доступны из выпадающего списка (используется механизм проверки данных (\foreignlanguage{english}{data validation}), доступный в Microsoft Excel; рис.~\ref{fig:precedent}). Если прецедент позволяет определить разбор однозначно, такая словоформа не учитывается при подсчёте "<порции"> словоформ, приходящейся на очередного студента, что позволяет увеличить её конечный объём.

% TODO Иллюстрация

\section{Опыт по внедрению прецедентной разметки}

В декабре 2018~г.\ в рамках зимней сессии автором совместно с Алексеевой~Е.",Л.\ был произведён опыт по практическому использованию нового формата разметки: студенты 2~курса образовательной программы бакалавриата "<Прикладная, компьютерная и математическая лингвистика"> СПбГУ в рамках филологической практики выполнили часть морфологической разметки жития Александра Свирского на материале вывода разработанного компонента.

Для тогдашней версии программы ещё не был создан перечень потенциальных грамматических омонимов, а учитывались только фактически имеющиеся в прецедентной базе~"--- вследствие чего, а также ввиду ограниченности объёма размеченной выборки множеству словоформ ошибочно приписывались однозначные разборы. Это сказалось на объёме фрагментов (в среднем он составил 524,7 при выставленном номинальном объёме 350) и потребовало их дополнительной экспертной предобработки.

Тем не менее, в результате проверяющим экспертом было отмечено, что совершённых экспериментальной группой ошибок было значительно меньше, чем обычно демонстрируют студенты второго курса. Можно выдвинуть следующие предположения, обусловившие данное обстоятельство:

\begin{compactitem}
    \item с точки зрения морфологии из синтетического строя церковнославянского языка, при котором один аффикс одновременно способен выражать целый ряд грамматических значений, следует их взаимная обусловленность~"--- становится проще предсказывать недостающие граммемы у не полностью размеченных словоформ исходя из уже имеющихся;
    \item с точки зрения синтаксиса важную роль следует отвести согласованию и координации словоформ: если, например, в сочетании прилагательного с существительным у первого известны все граммемы, а у последнего нет, но они с очевидностью составляют словосочетание, то заполнение недостающих граммем достигается тривиальным копированием.
\end{compactitem}

\section*{Выводы}
\addcontentsline{toc}{section}{Выводы}

Был разработан модуль для аннотирования текстов СКАТ с использованием прецедентов и описан предварительный опыт его внедрения в практику промежуточной аттестации. Формирование фрагментов, подлежащих разметке, с его помощью может вызвать не только количественный, но и качественный прирост мероприятий по её пополнению.
