\chapter{Прецедентная разметка текстов СКАТ}

\section{Опыт древнерусского подкорпуса НКРЯ}

Идея о том, что при определённом уровне накопленного материала дальнейшая лингвистическая разметка может осуществляться не с нуля, не нова. "<Базы данных текстовых прецедентов с приписанными вручную морфологическими пометами"> перечисляются в \autocite[47]{baranov:2015} первыми среди средств автоматизации разметки, распространённых в корпусной палеославистике.

Примером исторического корпуса русского языка, в котором данная идея успешно получила своё воплощение, может служить древнерусский подкорпус Национального корпуса русского языка (НКРЯ). Корпус составляют книжные тексты древнерусских рукописей XI--XIV~вв.\ суммарным объёмом порядка 500~тыс.\ словоупотреблений \autocite[101--102]{mishina_pichkhadze:2015}. Архаический характер их грамматики, огромная вариативность в орфографии и другие связанные проблемы исключают автоматизацию разметки, например, путём создания грамматических словарей (в отличие от более "<современных"> церковнославянского и старорусского подкорпусов, где данный вопрос либо решён \autocite[250--253]{polyakov:2014}, либо находится в процессе решения \autocite{lyashevskaya:2016}), поэтому их разметка ещё с середины 2000-х~гг.\ осуществляется вручную силами экспертов"=славистов. Очевидно, что затраты на столь скрупулёзный труд остаются крайне высокими.

В \autocite{archangel_mishina_pichkhadze:2014} описана среда Morphy, разработанная в Институте русского языка для грамматической разметки древних славянских текстов и используемая в древнерусском подкорпусе НКРЯ. Процесс аннотирования может осуществляться как вручную, так и полуавтоматически; в последнем случае "<программа использует информацию из уже размеченных текстов, т.",е.\ использует прецедентные разборы, а исследователь проверяет и редактирует предложенный вариант разбора. Если предложенных разборов несколько, а в данном контексте правильным является только один, исследователь убирает ненужные варианты разбора, возникающие из-за омонимии словоформ"> \autocite[29]{archangel_mishina_pichkhadze:2014}. В обоих случаях также привлекаются сведения из сводного словаря лемм: при вводе экспертом леммы, которая в нём присутствует, "<словарные грамматические признаки">, т.",е.\ граммемы классификационных категорий подставляются автоматически \autocite[28]{archangel_mishina_pichkhadze:2014}.

\section{Реализация прецедентной разметки для СКАТ}
\label{sec:precedent}

Реализованный компонент грамматического модуля СКАТ решает аналогичную задачу, но с поправкой на более привычный для коллектива СКАТ табличный формат представления разметки и ориентацией на студентов как конечных её исполнителей. С программной точки зрения компонент в свою очередь состоит из двух связанных между собой компонентов.

Первый подкомпонент интегрирован в конвертер для размеченных текстов как отдельный режим его работы~"--- \texttt{pkl} (см. \ref{sec:module}). В данном режиме обрабатываемые словоформы сериализуются в хранилище данных типа "<ключ~"--- значение">, где в качестве ключей выступают их нормализованные формы, в качестве значений~"--- массивы из зафиксированных в разметке кортежей из морфологических разборов (тегсетов).

Тегсеты записываются в хранилище в несколько упрощённом виде. В разметке СКАТ особо фиксируются словоформы, обнаруживающие переходные грамматические явления: так, развитие категории одушевлённости отражается значением падежа \texttt{вин/род}, где тег до косой черты обозначает ожидаемую граммему, а тег после~"--- фактическую; при записи в хранилище сохраняются только последние.

Личные формы глаголов в этом контексте также заслуживают отдельного упоминания. По умолчанию в хранилище для них фиксируются все релевантные граммемы: \begin{inparablank}
    \item наклонение,
    \item время (при наличии),
    \item лицо,
    \item число,
    \item класс (при наличии).
\end{inparablank} Однако для тех глаголов, которые могут выступать в составе аналитических форм, очевидно, такой подход не может быть оправдан, поскольку в полностью неразмеченных текстах не задан тот контекст, который позволил бы отличить синтетические формы от их омонимов в составе аналитических (см.\ \ref{sec:oppositions}); поэтому часть граммем таких глаголов при записи в хранилище опускается. Прежде всего это касается связок: \begin{inparaenum}[(1)]
    \item для глагола \textsc{быти} всегда опускаются наклонение и время;
    \item для \textsc{имѣти}, \textsc{начати}, \textsc{хотѣти}~"--- опускается время, если оно настоящее-будущее либо будущее~I.
\end{inparaenum} У полнозначных глаголов, выраженных эловыми причастиями, сохраняются только лицо и число; однако ввиду того что по сравнению с аористом и имперфектом они представлены в житиях достаточно редко, важной роли это не играет.

Второй дочерний компонент (\foreignlanguage{english}{\texttt{annotator.py}}) принимает на вход неразмеченный текст, уже сегментированный на токены.\footnote{%
    Для сегментации в несколько видоизменённом виде использован фрагмент модуля \texttt{texttoxml.py}, написанный для дипломной работы \autocite{alexeev:2009}.
} Каждая словоформа нормализуется и ищется в созданном ранее хранилище; её присутствие позволяет использовать ассоциированные с ней кортежи граммем для прецедентной разметки. Однако ограничиваться лишь теми тегсетами, которые содержатся в хранилище, нельзя: материал размеченных житий, на основе которых оно конструируется, весьма ограничен, и словоформы в его составе представлены парадигмой, далёкой от полной; привлечение только \textit{реально} существующих тегсетов не учитывает \textit{потенциальной} грамматической омонимии между членами словоизменительных парадигм.

Для решения этой проблемы был составлен перечень множеств тегсетов, план выражения которых омонимичен.\footnote{%
    Перечень имеет формат JSON и представлен файлом \href{https://github.com/vintagentleman/scat-v2/blob/master/src/utils/clusters.json}{\texttt{src/utils/clusters.json}}.
} При анализе словоформ все тегсеты последовательно сверяются с данным перечнем и при нахождении в одном из множеств последнее объединяется с множеством всех тегсетов, потенциально присущих словоформе.

Результат работы компонента~"--- таблица формата \texttt{.xlsx}, по содержимому столбцов соответствующий спецификации разметки СКАТ (\ref{sec:annotation}). При этом все однозначно определяемые граммемы заносятся в таблицу как есть; все те, в отношении которых зафиксирована потенциальная омонимия, явно не записываются, но соответствующие ячейки выделяются цветом фона, а при наведении все варианты грамматических значений становятся доступны из выпадающего списка (используется механизм проверки данных (\foreignlanguage{english}{data validation}), доступный в \foreignlanguage{english}{Microsoft Excel}; рис.~\ref{fig:precedent}).

В порядке организации разметки как учебной деятельности в рамках филологической практики таблица сегментирована на листы, на каждый из которых попадает ограниченное множество словоформ (предполагается, что количество листов соответствует количеству студентов в группе). При запуске компонента оба числа настраиваются; также подлежит конфигурации порядковый номер токена, вплоть до которого содержимое анализируемого жития следует игнорировать,~"--- это вызвано тем, что большинство текстов частично уже размечены и требуют доразметки не с начала.

Если путём прецедентной разметки та или иная словоформа размечается полностью и однозначно, то при подсчёте объёма "<порции"> словоформ, приходящейся на очередного студента, она не учитывается. Это позволяет существенно увеличить объём работы, подлежащей выполнению.

\begin{figure}[t]
    \centering
    \includegraphics[width=\textwidth]{precedent}
    \caption{Фрагмент прецедентной разметки жития Александра Свирского}
    \label{fig:precedent}
\end{figure}

\section{Опыт по внедрению прецедентной разметки}

В рамках промежуточной аттестации в декабре 2018~г.\ автором совместно с Е.",Л.~Алексеевой был проведён опыт по внедрению прецедентной разметки в учебную филологическую практику: студенты 2~курса образовательной программы бакалавриата "<Прикладная, компьютерная и математическая лингвистика"> СПбГУ в рамках филологической практики выполнили часть морфологической разметки жития Александра Свирского на материале вывода разработанного компонента.

Для версии программы, актуальной на тот момент, ещё не был создан перечень потенциальных грамматических омонимов~"--- учитывались только тегсеты, фактически имеющиеся в прецедентной базе. Ввиду ограниченности объёма размеченной выборки это привело к ожидаемому результату: многие словоформы ошибочно размечались как однозначные и игнорировались при расчёте объёма очередного фрагмента. Это в свою очередь сказалось на их объёме (в среднем он составил \num{524,7} при выставленном номинальном объёме \num{350}) и потребовало их дополнительной экспертной предобработки.

Тем не менее, в результате проверяющим экспертом было отмечено, что совершённых экспериментальной группой ошибок было значительно меньше, чем обычно демонстрируют студенты второго курса. Можно выдвинуть следующие предположения, обусловившие данное обстоятельство:

\begin{asparaitem}
    \item с точки зрения морфологии из синтетического строя церковнославянского языка, при котором один аффикс одновременно способен выражать целый ряд грамматических значений, следует их взаимная обусловленность~"--- становится проще предсказывать недостающие граммемы у не полностью размеченных словоформ исходя из уже имеющихся;
    \item с точки зрения синтаксиса важную роль следует отвести согласованию и координации словоформ: если, например, в сочетании прилагательного с существительным у первого известны все граммемы, а у последнего нет, но они с очевидностью составляют словосочетание, то заполнение недостающих граммем достигается тривиальным копированием.
\end{asparaitem}

\section*{Выводы}
\addcontentsline{toc}{section}{Выводы}

В данной главе был представлен обзор примера системы аннотирования исторических текстов с использованием прецедентов. На основании существующих практик был разработан аналогичный модуль для частичной автоматизации морфологической разметки текстов СКАТ, рассчитанный на то, что её окончательная доработка осуществляется студентами младших курсов. Предварительный опыт его внедрения продемонстрировал, что формирование фрагментов, подлежащих разметке, с его помощью может способствовать не только количественному увеличению их объёма, но и её качественному улучшению.
