\afterpage{
    \clearpage\vspace*{\fill}

    \begin{abstract}
        Выпускная квалификационная работа посвящена процессу разработки и программной реализации модуля для расширения грамматического слоя разметки Санкт"=Петербургского корпуса агиографических текстов (СКАТ). В теоретической части исследовано словоизменение церковнославянского глагола в аспекте проблем, которые оно представляет при решении такой задачи автоматического морфологического анализа, как лемматизация. Описан программный компонент, осуществляющий лемматизацию глагольных словоформ в размеченных текстах корпуса СКАТ и интегрированный в уже созданные для него инструменты. В дальнейшей практической части работы освещён компонент для частичной автоматизации грамматической разметки корпуса, опыт и перспективы его практического использования.

        \paragraph{\small Ключевые слова:} глагол, грамматическая разметка, русская агиография, компьютерная морфология, исторический корпус, словоизменение, церковнославянский язык
    \end{abstract}

    \selectlanguage{english}

    \begin{abstract}
        This graduation paper is dedicated to the development and programmatic implementation of a module aimed at extending the grammatical annotation layer of the Saint Petersburg Corpus of Hagiographic Texts (SCAT). The theoretical part explores the issues that Church Slavonic verbal inflection presents when dealing with lemmatization---an important task in computational morphology. It is followed by a description of a software component developed for lemmatizing verbs contained within morphologically annotated vitae comprising the SCAT corpus and integrated into already existing software developed for the latter. The further section of the experimental part presents a component for partially automating the grammatical annotation of SCAT, recounts an example of its application, and discusses the perspectives of its future use.

        \paragraph{\small Keywords:} Church Slavonic, computational morphology, grammatical annotation, historical corpus, inflection, Russian hagiography, verb
    \end{abstract}

    \selectlanguage{russian}
    \vspace*{\fill}\clearpage
}
