\chapter*{Заключение}
\addcontentsline{toc}{chapter}{Заключение}

В выпускной квалификационной работе проведено теоретическое обоснование и программное описание модуля для расширения грамматического слоя разметки Санкт"=Петербургского корпуса агиографических текстов. Практическое воплощение модуля выполнено на языке Python и выложено в открытый доступ.

Для достижения данной цели в работе решены следующие задачи.

\begin{asparaenum}
    \item Изучены основные проблемы церковнославянского глагольного словоизменения, связанные с задачей приведения словоформ к их леммам. Частные словоизменительные трудности рассмотрены с привлечением грамматического описания современного русского языка по \autocite{russ_gram}.
    \item Разработан программный компонент, реализующий практически безошибочную лемматизацию глагольных словоформ на материале работы. Его интеграция в режим экспорта размеченных житий в формат XML сделала их доступными для корпусных исследований морфологии церковнославянского глагола посредством платформы TXM.
    \item Реализован компонент для автоматизации части работ по морфологической разметке житий, ранее проводившейся исключительно вручную, с использованием прецедентной разметки. Её дальнейшее применение имеет потенциал для улучшения и ускорения дальнейших итераций этого кропотливого труда.
\end{asparaenum}
