\chapter*{Введение}
\addcontentsline{toc}{chapter}{Введение}

Санкт"=Петербургский корпус агиографических текстов (СКАТ)~"--- проект кафедры математической лингвистики Санкт"=Петербургского государственного университета (СПбГУ). Начатый ещё в конце 1970-х~гг.\ как фонд фото- и ксерокопий списков житий и похвальных слов, собранных по самым разным рукописным хранилищам (нынешние РНБ, БАН, древлехранилище ИРЛИ РАН им.~В.",И.~Малышева), СКАТ и поныне остаётся уникальным собранием памятников древнерусской агиографической литературы, фактически более нигде не доступных исследователю. Всего в базу данных введено порядка полусотни рукописей суммарным объёмом около полумиллиона словоупотреблений \autocite[5--6]{gerd_alexeeva_azarova_zakharova:2004}.

В более традиционном филологическом ключе деятельность проекта направлена на оцифровку церковнославянских рукописей и их последующее издание вкупе со словоуказателем и развёрнутым критическим аппаратом~"--- привычными инструментами текстологического анализа. Жития публикуются в серии "<Памятники русской агиографической литературы">. С другой стороны, в духе современной корпусной лингвистики для проекта разработаны либо разрабатываются форматы многоуровневой лингвистической разметки~"--- морфологической \autocite{ivanova:2006}, синтаксической \autocites{mikhailova:2012}{alexeeva:2014}{gorlov:2018}, дискурсивной \autocite{rogozina:2015}; все работы ведутся с опорой на общепринятый стандарт представления электронных изданий~"--- Text Encoding Initiative (TEI).

В рамках проекта СКАТ ранее уже было разработано и описано программное обеспечение для дополнения существующих ресурсов с грамматическими данными по ряду текстов СКАТ информацией по леммам, а также их интеграции в XML"=представление \autocite{sipunin:2018}; иначе говоря, для корпуса было произведено сведение материалов для грамматической разметки до разметки в собственном смысле слова. Однако в предыдущей работе рассматривались лишь словоформы именных частей речи, что обусловило неполноту её результатов; настоящая выпускная квалификационная работа призвана восполнить этот пробел.

Обогащение XML"=представления корпуса также слабо повлияло на тот факт, что объём размеченного подкорпуса: 5~житий объёмом порядка 50~тыс.\ словоупотреблений~"--- остаётся несопоставим с теми объёмами, которые по-прежнему подлежат разметке. Следовательно, грамматический модуль к корпусу, нормой для которого является исключительно ручная разметка, должен включать в себя средства по меньшей мере частичной её автоматизации.

\textbf{Цель} настоящей работы~"--- теоретическая разработка и практическая реализация компонентов грамматического модуля корпуса СКАТ.

Для достижения поставленной цели были выделены следующие \textbf{задачи}: \begin{compactenum}
    \item изучение проблем глагольного словоизменения в церковнославянском языке XV--XVII~вв.\ в рамках задачи лемматизации;
    \item программная реализация компонента для лемматизации глаголов, представленных на привлечённом материале;
    \item разработка программного компонента для прецедентной разметки текстов СКАТ, не обладающих грамматической разметкой.
\end{compactenum}

\textbf{Материал} работы~"--- морфологически размеченные тексты трёх житий в составе корпуса СКАТ (Димитрия Прилуцкого, Дионисия Глушицкого, Кирилла Новоезерского) суммарным объёмом около 30~тыс.\ словоупотреблений.

Корпус СКАТ~"--- единственный в своём роде лингвистический ресурс по языку древнерусской агиографии эпохи позднего Средневековья и Нового времени, чем обусловлена \textbf{актуальность} задачи его технологического развития. \textbf{Новизна} теоретической части исследования достигается за счёт систематического сопоставления словоизменительных классификаций глаголов в церковнославянском языке и современном русском, не предпринимавшегося ранее, и решения различных алгоритмических проблем лемматизации с опорой на результаты последнего; практической~"--- благодаря доведению существующей грамматической разметки СКАТ до относительной полноты, а также существенному ускорению темпов её дальнейшего пополнения.

\textbf{Структура} работы включает в себя введение, \total{chpnum}~главы, заключение, список литературы из \total{citnum}~наименований и \total{appnum}~приложение.
