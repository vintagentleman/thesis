\documentclass[xetex, aspectratio=169]{beamer}
\beamertemplatenavigationsymbolsempty
\setbeamertemplate{footline}[frame number]

\usepackage{fontspec}
\usepackage{xecyr}
\usepackage[russian]{babel}

\usepackage{fvextra}
\usepackage[babel]{microtype}
\usepackage[autostyle]{csquotes}
\usepackage[decimalsymbol=comma]{siunitx}

\defaultfontfeatures{Ligatures={TeX, Historic}, Mapping=tex-text}
\setmainfont{Times New Roman}
\setsansfont{Arial}
\setmonofont{Fira Code}
\newfontfamily{\agio}{AgioUnicode}

\usepackage{booktabs}
\usepackage{multirow}
\usepackage{makecell}
\usepackage{tabularx}
\usepackage[defblank]{paralist}
\renewcommand\theadfont{\bfseries}
\newcolumntype{P}[1]{>{\centering\arraybackslash}p{#1}}
\newcolumntype{R}[1]{>{\raggedleft\arraybackslash}p{#1}}

\usepackage{xcolor}
\usepackage{textcomp}
\usepackage{graphicx}
\graphicspath{{../fig/}}

\usepackage{schemata}
\usepackage{fancyvrb}

\usepackage[russian]{datetime2}

\begin{document}

\title{Разработка и реализация грамматического модуля \\ Санкт"=Петербургского корпуса агиографических текстов}
\author{%
    Исполнитель: Сипунин Константин Владимирович \\
    Научный руководитель: к.",ф.",н., доц.\ Алексеева Елена Леонидовна
}
\institute{%
    Уровень образования: магистратура \\
    Направление: 45.04.02 "<Лингвистика"> \\
    Основная образовательная программа: ВМ.5626 \\ "<Прикладная и~экспериментальная лингвистика">
}
\date{\DTMdisplaydate{2020}{06}{16}{5}}

\frame{\titlepage}

\section{Введение}

\frame{\tableofcontents[currentsection]}

\begin{frame}
    \frametitle{Санкт"=Петербургский корпус агиографических текстов}
    \framesubtitle{СКАТ}
    \begin{block}{Факты}
        \begin{itemize}
            \item материал~"--- списки житий древнерусских святых XV--XVII~вв.
            \item объём~"--- порядка 50~текстов, 500~тыс.\ словоупотреблений
            \item морфологически размеченных~"--- 5 (50~тыс.\ словоупотреблений)
            \item тексты в форматах PDF и TEI XML
        \end{itemize}
    \end{block}

    \begin{alertblock}{Проблемы}
        \begin{itemize}
            \item нет полной лемматизации (ранее были покрыты только имена)
            \item морфологическая разметка ручная~"--- следовательно, медленная
        \end{itemize}
    \end{alertblock}
\end{frame}

\begin{frame}{Цель и задачи работы}
    \begin{block}{Цель}
        теоретическая разработка и практическая реализация компонентов грамматического модуля корпуса СКАТ
    \end{block}

    \begin{block}{Задачи}
        \begin{enumerate}
            \item изучение проблем глагольного словоизменения в церковнославянском языке XV--XVII~вв.\ в рамках задачи лемматизации
            \item программная реализация компонента для лемматизации глаголов, представленных на привлечённом материале
            \item разработка программного компонента для прецедентной разметки текстов СКАТ, грамматической разметкой не обладающих
        \end{enumerate}
    \end{block}
\end{frame}

\begin{frame}{Материал работы}
    \centering
    \begin{tabularx}{\textwidth}{XR{1.5cm}R{1.5cm}R{1.5cm}R{1.5cm}}
        \toprule
        \thead{Частеречный тег}
        & \thead{ДП\footnote{
            Житие Димитрия Прилуцкого
        }}
        & \thead{ДГ\footnote{
            Житие Дионисия Глушицкого
        }}
        & \thead{КН\footnote{
            Житие Кирилла Новоезерского
        }}
        & \thead{Сумма} \\ \midrule\midrule
        \texttt{гл} & \num{396} & \num{1228} & \num{1352} & \num{2976} \\ \midrule
        \texttt{гл/в} & \num{55} & \num{132} & \num{217} & \num{404} \\ \midrule
        \texttt{прич} & \num{315} & \num{515} & \num{810} & \num{1640} \\ \midrule
        \texttt{прич/в} & \num{31} & \num{51} & \num{109} & \num{191} \\ \midrule\midrule
        Сумма & \num{797} & \num{1926} & \num{2488} & \num{5211} \\ \midrule
        Общий объём & \num{5039} & \num{9649} & \num{15122} & \num{29810} \\ \bottomrule
    \end{tabularx}
\end{frame}

\section{Лемматизация церковнославянских глаголов}

\frame{\tableofcontents[currentsection]}

\begin{frame}{Термины}
    \begin{columns}
        \column{0.5\textwidth}
        \begin{block}{Нормализация}
            идентификация графического инварианта
        \end{block}
        \column{0.5\textwidth}
        \begin{block}{Лемматизация}
            идентификация лексического инварианта
        \end{block}
    \end{columns}

    \begin{columns}
        \column{0.5\textwidth}
        \begin{block}{}
            \schema[close]{
                \schemabox{
                    \begin{minipage}{0.5\textwidth}
                        \begin{itemize}
                            \item {\agio бл҇ⷶженнаго}
                            \item {\agio бл҃женⷩаго}
                            \item {\agio бла҇ⷤнна҇ⷢ}
                        \end{itemize}
                    \end{minipage}
                }
            }{
                \schemabox{
                    \begin{minipage}{0.5\textwidth}
                        \begin{center}
                            {\agio блаженнаго}
                        \end{center}
                    \end{minipage}
                }
            }
        \end{block}

        \column{0.5\linewidth}
        \begin{block}{}
            \schema[close]{
                \schemabox{
                    \begin{minipage}{0.5\textwidth}
                        \begin{itemize}
                            \item {\agio блаженаго}
                            \item {\agio блаженомꙋ}
                            \item {\agio блаженныѧ}
                        \end{itemize}
                    \end{minipage}
                }
            }{
                \schemabox{
                    \begin{minipage}{0.5\textwidth}
                        \begin{center}
                            {\agio блаженныи}
                        \end{center}
                    \end{minipage}
                }
            }
        \end{block}
    \end{columns}
\end{frame}

\begin{frame}{Спецификация формата разметки}
    \begin{center}
        \includegraphics[width=0.9\linewidth]{spec}
    \end{center}
\end{frame}

\begin{frame}{Лемматизация}
    \begin{columns}
        \column{0.5\linewidth}
        \begin{block}{Алгоритм}
            \begin{enumerate}
                \item нормализация
                \item отсечение флексии и формативов
                \item \alert{преобразование основы}
                \item добавление словарной финали
            \end{enumerate}
        \end{block}

        \column{0.5\linewidth}
        \begin{block}{{\agio ѡбрѣтохомъ}}
            \begin{itemize}
                \item[] {\agio обрѣтохомъ}
                \item[] {\agio обрѣт}
                \item[] {\agio обрѣс}
                \item[] {\agio обрѣсти}
            \end{itemize}
        \end{block}
    \end{columns}
\end{frame}

\begin{frame}[fragile]{Парадигмы}
    \begin{columns}
        \column{0.5\linewidth}
        \begin{block}{Парадигма сигматического аориста}
            \begin{Verbatim}[fontsize=\small]
{
  ("1", "ед"): "Х?[ЪЬ]",
  ("2", "ед"): "[+Е]",
  ("3", "ед"): "[+Е]",
  ("1", "дв"): "Х?ОВ[+Е]",
  ("2", "дв"): "С?Т[+АЕ]",
  ("3", "дв"): "С?Т[+АЕ]",
  ("1", "мн"): "Х?ОМ[ЪЬ]",
  ("2", "мн"): "С?ТЕ",
  ("3", "мн"): "Ш?[АЯ]",
}
            \end{Verbatim}
        \end{block}

        \column{0.5\linewidth}
        \begin{block}{Парадигма эловых причастий}
            \begin{Verbatim}[fontsize=\small]
{
  ("м", "ед"): "Л?[ЪЬ]",
  ("м", "дв"): "Л?А",
  ("м", "мн"): "Л?И",
  ("ж", "ед"): "Л?А",
  ("ж", "дв"): "Л?[+И]",
  ("ж", "мн"): "Л?[ЫИ]",
  ("ср", "ед"): "Л?О",
  ("ср", "дв"): "Л?[+И]",
  ("ср", "мн"): "Л?[АИ]",
}
            \end{Verbatim}
        \end{block}
    \end{columns}
\end{frame}

\begin{frame}{Формы церковнославянского глагола по типу ФОС}
    \begin{columns}
        \column{0.5\linewidth}
        \begin{block}{От основы настоящего времени}
            \begin{enumerate}
                \item настоящее-будущее время
                \item причастия настоящего времени
                \item повелительное наклонение
            \end{enumerate}
        \end{block}

        \column{0.5\linewidth}
        \begin{block}{От основы инфинитива}
            \begin{enumerate}
                \item аорист
                \item имперфект
                \item причастия прошедшего времени
                \item эловые причастия
                \item \alert{инфинитив}
                \item супин
            \end{enumerate}
        \end{block}
    \end{columns}
\end{frame}

\begin{frame}
    \frametitle{Сопоставление классификаций глаголов}
    \framesubtitle{1--2~кл.}
    \centering
    \scriptsize
    \begin{tabularx}{\textwidth}{P{1.5cm}P{1.25cm}P{2cm}X}
        \toprule
        \thead{Класс \\ (ц.-сл.)} & \thead{Класс \\ (рус.)} & \thead{Подкласс \\ (рус.)} & \thead{Примеры \\ (рус.)} \\ \midrule\midrule

        & & 1/в & \textit{попрать}~"--- \textit{попрут}, \textit{рвать}~"--- \textit{рвут} \\ \cmidrule{3-4}
        & V & 1/г & \textit{брать}~"--- \textit{берут}, \textit{звать}~"--- \textit{зовут} \\ \cmidrule{3-4}
        & & 3 & \textit{реветь}~"--- \textit{ревут} \\ \cmidrule{2-4}
        1 & VI & & \textit{беречь}~"--- \textit{берегут}, \textit{грести}~"--- \textit{гребут}, \textit{умереть}~"--- \textit{умрут} \\ \cmidrule{2-4}
        & \multirow{2.5}{*}{VII} & 1 & \textit{блюcти}~"--- \textit{блюдут}, \textit{цвеcти}~"--- \textit{цветут} \\ \cmidrule{3-4}
        & & 2 & \textit{жить}~"--- \textit{живут}, \textit{слыть}~"--- \textit{слывут} \\ \cmidrule{2-4}
        & IX & & \textit{взять}~"--- \textit{возьмут}, \textit{начать}~"--- \textit{начнут} \\ \midrule

        & III & & \textit{двинуть}~"--- \textit{двинут}, \textit{рухнуть}~"--- \textit{рухнут} \\ \cmidrule{2-4}
        2 & IV & & \textit{разверзнуть}~"--- \textit{разверзнут}, \textit{вянуть}~"--- \textit{вянут} \\ \cmidrule{2-4}
        & VII & 3 & \textit{деть}~"--- \textit{денут}, \textit{стать}~"--- \textit{станут} \\ \midrule
    \end{tabularx}
\end{frame}

\begin{frame}
    \frametitle{Сопоставление классификаций глаголов}
    \framesubtitle{3--4~кл.}
    \centering
    \scriptsize
    \begin{tabularx}{\textwidth}{P{1.5cm}P{1.25cm}P{2cm}X}
        \toprule
        \thead{Класс \\ (ц.-сл.)} & \thead{Класс \\ (рус.)} & \thead{Подкласс \\ (рус.)} & \thead{Примеры \\ (рус.)} \\ \midrule\midrule

        \multirow{9}{*}{3} & I & & \textit{вожделеть}~"--- \textit{вожделеют}, \textit{уповать}~"--- \textit{уповают} \\ \cmidrule{2-4}
        & II & & \textit{даровать}~"--- \textit{даруют}, \textit{помиловать}~"--- \textit{помилуют} \\ \cmidrule{2-4}
        & & 1/а & \textit{кликать}~"--- \textit{кличут}, \textit{прятать}~"--- \textit{прячут} \\ \cmidrule{3-4}
        & V & 1/б & \textit{веять}~"--- \textit{веют}, \textit{надеяться}~"--- \textit{надеются} \\ \cmidrule{3-4}
        & & 2 & \textit{бороться}~"--- \textit{борются}, \textit{пороть}~"--- \textit{порют} \\ \cmidrule{2-4}
        & VIII & & \textit{давать}~"--- \textit{дают}, \textit{познавать}~"--- \textit{познают} \\ \midrule

        4 & X & & \textit{любить}~"--- \textit{любят}, \textit{видеть}~"--- \textit{видят}, \textit{бояться}~"--- \textit{боятся} \\ \bottomrule
    \end{tabularx}
\end{frame}

\begin{frame}{Несовпадение основ прошедшего времени и инфинитива}
    \centering
    \begin{tabularx}{0.9\textwidth}{P{1.5cm}X}
        \toprule
        \thead{Класс} & \thead{Примеры} \\ \midrule\midrule
        IV & \textsc{навыкоша}~"--- \textsc{навыкнути}, \textsc{погрrзоша}~"--- \textsc{погрrзнути} \\ \midrule
        VI/1 & \textsc{возмогоша}~"--- \textsc{возмощи}, \textsc{постригоша}~"--- \textsc{пострищи} \\ \midrule
        VI/2/а & \textsc{wскребоша}~"--- \textsc{wскрести}, \textsc{погребоша}~"--- \textsc{погрести} \\ \midrule
        VI/2/б & \textsc{прострошасr}~"--- \textsc{простретисr}, \textsc{uмроша}~"--- \textsc{uмрети} \\ \midrule
        VII/1 & \textsc{wбрѣтоша}~"--- \textsc{wбрѣсти}, \textsc{падоша}~"--- \textsc{пасти} \\ \bottomrule
    \end{tabularx}
\end{frame}

\begin{frame}{Оценка результатов}
    \begin{exampleblock}{Показатели}
        \begin{itemize}
            \item полнота 100\%
            \item точность \char`~99\%
        \end{itemize}
    \end{exampleblock}

    \begin{alertblock}{Ошибки}
        \begin{enumerate}
            \item формы с дистактным расположением возвратного постфикса \begin{itemize}
                \item \textsc{(на тебе сr) надѣемь} $\rightarrow$ \textsc{надѣяти}
                \item \textsc{боrще (бо сr)} $\rightarrow$ \textsc{бояти}
            \end{itemize}
            \item формы с чередованиями, вызванными префиксальной деривацией \begin{itemize}
                \item \textsc{изочтетъ} $\rightarrow$ \textsc{изочести}
                \item \textsc{во(з)ми} $\rightarrow$ \textsc{возяти}
            \end{itemize}
            \item формы с неразрешённой графико"=орфографической неоднозначностью \begin{itemize}
                \item \textsc{ликовствdюще} $\rightarrow$ \textsc{ликовствовати}
                \item \textsc{ликоствuю(т)} $\rightarrow$ \textsc{ликоствовати}
            \end{itemize}
        \end{enumerate}
    \end{alertblock}
\end{frame}

\begin{frame}{Работа с корпусом на платформе TXM}
    \begin{center}
        \includegraphics[width=\textwidth]{concordance}
    \end{center}
\end{frame}

\section{Прецедентная разметка текстов СКАТ}

\frame{\tableofcontents[currentsection]}

\begin{frame}{Прецедентная разметка}
    \begin{center}
        \includegraphics[width=0.6\textwidth]{precedent}
    \end{center}
\end{frame}

\begin{frame}[fragile]{Учёт потенциальной омонимии}
    \begin{columns}
        \column{0.5\linewidth}
    \begin{Verbatim}[fontsize=\scriptsize]
[
  ["a", "род", "ед", "м"],
  ["a", "им", "мн", "м"],
  ["a", "вин", "мн", "м"],
  ["a", "зв", "мн", "м"]
],
[
  ["a", "дат", "ед", "м"],
  ["a", "мест", "ед", "м"],
  ["a", "им", "дв", "м"],
  ["a", "вин", "дв", "м"],
  ["a", "зв", "дв", "м"]
],
    \end{Verbatim}

    \column{0.5\linewidth}
    \begin{Verbatim}[fontsize=\scriptsize]
[
  ["личн", "возвр", "род", "ед"],
  ["личн", "возвр", "вин", "ед"]
],
[
  ["личн", "возвр", "дат", "ед"],
  ["личн", "возвр", "мест", "ед"]
],
    \end{Verbatim}
    \end{columns}
\end{frame}

\begin{frame}{Опыт по внедрению}
    \begin{itemize}
        \item проведён в рамках зимней сессии 2018/19~учебного года
        \item омонимия ещё не учитывалась, что сказалось на длине фрагментов
        \item в среднем \alert{524,7} словоформ на студента при номинальном объёме 350
        \item качество работ возросло~"--- тексты стали проще для понимания \begin{itemize}
            \item взаимная обусловленность граммем в синтетическом языке
            \item согласование и координация граммем в словосочетаниях
        \end{itemize}
    \end{itemize}
\end{frame}

\end{document}
