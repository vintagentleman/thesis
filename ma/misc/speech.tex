\documentclass[12pt]{article}

\usepackage[%
    a4paper, includefoot,
    left=3.5cm, right=1cm,
    top=2cm, bottom=2cm,
    headsep=1cm, footskip=1cm
]{geometry}

\usepackage{fontspec}
\usepackage{xecyr}
\usepackage[russian]{babel}
\defaultfontfeatures{Ligatures = TeX, Mapping = tex-text, HyphenChar = "002D}
\setmainfont{Times New Roman}
\setsansfont{Arial}
\setmonofont{Consolas}

\usepackage[autostyle]{csquotes}
\usepackage[babel]{microtype}
\usepackage[defblank]{paralist}
\usepackage[russian]{datetime2}

\title{Разработка и реализация грамматического модуля Санкт-Петербургского корпуса агиографических текстов}
\author{%
    Исполнитель: Сипунин Константин Владимирович \\
    Научный руководитель: к. ф. н., доц.\ Алексеева Елена Леонидовна
}
\date{\DTMdisplaydate{2020}{06}{16}{5}}

% Аббревиатуры без переносов
\hyphenation{англ}
\hyphenation{подкл}
\hyphenation{СПбГУ}

\begin{document}

\maketitle

\textbf{[слайд 1]} Добрый день, уважаемые члены комиссии. Вашему вниманию представлена выпускная квалификационная работа, посвящённая разработке и реализации грамматического модуля Санкт"=Петербургского корпуса агиографических текстов (СКАТ).

\textbf{[слайд 3]} Проект СКАТ, уже более 40~лет разрабатываемый на кафедре математической лингвистики СПбГУ, посвящён критическому изданию и лингвистической обработке текстов древнерусских житий по спискам XV--XVII~вв.~"--- в~т.",ч.\ разметке на морфологическом и других языковых уровнях. Актуальность задач, поставленных перед работой, обоснована тем, что материалы морфологической разметки корпуса не содержат в себе лемм (ранее задача добавления в корпус информации по леммам была решена лишь частично~"--- только на материале имён), а сама разметка производится исключительно вручную и практически полностью силами студентов кафедры (что в свою очередь обусловливает её сравнительно низкий темп).

\textbf{[слайд 4]} Основная часть работы посвящена задаче лемматизации личных и неличных форм глагола, которая была декомпозирована на подзадачи теоретического изучения проблем церковнославянского глагольного словоизменения, сопряжённых с процедурой лемматизации, и собственно практической реализации соответствующего программного компонента. Задаче частичной автоматизации неразмеченных текстов СКАТ, призванной увеличить темпы её пополнения, отведена второстепенная роль.

\textbf{[слайд 5]} Материалом работы послужили размеченные тексты трёх житий: Димитрия Прилуцкого, Дионисия Глушицкого, Кирилла Новоезерского~"--- суммарным объёмом около 30~тыс.\ словоупотреблений. В презентации приведены показатели числа форм, размеченных как изменяемые формы глаголов, и общего числа словоформ в каждом из упомянутых житий.

\textbf{[слайд 7]} В начале теоретической части работы проводится терминологическое разграничение между \textit{нормализацией} и \textit{лемматизацией} как процедурами идентификации графического и лексического инвариантов словоформы соответственно. На материале исторических языков такое различение необходимо ввиду отсутствия кодифицированной орфографии и~значительной графико"=орфографической вариативности текстов, на них написанных. Морфологический уровень лингвистического анализа выше графического, поэтому лемматизация предполагает нормализацию как этап составной процедуры.

\textbf{[слайд 8]} В корпусе СКАТ принят формат позиционной разметки, формализованный в~спецификации. Разметка имеет табличный формат, где каждой словоформе сопоставлены её морфологические характеристики (в случае аналитических форм глаголов~"--- также часть синтаксических).

\textbf{[слайд 9]} В условиях того, что значения грамматических категорий, которыми наделена каждая словоформа, известны из разметки, процедура лемматизации сводится к следующим четырём этапам: \begin{inparaenum}[(1)]
    \item нормализация,
    \item отсечение флексии и формативов,
    \item преобразование основы,
    \item добавление словарной финали.
\end{inparaenum} Для нормализации в компонент интегрирован модуль Е.",Г.~Уфлянд, изначально предназначенный для уменьшения объёма сводного словоуказателя к~печатным изданиям житий. Определение того, какой финалью оканчивается лемма, в контексте глаголов не представляет трудностей: кроме исключений с основами на согласный, это всегда \textsc{-ти}. Таким образом, основная алгоритмическая нагрузка ложится на второй и третий этапы.

\textbf{[слайд 10]} Отсечение словоизменительных формантов (стемминг) основывается на формальных парадигмах, составленных с опорой на учебно"=научную литературу по старославянскому языку. В них цепочки граммем сопоставлены флексиям, с целью учёта неустранимой вариативности представленным в виде регулярных выражений. В ходе работы алгоритма они сопоставляются с абсолютными концами анализируемых словоформ; при успешном результате сопоставления найденные конечные подстроки отсекаются.

\textbf{[слайд 11]} Преобразование полученных таким образом основ до словарных~"--- наиболее трудоёмкий этап всей процедуры. Формообразующие основы настоящего времени в большинстве случаев имеют значительные отличия от основ инфинитива, а традиционная словоизменительная классификация глаголов, принятая в славистике и отражённая в разметке СКАТ, опирается на признак \textit{праславянской} темы, лишь косвенно релевантный для \textit{церковнославянского} языка ввиду значительных фонетических изменений в диахронии; как следствие, в один класс входят глаголы, по характеру словоизменения весьма различные между собой.

\textbf{[слайды 12--13]} В работе данная классификация сопоставлена с классификацией по "<Русской грамматике"> 1980~г.~"--- более дробной и пригодной для учёта всех словоизменительных особенностей глаголов (делается допущение, что системы церковнославянского и современного русского спряжения достаточно схожи, чтобы такое сближение было правомочно). Описание всех классов, представляющих проблемы, в работе снабжено историко"=фонетическим комментарием; отдельному рассмотрению также подвергнуты сложности, специфические для повелительного наклонения и причастий настоящего времени.

\textbf{[слайд 14]} Лемматизация форм с семантикой прошедшего времени затруднена в меньшей степени, поскольку в большинстве случаев их основа совпадает с основой инфинитива. Все классы исключений составлены с опорой на данные "<Русской грамматике"> и дополнительно выверены по обратному индексу к "<Материалам для словаря древнерусского языка"> И.",И.~Срезневского.

\textbf{[слайд 15]} Экспертная оценка результатов работы программного компонента позволяет судить о стопроцентном покрытии глагольных словоформ на материале работы и практически безошибочной лемматизации. Выделенные группы систематических ошибок связаны с процессами функционирования словоформ на синтаксическом и словообразовательном уровнях, которые в настоящей работе не рассматривались, а также с неполнотой процедуры нормализации. Исходный код на языке \foreignlanguage{english}{Python} и вывод компонента в табличном формате выложены в открытый доступ.

\textbf{[слайд 16]} После лемматизации размеченные тексты также доступны для экспорта в формат XML, совместимый с платформой TXM, что делает их доступными для проведения корпусных исследований морфологии церковнославянского глагола.

\textbf{[слайд 18]} Вторая задача работы~"--- частичная автоматизация разметки новых житийных текстов~"--- решена путём создания на материале трёх привлечённых житий базы прецедентных тегсетов. Компонент, который был реализован под данную задачу, при анализе неразмеченных текстов осуществляет нормализацию входных словоформ и далее обращается к~прецедентной базе: при нахождении в ней нормализованной словоформы все приписанные ей кортежи граммем приписываются и словоформе в новом тексте. Поскольку вывод компонента в конечном счёте предназначен для организации филологической практики студентов, он представляет собой таблицу формата \texttt{.xlsx}, соответствующую спецификации разметки текстов СКАТ. Параметры количества листов в таблице (соответствующего предполагаемому числу студентов в группе) и количества словоформ на каждом листе конфигурируемы.

\textbf{[слайд 19]} Для решения проблемы ограниченности материала житий, из которой, в частности, следует, что привлечение только \textit{реально} представленных в них тегсетов не учитывает \textit{потенциальной} грамматической омонимии между членами словоизменительных парадигм, был дополнительно составлен перечень множеств тегсетов (в формате JSON), план выражения которых омонимичен. При анализе словоформ все тегсеты последовательно сверяются с данным перечнем и при нахождении в одном из множеств последнее объединяется с множеством всех тегсетов, потенциально присущих словоформе.

\textbf{[слайд 20]} Предварительный опыт внедрения разработанного компонента в интродуктивную филологическую практику продемонстрировал, что формирование фрагментов, подлежащих разметке, с его помощью может способствовать не только количественному увеличению их объёма, но и её качественному улучшению.

\end{document}
