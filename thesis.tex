\documentclass[
    specialist,  % Тип документа
    href,        % Использовать пакет hyperref для создания гиперссылок
    subf,        % Использовать пакет subcaption для вложенной нумерации рисунков
    colorlinks   % Цветные гиперссылки
]{disser}

\usepackage[
    a4paper, includefoot,
    left = 2cm, right = 2cm,
    top = 2cm, bottom = 2cm,
    headsep = 1cm, footskip = 1cm
]{geometry}

% Настройка локализации и переносов
\usepackage[X2, T1, T2A]{fontenc}
\usepackage[utf8]{inputenc}
\usepackage[english, russian]{babel}
\usepackage[autostyle]{csquotes}

% Команда для доступа к расширенной кириллице
\newcommand{\old}{\fontencoding{X2}\selectfont}

% Глубина оглавления
\setcounter{tocdepth}{2}

% Нумерация страниц снизу и по центру
\pagestyle{footcenter}
\chapterpagestyle{footcenter}

% Спецсимволы
\usepackage{textcomp}

% Сноски
\usepackage[bottom]{footmisc}
\usepackage{chngcntr}
\counterwithout{footnote}{chapter}

% Списки
\usepackage{paralist}
\setdefaultleftmargin{5ex}{}{}{}{}{}

% Таблицы
\usepackage{booktabs}
\usepackage{tabularx}
\usepackage{makecell}
\renewcommand\theadfont{\bfseries}

% Графика
\usepackage{graphicx}
\usepackage{subcaption}
\graphicspath{{fig/}}

% Код
\usepackage{fancyvrb}

% Библиография
\usepackage{csquotes}
\usepackage[
    backend = biber,       % Движок
    bibencoding = utf8,    % Кодировка файлов с библиографическими БД
    style = gost-numeric,  % Стиль цитирования по ГОСТ
    language = auto,       % Получение языка из babel/polyglossia; ссылки на страницы на языке оригинала
    autolang = other,      % Многоязычная библиография
    clearlang = true,      % Внутренний сброс поля language при совпадении с языком из babel/polyglossia
    sortcites = true,      % Сортировка затекстовых ссылок при цитировании
    % minbibnames = 1,     % Число авторов, отображаемое при сокращении
    maxbibnames = 5        % Максимальное число авторов в препозиции
]{biblatex}

\addbibresource{./bib/scat.bib}
\addbibresource{./bib/stuff.bib}

% Сортировка библиографии
\DeclareSourcemap{
    \maps[datatype = bibtex]{
        \map{
            \step[fieldsource = langid, match = russian, final]
            \step[fieldset = presort, fieldvalue = {a}]
        }
        \map{
            \step[fieldsource = langid, notmatch = russian, final]
            \step[fieldset = presort, fieldvalue = {z}]
        }
    }
}

% Счётчик библиографических источников
\usepackage{totcount}
\newtotcounter{citnum}
\AtEveryBibitem{\stepcounter{citnum}}

\sloppy

\begin{document}

% Титульный лист, аннотация, оглавление
\institution{%
    Санкт-Петербургский государственный университет \\
    Филологический факультет \\
    Кафедра математической лингвистики
}

\apname{к.",ф.",н., доц.\ И.",С.~Николаев}

\title{Выпускная квалификационная работа}
\author{Сипунин Константин Владимирович}
\topic{Автоматическая лемматизация текстов в~корпусе СКАТ на~основе морфологической разметки}

\sa{Е.",Л.~Алексеева}
\sastatus{к.",ф.",н., доц.}

% \rev{И.",В.~Азарова}
% \revstatus{к.",ф.",н., доц.}

\city{Санкт-Петербург}
\date{\number\year}

\afterpage{
    \clearpage\vspace*{\fill}

    \begin{abstract}
        Выпускная квалификационная работа посвящена процессу разработки и программной реализации модуля для расширения грамматического слоя разметки Санкт"=Петербургского корпуса агиографических текстов (СКАТ). В теоретической части исследовано словоизменение церковнославянского глагола в аспекте проблем, которые оно представляет при решении такой задачи автоматического морфологического анализа, как лемматизация. Описан программный компонент, осуществляющий лемматизацию глагольных словоформ в размеченных текстах корпуса СКАТ и интегрированный в уже созданные для него инструменты. В дальнейшей практической части работы освещён компонент для частичной автоматизации грамматической разметки корпуса, опыт и перспективы его практического использования.

        \paragraph{\small Ключевые слова:} глагол, грамматическая разметка, русская агиография, компьютерная морфология, исторический корпус, словоизменение, церковнославянский язык
    \end{abstract}

    \selectlanguage{english}

    \begin{abstract}
        This graduation paper is dedicated to the development and programmatic implementation of a module aimed at extending the grammatical annotation layer of the Saint Petersburg Corpus of Hagiographic Texts (SCAT). The theoretical part explores the issues that Church Slavonic verbal inflection presents when dealing with lemmatization---an important task in computational morphology. It is followed by a description of a software component developed for lemmatizing verbs contained within morphologically annotated vitae comprising the SCAT corpus and integrated into already existing software developed for the latter. The further section of the experimental part presents a component for partially automating the grammatical annotation of SCAT, recounts an example of its application, and discusses the perspectives of its future use.

        \paragraph{\small Keywords:} Church Slavonic, computational morphology, grammatical annotation, historical corpus, inflection, Russian hagiography, verb
    \end{abstract}

    \selectlanguage{russian}
    \vspace*{\fill}\clearpage
}

\tableofcontents

% Текст работы
\intro

Славянская рукописная традиция зародилась уже более тысячи лет назад. От времён, последовавших за просветительской деятельностью преподобных Константина (Кирилла) и Мефодия в середине IX~в., до сегодняшних дней дошли десятки тычяч рукописей, созданных писцами и переписчиками в монастырях Восточной Европы,~"--- как списков Священного Писания, служебников, часословов и прочих богослужебных книг, непосредственно обслуживавших запросы церкви, так и оригинальных произведений, предназначенных для индивидуального чтения: поучений, сказаний, житий святых. Тем не менее, значительная доля данных текстовых массивов по сей день изучена недостаточно и по-прежнему нуждается во всесторонней исследовательской обработке~"--- исторической, этнографической, лингвистической.

Несколько десятилетий назад ситуация начала качественно преображаться в связи с появлением, а впоследствии и массовым распространением компьютеров и цифровых технологий: средства представления рукописей в электронном виде ознаменовали собой принципиально новые возможности их сохранения и изучения вне стен отдельных библиотек и архивных фондов, регулярным доступом к которым обладает далеко не каждый исследователь.

Соответствующие разработки начали появляться уже в конце третьей четверти XX~в.~"--- в том числе на кафедре математической лингвистики Ленинградского государственного университета. С конца 1970-х~гг.\ при участии сотрудников кафедры русского языка ЛГУ, а также ИРЛИ АН СССР и ГПБ им.~М.",Е.~Салтыкова-Щедрина на кафедре начал создаваться фонд фото- и ксерокопий списков древнерусских житий и похвальных слов XV--XVII вв.\ \autocite[512]{averina_alexeeva_gerd:1996}, впоследствии получивший название "<Санкт"=Петербургский корпус агиографических текстов"> (СКАТ). Для представления содержимого фонда в памяти ЭВМ каждую копию рукописного текста было необходимо транслитерировать~"--- перевести в машиночитаемый формат при помощи специальной системы кодирования. Однако в те годы фактически единственным средством ввода символьных цепочек в память компьютера являлись 8-битные кодировки на базе ASCII (\foreignlanguage{english}{American Standard Code for Information Interchange}), очевидно не предназначенные для размещения в диапазоне кодируемых символов знаков устаревших и экзотических систем письменности (в~т.",ч.\ кириллической). Вследствие этого для набора текстов, составляющих фонд, на кафедре была выработана собственная кодировка, в которой для вышедших из употребления символов кириллицы были введены замены (преимущественно буквы латинского алфавита): так, юс большой и юс малый обозначаются соответственно "<\textsc{g}"> и "<\textsc{r}">, кси~"--- "<\textsc{l}"> и~т.",д. Тексты вводимых в память ЭВМ рукописей набираются квалифицированными специалистами"=филологами вручную при помощи специально разработанного шрифта AGIO и затем автоматически переводятся в данную кодировку; при этом в текст вставляются словоразделы (в соответствии с принципами, разработанными проф.\ А.",А.~Алексеевым для издания серии "<Библиотека литературы Древней Руси">), а также маркируются границы составных частей рукописи~"--- строк, колонок и страниц. Всего к настоящему времени в базу данных введено более полусотни рукописей общим объёмом около полумиллиона словоупотреблений \autocite{gerd_alexeeva_azarova_zakharova:2004}.

Сегодня доступ к результатам работы коллектива проекта обеспечивается двояко. С одной стороны, с конца 1990-х~гг.\ издательством Санкт"=Петербургского государственного университета ведётся публикация изданий серии "<Памятники русской агиографической литературы">, в каждом из которых содержится один или несколько подготовленных к печати житийных текстов, набранных упомянутым выше шрифтом AGIO, полный словоуказатель словоформ, а также текстологические статьи об истории публикуемых житий, биографии святых, сведения об обителях. Последний, одиннадцатый выпуск увидел свет в 2012~г.; там же приведён перечень всех предыдущих публикаций серии \autocite[4]{coll:2012}.

С другой стороны, всё более повсеместное распространение онлайн"=технологий в 2000-х~гг.\ дало импульс к тому, чтобы обеспечить доступ к опубликованным материалам через интернет: был создан сайт проекта\footnote{\url{http://project.phil.spbu.ru/scat/} (дата обр.\ \today)}, а корпус получил своё нынешнее наименование. На сегодняшний день около полутора десятков житий доступны для загрузки с сайта в двух форматах: PDF, воспроизводящем их представление в печатных сборниках, и XML, где с помощью системы тегов производится формальное членение рукописей на структурные элементы. XML"=разметка текстов СКАТ соответствует международному стандарту оформления электронных изданий~"--- \foreignlanguage{english}{Text Encoding Initiative} (TEI).

\begin{figure}[t!]
    \centering
    \begin{subfigure}[t]{0.495\textwidth}
        \includegraphics[width=\linewidth]{scat_search}
        \caption{Выдача по запросу \textsc{бц} (режим нестрогого соответствия)}
        \label{fig:scat:1}
    \end{subfigure}
    \hfill
    \begin{subfigure}[t]{0.495\textwidth}
        \includegraphics[width=\linewidth]{scat_output}
        \caption{Контекстное окно вхождения словоформы \textsc{бголюбци\#} (ГП 323/19)}
        \label{fig:scat:2}
    \end{subfigure}
    \caption{Поиск по словоуказателю СКАТ}
\end{figure}

Также на сайте имеется возможность поиска по корпусу~"--- вернее, по той его части, для которой построен сводный словоуказатель. Это центральный компонент лингвистического обеспечения СКАТ, представляющий собой список словарных статей, в каждой из которых указана словоформа в нормализованном виде, абсолютная частота её встречаемости по всем проиндексированным рукописям и адреса вхождений. Адрес состоит из сокращённого наименования рукописи и сочетания порядковых номеров листа (с уточнением стороны~"--- лицевой либо оборотной), колонки и строки, разделённых косой чертой. При нажатии на адрес в поисковой выдаче (рис.~\ref{fig:scat:1}) пользователю предлагается "<нарезка"> из соответствующего PDF"=документа (рис.~\ref{fig:scat:2}), в которую попадает искомое вхождение; отыскивать его приходится самостоятельно~"--- путём отсчитывания от межстраничной либо межколонной границы с номером, указанным в адресе, необходимого числа строк.

Однако современный электронный корпус~"--- в отличие от простой коллекции текстов~"--- должен располагать определённым набором автоматизированных инструментов, применимых в ходе решения конкретных лингвистических задач. В ряде зарубежных работ по языкам с ограниченными ресурсами в последние годы вошло в обиход понятие BLARK~"--- \foreignlanguage{english}{Basic Language Resource Toolkit} (базовый набор лингвистических ресурсов), которое определяется как "<\foreignlanguage{english}{the minimal set of language resources that is necessary to do any precompetitive research and education at all}"> \autocite[11]{krauwer:2003} (минимальный набор лингвистических ресурсов, необходимый для любых базовых исследовательских и образовательных нужд). BLARK может включать в себя как традиционные одно- и двуязычные словари и грамматики, так и специфические ресурсы, вошедшие в лингвистический обиход лишь в последние десятилетия: модули распознавания и синтеза речи, морфосинтаксические анализаторы и~т.",д. Притом отмечается, что этот список не закрытый и может варьироваться от языка к языку: очевидно, для древнеписьменных языков, в число которых входит и церковнославянский, неактуальна задача обработки устной речи, однако вследствие некодифицированного характера орфографии зачастую требуются модули её нормализации.

\textcite[28]{passarotti:2010} предлагает вариант BLARK ("<\foreignlanguage{english}{a BLARK-like set}">) для латинского языка, который, как кажется, в равной степени приложим к другим древнеписьменным языкам. В нём предусмотрены инструменты, направленные на решение следующих основных задач: \begin{inparaenum}[(1)]
    \item предобработка текстовых данных: токенизация и распознавание именованных сущностей;
    \item морфологический анализ: лемматизация и разрешение морфосинтаксической неоднозначности;
    \item синтаксический анализ (поверхностный и глубинный);
    \item разрешение анафоры;
    \item семантический и прагматический анализ.
\end{inparaenum}

\textbf{Цель} настоящей работы заключается в том, чтобы в применении к корпусу СКАТ разработать комплекс инструментов для осуществления одной из подзадач морфологического анализа, специфицируемой в рамках базового набора лингвистических ресурсов,~"--- процедуры лемматизации. \textbf{Задачи}, которые необходимо решить для достижения поставленной цели, таковы:

\begin{compactenum}
    \item ознакомление с системами представления грамматических сведений (в~т.",ч.\ данных по леммам) в существующих восточнославянских исторических корпусах;
    \item изучение теоретических предпосылок алгоритма лемматизации церковнославянского языкового материала с учётом всех релевантных морфологических особенностей и его программная реализация;
    \item организация доступа к лемматизированному подкорпусу СКАТ (и шире~"--- ко всей оцифрованной части корпуса) с использованием общедоступных технологических средств.
\end{compactenum}

\textbf{Объект} основной части исследования~"--- словоизменение в церковнославянском языке XV--XVII вв. \textbf{Предмет} изучения~"--- проблемы формализации феноменов церковнославянского словоизменения в ходе алгоритмизации перехода от словоформ в несловарных парадигматических позициях к словарным (т.",е.\ леммам). \textbf{Материалом} послужили морфологически размеченные тексты трёх агиографических текстов в составе корпуса СКАТ: жития Димитрия Прилуцкого, Дионисия Глушицкого и Кирилла Новоезерского~"--- суммарным объёмом около 30~тыс.\ словоупотреблений.

\textbf{Актуальность} работы обоснована тем, что в рамках СКАТ~"--- единственного в своём роде источника сведений по языку древнерусской агиографии эпохи позднего Средневековья и Нового времени~"--- серьёзные попытки разработки составных частей BLARK в целом и подсистем морфологического анализа в частности фактически не предпринимались.

\textbf{Структура} работы включает в себя введение, \total{chpnum}~главы, заключение, список литературы из \total{citnum}~наименований и \total{appnum}~приложения.

\chapter{Представление грамматической информации в славянских исторических корпусах}

На сегодняшний день славянских диахронических корпусов существует крайне мало. Так, соответствующий
перечень, приведённый на сайте\footnote{\url{http://ruscorpora.ru}} Национального корпуса
русского языка (НКРЯ), состоит из всего четырёх наименований (не включая СКАТ):
\begin{inparaenum}[(1)]
    \item Регенсбургский диахронический корпус русского языка,
    \item Рукописные памятники Древней Руси,
    \item корпус "<Манускрипт"> Удмуртского государственного университета,
    \item корпус русских публицистических текстов второй половины XIX века Петрозаводского государственного университета (ввиду своей специфики он далее рассматриваться не будет);
\end{inparaenum}
помимо этого перечислены два старославянских корпуса: университетов Хельсинки и Южной Калифорнии.
Краткий обзор большинства названных корпусов (включая исторические подкорпуса самого НКРЯ) приведён
в статье \autocite{mitrenina:2014}; наше рассмотрение будет сосредоточено на реализованных в них
принципах и инструментах грамматической разметки и лемматизации. (Обсуждение технологий NLP~"---
в~т.",ч.\ морфологических модулей~"--- в зарубежных диахронических корпусах см.\ в седьмой главе
монографии \textcite[85--101]{passarotti:2010}).

\section{Исторические подкорпуса НКРЯ}


\chapter{Прецедентная разметка текстов СКАТ}

В настоящей главе пойдёт речь о втором компоненте разработанного грамматического модуля, направленном на частичную автоматизацию морфологической разметки текстов СКАТ.

\section{Опыт древнерусского подкорпуса НКРЯ}

Идея о том, что при определённом уровне накопленного материала дальнейшая лингвистическая разметка может осуществляться не с нуля, не нова. "<Базы данных текстовых прецедентов с приписанными вручную морфологическими пометами"> перечисляются в \autocite[47]{baranov:2015} первыми среди средств автоматизации разметки, распространённых в корпусной палеославистике.

Примером исторического корпуса русского языка, в котором данная идея успешно получила своё воплощение, может служить древнерусский подкорпус Национального корпуса русского языка (НКРЯ). Корпус составляют книжные тексты древнерусских рукописей XI--XIV~вв.\ суммарным объёмом порядка 500~тыс.\ словоупотреблений \autocite[101--102]{mishina_pichkhadze:2015}. Архаический характер их грамматики, огромная вариативность в орфографии и другие связанные проблемы исключают автоматизацию разметки, например, путём создания грамматических словарей (в отличие от более "<современных"> церковнославянского и старорусского подкорпусов, где данный вопрос либо решён \autocite[250--253]{polyakov:2014}, либо находится в процессе решения \autocite{lyashevskaya:2016}), поэтому их разметка ещё с середины 2000-х~гг.\ осуществляется вручную силами экспертов"=славистов. Очевидно, что затраты на столь скрупулёзный труд остаются крайне высокими.

В \autocite{archangel_mishina_pichkhadze:2014} описана среда Morphy, разработанная в Институте русского языка для грамматической разметки древних славянских текстов и используемая в древнерусском подкорпусе НКРЯ. Процесс аннотирования может осуществляться как вручную, так и полуавтоматически; в последнем случае "<программа использует информацию из уже размеченных текстов, т.",е.\ использует прецедентные разборы, а исследователь проверяет и редактирует предложенный вариант разбора. Если предложенных разборов несколько, а в данном контексте правильным является только один, исследователь убирает ненужные варианты разбора, возникающие из-за омонимии словоформ"> \autocite[29]{archangel_mishina_pichkhadze:2014}. В обоих случаях также привлекаются сведения из сводного словаря лемм: при вводе экспертом леммы, которая в нём присутствует, "<словарные грамматические признаки">, т.",е.\ граммемы классификационных категорий подставляются автоматически \autocite[28]{archangel_mishina_pichkhadze:2014}.

\section{Реализация прецедентной разметки для СКАТ}
\label{sec:precedent}

Реализованный компонент грамматического модуля СКАТ решает аналогичную задачу, но с поправкой на более привычный для коллектива СКАТ табличный формат представления разметки и ориентацией на студентов как конечных её исполнителей. С программной точки зрения компонент в свою очередь состоит из двух связанных между собой компонентов.

Первый подкомпонент интегрирован в конвертер для размеченных текстов как отдельный режим его работы~"--- \texttt{pkl} (см. \ref{sec:module}). В данном режиме обрабатываемые словоформы сериализуются в хранилище данных типа "<ключ~"--- значение">, где в качестве ключей выступают их нормализованные формы, в качестве значений~"--- массивы из зафиксированных в разметке кортежей из морфологических разборов (тегсетов).

Тегсеты записываются в хранилище в несколько упрощённом виде. В разметке СКАТ особо фиксируются словоформы, обнаруживающие переходные грамматические явления: так, развитие категории одушевлённости отражается значением падежа \texttt{вин/род}, где тег до косой черты обозначает ожидаемую граммему, а тег после~"--- фактическую; при записи в хранилище сохраняются только последние. Также отдельного упоминания заслуживают личные формы глаголов. Очевидно, что морфологически идентичные формы могут выражать разные грамматические значения наклонения и времени; однако в полностью неразмеченных текстах тот контекст, который позволил бы отличить синтетические глагольные формы от их омонимов в составе аналитических, априори не известен. В связи с этим для личных форм глаголов в хранилище фиксируются, помимо части речи, только граммемы лица (или рода в случае эловых причастий) и числа.

Второй дочерний компонент (\foreignlanguage{english}{\texttt{annotator.py}}) принимает на вход неразмеченный текст, уже сегментированный на токены.\footnote{%
    Для сегментации в несколько видоизменённом виде использован фрагмент модуля \texttt{texttoxml.py}, написанный для дипломной работы \autocite{alexeev:2009}.
} Каждая словоформа нормализуется и ищется в созданном ранее хранилище; её присутствие позволяет использовать ассоциированные с ней кортежи граммем для прецедентной разметки. Однако ограничиваться лишь теми тегсетами, которые содержатся в хранилище, нельзя: материал размеченных житий, на основе которых оно конструируется, весьма ограничен, и словоформы в его составе представлены парадигмой, далёкой от полной; привлечение только \textit{реально} существующих тегсетов не учитывает \textit{потенциальной} грамматической омонимии между членами словоизменительных парадигм.

Для решения этой проблемы был составлен перечень множеств тегсетов, план выражения которых омонимичен.\footnote{%
    Перечень в формате JSON приведён в файле \href{https://github.com/vintagentleman/scat-v2/blob/master/src/utils/clusters.json}{\texttt{src/utils/clusters.json}}.
} При анализе словоформ все тегсеты последовательно сверяются с данным перечнем и при нахождении в одном из множеств последнее объединяется с множеством всех тегсетов, потенциально присущих словоформе.

Результат работы компонента~"--- таблица формата \texttt{.xlsx}, по содержимому столбцов практически полностью соответствующий спецификации разметки СКАТ (\ref{sec:annotation}).\footnote{%
    Отступления касаются лишь тех личных форм глаголов, в грамматическом значении которых отсутствует время: в таком случае столбцы с граммемами лица (рода) и числа должны смещаться на единицу влево, однако поскольку время, как было сказано выше, не фиксируется в хранилище, нет возможности узнать о факте его отсутствия в размечаемом тексте. Полуавтоматическая разметка таких словоформ подлежит посткоррекции.
} При этом все однозначно определяемые граммемы заносятся в таблицу как есть; все те, в отношении которых зафиксирована потенциальная омонимия, явно не записываются, но соответствующие ячейки выделяются цветом фона, а при наведении все варианты грамматических значений становятся доступны из выпадающего списка (используется механизм проверки данных (\foreignlanguage{english}{data validation}), доступный в \foreignlanguage{english}{Microsoft Excel}; рис.~\ref{fig:precedent}).

В порядке организации разметки как учебной деятельности в рамках филологической практики таблица сегментирована на листы, на каждый из которых попадает ограниченное множество словоформ (предполагается, что количество листов соответствует количеству студентов в группе). При запуске компонента оба числа настраиваются; также подлежит конфигурации порядковый номер токена, вплоть до которого содержимое анализируемого жития следует игнорировать,~"--- это вызвано тем, что большинство текстов частично уже размечены и требуют доразметки не с начала.

Если путём прецедентной разметки та или иная словоформа размечается полностью и однозначно, то при подсчёте объёма "<порции"> словоформ, приходящейся на очередного студента, она не учитывается. Это позволяет существенно увеличить объём работы, подлежащей выполнению.

\begin{figure}[t]
    \centering
    \includegraphics[width=\textwidth]{precedent} % TODO
    \caption{Прецедентная разметка жития Александра Свирского}
    \label{fig:precedent}
\end{figure}

\section{Опыт по внедрению прецедентной разметки}

В рамках промежуточной аттестации в декабре 2018~г.\ автором совместно с Алексеевой~Е.",Л.\ был проведён опыт по внедрению прецедентной разметки в учебную филологическую практику: студенты 2~курса образовательной программы бакалавриата "<Прикладная, компьютерная и математическая лингвистика"> СПбГУ в рамках филологической практики выполнили часть морфологической разметки жития Александра Свирского на материале вывода разработанного компонента.

Для тогдашней версии программы ещё не был создан перечень потенциальных грамматических омонимов~"--- учитывались только тегсеты, фактически имеющиеся в прецедентной базе. Ввиду ограниченности объёма размеченной выборки это привело к ожидаемому результату: многие словоформы ошибочно размечались как однозначные и игнорировались при расчёте объёма очередного фрагмента. Это в свою очередь сказалось на их объёме (в среднем он составил 524,7 при выставленном номинальном объёме 350) и потребовало их дополнительной экспертной предобработки.

Тем не менее, в результате проверяющим экспертом было отмечено, что совершённых экспериментальной группой ошибок было значительно меньше, чем обычно демонстрируют студенты второго курса. Можно выдвинуть следующие предположения, обусловившие данное обстоятельство:

\begin{asparaitem}
    \item с точки зрения морфологии из синтетического строя церковнославянского языка, при котором один аффикс одновременно способен выражать целый ряд грамматических значений, следует их взаимная обусловленность~"--- становится проще предсказывать недостающие граммемы у не полностью размеченных словоформ исходя из уже имеющихся;
    \item с точки зрения синтаксиса важную роль следует отвести согласованию и координации словоформ: если, например, в сочетании прилагательного с существительным у первого известны все граммемы, а у последнего нет, но они с очевидностью составляют словосочетание, то заполнение недостающих граммем достигается тривиальным копированием.
\end{asparaitem}

\section*{Выводы}
\addcontentsline{toc}{section}{Выводы}

Был разработан модуль для аннотирования текстов СКАТ с использованием прецедентов. Предварительный опыт его внедрения в практику промежуточной аттестации продемонстрировал, что формирование фрагментов, подлежащих разметке, с его помощью способно вызвать не только количественный, но и качественный прирост мероприятий по её дальнейшему пополнению.

\chapter{Портирование корпуса СКАТ на~платформу TXM}

Платформа TXM\footnote{\url{http://textometrie.ens-lyon.fr} (дата обр.\ \today)}~"--- это свободно распространяемое программное обеспечение для работы с текстовыми корпусами, разработанное в лаборатории IHRIM (\foreignlanguage{french}{Institut d'Histoire des Repr\'{e}sentations et des Id\'{e}es dans les Modernit\'{e}s}) Национального центра научных исследований Франции \autocite{heiden:2010}. TXM предоставляет в распоряжение пользователя широкий набор инструментов количественного и качественного анализа текстов: получение конкордансов в формате KWIC и частотных списков лексических единиц на основе любого приписанного им параметра; построение частотных графиков динамики вхождений единиц, удовлетворяющих пользовательскому запросу (для статистических расчётов используется вычислительный движок R); сбор данных о совместной встречаемости, о лексических шаблонах и многое другое. Также платформа приспособлена для обработки текстовой метаинформации, что позволяет пользователю строить подкорпуса (\foreignlanguage{english}{subcorpora}) и разбиения (\foreignlanguage{english}{partitions}) корпусов, введённых в платформу, по различным метатекстовым основаниям. TXM поддерживает множество входных форматов (TXT, ODT/DOC/RTF, XML, различные проприетарные форматы), однако для внутреннего представления содержимого введённых корпусов используется XML"=представление.

По инициативе А.",М.~Лаврентьева, одного из главных разработчиков платформы, на протяжении нескольких лет активно сотрудничавшего с коллективом СКАТ и впервые написавшего программу для автоматической конвертации текстовых файлов житий в формат XML \autocite[21]{alexeeva_lavrentiev_azarova_zakharova:2004}, фрагмент корпуса СКАТ объёмом 12 житийных текстов (включая 2 похвальных слова) был загружен на демонстрационный портал TXM, открытый для пользования в режиме онлайн\footnote{\url{http://portal.textometrie.org/demo/} (дата обр.\ \today)}. Однако сотрудники СКАТ участия в этой работе фактически не принимали, вследствие чего корпус был не вполне качественно адаптирован к реалиям платформы: в частности, сами тексты доступны для чтения лишь в упрощённой графике и содержат ошибки перекодирования (в особенности это касается цифирных обозначений чисел).

Настоящий этап работы нацелен на устранение всех подобных недостатков и максимальное приспособление корпуса СКАТ к комфортному использованию при помощи стационарной версии платформы TXM, а также на внедрение в TXM"=совместимое представление текстов корпуса слоя грамматических данных и лемм.

\section{Режим импортирования XTZ}

Как было отмечено ранее, платформа TXM приспособлена к импорту текстовых корпусов во множестве различных форматов, однако де-факто стандартным и наиболее активно совершенствуемым в позднейших версиях платформы способом загрузки входных текстов в формате XML является режим XTZ~"--- \foreignlanguage{english}{XML TEI Zero} \autocite[76]{txm}.

Помимо универсальных средств обработки импортируемых документов (включая транспонирование различных уровней разметки во внутреннее TXM"=представление, благодаря которому пользователь получает возможность строить подкорпусы и разбиения по любым интересующим его размеченным текстовым структурам, многоаспектное индексирование словоформ и многое другое), режиму XTZ также присуща ориентированность на определённый минимальный ("<нулевой">) набор тегов, наиболее часто используемых при разметке текстовых данных с опорой на рекомендации консорциума TEI, и способность учитывать их семантику при конструировании HTML"=изданий, непосредственно доступных для чтения.

\begin{table}[t]
    \small
    \begin{tabularx}{\linewidth}{Xp{4cm}X}
        \toprule
        \thead{XML} & \thead{HTML} & \thead{Пояснение} \\ \midrule\midrule
        \texttt{<head>} & \texttt{<h2>} & Заголовок \\ \midrule
        \texttt{<p>} & \texttt{<p>} & Абзац \\ \midrule
        \texttt{<hi>} & \texttt{<b>} & Полужирное начертание \\ \midrule
        \texttt{<emph>} & \texttt{<i>} & Курсивное начертание \\ \midrule
        \texttt{<list type='unordered'>} & \texttt{<ul>} & Маркированный список \\ \midrule
        \texttt{<list type='ordered'>} & \texttt{<ol>} & Нумерованный список \\ \midrule
        \texttt{<item>} & \texttt{<li>} & Элемент списка \\ \midrule
        \texttt{<table>} & \texttt{<table>} & Таблица \\ \midrule
        \texttt{<row>} & \texttt{<tr>} & Табличная строка \\ \midrule
        \texttt{<cell>} & \texttt{<td>} & Табличная ячейка \\ \midrule
        \texttt{<graphic>} & \texttt{<img>} & Рисунок \\ \midrule
        \texttt{<ref>} & \texttt{<a>} & Гиперссылка \\ \midrule
        \texttt{<note>} & \texttt{<a>}, \texttt{<span>} & Сноска \\ \midrule
        \texttt{<w>} & \texttt{<span>} & Токен \\ \bottomrule
        \caption{Преобразования тегов при импорте в режиме XTZ (по \autocite[78--80]{txm})}
        \label{tab:edition}
    \end{tabularx}
\end{table}

Так, определяемые TEI маркеры начала новой строки~"--- \texttt{<lb/>} (\foreignlanguage{english}{line beginning})~"--- при генерации HTML преобразуются в теги \texttt{<br/>}, позволяющие форсировать разрыв строки в любом необходимом месте. Кроме того, если они дополнительно снабжены глобальным атрибутом \texttt{@n}, указывающим на порядковый номер соответствующей строки, то напротив строк через определённые интервалы автоматически вставляются их порядковые номера, подобно тому как нумеруются стихи в академических изданиях античной поэзии. Аналогично обрабатывается тег \texttt{<pb/>} (\foreignlanguage{english}{page beginning}); в тех случаях, когда пагинация текстов корпуса на уровне разметки не предусмотрена, TXM фрагментирует их самостоятельно, исходя из максимального числа токенов на каждой странице (этот параметр задаётся пользователем при импорте).

Перечень прочих XML"=тегов и их HTML"=эквивалентов приведён в таблице~\ref{tab:edition}.

\section{Адаптация XML"=представления СКАТ к режиму XTZ}
\label{sec:xml}

\subsection{Проблемы существующей XML"=структуры}

Последним, кто работал над СКАТ в рассматриваемом аспекте, был В.",А.~Алексеев. В рамках своей магистерской диссертации \autocite[41--54]{alexeev:2011} он предпринял ряд серьёзных мер, направленных на модернизацию XML"=представления текстов СКАТ в соответствии с современными стандартами электронного представления текстовых данных.

Нестандартные сущности, теги и атрибуты, ранее использовавшиеся для отображения графем, отсутствующих в современном русском языке, были заменены на символы Unicode~5.1. Данная мера была продиктована как нормативными, так и прагматическими соображениями, поскольку XML"=представление СКАТ образца нулевых было весьма громоздким и неудобочитаемым; так, результат преобразования в XML такой словоформы, как \textsc{ра(д)уасr}, в нём выглядел следующим образом:

\begin{Verbatim}[fontsize=\small, gobble=4, xleftmargin=5ex]
    ра<osl_letter type='overline'>
      д
    </osl_letter>уас&cyr-littleyus;
\end{Verbatim}

Те немногие графемы, которые не были определены в кодовой таблице Unicode~5.1, В.",А.~Алексеев предложил по-прежнему кодировать как сущности~"--- например, \texttt{\&i8-overline;} в случае выносного \textsc{и} восьмеричного. При этом все сущности были определены в отдельном файле определения типа документа (DTD), а также снабжены формальной декларацией (\texttt{<charDecl>}) на уровне описания кодировки TEI"=документа (\texttt{<encodingDesc>}). Этот механизм был впервые включён в рекомендации TEI в версии P5 \autocite[39, 192--201]{tei}, полноценное обновление до которой и строгое следование соответствующим нормам также входило в круг задач диссертационного исследования.

Тем не менее, разработанная В.",А.~Алексеевым версия XML"=представления СКАТ по ряду причин не является TXM"=совместимой. Во-первых, для разметки мельчайших структурных частей рукописи (страниц, колонок и строк) было предложено использовать сразу два синонимичных набора элементов:

\begin{compactenum}
    \item парные теги \texttt{<div2>}, \texttt{<div3>}\footnotemark, \texttt{<div4>}, \texttt{<l>};
    \item пустые теги \texttt{<pb/>}, \texttt{<cb/>}, \texttt{<lb/>}.
\end{compactenum}

\footnotetext{
    Тег \texttt{<div2>} маркирует лист, а \texttt{<div3>}~"--- страницу, т.",е.\ одну из сторон листа (лицевую либо оборотную).
}

Если первый набор нацелен на описание формально"=иерархической организации XML"=документа, то последний скорее предназначен для его семантического структурирования: вместо разбивки на строго непересекающиеся блоки в текст вносятся маркеры (\foreignlanguage{english}{milestones}), попросту указывающие на окончание одной структурной единицы и начало другой. Оба способа одновременно консорциум TEI предписывает задействовать лишь тогда, когда размечаемых структур более одной и они являются соперничающими \autocite[123--124]{tei}, т.",е.\ синонимичными, но не идентичными; в противном случае большей простотой и практичностью, невзирая на меньшую экспрессивность, обладает маркерная аннотация. Если же кодированию подлежит множество разнородных структур, то её использование для разметки таких базовых единиц, как строки, колонки и страницы, тем более предпочтительно. Кроме того, спецификации режима XTZ затрагивают именно пустые теги, а использования их парных аналогов (и шире~"--- всех элементов и атрибутов с целочисленными суффиксами) ввиду особенностей функционирования поисковой машины CQP, напротив, рекомендуется избегать \autocite[78]{txm}.

Во-вторых, XML"=разметка элементарных лексических единиц (токенов) была призвана учесть множество различных вариантов их графического представления. При этом все подобные варианты определялись как потомки базового тега \texttt{<w>}:

\begin{Verbatim}[fontsize=\small, gobble=4, xleftmargin=5ex]
    <w xml:id='CrlNvz.1'>
      <orig>мѣсѧца</orig>
      <reg>М+СЯЦА</reg>
      <src>М+СRЦА</src>
    </w>
\end{Verbatim}

Здесь внутри вложенного тега \texttt{<src>}\footnote{
    Это единственный случай отступления В.",А.~Алексеевым от рекомендаций TEI. Паронимический тег \texttt{<source>} имеет совершенно иную семантику и иное назначение \autocite[356]{tei}.
} первое слово жития Кирилла Новоезерского представлено в оригинальном 8-битном формате, внутри \texttt{<reg>}~"--- в упрощённой графике; наконец, в теге \texttt{<orig>} оно записано с использованием символов Unicode~5.1. В случае ошибочных написаний иерархия получает дальнейшее усложнение: \texttt{<orig>} в качестве потомка приобретает тег \texttt{<choice>}, обозначающий наличие альтернантов\footnote{
    Строго говоря, триада из \texttt{<orig>}, \texttt{<reg>} и \texttt{<src>} также требует обрамления тегом \texttt{<choice>}, однако подобный шаг, очевидно, ознаменовал бы собой ещё большее осложнение XML"=структуры.
}, а внутрь него в свою очередь заносится ошибка (\texttt{<sic>}) и исправление (\texttt{<corr>}). Например:

\begin{Verbatim}[fontsize=\small, gobble=4, xleftmargin=5ex]
    <w xml:id='CrlNvz.90'>
      <orig><choice>
          <sic>человѣчетвѡ</sic>
          <corr>человѣчествѡ</corr>
      </choice></orig>
      <reg>~ЧЕЛОВ+ЧЕТВО &lt;ЧЕЛОВ+ЧЕСТВО&gt;</reg>
      <src>~ЧЕЛОВ+ЧЕТВW &lt;ЧЕЛОВ+ЧЕСТВW&gt;</src>
    </w>
\end{Verbatim}

Между тем режим XTZ не предполагает наличия у ядерных лексических единиц столь развитой иерархической организации. Он ориентирован на обработку тегов \texttt{<w>} в простейшем виде, когда в качестве их содержимого выступает единственный вариант графического представления токена, а все альтернативные вкупе с прочими сопутствующими сведениями записаны в атрибуты \autocite[77]{txm}. Иначе говоря, предполагается, что элементы \texttt{<w>} являются терминальными узлами XML"=структуры и потомков не имеют; если в действительности это не так, то при импорте последние игнорируются, а содержимым родительского тега считается результат конкатенации содержимого всех дочерних.

Таким образом, при подготовке HTML"=издания первый пример из приведённых выше считался бы тождественным следующему (что было бы нежелательно):

\begin{Verbatim}[fontsize=\small, gobble=4, xleftmargin=5ex]
    <w xml:id='CrlNvz.1'>
      мѣсѧцаМ+СЯЦАМ+СRЦА
    </w>
\end{Verbatim}

\subsection{Структурные нововведения}

Предлагаемые нами нововведения в XML"=структуру текстов СКАТ, призванные обеспечить их полную совместимость с режимом импортирования XTZ, обобщены в таблице~\ref{tab:new_xml}.

\begin{table}[t]
    \footnotesize
    \begin{tabularx}{\linewidth}{XX}
        \toprule
        \thead{Старый тег} & \thead{Новый тег} \\ \midrule\midrule
        
        \texttt{<div1 type='part' n='1'></div1>} & \texttt{<ab></ab>} \\ \midrule
        
        \texttt{<div2 type='page' n='1'>} & \\
        \texttt{~~<div3 type='back'></div3>} & \texttt{<pb n='-1'/>} \\
        \texttt{</div2>} & \\ \midrule
        
        \texttt{<div3 type='front'>} & \\
        \texttt{~~<div4 type='col' n='1'></div4>} & \texttt{<pb n='1a'/>} \\
        \texttt{</div3>} & \\ \midrule
        
        \texttt{<l n='1'></l>} & \texttt{<lb n='1'/>} \\ \midrule
        
        \texttt{<w>} & \\
        \texttt{~~<orig>ѿ</orig>} & \\
        \texttt{~~<reg>О(Т)</reg>} & \texttt{<w reg='о(т)' src='W(Т)'>ѿ</w>} \\
        \texttt{~~<src>W(Т)</src>} & \\
        \texttt{</w>} & \\ \midrule
        
        \texttt{<w>} & \\
        \texttt{~~<orig><choice>} & \\
        \texttt{~~~~<sic>мъ</sic>} & \\
        \texttt{~~~~<corr>мѧ</corr>} & \texttt{<w reg='мя' src='\~{}МЪ \&lt;МR\&gt;'>мъ</w>} \\
        \texttt{~~</choice></orig>} & \texttt{<note type='corr'>мѧ</note>} \\
        \texttt{~~<reg>\~{}МЪ \&lt;МЯ\&gt;</reg>} & \\
        \texttt{~~<src>\~{}МЪ \&lt;МR\&gt;</src>} & \\
        \texttt{</w>} & \\ \midrule
        
        \texttt{<c type='punctuation'></c>} & \texttt{<pc></pc>} \\ \bottomrule
        \caption{Предлагаемые замены тегов}
        \label{tab:new_xml}
    \end{tabularx}
\end{table}

С чисто формальной точки зрения замещение структурных подразделений верхнего уровня \texttt{<div1>} анонимными блоками \texttt{<ab>} (\foreignlanguage{english}{anonymous block}) обусловлено обозначенным выше стремлением избавиться от тегов с целочисленными суффиксами; содержательная же подоплёка данного нововведения состоит в том, что так житийные тексты представляются как нерасчленённые,~"--- иначе говоря, делается имплицитное утверждение, что никаких промежуточных смысловых блоков внутри них не выделяется. Однако в будущем такое положение вещей, вероятно, изменится, поскольку разработки формата сюжетной разметки внутри коллектива СКАТ также ведутся \autocite{rogozina:2015}.

Формальную разбивку документов на листы (\texttt{<div2>}) и страницы (\texttt{<div3>}) предлагается полностью заменить смысловой и для маркировки границ между ними пользоваться исключительно тегом \texttt{<pb/>} с обязательным атрибутом \texttt{@n}, обозначающим порядковый номер соответствующей страницы. Номера лицевых и оборотных сторон листа в соответствии с транслитерационными соглашениями СКАТ отличаются между собой по наличию при них специального префикса~"--- дефиса.

Поскольку в режиме XTZ отсутствует поддержка специализированного тега"=разделителя между колонками (\texttt{<cb/>}, \foreignlanguage{english}{column beginning}), последние видится необходимым рассматривать как отдельные страницы и также отграничивать друг от друга при помощи элемента \texttt{<pb/>}. При этом формат атрибута \texttt{@n} получает дополнительное расширение в виде суффикса \texttt{a} для первой колонки или \texttt{b} для второй. Отметим, что рукописи с тремя колонками и более чрезвычайно редки и в корпусе СКАТ не представлены, а рукопись с двумя колонками всего одна (житие Александра Свирского; РНБ, Пог.~874, XVI~в.).

Замена элементов \texttt{<l>} на \texttt{<lb/>} осуществляется по аналогичному принципу; присваиваемые им порядковые номера являются простыми натуральными числами.

Направление преобразования тегов элементарных лексических единиц (\texttt{<w>}) было обосновано выше: в качестве их содержимого отныне выступает единственный вариант графического представления (совместимый с Unicode), а прочие конвертируются в одноимённые атрибуты. Для ошибочных написаний имеют место следующие спецификации: \begin{inparaenum}[(1)]
    \item в атрибут \texttt{@reg} записывается нормализованная форма исправленного варианта~"--- и только его;
    \item внутри тега \texttt{<w>} содержится Unicode"=совместимое представление оригинального написания;
    \item исправление обрамляется типизированным тегом \texttt{<note>}, непосредственно следующим за токеном.
\end{inparaenum} Далее это позволяет конструировать HTML"=издания житийных текстов в первозданном виде, исправления же отображать как сноски.

Наконец, \texttt{<pc>} введён вместо типизированного тега \texttt{<c>} в угоду краткости и соответствию нормам TEI \autocite[575--577]{tei}.

\subsection{Обновление до Unicode~6.1}

Выше было упомянуто, что стараниями В.",А.~Алексеева между собственной кодировкой исторических символов кириллицы, принятой в проекте СКАТ, и стандартом Unicode~5.1 было установлено практически полное взаимно однозначное соответствие. Исключение составляет ряд выносных букв (\textsc{ь}, \textsc{ы}, \textsc{у}, \textsc{u}, \textsc{и}, \textsc{i}, \textsc{w}, а также \textsc{е} широкое и его йотированный аналог), к моменту окончания диссертационного исследования В.",А.~Алексеева не успевших войти в Unicode; однако им отмечалось, что предложение по внесению соответствующих дополнений в стандарт к тому времени уже было составлено и находилось на рассмотрении одной из рабочих групп ISO (\foreignlanguage{english}{International Organization for Standardization}) \autocite[21]{alexeev:2011}.

В обновлении Unicode до версии~6.1, увидевшем свет в январе 2012~г., данное предложение \autocite{proposal:2010} было принято: все перечисленные выносные буквы стали доступны в составе блока \foreignlanguage{english}{Cyrillic Extended-B}. Следовательно, отныне кодировать недостающие символы как сущности и приписывать им формальную декларацию нет необходимости, и всем им были поставлены в соответствие их интернациональные эквиваленты.

\subsection{Нормализация и лемматизация}

В обновлённое XML"=представление были интегрированы все технологические наработки, составившие предмет обсуждения предыдущей главы. А именно: \begin{inparaenum}[(1)]
    \item в содержимое атрибута \texttt{@reg} словоформы отныне записываются не просто в упрощённой графике, но в нормализованном виде (см.\ \ref{sec:norm});
    \item морфологически размеченные словоформы дополнительно снабжаются атрибутом \texttt{@ana}, где позиции разметки последовательно перечислены через точку с запятой;
    \item леммы в случае их успешного определения попадают в атрибут \texttt{@lemma}.
\end{inparaenum}

Приложение~\ref{app:xml} иллюстрирует фрагмент XML"=представления начального фрагмента жития Димитрия Прилуцкого.

\section{Проблемы совместимости с TEI}



\section*{Выводы}
\addcontentsline{toc}{section}{Выводы}



\chapter*{Заключение}
\addcontentsline{toc}{chapter}{Заключение}

Итогом проделанной выпускной квалификационной работы стало решение следующих задач.

\begin{asparaenum}
    \item Был произведён обзор систем представления грамматических данных в существующих ныне восточнославянских исторических корпусах, в результате чего была обоснована актуальность проблемы лемматизации текстов в корпусе СКАТ для приведения последнего в соответствие с глобальным уровнем развития аналогичных проектов.

    \item Были изучены основные трудности церковнославянского именного словоизменения, сопряжённые с задачей корректного определения леммы по заданной словоформе, и разработаны способы их формализации в ходе программной разработки алгоритма лемматизации морфологически размеченных житий.

    \item Было усовершенствовано XML"=представление текстов корпуса, что позволило далее загрузить их на платформу TXM,~"--- не только выведя результаты работы алгоритма лемматизации на непосредственно практический уровень, но и расширив пользовательские возможности для практической работы с корпусом в целом.
\end{asparaenum}


% Библиография
\nocite{*}
\printbibliography[heading = bibintoc]

\end{document}
