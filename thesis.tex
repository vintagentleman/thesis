\documentclass[specialist, subf, substylefile = spbu.rtx]{disser}
\usepackage[a4paper, includefoot, left = 2cm, right = 2cm, top = 2cm, bottom = 2cm, headsep = 1cm, footskip = 1cm]{geometry}

% Локализация
\usepackage{fontspec}
\usepackage{xecyr}
\usepackage[polish, french, english, russian]{babel}

% Шрифты
\defaultfontfeatures{Ligatures = TeX, Mapping = tex-text, HyphenChar = "002D}
\setmainfont[Ligatures = {TeX, Historic}]{Times New Roman}
\setsansfont{Arial}
\setmonofont{Consolas}
\newfontfamily{\agio}{AgioUnicode}

% Оформление
\usepackage[babel]{microtype}
\usepackage[autostyle]{csquotes}
\usepackage[
    xetex,
    bookmarksnumbered,
    bookmarksopen,
    colorlinks,
    allcolors = blue,
    hyperfootnotes = false
]{hyperref}

% Глубина нумерации разделов и оглавления
\setcounter{secnumdepth}{3}
\setcounter{tocdepth}{3}

% Нумерация страниц снизу и по центру
\pagestyle{footcenter}
\chapterpagestyle{footcenter}

% Сноски
\usepackage[bottom, perpage]{footmisc}

% Списки
\usepackage{paralist}
\setdefaultleftmargin{5ex}{}{}{}{}{}

% Таблицы
\usepackage{booktabs}
\usepackage{multirow}
\usepackage{ltablex}
\keepXColumns
\renewcommand{\aboverulesep}{0.1pt}
\renewcommand{\belowrulesep}{0.1pt}
\newcommand*{\thead}[1]{\multicolumn{1}{c}{\bfseries #1}}

% Графика
\usepackage{textcomp}
\usepackage{graphicx}
\usepackage{subcaption}
\graphicspath{{fig/}}

% Код
\usepackage{fancyvrb}

% Библиография
\usepackage{csquotes}
\usepackage[
    backend = biber,       % Движок
    bibencoding = utf8,    % Кодировка файлов с библиографическими БД
    style = gost-numeric,  % Стиль цитирования по ГОСТ
    language = auto,       % Получение языка из babel/polyglossia; ссылки на страницы на языке оригинала
    autolang = other,      % Многоязычная библиография
    clearlang = true,      % Внутренний сброс поля language при совпадении с языком из babel/polyglossia
    sortcites = true,      % Сортировка затекстовых ссылок при цитировании
    movenames = false,     % Не менять местами заголовок и список авторов, если последних больше четырёх
    minbibnames = 3,       % Число авторов, отображаемое при сокращении
    maxbibnames = 5        % Максимальное число авторов в препозиции
]{biblatex}

\addbibresource{./bib/scat.bib}
\addbibresource{./bib/stuff.bib}

% Сортировка библиографии
\DeclareSourcemap{
    \maps[datatype = bibtex]{
        \map{
            \step[fieldsource = langid, match = russian, final]
            \step[fieldset = presort, fieldvalue = {a}]
        }
        \map{
            \step[fieldsource = langid, notmatch = russian, final]
            \step[fieldset = presort, fieldvalue = {z}]
        }
    }
}

% Счётчик библиографических источников
\usepackage{totcount}
\newtotcounter{citnum}
\AtEveryBibitem{\stepcounter{citnum}}

% Короткое тире во всех интервалах страниц
\DefineBibliographyExtras{russian}{\protected\def\bibrangedash{\textendash}}

\sloppy

\begin{document}

% Титульный лист, аннотация, оглавление
\institution{%
    Санкт-Петербургский государственный университет \\
    Филологический факультет \\
    Кафедра математической лингвистики
}

\apname{к.",ф.",н., доц.\ И.",С.~Николаев}

\title{Выпускная квалификационная работа}
\author{Сипунин Константин Владимирович}
\topic{Автоматическая лемматизация текстов в~корпусе СКАТ на~основе морфологической разметки}

\sa{Е.",Л.~Алексеева}
\sastatus{к.",ф.",н., доц.}

% \rev{И.",В.~Азарова}
% \revstatus{к.",ф.",н., доц.}

\city{Санкт-Петербург}
\date{\number\year}

\afterpage{
    \clearpage\vspace*{\fill}

    \begin{abstract}
        Выпускная квалификационная работа посвящена процессу разработки и программной реализации модуля для расширения грамматического слоя разметки Санкт"=Петербургского корпуса агиографических текстов (СКАТ). В теоретической части исследовано словоизменение церковнославянского глагола в аспекте проблем, которые оно представляет при решении такой задачи автоматического морфологического анализа, как лемматизация. Описан программный компонент, осуществляющий лемматизацию глагольных словоформ в размеченных текстах корпуса СКАТ и интегрированный в уже созданные для него инструменты. В дальнейшей практической части работы освещён компонент для частичной автоматизации грамматической разметки корпуса, опыт и перспективы его практического использования.

        \paragraph{\small Ключевые слова:} глагол, грамматическая разметка, русская агиография, компьютерная морфология, исторический корпус, словоизменение, церковнославянский язык
    \end{abstract}

    \selectlanguage{english}

    \begin{abstract}
        This graduation paper is dedicated to the development and programmatic implementation of a module aimed at extending the grammatical annotation layer of the Saint Petersburg Corpus of Hagiographic Texts (SCAT). The theoretical part explores the issues that Church Slavonic verbal inflection presents when dealing with lemmatization---an important task in computational morphology. It is followed by a description of a software component developed for lemmatizing verbs contained within morphologically annotated vitae comprising the SCAT corpus and integrated into already existing software developed for the latter. The further section of the experimental part presents a component for partially automating the grammatical annotation of SCAT, recounts an example of its application, and discusses the perspectives of its future use.

        \paragraph{\small Keywords:} Church Slavonic, computational morphology, grammatical annotation, historical corpus, inflection, Russian hagiography, verb
    \end{abstract}

    \selectlanguage{russian}
    \vspace*{\fill}\clearpage
}

\tableofcontents

% Текст работы
\intro

Славянская рукописная традиция зародилась уже более тысячи лет назад. От времён, последовавших за просветительской деятельностью преподобных Константина (Кирилла) и Мефодия в середине IX~в., до сегодняшних дней дошли десятки тычяч рукописей, созданных писцами и переписчиками в монастырях Восточной Европы,~"--- как списков Священного Писания, служебников, часословов и прочих богослужебных книг, непосредственно обслуживавших запросы церкви, так и оригинальных произведений, предназначенных для индивидуального чтения: поучений, сказаний, житий святых. Тем не менее, значительная доля данных текстовых массивов по сей день изучена недостаточно и по-прежнему нуждается во всесторонней исследовательской обработке~"--- исторической, этнографической, лингвистической.

Несколько десятилетий назад ситуация начала качественно преображаться в связи с появлением, а впоследствии и массовым распространением компьютеров и цифровых технологий: средства представления рукописей в электронном виде ознаменовали собой принципиально новые возможности их сохранения и изучения вне стен отдельных библиотек и архивных фондов, регулярным доступом к которым обладает далеко не каждый исследователь.

Соответствующие разработки начали появляться уже в конце третьей четверти XX~в.~"--- в том числе на кафедре математической лингвистики Ленинградского государственного университета. С конца 1970-х~гг.\ при участии сотрудников кафедры русского языка ЛГУ, а также ИРЛИ АН СССР и ГПБ им.~М.",Е.~Салтыкова-Щедрина на кафедре начал создаваться фонд фото- и ксерокопий списков древнерусских житий и похвальных слов XV--XVII вв.\ \autocite[512]{averina_alexeeva_gerd:1996}, впоследствии получивший название "<Санкт"=Петербургский корпус агиографических текстов"> (СКАТ). Для представления содержимого фонда в памяти ЭВМ каждую копию рукописного текста было необходимо транслитерировать~"--- перевести в машиночитаемый формат при помощи специальной системы кодирования. Однако в те годы фактически единственным средством ввода символьных цепочек в память компьютера являлись 8-битные кодировки на базе ASCII (\foreignlanguage{english}{American Standard Code for Information Interchange}), очевидно не предназначенные для размещения в диапазоне кодируемых символов знаков устаревших и экзотических систем письменности (в~т.",ч.\ кириллической). Вследствие этого для набора текстов, составляющих фонд, на кафедре была выработана собственная кодировка, в которой для вышедших из употребления символов кириллицы были введены замены (преимущественно буквы латинского алфавита): так, юс большой и юс малый обозначаются соответственно "<\textsc{g}"> и "<\textsc{r}">, кси~"--- "<\textsc{l}"> и~т.",д. Тексты вводимых в память ЭВМ рукописей набираются квалифицированными специалистами"=филологами вручную при помощи специально разработанного шрифта AGIO и затем автоматически переводятся в данную кодировку; при этом в текст вставляются словоразделы (в соответствии с принципами, разработанными проф.\ А.",А.~Алексеевым для издания серии "<Библиотека литературы Древней Руси">), а также маркируются границы составных частей рукописи~"--- строк, колонок и страниц. Всего к настоящему времени в базу данных введено более полусотни рукописей общим объёмом около полумиллиона словоупотреблений \autocite{gerd_alexeeva_azarova_zakharova:2004}.

Сегодня доступ к результатам работы коллектива проекта обеспечивается двояко. С одной стороны, с конца 1990-х~гг.\ издательством Санкт"=Петербургского государственного университета ведётся публикация изданий серии "<Памятники русской агиографической литературы">, в каждом из которых содержится один или несколько подготовленных к печати житийных текстов, набранных упомянутым выше шрифтом AGIO, полный словоуказатель словоформ, а также текстологические статьи об истории публикуемых житий, биографии святых, сведения об обителях. Последний, одиннадцатый выпуск увидел свет в 2012~г.; там же приведён перечень всех предыдущих публикаций серии \autocite[4]{coll:2012}.

С другой стороны, всё более повсеместное распространение онлайн"=технологий в 2000-х~гг.\ дало импульс к тому, чтобы обеспечить доступ к опубликованным материалам через интернет: был создан сайт проекта\footnote{\url{http://project.phil.spbu.ru/scat/} (дата обр.\ \today)}, а корпус получил своё нынешнее наименование. На сегодняшний день около полутора десятков житий доступны для загрузки с сайта в двух форматах: PDF, воспроизводящем их представление в печатных сборниках, и XML, где с помощью системы тегов производится формальное членение рукописей на структурные элементы. XML"=разметка текстов СКАТ соответствует международному стандарту оформления электронных изданий~"--- \foreignlanguage{english}{Text Encoding Initiative} (TEI).

\begin{figure}[t!]
    \centering
    \begin{subfigure}[t]{0.495\textwidth}
        \includegraphics[width=\linewidth]{scat_search}
        \caption{Выдача по запросу \textsc{бц} (режим нестрогого соответствия)}
        \label{fig:scat:1}
    \end{subfigure}
    \hfill
    \begin{subfigure}[t]{0.495\textwidth}
        \includegraphics[width=\linewidth]{scat_output}
        \caption{Контекстное окно вхождения словоформы \textsc{бголюбци\#} (ГП 323/19)}
        \label{fig:scat:2}
    \end{subfigure}
    \caption{Поиск по словоуказателю СКАТ}
\end{figure}

Также на сайте имеется возможность поиска по корпусу~"--- вернее, по той его части, для которой построен сводный словоуказатель. Это центральный компонент лингвистического обеспечения СКАТ, представляющий собой список словарных статей, в каждой из которых указана словоформа в нормализованном виде, абсолютная частота её встречаемости по всем проиндексированным рукописям и адреса вхождений. Адрес состоит из сокращённого наименования рукописи и сочетания порядковых номеров листа (с уточнением стороны~"--- лицевой либо оборотной), колонки и строки, разделённых косой чертой. При нажатии на адрес в поисковой выдаче (рис.~\ref{fig:scat:1}) пользователю предлагается "<нарезка"> из соответствующего PDF"=документа (рис.~\ref{fig:scat:2}), в которую попадает искомое вхождение; отыскивать его приходится самостоятельно~"--- путём отсчитывания от межстраничной либо межколонной границы с номером, указанным в адресе, необходимого числа строк.

Однако современный электронный корпус~"--- в отличие от простой коллекции текстов~"--- должен располагать определённым набором автоматизированных инструментов, применимых в ходе решения конкретных лингвистических задач. В ряде зарубежных работ по языкам с ограниченными ресурсами в последние годы вошло в обиход понятие BLARK~"--- \foreignlanguage{english}{Basic Language Resource Toolkit} (базовый набор лингвистических ресурсов), которое определяется как "<\foreignlanguage{english}{the minimal set of language resources that is necessary to do any precompetitive research and education at all}"> \autocite[11]{krauwer:2003} (минимальный набор лингвистических ресурсов, необходимый для любых базовых исследовательских и образовательных нужд). BLARK может включать в себя как традиционные одно- и двуязычные словари и грамматики, так и специфические ресурсы, вошедшие в лингвистический обиход лишь в последние десятилетия: модули распознавания и синтеза речи, морфосинтаксические анализаторы и~т.",д. Притом отмечается, что этот список не закрытый и может варьироваться от языка к языку: очевидно, для древнеписьменных языков, в число которых входит и церковнославянский, неактуальна задача обработки устной речи, однако вследствие некодифицированного характера орфографии зачастую требуются модули её нормализации.

\textcite[28]{passarotti:2010} предлагает вариант BLARK ("<\foreignlanguage{english}{a BLARK-like set}">) для латинского языка, который, как кажется, в равной степени приложим к другим древнеписьменным языкам. В нём предусмотрены инструменты, направленные на решение следующих основных задач: \begin{inparaenum}[(1)]
    \item предобработка текстовых данных: токенизация и распознавание именованных сущностей;
    \item морфологический анализ: лемматизация и разрешение морфосинтаксической неоднозначности;
    \item синтаксический анализ (поверхностный и глубинный);
    \item разрешение анафоры;
    \item семантический и прагматический анализ.
\end{inparaenum}

\textbf{Цель} настоящей работы заключается в том, чтобы в применении к корпусу СКАТ разработать комплекс инструментов для осуществления одной из подзадач морфологического анализа, специфицируемой в рамках базового набора лингвистических ресурсов,~"--- процедуры лемматизации. \textbf{Задачи}, которые необходимо решить для достижения поставленной цели, таковы:

\begin{compactenum}
    \item ознакомление с системами представления грамматических сведений (в~т.",ч.\ данных по леммам) в существующих восточнославянских исторических корпусах;
    \item изучение теоретических предпосылок алгоритма лемматизации церковнославянского языкового материала с учётом всех релевантных морфологических особенностей и его программная реализация;
    \item организация доступа к лемматизированному подкорпусу СКАТ (и шире~"--- ко всей оцифрованной части корпуса) с использованием общедоступных технологических средств.
\end{compactenum}

\textbf{Объект} основной части исследования~"--- словоизменение в церковнославянском языке XV--XVII вв. \textbf{Предмет} изучения~"--- проблемы формализации феноменов церковнославянского словоизменения в ходе алгоритмизации перехода от словоформ в несловарных парадигматических позициях к словарным (т.",е.\ леммам). \textbf{Материалом} послужили морфологически размеченные тексты трёх агиографических текстов в составе корпуса СКАТ: жития Димитрия Прилуцкого, Дионисия Глушицкого и Кирилла Новоезерского~"--- суммарным объёмом около 30~тыс.\ словоупотреблений.

\textbf{Актуальность} работы обоснована тем, что в рамках СКАТ~"--- единственного в своём роде источника сведений по языку древнерусской агиографии эпохи позднего Средневековья и Нового времени~"--- серьёзные попытки разработки составных частей BLARK в целом и подсистем морфологического анализа в частности фактически не предпринимались.

\textbf{Структура} работы включает в себя введение, \total{chpnum}~главы, заключение, список литературы из \total{citnum}~наименований и \total{appnum}~приложения.

\chapter{Представление грамматической информации в славянских исторических корпусах}

На сегодняшний день славянских диахронических корпусов существует крайне мало. Так, соответствующий
перечень, приведённый на сайте\footnote{\url{http://ruscorpora.ru}} Национального корпуса
русского языка (НКРЯ), состоит из всего четырёх наименований (не включая СКАТ):
\begin{inparaenum}[(1)]
    \item Регенсбургский диахронический корпус русского языка,
    \item Рукописные памятники Древней Руси,
    \item корпус "<Манускрипт"> Удмуртского государственного университета,
    \item корпус русских публицистических текстов второй половины XIX века Петрозаводского государственного университета (ввиду своей специфики он далее рассматриваться не будет);
\end{inparaenum}
помимо этого перечислены два старославянских корпуса: университетов Хельсинки и Южной Калифорнии.
Краткий обзор большинства названных корпусов (включая исторические подкорпуса самого НКРЯ) приведён
в статье \autocite{mitrenina:2014}; наше рассмотрение будет сосредоточено на реализованных в них
принципах и инструментах грамматической разметки и лемматизации. (Обсуждение технологий NLP~"---
в~т.",ч.\ морфологических модулей~"--- в зарубежных диахронических корпусах см.\ в седьмой главе
монографии \textcite[85--101]{passarotti:2010}).

\section{Исторические подкорпуса НКРЯ}


\chapter{Прецедентная разметка текстов СКАТ}

В настоящей главе пойдёт речь о втором компоненте разработанного грамматического модуля, направленном на частичную автоматизацию морфологической разметки текстов СКАТ.

\section{Опыт древнерусского подкорпуса НКРЯ}

Идея о том, что при определённом уровне накопленного материала дальнейшая лингвистическая разметка может осуществляться не с нуля, не нова. "<Базы данных текстовых прецедентов с приписанными вручную морфологическими пометами"> перечисляются в \autocite[47]{baranov:2015} первыми среди средств автоматизации разметки, распространённых в корпусной палеославистике.

Примером исторического корпуса русского языка, в котором данная идея успешно получила своё воплощение, может служить древнерусский подкорпус Национального корпуса русского языка (НКРЯ). Корпус составляют книжные тексты древнерусских рукописей XI--XIV~вв.\ суммарным объёмом порядка 500~тыс.\ словоупотреблений \autocite[101--102]{mishina_pichkhadze:2015}. Архаический характер их грамматики, огромная вариативность в орфографии и другие связанные проблемы исключают автоматизацию разметки, например, путём создания грамматических словарей (в отличие от более "<современных"> церковнославянского и старорусского подкорпусов, где данный вопрос либо решён \autocite[250--253]{polyakov:2014}, либо находится в процессе решения \autocite{lyashevskaya:2016}), поэтому их разметка ещё с середины 2000-х~гг.\ осуществляется вручную силами экспертов"=славистов. Очевидно, что затраты на столь скрупулёзный труд остаются крайне высокими.

В \autocite{archangel_mishina_pichkhadze:2014} описана среда Morphy, разработанная в Институте русского языка для грамматической разметки древних славянских текстов и используемая в древнерусском подкорпусе НКРЯ. Процесс аннотирования может осуществляться как вручную, так и полуавтоматически; в последнем случае "<программа использует информацию из уже размеченных текстов, т.",е.\ использует прецедентные разборы, а исследователь проверяет и редактирует предложенный вариант разбора. Если предложенных разборов несколько, а в данном контексте правильным является только один, исследователь убирает ненужные варианты разбора, возникающие из-за омонимии словоформ"> \autocite[29]{archangel_mishina_pichkhadze:2014}. В обоих случаях также привлекаются сведения из сводного словаря лемм: при вводе экспертом леммы, которая в нём присутствует, "<словарные грамматические признаки">, т.",е.\ граммемы классификационных категорий подставляются автоматически \autocite[28]{archangel_mishina_pichkhadze:2014}.

\section{Реализация прецедентной разметки для СКАТ}
\label{sec:precedent}

Реализованный компонент грамматического модуля СКАТ решает аналогичную задачу, но с поправкой на более привычный для коллектива СКАТ табличный формат представления разметки и ориентацией на студентов как конечных её исполнителей. С программной точки зрения компонент в свою очередь состоит из двух связанных между собой компонентов.

Первый подкомпонент интегрирован в конвертер для размеченных текстов как отдельный режим его работы~"--- \texttt{pkl} (см. \ref{sec:module}). В данном режиме обрабатываемые словоформы сериализуются в хранилище данных типа "<ключ~"--- значение">, где в качестве ключей выступают их нормализованные формы, в качестве значений~"--- массивы из зафиксированных в разметке кортежей из морфологических разборов (тегсетов).

Тегсеты записываются в хранилище в несколько упрощённом виде. В разметке СКАТ особо фиксируются словоформы, обнаруживающие переходные грамматические явления: так, развитие категории одушевлённости отражается значением падежа \texttt{вин/род}, где тег до косой черты обозначает ожидаемую граммему, а тег после~"--- фактическую; при записи в хранилище сохраняются только последние. Также отдельного упоминания заслуживают личные формы глаголов. Очевидно, что морфологически идентичные формы могут выражать разные грамматические значения наклонения и времени; однако в полностью неразмеченных текстах тот контекст, который позволил бы отличить синтетические глагольные формы от их омонимов в составе аналитических, априори не известен. В связи с этим для личных форм глаголов в хранилище фиксируются, помимо части речи, только граммемы лица (или рода в случае эловых причастий) и числа.

Второй дочерний компонент (\foreignlanguage{english}{\texttt{annotator.py}}) принимает на вход неразмеченный текст, уже сегментированный на токены.\footnote{%
    Для сегментации в несколько видоизменённом виде использован фрагмент модуля \texttt{texttoxml.py}, написанный для дипломной работы \autocite{alexeev:2009}.
} Каждая словоформа нормализуется и ищется в созданном ранее хранилище; её присутствие позволяет использовать ассоциированные с ней кортежи граммем для прецедентной разметки. Однако ограничиваться лишь теми тегсетами, которые содержатся в хранилище, нельзя: материал размеченных житий, на основе которых оно конструируется, весьма ограничен, и словоформы в его составе представлены парадигмой, далёкой от полной; привлечение только \textit{реально} существующих тегсетов не учитывает \textit{потенциальной} грамматической омонимии между членами словоизменительных парадигм.

Для решения этой проблемы был составлен перечень множеств тегсетов, план выражения которых омонимичен.\footnote{%
    Перечень в формате JSON приведён в файле \href{https://github.com/vintagentleman/scat-v2/blob/master/src/utils/clusters.json}{\texttt{src/utils/clusters.json}}.
} При анализе словоформ все тегсеты последовательно сверяются с данным перечнем и при нахождении в одном из множеств последнее объединяется с множеством всех тегсетов, потенциально присущих словоформе.

Результат работы компонента~"--- таблица формата \texttt{.xlsx}, по содержимому столбцов практически полностью соответствующий спецификации разметки СКАТ (\ref{sec:annotation}).\footnote{%
    Отступления касаются лишь тех личных форм глаголов, в грамматическом значении которых отсутствует время: в таком случае столбцы с граммемами лица (рода) и числа должны смещаться на единицу влево, однако поскольку время, как было сказано выше, не фиксируется в хранилище, нет возможности узнать о факте его отсутствия в размечаемом тексте. Полуавтоматическая разметка таких словоформ подлежит посткоррекции.
} При этом все однозначно определяемые граммемы заносятся в таблицу как есть; все те, в отношении которых зафиксирована потенциальная омонимия, явно не записываются, но соответствующие ячейки выделяются цветом фона, а при наведении все варианты грамматических значений становятся доступны из выпадающего списка (используется механизм проверки данных (\foreignlanguage{english}{data validation}), доступный в \foreignlanguage{english}{Microsoft Excel}; рис.~\ref{fig:precedent}).

В порядке организации разметки как учебной деятельности в рамках филологической практики таблица сегментирована на листы, на каждый из которых попадает ограниченное множество словоформ (предполагается, что количество листов соответствует количеству студентов в группе). При запуске компонента оба числа настраиваются; также подлежит конфигурации порядковый номер токена, вплоть до которого содержимое анализируемого жития следует игнорировать,~"--- это вызвано тем, что большинство текстов частично уже размечены и требуют доразметки не с начала.

Если путём прецедентной разметки та или иная словоформа размечается полностью и однозначно, то при подсчёте объёма "<порции"> словоформ, приходящейся на очередного студента, она не учитывается. Это позволяет существенно увеличить объём работы, подлежащей выполнению.

\begin{figure}[t]
    \centering
    \includegraphics[width=\textwidth]{precedent} % TODO
    \caption{Прецедентная разметка жития Александра Свирского}
    \label{fig:precedent}
\end{figure}

\section{Опыт по внедрению прецедентной разметки}

В рамках промежуточной аттестации в декабре 2018~г.\ автором совместно с Алексеевой~Е.",Л.\ был проведён опыт по внедрению прецедентной разметки в учебную филологическую практику: студенты 2~курса образовательной программы бакалавриата "<Прикладная, компьютерная и математическая лингвистика"> СПбГУ в рамках филологической практики выполнили часть морфологической разметки жития Александра Свирского на материале вывода разработанного компонента.

Для тогдашней версии программы ещё не был создан перечень потенциальных грамматических омонимов~"--- учитывались только тегсеты, фактически имеющиеся в прецедентной базе. Ввиду ограниченности объёма размеченной выборки это привело к ожидаемому результату: многие словоформы ошибочно размечались как однозначные и игнорировались при расчёте объёма очередного фрагмента. Это в свою очередь сказалось на их объёме (в среднем он составил 524,7 при выставленном номинальном объёме 350) и потребовало их дополнительной экспертной предобработки.

Тем не менее, в результате проверяющим экспертом было отмечено, что совершённых экспериментальной группой ошибок было значительно меньше, чем обычно демонстрируют студенты второго курса. Можно выдвинуть следующие предположения, обусловившие данное обстоятельство:

\begin{asparaitem}
    \item с точки зрения морфологии из синтетического строя церковнославянского языка, при котором один аффикс одновременно способен выражать целый ряд грамматических значений, следует их взаимная обусловленность~"--- становится проще предсказывать недостающие граммемы у не полностью размеченных словоформ исходя из уже имеющихся;
    \item с точки зрения синтаксиса важную роль следует отвести согласованию и координации словоформ: если, например, в сочетании прилагательного с существительным у первого известны все граммемы, а у последнего нет, но они с очевидностью составляют словосочетание, то заполнение недостающих граммем достигается тривиальным копированием.
\end{asparaitem}

\section*{Выводы}
\addcontentsline{toc}{section}{Выводы}

Был разработан модуль для аннотирования текстов СКАТ с использованием прецедентов. Предварительный опыт его внедрения в практику промежуточной аттестации продемонстрировал, что формирование фрагментов, подлежащих разметке, с его помощью способно вызвать не только количественный, но и качественный прирост мероприятий по её дальнейшему пополнению.

\chapter{Портирование корпуса СКАТ на~платформу TXM}

Платформа TXM\footnote{\url{http://textometrie.ens-lyon.fr} (дата обр.\ \today)}~"--- это свободно распространяемое программное обеспечение для работы с текстовыми корпусами, разработанное в лаборатории IHRIM (\foreignlanguage{french}{Institut d'Histoire des Repr\'{e}sentations et des Id\'{e}es dans les Modernit\'{e}s}) Национального центра научных исследований Франции \autocite{heiden:2010}. TXM предоставляет в распоряжение пользователя широкий набор инструментов количественного и качественного анализа текстов: получение конкордансов в формате KWIC и частотных списков лексических единиц на основе любого приписанного им параметра; построение частотных графиков динамики вхождений единиц, удовлетворяющих пользовательскому запросу (для статистических расчётов используется вычислительный движок R); сбор данных о совместной встречаемости, о лексических шаблонах и многое другое. Также платформа приспособлена для обработки текстовой метаинформации, что позволяет пользователю строить подкорпуса (\foreignlanguage{english}{subcorpora}) и разбиения (\foreignlanguage{english}{partitions}) корпусов, введённых в платформу, по различным метатекстовым основаниям. TXM поддерживает множество входных форматов (TXT, ODT/DOC/RTF, XML, различные проприетарные форматы), однако для внутреннего представления содержимого введённых корпусов используется XML"=представление.

По инициативе А.",М.~Лаврентьева, одного из главных разработчиков платформы, на протяжении нескольких лет активно сотрудничавшего с коллективом СКАТ и впервые написавшего программу для автоматической конвертации текстовых файлов житий в формат XML \autocite[21]{alexeeva_lavrentiev_azarova_zakharova:2004}, фрагмент корпуса СКАТ объёмом 12 житийных текстов (включая 2 похвальных слова) был загружен на демонстрационный портал TXM, открытый для пользования в режиме онлайн\footnote{\url{http://portal.textometrie.org/demo/} (дата обр.\ \today)}. Однако сотрудники СКАТ участия в этой работе фактически не принимали, вследствие чего корпус был не вполне качественно адаптирован к реалиям платформы: в частности, сами тексты доступны для чтения лишь в упрощённой графике и содержат ошибки перекодирования (в особенности это касается цифирных обозначений чисел).

Настоящий этап работы нацелен на устранение всех подобных недостатков и максимальное приспособление корпуса СКАТ к комфортному использованию при помощи стационарной версии платформы TXM, а также на внедрение в TXM"=совместимое представление текстов корпуса слоя грамматических данных и лемм.

\section{Режим импортирования XTZ}

Как было отмечено ранее, платформа TXM приспособлена к импорту текстовых корпусов во множестве различных форматов, однако де-факто стандартным и наиболее активно совершенствуемым в позднейших версиях платформы способом загрузки входных текстов в формате XML является режим XTZ~"--- \foreignlanguage{english}{XML TEI Zero} \autocite[76]{txm}.

Помимо универсальных средств обработки импортируемых документов (включая транспонирование различных уровней разметки во внутреннее TXM"=представление, благодаря которому пользователь получает возможность строить подкорпусы и разбиения по любым интересующим его размеченным текстовым структурам, многоаспектное индексирование словоформ и многое другое), режиму XTZ также присуща ориентированность на определённый минимальный ("<нулевой">) набор тегов, наиболее часто используемых при разметке текстовых данных с опорой на рекомендации консорциума TEI, и способность учитывать их семантику при конструировании HTML"=изданий, непосредственно доступных для чтения.

\begin{table}[t]
    \small
    \begin{tabularx}{\linewidth}{Xp{4cm}X}
        \toprule
        \thead{XML} & \thead{HTML} & \thead{Пояснение} \\ \midrule\midrule
        \texttt{<head>} & \texttt{<h2>} & Заголовок \\ \midrule
        \texttt{<p>} & \texttt{<p>} & Абзац \\ \midrule
        \texttt{<hi>} & \texttt{<b>} & Полужирное начертание \\ \midrule
        \texttt{<emph>} & \texttt{<i>} & Курсивное начертание \\ \midrule
        \texttt{<list type='unordered'>} & \texttt{<ul>} & Маркированный список \\ \midrule
        \texttt{<list type='ordered'>} & \texttt{<ol>} & Нумерованный список \\ \midrule
        \texttt{<item>} & \texttt{<li>} & Элемент списка \\ \midrule
        \texttt{<table>} & \texttt{<table>} & Таблица \\ \midrule
        \texttt{<row>} & \texttt{<tr>} & Табличная строка \\ \midrule
        \texttt{<cell>} & \texttt{<td>} & Табличная ячейка \\ \midrule
        \texttt{<graphic>} & \texttt{<img>} & Рисунок \\ \midrule
        \texttt{<ref>} & \texttt{<a>} & Гиперссылка \\ \midrule
        \texttt{<note>} & \texttt{<a>}, \texttt{<span>} & Сноска \\ \midrule
        \texttt{<w>} & \texttt{<span>} & Токен \\ \bottomrule
        \caption{Преобразования тегов при импорте в режиме XTZ (по \autocite[78--80]{txm})}
        \label{tab:edition}
    \end{tabularx}
\end{table}

Так, определяемые TEI маркеры начала новой строки~"--- \texttt{<lb/>} (\foreignlanguage{english}{line beginning})~"--- при генерации HTML преобразуются в теги \texttt{<br/>}, позволяющие форсировать разрыв строки в любом необходимом месте. Кроме того, если они дополнительно снабжены глобальным атрибутом \texttt{@n}, указывающим на порядковый номер соответствующей строки, то напротив строк через определённые интервалы автоматически вставляются их порядковые номера, подобно тому как нумеруются стихи в академических изданиях античной поэзии. Аналогично обрабатывается тег \texttt{<pb/>} (\foreignlanguage{english}{page beginning}); в тех случаях, когда пагинация текстов корпуса на уровне разметки не предусмотрена, TXM фрагментирует их самостоятельно, исходя из максимального числа токенов на каждой странице (этот параметр задаётся пользователем при импорте).

Перечень прочих XML"=тегов и их HTML"=эквивалентов приведён в таблице~\ref{tab:edition}.

\section{Адаптация XML"=представления СКАТ к режиму XTZ}
\label{sec:xml}

\subsection{Проблемы существующей XML"=структуры}

Последним, кто работал над СКАТ в рассматриваемом аспекте, был В.",А.~Алексеев. В рамках своей магистерской диссертации \autocite[41--54]{alexeev:2011} он предпринял ряд серьёзных мер, направленных на модернизацию XML"=представления текстов СКАТ в соответствии с современными стандартами электронного представления текстовых данных.

Нестандартные сущности, теги и атрибуты, ранее использовавшиеся для отображения графем, отсутствующих в современном русском языке, были заменены на символы Unicode~5.1. Данная мера была продиктована как нормативными, так и прагматическими соображениями, поскольку XML"=представление СКАТ образца нулевых было весьма громоздким и неудобочитаемым; так, результат преобразования в XML такой словоформы, как \textsc{ра(д)уасr}, в нём выглядел следующим образом:

\begin{Verbatim}[fontsize=\small, gobble=4, xleftmargin=5ex]
    ра<osl_letter type='overline'>
      д
    </osl_letter>уас&cyr-littleyus;
\end{Verbatim}

Те немногие графемы, которые не были определены в кодовой таблице Unicode~5.1, В.",А.~Алексеев предложил по-прежнему кодировать как сущности~"--- например, \texttt{\&i8-overline;} в случае выносного \textsc{и} восьмеричного. При этом все сущности были определены в отдельном файле определения типа документа (DTD), а также снабжены формальной декларацией (\texttt{<charDecl>}) на уровне описания кодировки TEI"=документа (\texttt{<encodingDesc>}). Этот механизм был впервые включён в рекомендации TEI в версии P5 \autocite[39, 192--201]{tei}, полноценное обновление до которой и строгое следование соответствующим нормам также входило в круг задач диссертационного исследования.

Тем не менее, разработанная В.",А.~Алексеевым версия XML"=представления СКАТ по ряду причин не является TXM"=совместимой. Во-первых, для разметки мельчайших структурных частей рукописи (страниц, колонок и строк) было предложено использовать сразу два синонимичных набора элементов:

\begin{compactenum}
    \item парные теги \texttt{<div2>}, \texttt{<div3>}\footnotemark, \texttt{<div4>}, \texttt{<l>};
    \item пустые теги \texttt{<pb/>}, \texttt{<cb/>}, \texttt{<lb/>}.
\end{compactenum}

\footnotetext{
    Тег \texttt{<div2>} маркирует лист, а \texttt{<div3>}~"--- страницу, т.",е.\ одну из сторон листа (лицевую либо оборотную).
}

Если первый набор нацелен на описание формально"=иерархической организации XML"=документа, то последний скорее предназначен для его семантического структурирования: вместо разбивки на строго непересекающиеся блоки в текст вносятся маркеры (\foreignlanguage{english}{milestones}), попросту указывающие на окончание одной структурной единицы и начало другой. Оба способа одновременно консорциум TEI предписывает задействовать лишь тогда, когда размечаемых структур более одной и они являются соперничающими \autocite[123--124]{tei}, т.",е.\ синонимичными, но не идентичными; в противном случае большей простотой и практичностью, невзирая на меньшую экспрессивность, обладает маркерная аннотация. Если же кодированию подлежит множество разнородных структур, то её использование для разметки таких базовых единиц, как строки, колонки и страницы, тем более предпочтительно. Кроме того, спецификации режима XTZ затрагивают именно пустые теги, а использования их парных аналогов (и шире~"--- всех элементов и атрибутов с целочисленными суффиксами) ввиду особенностей функционирования поисковой машины CQP, напротив, рекомендуется избегать \autocite[78]{txm}.

Во-вторых, XML"=разметка элементарных лексических единиц (токенов) была призвана учесть множество различных вариантов их графического представления. При этом все подобные варианты определялись как потомки базового тега \texttt{<w>}:

\begin{Verbatim}[fontsize=\small, gobble=4, xleftmargin=5ex]
    <w xml:id='CrlNvz.1'>
      <orig>мѣсѧца</orig>
      <reg>М+СЯЦА</reg>
      <src>М+СRЦА</src>
    </w>
\end{Verbatim}

Здесь внутри вложенного тега \texttt{<src>}\footnote{
    Это единственный случай отступления В.",А.~Алексеевым от рекомендаций TEI. Паронимический тег \texttt{<source>} имеет совершенно иную семантику и иное назначение \autocite[356]{tei}.
} первое слово жития Кирилла Новоезерского представлено в оригинальном 8-битном формате, внутри \texttt{<reg>}~"--- в упрощённой графике; наконец, в теге \texttt{<orig>} оно записано с использованием символов Unicode~5.1. В случае ошибочных написаний иерархия получает дальнейшее усложнение: \texttt{<orig>} в качестве потомка приобретает тег \texttt{<choice>}, обозначающий наличие альтернантов\footnote{
    Строго говоря, триада из \texttt{<orig>}, \texttt{<reg>} и \texttt{<src>} также требует обрамления тегом \texttt{<choice>}, однако подобный шаг, очевидно, ознаменовал бы собой ещё большее осложнение XML"=структуры.
}, а внутрь него в свою очередь заносится ошибка (\texttt{<sic>}) и исправление (\texttt{<corr>}). Например:

\begin{Verbatim}[fontsize=\small, gobble=4, xleftmargin=5ex]
    <w xml:id='CrlNvz.90'>
      <orig><choice>
          <sic>человѣчетвѡ</sic>
          <corr>человѣчествѡ</corr>
      </choice></orig>
      <reg>~ЧЕЛОВ+ЧЕТВО &lt;ЧЕЛОВ+ЧЕСТВО&gt;</reg>
      <src>~ЧЕЛОВ+ЧЕТВW &lt;ЧЕЛОВ+ЧЕСТВW&gt;</src>
    </w>
\end{Verbatim}

Между тем режим XTZ не предполагает наличия у ядерных лексических единиц столь развитой иерархической организации. Он ориентирован на обработку тегов \texttt{<w>} в простейшем виде, когда в качестве их содержимого выступает единственный вариант графического представления токена, а все альтернативные вкупе с прочими сопутствующими сведениями записаны в атрибуты \autocite[77]{txm}. Иначе говоря, предполагается, что элементы \texttt{<w>} являются терминальными узлами XML"=структуры и потомков не имеют; если в действительности это не так, то при импорте последние игнорируются, а содержимым родительского тега считается результат конкатенации содержимого всех дочерних.

Таким образом, при подготовке HTML"=издания первый пример из приведённых выше считался бы тождественным следующему (что было бы нежелательно):

\begin{Verbatim}[fontsize=\small, gobble=4, xleftmargin=5ex]
    <w xml:id='CrlNvz.1'>
      мѣсѧцаМ+СЯЦАМ+СRЦА
    </w>
\end{Verbatim}

\subsection{Структурные нововведения}

Предлагаемые нами нововведения в XML"=структуру текстов СКАТ, призванные обеспечить их полную совместимость с режимом импортирования XTZ, обобщены в таблице~\ref{tab:new_xml}.

\begin{table}[t]
    \footnotesize
    \begin{tabularx}{\linewidth}{XX}
        \toprule
        \thead{Старый тег} & \thead{Новый тег} \\ \midrule\midrule
        
        \texttt{<div1 type='part' n='1'></div1>} & \texttt{<ab></ab>} \\ \midrule
        
        \texttt{<div2 type='page' n='1'>} & \\
        \texttt{~~<div3 type='back'></div3>} & \texttt{<pb n='-1'/>} \\
        \texttt{</div2>} & \\ \midrule
        
        \texttt{<div3 type='front'>} & \\
        \texttt{~~<div4 type='col' n='1'></div4>} & \texttt{<pb n='1a'/>} \\
        \texttt{</div3>} & \\ \midrule
        
        \texttt{<l n='1'></l>} & \texttt{<lb n='1'/>} \\ \midrule
        
        \texttt{<w>} & \\
        \texttt{~~<orig>ѿ</orig>} & \\
        \texttt{~~<reg>О(Т)</reg>} & \texttt{<w reg='о(т)' src='W(Т)'>ѿ</w>} \\
        \texttt{~~<src>W(Т)</src>} & \\
        \texttt{</w>} & \\ \midrule
        
        \texttt{<w>} & \\
        \texttt{~~<orig><choice>} & \\
        \texttt{~~~~<sic>мъ</sic>} & \\
        \texttt{~~~~<corr>мѧ</corr>} & \texttt{<w reg='мя' src='\~{}МЪ \&lt;МR\&gt;'>мъ</w>} \\
        \texttt{~~</choice></orig>} & \texttt{<note type='corr'>мѧ</note>} \\
        \texttt{~~<reg>\~{}МЪ \&lt;МЯ\&gt;</reg>} & \\
        \texttt{~~<src>\~{}МЪ \&lt;МR\&gt;</src>} & \\
        \texttt{</w>} & \\ \midrule
        
        \texttt{<c type='punctuation'></c>} & \texttt{<pc></pc>} \\ \bottomrule
        \caption{Предлагаемые замены тегов}
        \label{tab:new_xml}
    \end{tabularx}
\end{table}

С чисто формальной точки зрения замещение структурных подразделений верхнего уровня \texttt{<div1>} анонимными блоками \texttt{<ab>} (\foreignlanguage{english}{anonymous block}) обусловлено обозначенным выше стремлением избавиться от тегов с целочисленными суффиксами; содержательная же подоплёка данного нововведения состоит в том, что так житийные тексты представляются как нерасчленённые,~"--- иначе говоря, делается имплицитное утверждение, что никаких промежуточных смысловых блоков внутри них не выделяется. Однако в будущем такое положение вещей, вероятно, изменится, поскольку разработки формата сюжетной разметки внутри коллектива СКАТ также ведутся \autocite{rogozina:2015}.

Формальную разбивку документов на листы (\texttt{<div2>}) и страницы (\texttt{<div3>}) предлагается полностью заменить смысловой и для маркировки границ между ними пользоваться исключительно тегом \texttt{<pb/>} с обязательным атрибутом \texttt{@n}, обозначающим порядковый номер соответствующей страницы. Номера лицевых и оборотных сторон листа в соответствии с транслитерационными соглашениями СКАТ отличаются между собой по наличию при них специального префикса~"--- дефиса.

Поскольку в режиме XTZ отсутствует поддержка специализированного тега"=разделителя между колонками (\texttt{<cb/>}, \foreignlanguage{english}{column beginning}), последние видится необходимым рассматривать как отдельные страницы и также отграничивать друг от друга при помощи элемента \texttt{<pb/>}. При этом формат атрибута \texttt{@n} получает дополнительное расширение в виде суффикса \texttt{a} для первой колонки или \texttt{b} для второй. Отметим, что рукописи с тремя колонками и более чрезвычайно редки и в корпусе СКАТ не представлены, а рукопись с двумя колонками всего одна (житие Александра Свирского; РНБ, Пог.~874, XVI~в.).

Замена элементов \texttt{<l>} на \texttt{<lb/>} осуществляется по аналогичному принципу; присваиваемые им порядковые номера являются простыми натуральными числами.

Направление преобразования тегов элементарных лексических единиц (\texttt{<w>}) было обосновано выше: в качестве их содержимого отныне выступает единственный вариант графического представления (совместимый с Unicode), а прочие конвертируются в одноимённые атрибуты. Для ошибочных написаний имеют место следующие спецификации: \begin{inparaenum}[(1)]
    \item в атрибут \texttt{@reg} записывается нормализованная форма исправленного варианта~"--- и только его;
    \item внутри тега \texttt{<w>} содержится Unicode"=совместимое представление оригинального написания;
    \item исправление обрамляется типизированным тегом \texttt{<note>}, непосредственно следующим за токеном.
\end{inparaenum} Далее это позволяет конструировать HTML"=издания житийных текстов в первозданном виде, исправления же отображать как сноски.

Наконец, \texttt{<pc>} введён вместо типизированного тега \texttt{<c>} в угоду краткости и соответствию нормам TEI \autocite[575--577]{tei}.

\subsection{Обновление до Unicode~6.1}

Выше было упомянуто, что стараниями В.",А.~Алексеева между собственной кодировкой исторических символов кириллицы, принятой в проекте СКАТ, и стандартом Unicode~5.1 было установлено практически полное взаимно однозначное соответствие. Исключение составляет ряд выносных букв (\textsc{ь}, \textsc{ы}, \textsc{у}, \textsc{u}, \textsc{и}, \textsc{i}, \textsc{w}, а также \textsc{е} широкое и его йотированный аналог), к моменту окончания диссертационного исследования В.",А.~Алексеева не успевших войти в Unicode; однако им отмечалось, что предложение по внесению соответствующих дополнений в стандарт к тому времени уже было составлено и находилось на рассмотрении одной из рабочих групп ISO (\foreignlanguage{english}{International Organization for Standardization}) \autocite[21]{alexeev:2011}.

В обновлении Unicode до версии~6.1, увидевшем свет в январе 2012~г., данное предложение \autocite{proposal:2010} было принято: все перечисленные выносные буквы стали доступны в составе блока \foreignlanguage{english}{Cyrillic Extended-B}. Следовательно, отныне кодировать недостающие символы как сущности и приписывать им формальную декларацию нет необходимости, и всем им были поставлены в соответствие их интернациональные эквиваленты.

\subsection{Нормализация и лемматизация}

В обновлённое XML"=представление были интегрированы все технологические наработки, составившие предмет обсуждения предыдущей главы. А именно: \begin{inparaenum}[(1)]
    \item в содержимое атрибута \texttt{@reg} словоформы отныне записываются не просто в упрощённой графике, но в нормализованном виде (см.\ \ref{sec:norm});
    \item морфологически размеченные словоформы дополнительно снабжаются атрибутом \texttt{@ana}, где позиции разметки последовательно перечислены через точку с запятой;
    \item леммы в случае их успешного определения попадают в атрибут \texttt{@lemma}.
\end{inparaenum}

Приложение~\ref{app:xml} иллюстрирует фрагмент XML"=представления начального фрагмента жития Димитрия Прилуцкого.

\section{Проблемы совместимости с TEI}



\section*{Выводы}
\addcontentsline{toc}{section}{Выводы}



\chapter*{Заключение}
\addcontentsline{toc}{chapter}{Заключение}

Итогом проделанной выпускной квалификационной работы стало решение следующих задач.

\begin{asparaenum}
    \item Был произведён обзор систем представления грамматических данных в существующих ныне восточнославянских исторических корпусах, в результате чего была обоснована актуальность проблемы лемматизации текстов в корпусе СКАТ для приведения последнего в соответствие с глобальным уровнем развития аналогичных проектов.

    \item Были изучены основные трудности церковнославянского именного словоизменения, сопряжённые с задачей корректного определения леммы по заданной словоформе, и разработаны способы их формализации в ходе программной разработки алгоритма лемматизации морфологически размеченных житий.

    \item Было усовершенствовано XML"=представление текстов корпуса, что позволило далее загрузить их на платформу TXM,~"--- не только выведя результаты работы алгоритма лемматизации на непосредственно практический уровень, но и расширив пользовательские возможности для практической работы с корпусом в целом.
\end{asparaenum}


% Библиография
\printbibliography[heading = bibintoc]

% Приложения
\appendix
\chapter{Выдача компонента для лемматизации}
\label{app:output}

\renewcommand{\aboverulesep}{0.1pt}\renewcommand{\belowrulesep}{0.1pt}\small

\begin{landscape}
    \begin{tabularx}{\textwidth}{p{5cm}p{1.5cm}p{1.5cm}p{1.5cm}p{1.5cm}p{1.5cm}p{1.5cm}p{5cm}}
        \toprule \endfirsthead
        \midrule \endhead
        ПОЖИВШИ(Х), & прич & м & прош & род & мн & м & ПОЖИТИ \\ \midrule
        W(Т)ВЕРГЬ\&ШИ(Х). & прич & м & прош & род & мн & м & ОТВЕРГНУТИ \\ \midrule
        БUДUЩАА & прич & м & наст & род & ед & ж & БЫТИ \\ \midrule
        UГОТОВА & гл & изъяв & аор гл & 3 & ед &  & УГОТОВАТИ \\ \midrule
        МОЖААХU\& & гл & изъяв & имп & 3 & мн &  & МОЩИ \\ \midrule
        ТЩАЩЕСR, & прич/в & en & наст & им & мн & м & ТЩИТИСЯ \\ \midrule
        ПРЕ(Д)ЛАГАХU. & гл & изъяв & имп & 3 & мн &  & ПРЕДЛАГАТИ \\ \midrule
        ПОС+ЩЕНЫМЪ. & прич & тв & прош & дат & мн & м & ПОС+ТИТИ \\ \midrule
        ПОДОБАЕТЬ & гл & изъяв & н/б & 3 & ед & 3 & ПОДОБАТИ \\ \midrule
        ДАРОВАННАА & прич & тв & прош & вин & мн & ср & ДАРОВАТИ \\ \midrule
        СЛGЖАЩЕ & прич & jo/en & наст & им & мн & м & СЛУЖИТИ \\ \midrule
        ПОДОБАЕТЪ & гл & изъяв & н/б & 3 & ед & 3 & ПОДОБАТИ \\ \midrule
        ВИД+ХОМЪ.\& & гл & изъяв & аор гл & 1 & мн &  & ВИД+ТИ \\ \midrule
        СЛЫШАХОМЪ. & гл & изъяв & аор гл & 1 & мн &  & СЛЫШАТИ \\ \midrule
        БGДUЩИ(Х) & прич & м & наст & род & мн & ср & БЫТИ \\ \midrule
        СВ+ДUЩИМЪ. & прич & м & наст & дат & мн & м & СВ+ДАТИ \\ \midrule
        ИМUЩИМЪ. & прич & м & наст & дат & мн & м & ИМАТИ \\ \midrule
        ИСПРАВЛRЮЩИ & прич & jo & наст & им & мн & м & ИСПРАВЛЯТИ \\ \midrule
        ПОМRНGТАА\& & прич & тв & прош & вин & мн & ср & ПОМЯНУТИ \\ \midrule
        ИЗЫДЕТЪ & гл & изъяв & н/б & 3 & ед & 1 & ИЗЫТИ \\ \midrule
        ВЪ\&СПОМRНЕМЬ. & гл & изъяв & н/б & 1 & мн & 2 & ВЪСПОМЯНУТИ \\ \midrule
        ЛЕНRЩИСR\& & прич/в & jo & наст & им & мн & м & ЛЕНИТИСЯ \\ \midrule
        UМОЛЧИМЪ. & гл & изъяв & н/б & 1 & мн & 4 & УМОЛЧАТИ \\ \midrule
        WДАРИ.\& & гл & изъяв & аор гл & 3 & ед &  & ОДАРИТИ \\ \midrule
        ВЪПIЮТЪ. & гл & изъяв & н/б & 3 & мн & 3 & ВЪПИТИ \\ \midrule
        ДЕ\&РЗНЕТЬ & гл & изъяв & н/б & 3 & ед & 2 & ДЕРЗНУТИ \\ \midrule
        ИМUЩЕ. & прич & jo/en & наст & им & мн & м & ИМАТИ \\ \midrule
        ПРЕ(Д)ЛЕЖИТЪ. & гл & изъяв & н/б & 3 & ед & 4 & ПРЕДЛЕЖАТИ \\ \midrule
        UДИВИШАСR, & гл/в & изъяв & аор гл & 3 & мн &  & УДИВИТИСЯ \\ \midrule
        ПОХВАЛИША & гл & изъяв & аор гл & 3 & мн &  & ПОХВАЛИТИ \\ \midrule
        НАПИСАНА & прич & o & прош & им & мн & ср & НАПИСАТИ \\ \midrule
        СUТЬ & гл & изъяв & н/б & 3 & мн & 5 & БЫТИ \\ \midrule
        ВЗЕМШЕ. & прич & en & прош & им & мн & м & ВЗЯТИ \\ \midrule
        ШЕСТВОВАША, & гл & изъяв & аор гл & 3 & мн &  & ШЕСТВОВАТИ \\ \midrule
        ВНИДОША\& & гл & изъяв & аор нов & 3 & мн &  & ВНИТИ \\ \midrule
        ПИШЕ(Т)\& & гл & изъяв & н/б & 3 & ед & 3 & ПИСАТИ \\ \midrule
        БUДИТЕ & гл & повел & 2 & мн & 1 &  & БЫТИ \\ \midrule
        ПРАВW\&ЖИВUЩИМЪ. & прич & м & наст & дат & мн & м & ПРАВОЖИТИ \\ \midrule
        НАПИШИ & гл & повел & 2 & ед & 3 &  & НАПИСАТИ \\ \midrule
        РЕЧЕ & гл & изъяв & аор пр & 3 & ед &  & РЕЩИ \\ \midrule
        СТРАШИТЪ & гл & изъяв & н/б & 3 & ед & 4 & СТРАШИТИ \\ \midrule
        СТRЖА(Х) & гл & изъяв & аор гл & 1 & ед &  & СТЯЖАТИ \\ \midrule
        НЕИСПРАВЛЕ\&НА & прич & o & прош & вин & мн & ср & ИСПРАВИТИ \\ \midrule
        ДОСТИЖЕ & гл & изъяв & аор пр & 3 & ед &  & ДОСТИГНУТИ \\ \midrule
        ИМ+А; & прич & jo & наст & им & ед & м & ИМ+ТИ \\ \midrule
        ПОС+ТИ & гл & изъяв & аор гл & 3 & ед &  & ПОС+ТИТИ \\ \midrule
        ПРОЯВИ\& & гл & изъяв & аор гл & 3 & ед &  & ПРОЯВИТИ \\ \midrule
        ВИДRЩЕ & прич & jo/en & наст & им & мн & м & ВИТИ \\ \midrule
        ~РА(З)СТRЩU <РАСТRЩU> & прич & ja & наст & вин & ед & ж & РАСТИТИ \\ \midrule
        ПОДАА, & прич & jo & наст & им & ед & м & ПОДАТИ \\ \midrule
        НАРИЦАЮЩИМЪ. & прич & м & наст & дат & мн & м & НАРИЦАТИ \\ \midrule
        ПРИВОДRЩИМЪ & прич & м & наст & дат & мн & м & ПРИВОДИТИ \\ \midrule
        ПОКАЖЕТЪ; & гл & изъяв & н/б & 3 & ед & 3 & ПОКАЗАТИ \\ \midrule
        ВСЫЛАЮЩЕ, & прич & jo/en & наст & им & мн & м & ВСЫЛАТИ \\ \midrule
        W(Т)ВЕРЗЕШИ. & гл & изъяв & н/б & 2 & ед & 1 & ОТВЕРЗТИ \\ \midrule
        ВЪЗВ+СТRТЪ & гл & изъяв & н/б & 3 & мн & 4 & ВЪЗВ+СТИТИ \\ \midrule
        НАСТОИТЪ & гл & изъяв & н/б & 3 & ед & 4 & НАСТОЯТИ \\ \midrule
        ПРIИД+ТЕ\& & гл & повел & 2 & мн & 1 &  & ПРИИТИ \\ \midrule
        СНИДЕТЕСR & гл/в & повел & 2 & мн & 1 &  & СНИТИСЯ \\ \midrule
        ПРIИ\&Д+ТЕ & гл & повел & 2 & мн & 1 &  & ПРИИТИ \\ \midrule
        ПРIИД+ТЕ & гл & повел & 2 & мн & 1 &  & ПРИИТИ \\ \midrule
        W(Т)ВЕРГШЕ. & прич & jo/en & прош & им & мн & м & ОТВЕРГНУТИ \\ \midrule
        W(Т)РИНUВШЕ. & прич & jo/en & прош & им & мн & м & ОТРИНУТИ \\ \midrule
        WЧИZ 203 СТИВШЕСR & прич/в & jo/en & прош & им & мн & м & ОЧИСТИТИСЯ \\ \midrule
        ПРI\&КЛОНИТЕ & гл & повел & 2 & мн & 4 &  & ПРИКЛОНИТИ \\ \midrule
        ПРIИДЕ & гл & изъяв & аор пр & 3 & ед &  & ПРИИТИ \\ \midrule
        ВОСIА. & гл & изъяв & аор гл & 3 & ед &  & ВОСИЯТИ \\ \midrule
        UВ+МЫ & гл & изъяв & н/б & 1 & мн & 5 & УВ+ДАТИ \\ \midrule
        ПОХОДИВЫИ & прич & м & прош & им & ед & м & ПОХОДИТИ \\ \midrule
        WБ+\&ТОВАННАГО & прич & тв & прош & род & ед & м & ОБ+ТОВАТИ \\ \midrule
        ПРОСВ+ТИСR & гл/в & изъяв & аор гл & 3 & ед &  & ПРОСВ+ТИТИСЯ \\ \midrule
        ИЗРАСТЕ & гл & изъяв & аор пр & 3 & ед &  & ИЗРАСТНУТИ \\ \midrule
        ЦВ+ТD\&ЩИМЪ. & прич & м & наст & дат & мн & м & ЦВЕСТИ \\ \midrule
        РОДИСR & гл/в & изъяв & аор гл & 3 & ед &  & РОДИТИСЯ \\ \midrule
        ВЪ\&СПИТАНЪ & прич & o & прош & им & ед & м & ВЪСПИТАТИ \\ \midrule
        БЫ(с) & гл & изъяв & аор гл & 3 & ед &  & БЫТИ \\ \midrule
        НЕГЫБЛЮЩИМЪ & прич & м & наст & тв & ед & ср & ГИБАТИ \\ \midrule
        СЫИ & прич & м & наст & им & ед & м & БЫТИ \\ \midrule
        ВНИМАШЕ & гл & изъяв & имп & 3 & ед &  & ВНИМАТИ \\ \midrule
        ТВОРRЩИХЬ\& & прич & м & наст & род & мн & м & ТВОРИТИ \\ \midrule
        ТРЕ\&БОВАВШИ(Х). & прич & м & прош & род & мн & м & ТРЕБОВАТИ \\ \midrule
        ВЗЫСЬ\&КАА. & прич & jo & наст & им & ед & м & ВЗЫСЬКАТИ \\ \midrule
        ПОDЧААСR\& & прич/в & jo & наст & им & ед & м & ПОУЧАТИСЯ \\ \midrule
        ВЪЗЛЮБИ, & гл & изъяв & аор гл & 3 & ед &  & ВЪЗЛЮБИТИ \\ \midrule
        ПОЗНА & гл & изъяв & аор гл & 3 & ед &  & ПОЗНАТИ \\ \midrule
        ВЪ(З)ЛЮ\&БИ. & гл & изъяв & аор гл & 3 & ед &  & ВЪЗЛЮБИТИ \\ \midrule
        ПОВИНGRСR & прич/в & jo & наст & им & ед & м & ПОВИНОВАТИСЯ \\ \midrule
        ЖИВЫИ & прич & м & наст & им & ед & м & ЖИТИ \\ \midrule
        НЕБРЕЖАШЕ.\& & гл & изъяв & имп & 3 & ед &  & НЕБРЕЩИ \\ \midrule
        ТВОРRХU & гл & изъяв & имп & 3 & мн &  & ТВОРИТИ \\ \midrule
        СЪБI\&РАЮЩЕ & прич & jo/en & наст & им & мн & м & СЪБИРАТИ \\ \midrule
        ВЪЗЫСКАR,\& & прич & jo & наст & им & ед & м & ВЪЗЫСКАТИ \\ \midrule
        РА(З)СМОТРИВЪ & прич & jo & прош & им & ед & м & РАЗСМОТРИТИ \\ \midrule
        СКОРОМИНUЩЕЕ & прич & м & наст & вин & ед & ср & СКОРОМИНУТИ \\ \midrule
        ПОМИНАА & прич & jo & наст & им & ед & м & ПОМИНАТИ \\ \midrule
        РЕКШАГО & прич & м & прош & вин/род & ед & м & РЕЩИ \\ \midrule
        WСТАВИТЬ & гл & изъяв & н/б & 3 & ед & 4 & ОСТАВИТИ \\ \midrule
        ПРIИМЕТЪ. & гл & изъяв & н/б & 3 & ед & 1 & ПРИЯТИ \\ \midrule
        НАСЛ+ДИТЬ. & гл & изъяв & н/б & 3 & ед & 4 & НАСЛ+ДИТИ \\ \midrule
        РЕЧЕ & гл & изъяв & аор пр & 3 & ед &  & РЕЩИ \\ \midrule
        WСТАВИТСR & гл/в & изъяв & н/б & 3 & ед & 4 & ОСТАВИТИСЯ \\ \midrule
        ПРЕДИРЕЧЕННЫ(Х) & прич & тв & прош & род & мн & ср & ПРЕДИРЕЩИ \\ \midrule
        МОЖЕТЪ & гл & изъяв & н/б & 3 & ед & 1 & МОЩИ \\ \midrule
        ЛЕ\&ЖАЩАА. & прич & м & наст & вин & мн & ср & ЛЕЖАТИ \\ \midrule
        ПРIИДЕ & гл & изъяв & аор пр & 3 & ед &  & ПРИИТИ \\ \midrule
        ВЪЗЛЮБИ & гл & изъяв & аор гл & 3 & ед &  & ВЪЗЛЮБИТИ \\ \midrule
        ВЪСПРIЕМЛЕТЪ & гл & изъяв & н/б & 3 & ед & 3 & ВЪСПРИЯТИ \\ \midrule
        ПОНЕСЫИ. & прич & м & прош & им & ед & м & ПОНЕСТИ \\ \midrule
        ХО\&ТR & прич & jo & наст & им & ед & м & ХОТ+ТИ \\ \midrule
        ВМЕНRА, & прич & jo & наст & им & ед & м & ВМЕНЯТИ \\ \midrule
        WСТАВИ. & гл & изъяв & аор гл & 3 & ед &  & ОСТАВИТИ \\ \midrule
        СЫ(И)\& & прич & м & наст & им & ед & м & БЫТИ \\ \midrule
        ВХОДИТЪ & гл & изъяв & н/б & 3 & ед & 4 & ВХОДИТИ \\ \midrule
        ЗОВОМЫИ.\& & прич & тв & наст & им & ед & м & ЗВАТИ \\ \midrule
        ПОСТРИ\&ЗАЕТЪ. & гл & изъяв & н/б & 3 & ед & 3 & ПОСТРИЗАТИ \\ \midrule
        ЛЮБЛRШЕ.\& & гл & изъяв & имп & 3 & ед &  & ЛЮБИТИ \\ \midrule
        ПОСЛ+ДGА & прич & jo & наст & им & ед & м & ПОСЛ+ДОВАТИ \\ \midrule
        ПОDЧААСR & прич/в & jo & наст & им & ед & м & ПОУЧАТИСЯ \\ \midrule
        БЫ(с) & гл & изъяв & аор гл & 3 & ед &  & БЫТИ \\ \midrule
        ХОТRЩАА & прич & м & наст & им & мн & ж & ХОТ+ТИ \\ \midrule
        ПОРUЧЕННЫИ & прич & тв & прош & вин & ед & м & ПОРУЧИТИ \\ \midrule
        ПРIИМАЕТЪ.\& & гл & изъяв & н/б & 3 & ед & 3 & ПРИИМАТИ \\ \midrule
        ТВОРR & прич & jo & наст & им & ед & м & ТВОРИТИ \\ \midrule
        НЕU\&КРЫВАЕМЫИ & прич & тв & наст & им & ед & м & УКРЫВАТИ \\ \midrule
        БЫ(с) & гл & изъяв & аор гл & 3 & ед &  & БЫТИ \\ \midrule
        СТО(А).\& & прич & jo & наст & им & ед & м & СТОЯТИ \\ \midrule
        СОСТАВЛRЕТЪ & гл & изъяв & н/б & 3 & ед & 3 & СОСТАВЛЯТИ \\ \midrule
        НАРИЦАЕМ+ & прич & o & наст & мест & ед & ср & НАРИЦАТИ \\ \midrule
        ПОСТАВИ & гл & изъяв & аор гл & 3 & ед &  & ПОСТАВИТИ \\ \midrule
        UСТРОИ. & гл & изъяв & аор гл & 3 & ед &  & УСТРОИТИ \\ \midrule
        WСЩАА\#. & прич & jo & наст & им & ед & м & ОСВЯЩАТИ \\ \midrule
        БЫВАШЕ. & гл & изъяв & имп & 3 & ед &  & БЫВАТИ \\ \midrule
        ЛЮБЛRШЕ. & гл & изъяв & имп & 3 & ед &  & ЛЮБИТИ \\ \midrule
        ПОUЧАR.\& & прич & jo & наст & им & ед & м & ПОУЧАТИ \\ \midrule
        ПРИХОДR\&ЩЕ & прич & jo/en & наст & им & мн & м & ПРИХОДИТИ \\ \midrule
        ХОТRЩIИ\& & прич & м & наст & им & мн & м & ХОТ+ТИ \\ \midrule
        WСТАВЛЬШИ(Х)\& & прич & м & прош & вин/род & мн & м & ОСТАВИТИ \\ \midrule
        WБЛАЧАR. & прич & jo & наст & им & ед & м & ОБЛАЧАТИ \\ \midrule
        ЧАЮЩИХЪ\& & прич & м & наст & вин/род & мн & м & ЧАЯТИ \\ \midrule
        ПОВИНUЮЩЕСR & прич/в & jo/en & наст & им & мн & м & ПОВИНОВАТИСЯ \\ \midrule
        БЫВШU & прич & jo & прош & дат & ед & м & БЫТИ \\ \midrule
        РОДИСR.\& & гл/в & изъяв & аор гл & 3 & ед &  & РОДИТИСЯ \\ \midrule
        ПРИЛUЧИСR & гл/в & изъяв & аор гл & 3 & ед &  & ПРИЛУЧИТИСЯ \\ \midrule
        БRШЕ\& & гл & изъяв & имп & 3 & ед &  & БЫТИ \\ \midrule
        СIАШЕ & гл & изъяв & имп & 3 & ед &  & СИЯТИ \\ \midrule
        ИЗВОЛИ. & гл & изъяв & аор гл & 3 & ед &  & ИЗВОЛИТИ \\ \midrule
        ХОТR & прич & jo & наст & им & ед & м & ХОТ+ТИ \\ \midrule
        ИМ+АШЕ & гл & изъяв & имп & 3 & ед &  & ИМ+ТИ \\ \midrule
        ПРОСВ+ЩАШЕ(с). & гл/в & изъяв & имп & 3 & ед &  & ПРОСВ+ТИТИСЯ \\ \midrule
        ЦВ+ТUЩЕ. & прич & jo & наст & им/вин & ед & ср & ЦВЕСТИ \\ \midrule
        ИМ+R. & прич & jo & наст & им & ед & м & ИМ+ТИ \\ \midrule
        ПОКРЫВАR & прич & jo & наст & им & ед & м & ПОКРЫВАТИ \\ \midrule
        ХОЖАШЕ. & гл & изъяв & имп & 3 & ед &  & ХОДИТИ \\ \midrule
        БЕС+ДUR. & прич & jo & наст & им & ед & м & БЕС+ДОВАТИ \\ \midrule
        НОШАШЕСR. & гл/в & изъяв & имп & 3 & ед &  & НОСИТИСЯ \\ \midrule
        БRШЕ. & гл & изъяв & имп & 3 & ед &  & БЫТИ \\ \midrule
        ДИВRЩЕСR & прич/в & jo/en & наст & им & мн & м & ДИВИТИСЯ \\ \midrule
        ГЛА\&ГОЛАШЕ. & гл & изъяв & имп & 3 & ед &  & ГЛАГОЛАТИ \\ \midrule
        ПОГUБИ & гл & изъяв & аор гл & 3 & ед &  & ПОГУБИТИ \\ \midrule
        БЫВА\&ШЕ & гл & изъяв & имп & 3 & ед &  & БЫВАТИ \\ \midrule
        ПРАВRЩЕМU & прич & м & наст & дат & ед & м & ПРАВИТИ \\ \midrule
        ПРИХОЖАШЕ. & гл & изъяв & имп & 3 & ед &  & ПРИХОДИТИ \\ \midrule
        ПРIИМАШЕ. & гл & изъяв & имп & 3 & ед &  & ПРИИМАТИ \\ \midrule
        ИМU\&ЩЕ. & прич & jo/en & наст & им & мн & м & ИМАТИ \\ \midrule
        СUЩЕ & прич & jo/en & наст & им & мн & м & БЫТИ \\ \midrule
        W(Т)ХОЖАШЕ & гл & изъяв & имп & 3 & ед &  & ОТХОДИТИ \\ \midrule
        ВЪ\&НИМАЮЩЕ & прич & jo/en & прош & им & мн & м & ВЪНИМАЮЩИТИ \\ \midrule
        ХОДRЩЕИ.\& & прич & м & наст & им & мн & м & ХОДИТИ \\ \midrule
        БRШЕ & гл & изъяв & имп & 3 & ед &  & БЫТИ \\ \midrule
        ПРИХО\&ДRЩЕ. & прич & jo/en & наст & им & мн & м & ПРИХОДИТИ \\ \midrule
        ЕСТЬ & гл & изъяв & н/б & 3 & ед & 5 & БЫТИ \\ \midrule
        ПРИХОДRИ & прич & м & наст & им & ед & м & ПРИХОДИТИ \\ \midrule
        МО\&ЖЕ & гл & изъяв & аор пр & 3 & ед &  & МОЩИ \\ \midrule
        ПРЕДИРЕКОХО(М) & гл & изъяв & аор нов & 1 & мн &  & ПРЕДИРЕЩИ \\ \midrule
        ТААШЕ & гл & изъяв & имп & 3 & ед &  & ТАЯТИ \\ \midrule
        СЛЫШАШЕ & гл & изъяв & имп & 3 & ед &  & СЛЫШАТИ \\ \midrule
        БЫ\&ВШЕЕ & прич & м & прош & вин & ед & ср & БЫТИ \\ \midrule
        ПРИЛUЧИ & гл & изъяв & аор гл & 3 & ед &  & ПРИЛУЧИТИ \\ \midrule
        ВИД+ & гл & изъяв & аор гл & 3 & ед &  & ВИД+ТИ \\ \midrule
        ГОТОВRЩUСR & прич/в & jo & наст & дат & ед & м & ГОТОВИТИСЯ \\ \midrule
        ВИД+ & гл & изъяв & аор гл & 3 & ед &  & ВИД+ТИ \\ \midrule
        WСКОРБИСR & гл/в & изъяв & аор гл & 3 & ед &  & ОСКОРБИТИСЯ \\ \midrule
        WСIА & гл & изъяв & аор гл & 3 & ед &  & ОСИЯТИ \\ \midrule
        ЦВ+Z -206 ТUЩАГО. & прич & м & наст & род & ед & ср & ЦВЕСТИ \\ \midrule
        НАПАДЕ & гл & изъяв & аор пр & 3 & ед &  & НАПАСТИ \\ \midrule
        БЫСТЬ & гл & изъяв & аор гл & 3 & ед &  & БЫТИ \\ \midrule
        РАСЛАБЛЕНА & прич & a & прош & им & ед & ж & РАСЛАБИТИ \\ \midrule
        В+МЫ & гл & изъяв & н/б & 1 & мн & 5 & В+ДАТИ \\ \midrule
        ХОТ+ & гл & изъяв & аор гл & 3 & ед &  & ХОТ+ТИ \\ \midrule
        ПРИВЕДОША & гл & изъяв & аор нов & 3 & мн &  & ПРИВЕСТИ \\ \midrule
        СUЩU. & прич & ja & наст & вин & ед & ж & БЫТИ \\ \midrule
        МОЛИША & гл & изъяв & аор гл & 3 & мн &  & МОЛИТИ \\ \midrule
        КАЮЩИСR & прич/в & ja & наст & им & ед & ж & КАЯТИСЯ \\ \midrule
        ПОЧИВАЮЩАГО. & прич & м & наст & род & ед & м & ПОЧИВАТИ \\ \midrule
        ТВОРRШЕ. & гл & изъяв & имп & 3 & ед &  & ТВОРИТИ \\ \midrule
        ПРИШЕДЪ\& & прич & jo & прош & им & ед & м & ПРИИТИ \\ \midrule
        ГЛА\# & гл & изъяв & аор гл & 3 & ед &  & ГЛАГОЛАТИ \\ \midrule
        ВЪСХОТ+ & гл & изъяв & аор гл & 2 & ед &  & ВЪСХОТ+ТИ \\ \midrule
        ПО\&ДОБАЕТЪ. & гл & изъяв & н/б & 3 & ед & 3 & ПОДОБАТИ \\ \midrule
        ПОUЧИВЪ & прич & jo & прош & им & ед & м & ПОУЧИТИ \\ \midrule
        ИМU\&ЩЕ & прич & jo & наст & им & ед & ж/ср & None \\ \midrule
        ПОДАСТЬ & гл & изъяв & аор гл & 3 & ед &  & ПОДАТИ \\ \midrule
        СЪТВОРИ & гл & изъяв & аор гл & 3 & ед &  & СЪТВОРИТИ \\ \midrule
        БЫ(с) & гл & изъяв & аор гл & 3 & ед &  & БЫТИ \\ \midrule
        ВЪСТАВЪ & прич & jo & прош & им & ед & м & ВЪСТАТИ \\ \midrule
        W(Т)ИДЕ & гл & изъяв & аор пр & 3 & ед &  & ОТИТИ \\ \midrule
        СЛА\&ВR & прич & jo & наст & им & ед & м & СЛАВИТИ \\ \midrule
        ЗА\&ЗИРАЮЩЕ & прич & ja/en & наст & им & ед/мн & ж/м & ЗАЗИРАТИ \\ \midrule
        ПРИЛU\&ЧИВШИ(Х)СR. & прич/в & м & прош & род & мн & ср & ПРИЛУЧИТИСЯ \\ \midrule
        СЛЫШАВШЕ & прич & jo/en & прош & им & мн & м & СЛЫШАТИ \\ \midrule
        РАДОВАШЕСR & гл/в & изъяв & имп & 3 & ед &  & РАДОВАТИСЯ \\ \midrule
        СЛАВR\&ЩЕ & прич & jo/en & наст & им & мн & м & СЛАВИТИ \\ \midrule
        ДАВШАГО & прич & м & прош & вин/род & ед & м & ДАТИ \\ \midrule
        БОЛRЩИМЪ. & прич & м & наст & дат & мн & м & БОЛ+ТИ \\ \midrule
        ПРОХОЖАШЕ. & гл & изъяв & имп & 3 & ед &  & ПРОХОДИТИ \\ \midrule
        ПРЕДЕРЖАИ & прич & м & наст & им & ед & м & ПРЕДЕРЖАТИ \\ \midrule
        СЫ(И). & прич & м & наст & им & ед & м & БЫТИ \\ \midrule
        ПОКАЗАВЫИ & прич & м & прош & им & ед & м & ПОКАЗАТИ \\ \midrule
        СЛЫШАВЪ & прич & jo & прош & им & ед & м & СЛЫШАТИ \\ \midrule
        РАДОВАШЕ(с).\& & гл/в & изъяв & имп & 3 & ед &  & РАДОВАТИСЯ \\ \midrule
        ПОЧИТАШЕ & гл & изъяв & имп & 3 & ед &  & ПОЧИТАТИ \\ \midrule
        ПРIЕМЛR. & прич & jo & наст & им & ед & м & ПРИЯТИ \\ \midrule
        UМОЛИ & гл & изъяв & аор гл & 3 & ед &  & УМОЛИТИ \\ \midrule
        ПРОСВ+ТИТЪ & гл & изъяв & н/б & 3 & ед & 4 & ПРОСВ+ТИТИ \\ \midrule
        ПОДО\&БАШЕ & гл & изъяв & имп & 3 & ед &  & ПОДОБИТИ \\ \midrule
        ПОЧИТАШЕ.\& & гл & изъяв & имп & 3 & ед &  & ПОЧИТАТИ \\ \midrule
        ВЪ\&ЗДАА. & прич & jo & наст & им & ед & м & ВЪЗДАТИ \\ \midrule
        ПРЕМОЖЕ & гл & изъяв & аор пр & 3 & ед &  & ПРЕМОЩИ \\ \midrule
        ВИДR & прич & jo & наст & им & ед & м & ВИТИ \\ \midrule
        НАХОДRЩЕИ & прич & м & наст & мест & ед & ж & НАХОДИТИ \\ \midrule
        ВМ+\&НRШЕ. & гл & изъяв & имп & 3 & ед &  & ВМ+НЯТИ \\ \midrule
        РЕЧЕ; & гл & изъяв & аор пр & 3 & ед &  & РЕЩИ \\ \midrule
        ЕСТЬ & гл & изъяв & н/б & 3 & ед & 5 & БЫТИ \\ \midrule
        СМИРRЮЩЕСR & прич/в & jo & наст & им & мн & м & СМИРЯТИСЯ \\ \midrule
        НАРЕКUТСR. & гл/в & изъяв & н/б & 3 & мн & 1 & НАРЕЩИСЯ \\ \midrule
        ИЗБЫ\&ВАША & гл & изъяв & аор гл & 3 & мн &  & ИЗБЫВАТИ \\ \midrule
        ИЗБ+ЖА. & гл & изъяв & аор гл & 3 & ед &  & ИЗБ+ЖАТИ \\ \midrule
        WСТА\&ВЛRЕ(Т). & гл & изъяв & н/б & 3 & ед & 3 & ОСТАВЛЯТИ \\ \midrule
        ИЗЫ\&ДЕ. & гл & изъяв & аор пр & 3 & ед &  & ИЗЫТИ \\ \midrule
        ПОR & гл & изъяв & аор гл & 3 & ед &  & ПОЯТИ \\ \midrule
        СВИД+ТЕЛЬСТВОВА & гл & изъяв & аор гл & 3 & ед &  & СВИД+ТЕЛЬСТВОВАТИ \\ \midrule
        БЫ(с)\& & гл & изъяв & аор гл & 3 & ед &  & БЫТИ \\ \midrule
        ПРОХОДR. & прич & jo & наст & им & ед & м & ПРОХОДИТИ \\ \midrule
        БЫ(с) & гл & изъяв & аор гл & 3 & ед &  & БЫТИ \\ \midrule
        ПРIАСТА; & гл & изъяв & аор гл & 3 & дв &  & ПРИЯТИ \\ \midrule
        ПРЕ(Д)ЛЕЖАЩЕЕ\& & прич & м & наст & вин & ед & ср & ПРЕДЛЕЖАТИ \\ \midrule
        ВЪЗВРАТИМСR, & гл/в & изъяв & н/б & 1 & мн & 4 & ВЪЗВРАТИТИСЯ \\ \midrule
        ИЗЫДЕ & гл & изъяв & аор пр & 3 & ед &  & ИЗЫТИ \\ \midrule
        UСТРЕМИВШЕСR & прич/в & jo/en & прош & им & мн & м & УСТРЕМИТИСЯ \\ \midrule
        ПРОХОДИZ -208 ВШЕ & прич & jo/en & прош & им & мн & м & ПРОХОДИТИ \\ \bottomrule
    \end{tabularx}
\end{landscape}

\normalsize

\chapter{Пример обновлённого XML-представления жития Димитрия Прилуцкого}
\label{app:xml}

\begin{Verbatim}[fontsize=\scriptsize, gobble=4, breaklines, numbers=left]
    <?xml version="1.0" encoding="UTF-8"?>
    <TEI xmlns="http://www.tei-c.org/ns/1.0">
      <teiHeader>
        <fileDesc>
          <titleStmt>
            <title>Житие Димитрия Прилуцкого</title>
            <respStmt>
              <resp>редактор</resp>
              <name>А. С. Герд</name>
            </respStmt>
            <respStmt>
              <resp>составитель</resp>
              <name>И. В. Азарова</name>
            </respStmt>
            <respStmt>
              <resp>составитель</resp>
              <name>Е. Л. Алексеева</name>
            </respStmt>
            <respStmt>
              <resp>составитель</resp>
              <name>Л. А. Захарова</name>
            </respStmt>
            <respStmt>
              <resp>составитель</resp>
              <name>К. Н. Лемешев</name>
            </respStmt>
            <respStmt>
              <resp>конвертация в формат XML-TEI</resp>
              <name>К. В. Сипунин</name>
            </respStmt>
          </titleStmt>
          <publicationStmt>
            <publisher>Издательство Санкт-Петербургского университета</publisher>
            <pubPlace>Санкт-Петербург</pubPlace>
            <date>2003</date>
            <idno type="ISBN">5-288-03308-0</idno>
          </publicationStmt>
          <sourceDesc>
            <bibl>РНБ, Соф. 1361; XVI в.</bibl>
          </sourceDesc>
        </fileDesc>
      </teiHeader>
      <text><body><ab>
        <pb n="201"/><lb n="1"/>
        <w xml:id="DmPrlc.1" ana="сущ;jo;род;ед;м" lemma="месяць" reg="месяца" src="М(с)ЦА">м҇ⷭца</w>
        <w xml:id="DmPrlc.2" ana="сущ;jo;род;ед;м" lemma="февраль" reg="февраля" src="FЕВРАЛR">ѳевралѧ</w>
        <pc xml:id="DmPrlc.3">.</pc>
        <num><w xml:id="DmPrlc.4" reg="11" src="АI#">аї҃</w></num>
        <pc xml:id="DmPrlc.5">.</pc>
        <w xml:id="DmPrlc.6" ana="сущ;jo;им;ед;ср" lemma="житие" reg="житие" src="ЖИТIЕ">житїе</w>
        <w xml:id="DmPrlc.7" ana="прил;тв;род;ед;м" lemma="преподобныи" reg="преподобнаго" src="ПРПW(ДО)&amp;БНАГО">прпѡⷣⷪ<lb n="2"/>бнаго</w>
        <w xml:id="DmPrlc.8" ana="сущ;jo;род;ед;м" lemma="отець" reg="о(т)ца" src="W(Т)ЦА">ѿца</w>
        <w xml:id="DmPrlc.9" ana="мест;м;род;ед;м" lemma="нашь" reg="нашего" src="НШЕГО#">нш҃его</w>
        <name><w xml:id="DmPrlc.10" ana="сущ;jo;род;ед;м" lemma="*димитрии" reg="*димитрия" src="*ДIМИТРIА">дїмитрїа</w></name>
        <pc xml:id="DmPrlc.11">,</pc>
        <w xml:id="DmPrlc.12" ana="сущ;o;род;ед;м" lemma="игуменъ" reg="игумена" src="ИГU&amp;МЕНА">игѹ<lb n="3"/>мена</w>
        <name><w xml:id="DmPrlc.13" ana="прил;тв;род;ед;м" lemma="*прилуцкии" reg="*прилуцкаго" src="*ПРИЛUЦКАГО">прилѹцкаго</w></name>
        <pc xml:id="DmPrlc.14">.</pc>
        <name><w xml:id="DmPrlc.15" ana="прил;тв;род;ед;м" lemma="*вологоцкии" reg="*вологоцкаго" src="*ВОЛОГОЦКАГW">вологоцкагѡ</w></name>
        <lb n="4"/>
        <w xml:id="DmPrlc.16" ana="сущ;jo;род;ед;м" lemma="чюдотворець" reg="чюдотворца" src="ЧЮДОТВОРЦА">чюдотворца</w>
        <pc xml:id="DmPrlc.17">.</pc>
        <w xml:id="DmPrlc.18" ana="сущ;jo;им;ед;ср" lemma="творение" reg="творение" src="ТВОРЕНIЕ">творенїе</w>
        <w xml:id="DmPrlc.19" ana="мест;тв;род;ед;ж" lemma="тои" reg="тоя" src="ТОА">тоа</w>
        <w xml:id="DmPrlc.20" ana="част" lemma="же" reg="же" src="ЖЕ">же</w>
        <w xml:id="DmPrlc.21" ana="сущ;i;род;ед;ж" lemma="обитель" reg="обители" src="W&amp;БИТЕЛИ">ѡ<lb n="5"/>бители</w>
        <pc xml:id="DmPrlc.22">,</pc>
        <w xml:id="DmPrlc.23" ana="сущ;o;род;ед;м" lemma="игуменъ" reg="игумена" src="ИГUМЕНА">игѹмена</w>
        <name><w xml:id="DmPrlc.24" ana="сущ;jo;род;ед;м" lemma="*макарии" reg="*макария" src="*МАКАРIА">макарїа</w></name>
        <pc xml:id="DmPrlc.25">.</pc>
        <w xml:id="DmPrlc.26" ana="гл;повел;2;ед;4" reg="благослови" src="БЛ(с)ВИ">бл҇ⷭви</w>
        <w xml:id="DmPrlc.27" ana="сущ;jo/o;зв;ед;м" lemma="отець" reg="о(ч)" src="W(ч)">ѡ҇ⷱ</w>
        <pc xml:id="DmPrlc.28">;</pc>
        <lb n="6"/>
        <w xml:id="DmPrlc.29" ana="прил;тв;вин;ед;ср" lemma="святыи" reg="святое" src="СТОЕ#">ст҃ое</w>
        <w xml:id="DmPrlc.30" ana="сущ;jo;вин;ед;ср" lemma="житие" reg="житие" src="ЖИТIЕ">житїе</w>
        <w xml:id="DmPrlc.31" ana="прич;м;прош;род;мн;м" lemma="пожити" reg="пожившихъ" src="ПОЖИВШИ(Х)">пожившиⷯ</w>
        <pc xml:id="DmPrlc.32">,</pc>
        <w xml:id="DmPrlc.33" ana="прил;тв;род;мн;м" lemma="святыи" reg="святыхъ" src="СТЫ(Х)#">ст҃ыⷯ</w>
        <w xml:id="DmPrlc.34" ana="прил;тв;род;мн;м" lemma="преподобныи" reg="преподобныхъ" src="ПРЕПО&amp;ДОБНЫ(Х)">препо<lb n="7"/>добныⷯ</w>
        <w xml:id="DmPrlc.35" ana="сущ;jo/i;род;мн;м" lemma="мужь" reg="мужеи" src="МUЖЕИ">мѹжеи</w>
        <w xml:id="DmPrlc.36" ana="прил;тв;род;мн;м" lemma="великии" reg="великихъ" src="ВЕЛИКЫ(Х)">великыⷯ</w>
        <pc xml:id="DmPrlc.37">.</pc>
        <w xml:id="DmPrlc.38" ana="мест;м;им;мн;м" lemma="иже" reg="иже" src="ИЖЕ">иже</w>
        <w xml:id="DmPrlc.39" ana="пред" lemma="въ" reg="въ" src="В">в</w>
        <w xml:id="DmPrlc.40" ana="прил;тв;мест;мн;м" lemma="постныи" reg="постныхъ" src="ПО&amp;СТНЫ(Х)">по<lb n="8"/>стныⷯ</w>
        <w xml:id="DmPrlc.41" ana="сущ;o;мест;мн;м;*" lemma="подвигъ" reg="подвиз+хъ" src="ПОДВИЗ+(Х)">подвизѣⷯ</w>
        <w xml:id="DmPrlc.42" ana="прич;м;прош;род;мн;м" lemma="просияти" reg="просиявшихъ" src="ПРОСIАВШИ(Х)">просїавшиⷯ</w>
        <pc xml:id="DmPrlc.43">,</pc>
        <w xml:id="DmPrlc.44" ana="союз" lemma="и" reg="и" src="И">и</w>
        <w xml:id="DmPrlc.45" ana="пред" lemma="въ" reg="въ" src="В">в</w>
        <w xml:id="DmPrlc.46" ana="прил;тв;мест;мн;ж" lemma="непрестанныи" reg="непрестанныхъ" src="НЕ&amp;ПРЕСТАННЫ(Х)">не<lb n="9"/>престанныⷯ</w>
        <w xml:id="DmPrlc.47" ana="сущ;a;мест;мн;ж" lemma="молитва" reg="молитвахъ" src="МЛТВА(Х)#">мл҃тваⷯ</w>
        <w xml:id="DmPrlc.48" ana="пред" lemma="къ" reg="къ" src="КЪ">къ</w>
        <w xml:id="DmPrlc.49" ana="сущ;o;дат;ед;м" lemma="богъ" reg="богу" src="БU#">бѹ҃</w>
        <pc xml:id="DmPrlc.50">.</pc>
        <w xml:id="DmPrlc.51" ana="прил;а;вин;ед;ж" lemma="великии" reg="велику" src="ВЕЛИКD">великꙋ</w>
        <lb n="10"/>
        <w xml:id="DmPrlc.52" ana="сущ;a;вин;ед;ж" lemma="поб+да" reg="поб+ду" src="ПОБ+ДU">побѣдѹ</w>
        <w xml:id="DmPrlc.53" ana="пред" lemma="на" reg="на" src="НА">на</w>
        <w xml:id="DmPrlc.54" ana="сущ;o;вин;мн;м" lemma="врагъ" reg="враги" src="ВРАГЫ">врагы</w>
        <w xml:id="DmPrlc.55" ana="прич;jo/en;прош;им;мн;м" lemma="показати" reg="показавше" src="ПОКАЗАВШЕ">показавше</w>
        <pc xml:id="DmPrlc.56">.</pc>
        <w xml:id="DmPrlc.57" ana="прил;тв;вин;мн;ср" lemma="красныи" reg="красная" src="КРА&amp;СНАА">кра<lb n="11"/>снаа</w>
        <w xml:id="DmPrlc.58" ana="част" lemma="же" reg="же" src="ЖЕ">же</w>
        <w xml:id="DmPrlc.59" ana="союз" lemma="и" reg="и" src="И">и</w>
        <w xml:id="DmPrlc.60" ana="прил;тв;вин;мн;ср" lemma="суетныи" reg="суетная" src="СUЕТНАА">сѹетнаа</w>
        <w xml:id="DmPrlc.61" ana="сущ;o;род;ед;м" lemma="миръ" reg="мира" src="МИРА">мира</w>
        <w xml:id="DmPrlc.62" ana="мест;м;род;ед;м" lemma="сеи" reg="сего" src="СЕГО">сего</w>
        <w xml:id="DmPrlc.63" ana="прич;м;прош;род;мн;м" lemma="отвергнути" reg="о(т)вергшихъ" src="W(Т)ВЕРГЬ&amp;ШИ(Х)">ѿвергь<lb n="12"/>шиⷯ</w>
        <pc xml:id="DmPrlc.64">.</pc>
        <w xml:id="DmPrlc.65" ana="прич;м;наст;род;ед;ж" reg="будущая" src="БUДUЩАА">бѹдѹщаа</w>
        <w xml:id="DmPrlc.66" ana="посл" lemma="ради" reg="ради" src="РАДИ">ради</w>
        <w xml:id="DmPrlc.67" ana="сущ;i;род;ед;ж" lemma="жизнь" reg="жизни" src="ЖИЗНИ">жизни</w>
        <pc xml:id="DmPrlc.68">.</pc>
        <w xml:id="DmPrlc.69" ana="мест;м;вин;мн;ср" lemma="иже" reg="яже" src="ЯЖЕ">ꙗже</w>
        <lb n="13"/>
        <w xml:id="DmPrlc.70" ana="сущ;o;им;ед;м" lemma="богъ" reg="богъ" src="БЪ#">бъ҃</w>
        <w xml:id="DmPrlc.71" ana="гл;изъяв;аор гл;3;ед" lemma="уготовати" reg="уготова" src="UГОТОВА">ѹготова</w>
        <w xml:id="DmPrlc.72" ana="прич;м;наст;дат;мн;м" reg="любящимъ" src="ЛЮБRЩИМЪ">любѧщимъ</w>
        <w xml:id="DmPrlc.73" ana="мест;м;вин/род;ед;м" lemma="и" reg="его" src="ЕГО">его</w>
        <pc xml:id="DmPrlc.74">.</pc>
        <w xml:id="DmPrlc.75" ana="союз" lemma="аще" reg="аще" src="АЩЕ">аще</w>
        <lb n="14"/>
        <w xml:id="DmPrlc.76" ana="союз" lemma="бо" reg="бо" src="БО">бо</w>
        <w xml:id="DmPrlc.77" ana="прил;тв;им;мн;м;*" lemma="еллинскии" reg="еллинстии" src="ЕЛЛИНСТIИ">еллинстїи</w>
        <w xml:id="DmPrlc.78" ana="сущ;jo;им;мн;м" lemma="баснотворець" reg="баснотворцы" src="БАСНОТВОРЦИ">баснотворци</w>
        <pc xml:id="DmPrlc.79">.</pc>
        <w xml:id="DmPrlc.80" ana="союз" lemma="и" reg="и" src="И">и</w>
        <w xml:id="DmPrlc.81" ana="мест;м;им;мн;м" lemma="иже" reg="иже" src="ИЖЕ">иже</w>
        <lb n="15"/>
        <w xml:id="DmPrlc.82" ana="пред" lemma="съ" reg="съ" src="С">с</w>
        <w xml:id="DmPrlc.83" ana="мест;м;тв;мн;м" lemma="и" reg="ними" src="НИМИ">ними</w>
        <w xml:id="DmPrlc.84" ana="прил;тв;им;мн;м" lemma="нечестивыи" reg="нечестивии" src="НЕЧЕСТИВIИ">нечестивїи</w>
        <w xml:id="DmPrlc.85" ana="сущ;o;им;мн;м;*" lemma="языкъ" reg="языцы" src="RЗЫЦИ">ѧзыци</w>
        <pc xml:id="DmPrlc.86">.</pc>
        <w xml:id="DmPrlc.87" ana="част" lemma="не" reg="не" src="НЕ">не</w>
        <w xml:id="DmPrlc.88" ana="прич;jo;наст;им;мн;м" reg="знающи" src="ЗНА&amp;ЮЩИ">зна<lb n="16"/>ющи</w>
        <w xml:id="DmPrlc.89" ana="сущ;o;вин/род;ед;м" lemma="богъ" reg="бога" src="БА#">ба҃</w>
        <w xml:id="DmPrlc.90" ana="сущ;jo;вин/род;ед;м" lemma="творець" reg="творца" src="ТВОРЦА">творца</w>
        <w xml:id="DmPrlc.91" ana="прил;тв;дат;мн;ср" lemma="всяческии" reg="всяческимъ" src="ВСRЧЕСКИМЪ">всѧческимъ</w>
        <pc xml:id="DmPrlc.92">.</pc>
        <w xml:id="DmPrlc.93" ana="сущ;o;вин;мн;м" lemma="праздникъ" reg="праз(д)ники" src="ПРА&amp;З(Д)НИКИ">пра<lb n="17"/>зⷣники</w>
        <w xml:id="DmPrlc.94" ana="союз" lemma="и" reg="и" src="И">и</w>
        <w xml:id="DmPrlc.95" ana="сущ;jo;вин;ед;ср" lemma="поклонение" reg="поклонение" src="ПОКЛОНЕНIЕ">поклоненїе</w>
        <w xml:id="DmPrlc.96" ana="сущ;o;дат;мн;м" lemma="идолъ" reg="идоломъ" src="ИДОЛОМЪ">идоломъ</w>
        <w xml:id="DmPrlc.97" ana="прич;jo/en;наст;им;мн;м" reg="приносяще" src="ПРИНО&amp;СRЩЕ">прино<lb n="18"/>сѧще</w>
        <pc xml:id="DmPrlc.98">.</pc>
        <w xml:id="DmPrlc.99" ana="мест;тв;вин;ед;ср" lemma="елико" reg="елико" src="ЕЛИКО">елико</w>
        <w xml:id="DmPrlc.100" ana="мест;тв;им;ед;м" lemma="кождо" reg="кож(д)о" src="КОЖ(Д)О">кожⷣо</w>
        <w xml:id="DmPrlc.101" ana="мест;м;род;мн;м" lemma="и" reg="ихъ" src="ИХЪ">ихъ</w>
        <w xml:id="DmPrlc.102" ana="гл;изъяв;имп;3;мн" reg="можаху" src="МОЖААХU">можаахѹ</w>
        <lb n="19"/>
        <w xml:id="DmPrlc.103" ana="прич/в;en;наст;им;мн;м" reg="тщащеся" src="ТЩАЩЕСR">тщащесѧ</w>
        <pc xml:id="DmPrlc.104">,</pc>
        <w xml:id="DmPrlc.105" ana="пред" lemma="отъ" reg="о(т)" src="W(Т)">ѿ</w>
        <w xml:id="DmPrlc.106" ana="сущ;jo;род;мн;ср" lemma="им+ние" reg="им+нии" src="ИМ+НIИ">имѣнїи</w>
        <w xml:id="DmPrlc.107" ana="мест;м;род;мн;ср" lemma="свои" reg="своихъ" src="СВОИ(Х)">своиⷯ</w>
        <pc xml:id="DmPrlc.108">.</pc>
        <w xml:id="DmPrlc.109" ana="мест;тв;дат;мн;м" lemma="тои" reg="т+мъ" src="Т+(М)">тѣⷨ</w>
        <lb n="20"/>
        <w xml:id="DmPrlc.110" ana="пред" lemma="безъ" reg="безъ" src="БЕЗ">без</w>
        <w xml:id="DmPrlc.111" ana="сущ;o;род;ед;м" lemma="умъ" reg="ума" src="UМА">ѹма</w>
        <w xml:id="DmPrlc.112" ana="гл;изъяв;имп;3;мн" reg="пре(д)лагаху" src="ПРЕ(Д)ЛАГАХU">преⷣлагахѹ</w>
        <pc xml:id="DmPrlc.113">.</pc>
        <w xml:id="DmPrlc.114" ana="нар" lemma="колми" reg="колми" src="КОЛМИ">колми</w>
        <w xml:id="DmPrlc.115" ana="част" lemma="же" reg="же" src="ЖЕ">же</w>
        <w xml:id="DmPrlc.116" ana="мест;личн;1;дат;мн" lemma="мы" reg="намъ" src="НА(М)">наⷨ</w>
        <pc xml:id="DmPrlc.117">.</pc>
        <lb n="21"/>
        <w xml:id="DmPrlc.118" ana="мест;м;им;мн;м" lemma="иже" reg="иже" src="ИЖЕ">иже</w>
        <w xml:id="DmPrlc.119" ana="прил;тв;тв;ед;ж" lemma="божественыи" reg="божественою" src="БЖ(с)ТВЕНОЮ">бж҇ⷭтвеною</w>
        <w xml:id="DmPrlc.120" ana="сущ;i;тв;ед;ж" lemma="благодать" reg="благодатию" src="БЛГОДАТIЮ#">бл҃годатїю</w>
        <w xml:id="DmPrlc.121" ana="пред" lemma="отъ" reg="о(т)" src="W(Т)">ѿ</w>
        <w xml:id="DmPrlc.122" ana="сущ;i/о;род;ед;м" lemma="господь" reg="господа" src="ГА#">га҃</w>
        <w xml:id="DmPrlc.123" ana="сущ;o;род;ед;м" lemma="богъ" reg="бога" src="БА#">ба҃</w>
        <lb n="22"/>
        <w xml:id="DmPrlc.124" ana="прич;тв;прош;дат;мн;м" lemma="пос+тити" reg="пос+щенымъ" src="ПОС+ЩЕНЫМЪ">посѣщенымъ</w>
        <pc xml:id="DmPrlc.125">.</pc>
        <w xml:id="DmPrlc.126" ana="гл;изъяв;н/б;3;ед;3" reg="подобаеть" src="ПОДОБАЕТЬ">подобаеть</w>
        <w xml:id="DmPrlc.127" ana="нар" lemma="достоино" reg="достоино" src="ДОСТО&amp;ИНО">досто<lb n="23"/>ино</w>
        <w xml:id="DmPrlc.128" ana="сущ;i;вин;мн;ж" lemma="память" reg="памяти" src="ПАМRТИ">памѧти</w>
        <w xml:id="DmPrlc.129" ana="прил;тв;род;мн;м" lemma="святыи" reg="святыхъ" src="СТЫ(Х)#">ст҃ыⷯ</w>
        <w xml:id="DmPrlc.130" ana="сущ;jo;род;мн;м" lemma="отець" reg="отець" src="W(Т)ЦЬ#">ѿ҃ць</w>
        <w xml:id="DmPrlc.131" ana="нар" lemma="духовно" reg="духовно" src="ДХОВНО#">дх҃овно</w>
        <w xml:id="DmPrlc.132" ana="инф" lemma="празновати" reg="празновати" src="ПРАZ -201 ЗНОВАТИ">пра<pb n="-201"/><lb n="1"/>зновати</w>
        <w xml:id="DmPrlc.133" ana="союз" lemma="и" reg="и" src="И">и</w>
        <w xml:id="DmPrlc.134" ana="пред" lemma="отъ" reg="о(т)" src="W(Т)">ѿ</w>
        <name><w xml:id="DmPrlc.135" ana="сущ;o;род;ед;м" lemma="*христосъ" reg="*христа" src="*ХА#">ха҃</w></name>
        <w xml:id="DmPrlc.136" ana="сущ;o;род;ед;м" lemma="богъ" reg="бога" src="БА#">ба҃</w>
        <w xml:id="DmPrlc.137" ana="прич;тв;прош;вин;мн;ср" lemma="даровати" reg="дарованная" src="ДАРОВАННАА">дарованнаа</w>
        <w xml:id="DmPrlc.138" ana="мест;тв;дат;мн;м" lemma="тои" reg="т+мъ" src="Т+(М)">тѣⷨ</w>
        <pc xml:id="DmPrlc.139">,</pc>
        <lb n="2"/>
        <w xml:id="DmPrlc.140" ana="сущ;es;вин;мн;ср" lemma="чюдо" reg="чюдеса" src="ЧЮДЕСА">чюдеса</w>
        <w xml:id="DmPrlc.141" ana="инф" lemma="похвалити" reg="похвалити" src="ПОХВАЛИТИ">похвалити</w>
        <w xml:id="DmPrlc.142" ana="пред" lemma="въ" reg="въ" src="ВЪ">въ</w>
        <w xml:id="DmPrlc.143" ana="сущ;о;мест;мн;м;+о" lemma="псаломъ" reg="псалм+хъ" src="QАЛМ+(Х)">ѱалмѣⷯ</w>
        <w xml:id="DmPrlc.144" ana="союз" lemma="и" reg="и" src="И">и</w>
        <w xml:id="DmPrlc.145" ana="сущ;jo;мест;мн;ср" lemma="п+ние" reg="п+ниихъ" src="П+&amp;НIИХЪ">пѣ<lb n="3"/>нїихъ</w>
        <w xml:id="DmPrlc.146" ana="прич;jo/en;наст;им;мн;м" reg="служаще" src="СЛGЖАЩЕ">слѫжаще</w>
        <w xml:id="DmPrlc.147" ana="сущ;i/u;дат;ед;м" lemma="господь" reg="господеви" src="ГВИ#">гв҃и</w>
        <pc xml:id="DmPrlc.148">.</pc>
        <w xml:id="DmPrlc.149" ana="сущ;i;род;ед;ж" lemma="память" reg="памяти" src="ПАМRТИ">памѧти</w>
        <w xml:id="DmPrlc.150" ana="мест;м;род;мн;м" lemma="и" reg="ихъ" src="И(Х)">иⷯ</w>
        <lb n="4"/>
        <w xml:id="DmPrlc.151" ana="посл" lemma="ради" reg="ради" src="РАДИ">ради</w>
        <w xml:id="DmPrlc.152" ana="гл;изъяв;н/б;3;ед;3" reg="подобаетъ" src="ПОДОБАЕТЪ">подобаетъ</w>
        <w xml:id="DmPrlc.153" ana="мест;м;вин;мн;ср" lemma="иже" reg="яже" src="ЯЖЕ">ꙗже</w>
        <w xml:id="DmPrlc.154" ana="гл;изъяв;аор гл;1;мн" lemma="вид+ти" reg="вид+хомъ" src="ВИД+ХОМЪ">видѣхомъ</w>
        <pc xml:id="DmPrlc.155">.</pc>
        <lb n="5"/>
        <w xml:id="DmPrlc.156" ana="союз" lemma="и" reg="и" src="И">и</w>
        <w xml:id="DmPrlc.157" ana="союз" lemma="аще" reg="аще" src="АЩЕ">аще</w>
        <w xml:id="DmPrlc.158" ana="союз" lemma="и" reg="и" src="И">и</w>
        <w xml:id="DmPrlc.159" ana="прил;м;вин;мн;ср" lemma="прежнии" reg="прежняя" src="ПРЕЖНRА">прежнѧа</w>
        <w xml:id="DmPrlc.160" ana="союз" lemma="и" reg="и" src="И">и</w>
        <w xml:id="DmPrlc.161" ana="гл;изъяв;аор гл;1;мн" lemma="слышати" reg="слышахомъ" src="СЛЫШАХОМЪ">слышахомъ</w>
        <pc xml:id="DmPrlc.162">.</pc>
        <w xml:id="DmPrlc.163" ana="сущ;jo;дат;ед;ср" lemma="писание" reg="писанию" src="ПИ&amp;САНIЮ">пи<lb n="6"/>санїю</w>
        <w xml:id="DmPrlc.164" ana="прил;тв;вин;мн;ср" lemma="истинныи" reg="истинная" src="ИСТИННАА">истиннаа</w>
        <w xml:id="DmPrlc.165" ana="инф" lemma="положити" reg="положити" src="ПОЛОЖИТИ">положити</w>
        <pc xml:id="DmPrlc.166">.</pc>
      </ab></body></text>
    </TEI>
\end{Verbatim}


\end{document}
